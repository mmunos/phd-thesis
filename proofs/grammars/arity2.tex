\paragraph{Conditions 1 and 2: removing useless nonterminals}
We first perform a linear-time exploration from the terminals to mark the
nonterminals $X$ that can derive some string of terminals. The base case is if a nonterminal
$X$ has a rule $X \rightarrow \alpha$ where $\alpha$ only consists of terminals
(in particular $\alpha = \epsilon$), then we mark it.
The induction is that whenever a nonterminal $X$ has a rule $X
\rightarrow \alpha$ where $\alpha$ only consists of terminals and of
marked nonterminals, then we mark~$X$. At the end of this process, it is clear that any nonterminal
that is not known to derive a string of terminals indeed does not derive any
string, because any derivation of a string of terminals from a nonterminal $X$
would witness that all nonterminals in this derivation, including $X$, should
have been marked, which is impossible. Hence, we can remove the nonterminals
that are not marked without changing the language or
successful derivations of the grammar, and satisfy condition 1 in linear time.

Second, we perform a linear-time exploration from the start symbol $S$ to mark the
nonterminals $X$ that can be reached in a derivation from $S$. The base case is
that $S$ is marked. The induction is that whenever a nonterminal $Y$ occurs in
the right-hand side of a rule having $X$ as its left-hand side, and $X$ is
marked, then we mark $Y$. At the end of the process, if a nonterminal $X$ is
not marked, then indeed there is no derivation from $S$ that produces a string
featuring $X$, as otherwise it would witness that $X$ is marked, which is
impossible. Hence, we can again remove the nonterminals that are not marked,
the grammar and successful derivations are again unchanged, and we
satisfy condition 2 in linear time.

As the transformations here only remove nonterminals and rules that cannot
appear in a derivation, they clearly preserve unambiguity as well as rigidity.

\paragraph{Condition 3: shape of rules}
We first ensure that every right-hand side of a rule is of size $\leq 2$.
Given the annotated grammar $\mathcal{G}$, for every rule $X \rightarrow \alpha$ where
$|\alpha| > 2$, letting $\alpha = \alpha_1\cdots\alpha_n$,
we introduce $n-2$ fresh nonterminals $X_{\alpha,1},
\ldots, X_{\alpha,n-2}$, and replace the rule by the following: $X
\rightarrow \alpha_1 X_{\alpha,1}$, $X_{\alpha,1} \rightarrow \alpha_2
X_{\alpha,2}$, ..., $X_{\alpha,n-2} \rightarrow \alpha_{n-1} \alpha_{n}$.

We make sure that the right-hand side of rules of size 2 consist only of
nonterminals by introducing fresh intermediate nonterminals whenever
necessary, which rewrite to the requisite terminal.

%
%
%
It is then clear that the result satisfies condition 3, and that there is
a one-to-one correspondence between derivations in the original grammar and derivations in the rewritten grammar. To see this, note that there is an obvious one-to-one function which maps derivations from the original grammar into derivations in the new grammar, and that there is a slightly more involved function which receives a derivation in the new grammar, and builds a derivation in the original grammar by following the steps detailed above (and using the fact that each fresh nonterminal is associated to exactly one rule), which is also one-to-one. We conclude that the original grammar is unambiguous if and only if the new grammar is unambiguous.

%
The last point to check is that the arity-2 transformation preserves rigidity,
i.e., if the original annotated grammar is rigid then so is the image of the
transformation. Let $X$ be some symbol of the original grammar $\cG$, and
$w \in \Sigma^*$ be a string. Let us show that all derivations from the
corresponding
symbol $X'$ of the rewritten grammar $\cG'$ have same shape. We do so by
induction on the length of~$w$ and then on the topological order on
nonterminals. The base case of $w$ of length $0$ is clear: the possible
derivations are sequences of applications of rules of the form $Y \rightarrow Z$
in a sequence of some fixed length, followed by a rule of the form $Y
\rightarrow \epsilon$, and what can happen in the rewritten grammar is the same.

For the inductive case, as $\cG$ is rigid, we know that there must be one
fixed profile $\pi \in \{0, 1\}^k$ such that all derivations of~$w$ from~$X$ start by the
application of a rule $X \rightarrow \alpha$ where $\alpha$ corresponds to
profile~$\pi$, i.e., it has length~$k$ and its $i$-th character is a nonterminal
or terminal according to the value of the $i$-th bit of~$\pi$.
Otherwise the existence of two different right-hand-side profiles would
contradict rigidity. Furthermore, by considering the possible sub-derivations
from~$\alpha_1$ (including the empty derivation if $\alpha_1$ is a terminal), we
know that $\alpha_1$ derives some fixed prefix of~$w$ and that all
such derivations have the same sequence of profiles; otherwise we would witness
a contradiction to rigidity. By applying the same argument successively to
$\alpha_2,
\ldots, \alpha_k$, we deduce
that there must be a partition of $w =
w_1 \cdots w_k$ such that, in all derivations of~$w$ from~$X$, the derivation
applies a rule with right-hand having profile $\pi$ to produce some string $\alpha_1 \cdots
\alpha_k$, and then each $\alpha_i$ derives an annotation of~$w_i$ and
for each $i$ all possible derivations of some annotation of~$w_i$ by some $i$-th
element in the right-hand size of such a rule has the same sequence of
profiles.

As the string is nonempty we know that $k > 0$. Further, if $k = 1$ then $X$ and
the productions involving $X$ were not rewritten so we immediately conclude
either with the case of a rule $X \rightarrow \tau$ for a terminal $\tau$ or by
induction hypothesis on the nonterminals in the topological order for the case
of a rule of the form $X \rightarrow Y$. Hence, we assume that $k \geq 2$.

We know by induction that, in the rewritten grammar, the derivation from $X$
will start by rewriting $X$ to $Y_1 X_{\alpha, 1}$, the $X_{\alpha, 1}$ being
itself rewritten to $Y_2 X_{\alpha, 2}$, and so on, for some right-hand size
$\alpha$ of a rule $X \rightarrow \alpha$ having profile $\pi$. Clearly each
$Y_i$ will have to derive an annotation of the $w_i$ in the partitioning of~$w$, as
a derivation following a different partitioning would witness a derivation in
the original grammar that contradicts rigidity.
Now, the profile
$\pi$ indicates if each $Y_i$ is a nonterminal of the initial grammar or a fresh
nonterminal introduced to rewrite to a terminal. In the latter case, there is no
possible deviation in profiles. In the former case, we conclude by induction
hypothesis that each $Y_i$ derives annotations of its~$w_i$ that all have the
same profile, and we conclude that all derivations in the rewritten grammar
indeed have the same profile, concluding the proof.~\qedhere


%
%
%
%
%
%
%
%
%
%
%
%
%
%
%
%
%
%
%
%
%
%
%
%
%
%
%
%
%
%
%
%
%
%
%
%
%
%
%
%
%
%
%
%
%
%
%
%
%
%
%
%
%
%
%
%
%
%
%
%

%
%


%
%
%
%
%
%
%
%
%
%
%
%
%
%
%
%
%
%
%
%

%
%
%
%
%
%
%
%
%

%

%

