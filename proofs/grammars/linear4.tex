
Consider a deterministically profiled PDAnn $\cP$. To prove that it is profiled, consider an input string $w\in \Sigma^*$. We will prove by a simple induction argument that any two runs of $\cP$ over $w$ have the same profile. The base case is trivial since the run is of length 0, and the profile up to now is composed simply of the stack size 0. Assume now that for each pair of runs $\rho$ and $\rho'$ of $\cP$ over $w$ of size $k$, that they have the same profile. We will show that for every pair of runs $\rho_1$ and $\rho_2$ over $w$ of size $k+1$, they have the same profile as well. Note that the runs $\rho_1^-$ and $\rho_2^-$ that are obtained by removing the last step have the same profile, by the hypothesis. From the definition of deterministically profiled it can be directly seen that if (1) the last transition in $\rho_1$ is a read or read-write transition, then for the runs $\rho_1^-$ and $\rho_2^-$, the only choices are read or read-write transitions, from which we deduce that the last transition in $\rho_2$ is a read or read-write transition as well, if (2) the last transition in $\rho_1$ is a push transition, then for the run $\rho_1^-$ and $\rho_2^-$ the only choice are push transitions, and therefore the last transition in $\rho_2$ has to be a push transition as well, and if (3) the last transition in $\rho_1$ is a pop transition, then for $\rho_1^-$ and $\rho_2^-$ the only choices are pop transitions, so the last transition in $\rho_2$ must be a pop transition as well. We obtain that $\rho_1$ and $\rho_2$ have the same profile, and from the induction argument, we conclude that $\cP$ is profiled.

Now, consider a deterministically profiled PDAnn $\cP$ and an input string $w$. We will prove that the unique profile of accepting runs of $\cP$ over $w$ can be computed in linear time in $|w|$. The way we do this is by using the pushdown automaton $\cA$ that was constructed in Proposition~\ref{gram:prp:basedeterm}. By inspecting the proof, it can be seen that the unique profile of $\cP$ over $w$ is maintained throughout the construction. Indeed, the first construction simply removes the output symbols, which does not affect the profile, and the second construction has an invariant that keeps the profile intact as well. Therefore, by running the automaton $\cA$ over $w$, and storing the stack sizes at each step, we obtain a profile $\pi$ which is exactly the same profile of the accepting runs of $\cP$ over $w$. To finish the proof, we only need to argue that this profile has linear size on $|w|$ (from a data complexity perspective). This follows from the fact that any run of a deterministic pushdown automaton $\cA$ over a string $w$ has $\cO(f(\cA)\times|w|)$ length, for some computable function $f$. This can be seen from a counting argument: (1) There is a maximum stack size $k$ that can be reached in an accepting run of $\cP$ over $w$ from an empty stack through $\eps$-transitions, which is given by the number of states in $\cA$. Otherwise, there are two configurations which are reachable from one another in a way such the stack, as it was at the first configuration, is not seen. This implies that there is a loop, and since $\cA$ is deterministic, $\cA$ does not accept $w$. (2) From a given stack, the maximum numbers of steps that can be taken without reading from $w$, and without seeing the topmost symbol on the stack is given by the number of possible stacks of size $k$. (3) Between a read (or read-write) transition and the next one, the maximum height difference is $k$, and if we move out of a read (or read-write) transition with a certain stack, from (2) we can see that we can only do a fixed number of steps before consuming some symbol from this stack, and therefore, the number of steps is bounded by a factor depending on $\cA$ multiplied by the size of the stack up until this point, which is linear on the number of symbols in $w$ read so far. We conclude that $w'$ has size linear on $w$, from a data complexity point of view.

