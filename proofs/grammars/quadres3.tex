We first show the undecidability of checking if an annotated grammar has an
equivalent rigid annotated grammar:

\begin{claim}
  \label{gram:prp:unambundec}
  Consider the problem, given an annotated grammar~$\cG$, of determining whether
  there exists some equivalent rigid annotated grammar equivalent to~$\cG$.
  This problem is undecidable.
\end{claim}

\begin{proof}
  We reduce from the problem of deciding whether the language $L_2$ of an input
  (unannotated) context-free grammar $\cG_2$ can be recognized by an unambiguous
  context-free grammar: this task is known to be
  undecidable~\cite{ginsburg1966ambiguity}. Consider $\cG_2$ as an annotated
  grammar (with empty annotations). Let us show that $L_2$ can be recognized by
  a rigid annotated grammar iff it can be recognized by an
  unambiguous context-free grammar, which concludes. For the forward direction,
  if $L_2$ can be recognized
  by an unambiguous context-free grammar, then that grammar is in particular
  rigid. 
  For the backward direction, if $L_2$ can be recognized by a rigid
  grammar, then
  Proposition~\ref{gram:prp:baseunambig} implies that $L_2$ can also be recognized by
  an unambiguous context-free grammar.
  Thus, we have showed that the (trivial) reduction is correct.
\end{proof}

We next show that it is undecidable to check if an input annotated grammar is
rigid:

\begin{claim}
  \label{gram:prp:unambundecb}
  Consider the problem, given an annotated grammar~$\cG$, of determining whether
  it is rigid. This problem is undecidable.
\end{claim}


\begin{proof}
  We adapt the standard proof of undecidability~\cite[Ambiguity Theorem
  2]{chomsky1959algebraic} for the problem of deciding, given an input
  unannotated grammar~$\cG$, if it is unambiguous. The reduction is from the Post
  Correspondence Problem (PCP), which is undecidable: we are given as input sequences $\alpha_1, \ldots, \alpha_n$
  and $\beta_1, \ldots, \beta_n$ of strings over some alphabet~$\Sigma$, and we ask whether
  there is a non-empty sequence of indices $i_1, \ldots, i_m$ of integers in $[1, n]$ such that
  $\alpha_{i_1} \ldots \alpha_{i_m} = \beta_{i_1} \ldots \beta_{i_m}$. Given the
  input sequences $\alpha_1, \ldots, \alpha_n$ and $\beta_1, \ldots, \beta_n$ to
  the PCP, we
  consider the alphabet $\Sigma' = \Sigma \cup \{1, \ldots, n\}$, and we
  consider the CFG having nonterminals $S$, $S_1$, and $S_2$, start symbol $S$,
  and rules $S \rightarrow S_1$, $S \rightarrow S_2$, $S_1 \rightarrow
  \epsilon$, $S_2\rightarrow \epsilon$, and for each $1 \leq
  i \leq n$ the productions $S_1 \rightarrow \alpha_i S_1 i$ and $S_2
  \rightarrow \beta_i S_2 i$.

  We claim that this grammar is ambiguous iff there is a solution to the Post
  correspondence problem. Indeed, given any solution $\alpha_{i_1} \cdots
  \alpha_{i_m} = \beta_{i_1} \ldots \beta_{i_m}$, considering the string 
$\alpha_{i_1} \cdots
  \alpha_{i_m} i_m \cdots i_1 = \beta_{i_1} \ldots \beta_{i_m} i_m \cdots i_1$,
  we can parse it with one derivation  featuring $S_1$ and one derivation
  featuring $S_2$. Conversely, if we can parse a string $w \in \Sigma^*$ with two
  different derivations, we know that there cannot be two different derivations
  featuring $S_1$. Indeed, reading the string from right to left uniquely
  identifies the possible derivations from $S_1$. The same argument applies to
  derivations featuring $S_2$. Hence, if the grammar is ambiguous, then there is
  exactly one derivation featuring $S_1$ and exactly one derivation featuring
  $S_2$. These two derivations can be used to find a solution to the Post
  correspondence problem.

  We now adapt this proof to show the undecidability of rigidity. We
  say that an input to the PCP is \emph{trivial} if there is $i$ such that
  $\alpha_i = \beta_i$. We can clearly decide in linear time, given the input to
  the PCP, if it is trivial. Hence, the PCP is also undecidable in the case
  where the PCP is non-trivial. Now, when doing the reduction above on a PCP
  instance that is not trivial, we observe that two derivations of the same string
  can never have the same sequences of shapes. Indeed, if we have two derivations of the same
  string, then as we explained one must feature $S_1$ and the other must feature
  $S_2$, and they give a solution $\alpha_{i_1} \cdots \alpha_{i_m} =
  \beta_{i_1} \ldots \beta_{i_m}$ to the PCP. Assume by contradiction that both
  derivations have the same sequences of shapes. Then, 
  it means that we have $\card{\alpha_{i_j}} = \card{\beta_{i_j}}$ for every $1
  \leq j \leq m$. In particular we have $\card{\alpha_{i_1}} =
  \card{\beta_{i_1}}$, and so we know that $\alpha_{i_1} =
  \beta_{i_1}$ and the PCP instance was trivial, a contradiction.

  Hence, let us reduce from the PCP on non-trivial instances to the problem of
  deciding whether an input annotated grammar is not rigid. Given a
  non-trivial PCP instance, we construct $\cG$ as above, but seeing it as an
  annotated grammar with no outputs. Then $\cG$ is not rigid iff there is a
  string $w$ such that the empty annotation of $w$ has two derivations that do not
  have the same sequence of shapes. But this is equivalent to $\cG$ being
  unambiguous when seen as a CFG. Indeed, for the forward direction, if $\cG$ has
  two such derivations on a string~$w$ then clearly $w$ witnesses that $\cG$ is
  ambiguous when seen as a CFG. Conversely, if $\cG$ is ambiguous when seen as a
  CFG, we have explained in the previous paragraph that the two derivations must
  have different sequences of shapes, so $\cG$ is not rigid. Hence, we
  conclude that there is a solution to the input non-trivial PCP instance iff
  $\cG$ is not rigid. This establishes that the problem is undecidable
  and concludes the proof.
\end{proof}
