%

We know from~\cite{AbboudBW18} that for any $c>0$, there exists a fixed
grammar $\cG$ such that determining whether a string $w$ is
derived by $\cG$, cannot be solved in time $\cO(|w|^{\omega-c})$, unless the conjecture in graph algorithms mentioned in~\cite{AbboudBW18} is false.


We will see that this conditional lower bound translates to unambiguous
annotated grammars. Indeed, we will show that for each grammar
$\cG$ there exists an unambiguous annotated grammar
$\mathcal{G}'$ such that $w$ is derived by $\cG$ if and only
if $\sem{\mathcal{G}'}(w)$ is non-empty. Therefore after the
preprocessing of $w$ for $\mathcal{G}'$, we know in constant time
whether $w$ is derived by $\cG$ which proves that the
preprocessing of $\mathcal{G}'$ on $w$ requires $\cO(|w|^{\omega-c})$
time, assuming the conjecture is true.

Now let us show how to translate a grammar $\cG$ into an
unambiguous annotated grammar $\mathcal{G}'$. This can be challenging, because
$\cG$ is not necessarily unambiguous: for this reason we need to define
$\mathcal{G}'$ intuitively by adding annotations that disambiguate the various
possible derivations of $\cG$, to guarantee that the result is
unambiguous. As this is cumbersome to do on grammars, we use the
correspondence between annotated grammars and pushdown annotators
(Proposition~\ref{gram:prop:grammar-pdann}), shown later in the article. 

In this proof, we will use the notion of \emph{pushdown automata} (PDA); see
Definition~\ref{gram:def:pda} for the formal definition.
Let us consider a PDA $\mathcal{P}$ 
which is equivalent to $\cG$.
As is standard with PDAs, we can change the given definition to suppose without
loss of generality that no transition in $\mathcal{P}$ is an
$\epsilon$-transition. Specifically, we consider PDAs in a slightly different
model where transitions 
are of the form
$(q_1,a,s_1,q_2,s_2)\in Q\times \Sigma \times \Gamma^+ \times
Q \times \Gamma^+$: such a transition means that in state $q_1$, when the top
stack symbols are $s_1$ and the next letter to read is~$a$, the automaton can
read the letter, move to state
$q_2$ and replace $s_1$ by $s_2$ on the stack. We create our
unambiguous PDAnn $\mathcal{P}'$ from $\mathcal{P}$ by replacing
each transition $t=(q_1,a,s_1,q_2,s_2)$ to a set of transitions that
first pop the symbols of $s_1$ from the stack, then reads $a$, then
pushes the symbols of $s_2$ onto the stack. The first state of this
transition is $q_1$, the last state is $q_2$ but we make sure that
each of the intermediate states are unique to $t$. Furthermore, the
transition that reads the letter $a$ outputs a symbol unique to the
transition $t$. Therefore, by construction there is a bijection
between runs of $\mathcal{P}$ and runs of $\mathcal{P}'$ and the 
PDAnn $\mathcal{P}'$ is unambiguous because the run used for each
output can be retrieved from that output.

We conclude by using Proposition~\ref{gram:prop:grammar-pdann} to obtain an equivalent annotated
grammar $\cG'$, which is also unambiguous. Thus, we know that on any unannotated
string $w$, the set $\sem{\cG'}(w)$ is empty if $\cG$ does not derive
$w$, and non-empty if it does. Thus, we know that, if we assume the conjecture is true, we cannot determine
in $\cO(|w|^{\omega-c})$ whether $\sem{\cG'}(w)$ is empty or not. But if we have
an algorithm to enumerate $\sem{\cG'}(w)$  with output-linear delay, as any
output has size $\cO(|w|)$ in~$|w|$, we can do this with a complexity linear in~$|w|$
which is that of the preprocessing of the enumeration algorithm. Thus, we
conclude that the preprocessing conditionally requires $\Omega(|w|^{\omega-c})$ time.

%
%
%
%
%
%
%

%
%

%
%
%
%
%
%


%
%
%

%
%
%
%
%
%
