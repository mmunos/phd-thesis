Recall that, in the statement of this result, the formal notion of an
\emph{equivalent annotated grammar} is the one defined above.
% (Definition~\ref{def:equivalentannotated}). 
Recall also the formal definition of
\emph{ref-words} (see Section~\ref{prelims:sec:spanners}).

%
%
%
%
%
%

Let $r$ be a ref-word in $\Sigma \cup \captures_\var$ and let $\hat{w}$ be an annotated string in $\Sigma \cup (\Sigma \times \Omega_\var)$. We say that $r$ and $\hat{w}$ are \emph{equivalent} if $\splain(r\cdot\#) = \str(\hat{w})$ and $\outf(\eta^{r}) = \ann(\hat{w})$. 
For example, the ref-word $r_1 = \ \vdash_x\!\texttt{a}\texttt{a}\!\dashv_x\vdash_y\!\texttt{b}\texttt{b}\!\dashv_y\!\texttt{b}$ is equivalent to $\hat{w}_1 = (\texttt{a},\{\vdash_x\})\,\texttt{a}\,(\texttt{b},\{\dashv_x,\vdash_y\})\,\texttt{b}\,\texttt{b}\,(\texttt{b},\{\dashv_y\})\,\#$.

The overall strategy of this proof is going to be to construct an annotated grammar $\cG$ in a way such that for every ref-word $r\in L(\cH)$ there exists an equivalent annotated string $\hat{w}\in L(\cG)$, and vice versa. It is clear that this implies that $\cG$ and $\cH$ are equivalent.

The way we build $\cG$ will look like we are ``pushing'' the variable operations to the next terminal to the right. 
We will do this process one variable operation at a time.

First, we need to define an intermediate model between those of extraction grammars and annotated grammars. 
We define \emph{extraction grammars with annotations} as a straightforward
extension of extraction grammars which allow annotations on terminals that are
not variable operations. 
For the set of variables $\var$, recall from Section~\ref{prelims:sec:spanners} that
we define the variable operations  of $\var$ by $\captures_\var = \{ {\vdash_x},
{\dashv_x} \mid x\in \var\}$,
and recall from Definition~\ref{gram:def:outputset} that we define $\Omega_\var = 2^{\captures_\var}$. An \emph{extraction grammar with annotations} is a tuple
$\cF = (V, \Sigma, \var, P, S)$ where 
$V$ is a finite set of nonterminal symbols, $\Sigma$ is an alphabet and $\var$ is a set of variables, such that $V$, $\Sigma$, $\Sigma\times\Omega_\var$, and $\captures_\var$ are pairwise disjoint, 
$P$ is a finite set of rules of the form $A \to \alpha$ with $A \in
V$ and $\alpha \in (V \cup \Sigma \cup (\Sigma\times\Omega_\var) \cup \captures_\var)^*$, and $S \in V$ is the
start symbol. As in the other models, the semantic of extraction grammars is defined through derivations. Specifically,  the set $P$ defines the (left) derivation relation $\der{\cF} \ \subseteq \, (V \cup \Sigma  \cup (\Sigma\times\Omega_\var) \cup \captures_\var)^* \times (V \cup \Sigma  \cup (\Sigma\times\Omega_\var) \cup  \captures_\var)^*$ such that $\hat{u} A \beta \der{\cF} \hat{u} \alpha \beta$ iff $\hat{u} \in (\Sigma  \cup (\Sigma\times\Omega_\var) \cup \captures_\var)^*$, $A \in V$, $\alpha, \beta \in (V \cup \Sigma \cup (\Sigma\times\Omega_\var) \cup \captures_\var)^*$, and $A \rightarrow \alpha \in P$. We denote by $\der{\cF}^*$ the reflexive and transitive closure of $\der{\cF}$. Then the language defined by $\cF$ is $L(\cF) = \{\hat{w} \in (\Sigma \cup (\Sigma\times\Omega_\var) \cup \captures_\var)^* \mid S \der{\cF}^* \hat{w}\}$. In addition, we assume that no string $\hat{w}$ in $L(\cF)$ has a variable operation as its last symbol.

An extraction grammar with annotations generates
strings over $\Sigma \cup (\Sigma\times\Omega_\var) \cup \captures_\var$, which we now refer to as \emph{annotated ref-words}, and each annotated ref-word defines an output.
We will define the semantics of extraction grammars with annotations recursively by using the semantics of annotated grammars as a starting point, that is, by extending the function $\ann$ to receive strings over $\Sigma \cup (\Sigma\times\Omega_\var) \cup \captures_\var$. In particular, for an annotated ref-word $\hat{r}\in (\Sigma \cup (\Sigma\times\Omega_\var))^*$ the result of $\ann(\hat{r})$ stays the same. For a string $\hat{r} = \hat{u} \kappa a \hat{v} \in (\Sigma \cup (\Sigma\times\Omega_\var) \cup \captures_\var)^*$ where $\kappa\in \captures_{\var}$ and $a\in\Sigma$, we define $\ann(\hat{r}) = \ann(\hat{u} (a,\{\kappa\}) \hat{v})$, and for a string $\hat{r} = \hat{u} \kappa (a,\oout) \hat{v} \in (\Sigma \cup (\Sigma\times\Omega_\var) \cup \captures_\var)^*$ where $\oout\in\Omega_\var$, we define $\ann(\hat{r}) = \ann(\hat{u} (a,\oout\cup\{\kappa\}) \hat{v})$.

Further, we extend the function $\str$ to receive strings over $\Sigma \cup (\Sigma\times\Omega_\var) \cup \captures_\var$ as $\str(\hat{r}) = \str(\splain(\hat{r}))$. Therefore, for an extraction grammar with annotations $\cF$ and a string $w\in\Sigma^*$ we define the function $\sem{\cF}$ as: $\sem{\cF}(w) \ := \  \{\ann(\hat{r}) \mid \hat{r} \in L(\cF) \wedge \str(\hat{r}) = w\}$.
We maintain the notions of equivalency between annotated grammars, extraction grammars, and extraction grammars with annotations. Likewise, we define equality between annotated strings, ref-words and  annotated ref-words in the obvious way. Further, we note that any extraction grammar $\cH = (V, \Sigma, \var, P, S)$ is equivalent to the extraction grammar with annotations $\cF = (V', \Sigma\cup\{\#\}, \var, P', S')$ where $V' = V\cup\{S'\}$ for some $S'\not\in V$, and $P' = P \cup\{S'\to S\# \}$. It is obvious that $\cF$ is unambiguous if and only if $\cH$ is unambiguous.

We proceed as follows. First we convert $\cH$ into a functional extraction grammar.
As detailed in Peterfreund's work
\cite[Propositions 10 and 12]{Peterfreund21}, this takes running time 
$\cO(3^{2k}|\cH|^2)$, and if the initial grammar is unambiguous then so is
the resulting grammar.
Note that this implies that every ref-word $r$ which is derivable from $S$ contains each variable operation at most once. Hence we can build an equivalent extraction grammar with annotations $\cF = (V, \Sigma, \var, P, S)$ using the technique above. We then convert $\cF$ into a version of CNF which is slightly more restrictive than arity-two normal form: We allow rules of the form $X\to YZ$, $X\to \eps$ and $X\to \tau$, for nonterminals $X, Y$ and $Z$ and a terminal $\tau$, but rules of the form $X\to Y$ are not permitted.
Converting to this formalism can be done in linear time in $|\cF|$ while preserving unambiguity,
%
e.g., by transforming rules of the form $X \rightarrow Y$ to $X \rightarrow E Y$
for some fresh nonterminal $E$ with a rule $E \rightarrow \epsilon$, and
otherwise applying our result on arity-2 normal form (Proposition~\ref{gram:prp:2nf}).
We pick an order over the variable operations in $\captures_\var$ and for each $\kappa\in\captures_\var$ we do the following:

Define a function $\proc_{\kappa}$ that receives an annotated ref-word
$\hat{r}\in (\Sigma \cup (\Sigma\times2^{\captures_\var}) \cup
\captures_\var)^*$ and:
\begin{enumerate}
  \item if $\hat{r}\in (\Sigma \cup (\Sigma\times2^{\captures_\var}) \cup (\captures_\var\setminus\{\kappa\}))^*$, then $\proc_{\kappa}(\hat{r}) = \hat{r}$,
  \item if $\hat{r} = \hat{u}\kappa \beta a\hat{v}$ for some $\hat{u}, \hat{v}\in (\Sigma \cup (\Sigma\times2^{\captures_\var}) \cup (\captures_\var\setminus\{\kappa\}))^*$, $\beta\in (\captures_\var\setminus\{\kappa\})^*$ and $a\in\Sigma$, then $\proc_{\kappa}(\hat{r}) = \hat{u}\kappa \beta (a,\{\kappa\})\hat{v}$,
  \item if $\hat{r} = \hat{u}\kappa \beta (a,T)\hat{v}$, with $T\subseteq\captures_\var\setminus\{\kappa\}$, then  $\proc_{\kappa}(\hat{r}) = \hat{u}\kappa \beta (a,T\cup\{\kappa\})\hat{v}$, and
  \item $\proc_{\kappa}$ is undefined in any other case.
\end{enumerate}
It is straightforward to see that the annotated ref-words $\hat{r}$ and $\hat{t} = \proc_\kappa(\hat{r})$ are equivalent whenever $\proc_\kappa(\hat{r})$ is defined.\\


We build an extraction grammar with annotations $\cF' = (V', \Sigma, \var, P', S')$ where $V' = V_\sfo \cup V_\sfi \cup V_\sfl \cup V_\sfm \cup V_\sfr\cup\{S'\}$, and $V_\mathsf{scr} = \{A^\mathsf{scr} \mid A\in P\}$ for $\mathsf{scr}\in\{\sfo, \sfi, \sfl, \sfm, \sfr\}$, and $P'$ is defined by the following rules:
\begin{itemize}
	\item For each rule $S\to AB$ in $P$, we add the rules $S'\to A^{\sfo}B^{\sfo}$, $S'\to A^{\sfi}B^{\sfo}$ and $S'\to A^{\sfl}B^{\sfr}$ to $P'$.
	\item For each rule $A\to BC$, we add the rules $A^{\sfo} \to B^{\sfo}C^{\sfo}$, $A^{\sfi}\to B^{\sfi}C^{\sfo}$, $A^{\sfi}\to B^{\sfo}C^{\sfi}$, $A^{\sfi}\to B^{\sfl}C^{\sfr}$, $A^{\sfl}\to B^{\sfo}C^{\sfl}$, $A^{\sfl}\to B^{\sfl}C^{\sfm}$, $A^{\sfm}\to B^{\sfm}C^{\sfm}$, $A^{\sfr}\to B^{\sfr}C^{\sfo}$ and $A^{\sfr}\to B^{\sfm}C^{\sfr}$ to $P'$.
	\item For each rule $A\to a$, $a\in \Sigma$, we add $A^{\sfo}\to a$ and $A^{\sfr}\to(a, \{\kappa\})$.
\item For each rule $A\to (a, T)$, $a\in \Sigma$ and $T\subseteq\captures_\var\setminus\{\kappa\}$, we add $A^{\sfo}\to (a, T)$ and $A^{\sfr}\to(a, T\cup\{\kappa\})$.
\item For each rule $A\to \kappa'$, $\kappa'\in\captures_\var\setminus\{\kappa\}$, we add the rules $A^{\sfo}\to\kappa'$ and $A^{\sfm}\to\kappa'$.
\item For each rule $A\to \kappa$, we add the rule $A^{\sfl}\to\eps$.
\item For each rule $A\to\eps$, we add the rules $A^{\sfo}\to\eps$ and $A^{\sfm}\to\eps$.
\end{itemize}

%

Let $\hat{\Sigma}_{\kappa} = \Sigma \cup (\Sigma\times2^{\captures_\var}) \cup \captures_\var\setminus\{\kappa\}$. For each nonterminal $A\in V$ these hold:
\begin{align*}
	L(A^{\sfo}) &= \{\hat{w}\mid 
	A\der{\cF}^* \hat{w}, 
	\ \text{where } \hat{w} \in  \hat{\Sigma}_{\kappa}^*\}\\
	L(A^{\sfi}) &= \{\proc_{\kappa}(\hat{w}) \mid 
	A\der{\cF}^* \hat{w},
	\ \text{where } \hat{w} = \hat{u}\kappa\beta a \hat{v} \text{ or }\hat{w} = \hat{u}\kappa\beta (a, T) \hat{v},\\ 
	&\ \ \ \ \ \ \ \ \ \ \ \ \ \ \ \ \  \hat{u}, \hat{v} \in \hat{\Sigma}_{\kappa}^*, \beta\in (\captures_\var\setminus\{\kappa\})^*, a\in\Sigma, T\subseteq\captures_\var\setminus\{\kappa\}\}\\
	L(A^{\sfl}) &= \{\hat{w}\beta \mid 
	A\der{\cF}^* \hat{w}\kappa\beta,
	\ \text{where } \hat{w}\in\hat{\Sigma}_{\kappa}^*, \beta\in(\captures_\var\setminus\{\kappa\})^*\}\\
	L(A^{\sfr}) &= \{\beta(a, \{\kappa\})\hat{w} \mid 
	A\der{\cF}^* \beta a \hat{w},
	\ \text{where } \hat{w}\in \hat{\Sigma}_\kappa^*, \beta\in(\captures_\var\setminus\{\kappa\})^*
	\} \ \cup\\ 
	& \!\!\!\{\beta(a, T\cup \{\kappa\})\hat{w} \mid 
	A\der{\cF}^* \beta (a,T) \hat{w},
	\ \text{where }\hat{w}\in\hat{\Sigma}_\kappa^*, \beta\in(\captures_\var\setminus\{\kappa\})^*, T\subseteq\captures_\var\setminus\{\kappa\}\}\\
	L(A^{\sfm}) &= \{\beta\mid A \der{\cF}^*\beta,
	\ \text{where }\beta\in(\captures_\var\setminus\{\kappa\})^*\}
\end{align*}

These equalities are given without proof since they are not used, and are just for illustrating the idea behind the proof.

For the rest of our proof we will represent derivations $X\der{}^* \delta$ as the sequence of productions $X_1\der{}\gamma_1, X_2\der{}\gamma_2, \ldots,X_m\der{}\gamma_m$, where $X_1 = X$ and $\gamma_m = \delta$, which uniquely determines the derivation by doing it in the leftmost way. We use this representation to state exactly how derivations in $\cF$ are translated to derivations in $\cF'$ and vice versa.

Another notion we need to address is how, in a given derivation $X\der{}^* \delta$, instances of nonterminals are located with respect to each other. By an \emph{instance of a nonterminal} (or just \emph{instance}), we mean an $X_i$ along with some specific derivation $X_i\der{} \gamma_i$ in the sequence. For some instances $X_i$ and $X_j$, we say that $X_i$ is a \emph{descendant} of $X_j$ if $X_i = X_j$, or if $Y \der{} X_i Z$, or $Y\der{} Z X_i$ for some $Y$ which is a descendant of $X_j$. We say that $X_i$ is \emph{to the left} of $X_j$ (or $X_j$ is \emph{to the right} of $X_i$) if there is a derivation $X\der{} Y Z$ in the sequence such that $X_i$ is a descendant of $Y$ and $X_j$ is a descendant of $Z$. 

We note a few things in our construction: (1) Each $X\in V_\sfo$ only produces terminals (which do not include $\kappa$) and nonterminals in $V_\sfo$; furthermore, every rule $X\to YZ$ in $P$ is copied into $P'$ as $X^\sfo \to Y^\sfo Z^\sfo$. (2) Nonterminals in $V_\sfi$ do not produce any terminals or $\eps$ directly, so they need to derive into some $X\in V_\sfi$ and some $Y\in V_\sfr$ to derive some string. (3) As with $V_\sfo$, each $X\in V_\sfm$ only produces terminals in $\captures_\var\setminus\{\kappa\}$ and nonterminals in $V_\sfm$. (4) Each $X\in V_\sfl$ (resp. $V_\sfr$) produces exactly one nonterminal $X'\in V_\sfl$ (resp. $V_\sfr$), or $\eps$ (resp. $(a, T)$ for some $a\in \Sigma$ and $T\subseteq\captures_\var$ such that $\kappa\in T$); this, as a consequence, means that on each derivation from $\cF'$ where the first production is not $S' \der{} X^\sfo Y^\sfo$ there is exactly one derivation $X\der{}\eps$ such that $X\in V_\sfl$ (resp., exactly one derivation $X\der{} (a, T)$, such that $X\in V_\sfr$).

From point (1) we see that each annotated ref-word $\hat{r}\in L(\cF)$ such that $\hat{r}\in\hat{\Sigma}_\kappa^*$ (this is, which does not mention $\kappa$ at all) can be derived by $S'$ starting by $S'\der{} A^{\sfo}B^{\sfo}$, $S'\der{} a$, $S'\der{} \eps$ or $S'\der{} \kappa'$. 

On the other hand, each annotated ref-word $\hat{r}\in L(\cF')$ which does not have $\kappa$ on any annotation set was necessarily derived through rules of the form $X^\sfo\to Y^\sfo Z^\sfo$ which correspond to the rule $X\to YZ$ in $P$, so we deduce that $\hat{r}\in L(\cF)$.

We shall now prove that for any string $\hat{r} = \hat{u}\kappa \beta a\hat{v}$, or $\hat{r} = \hat{u}\kappa \beta (a,T)\hat{v}$, where $\hat{u}, \hat{v}\in\hat{\Sigma}_\kappa^*$, $\beta\in\captures_\var\setminus\{\kappa\}$, $a\in\Sigma$ and $T\subseteq\captures_\var\setminus\{\kappa\}$ such that $\hat{r}\in L(\cF)$, it holds that $\proc(\hat{r})\in L(\cF')$. 
W.l.o.g., let $\hat{r} =  \hat{u}\kappa \beta a\hat{v}$ and consider
some leftmost derivation of $\hat{r}$ from $\cF$:
\begin{align*}
	S  &\der{\cF}^* 
	\hat{u}_1 A \delta \\
	&\der{\cF} 
	\hat{u}_1 B C \delta \\
	&\der{\cF}^* 
	\hat{u}_1 \hat{u}_2 \kappa \beta_1 C \delta \\
	&\der{\cF}^*
	\hat{u}_1 \hat{u}_2 \kappa \beta_1 \beta_2 a \hat{v}_1 \delta \\
	&\der{\cF}^* 
	\hat{u}_1 \hat{u}_2 \kappa \beta_1 \beta_2 a \hat{v}_1 \hat{v}_2 = \hat{r},
\end{align*}
where we have that $\beta = \beta_1\beta_2$, $\hat{u} = \hat{u}_1\hat{u}_2$ and $\hat{v} = \hat{v}_1\hat{v}_2$. Note that this is an arbitrary derivation, and we are merely identifying these nonterminals $A$, $B$ and $C$. We also identify the nonterminals $D$, which produces $\kappa$, and $E$, which produces $a$. For the rest of the current part of the proof, we only refer to the \emph{instances} of these nonterminals. Using this, we build a derivation from $\cF'$ step by step:
\begin{enumerate}
	\item We have $S' \der{\cF'}^* \hat{u}_1 A^{\sfi} \delta'$, where $\delta'$ is obtained by replacing each nonterminal $X$ in $\delta$ by $X^{\sfo}$. We get this by starting with the derivation $S \der{\cF}^* \hat{u}_1 A \delta$, and replacing $X\der{\cF} YZ$ by $X^{\sfi}\der{\cF'} Y^{\sfi}Z^{\sfo}$ if $A$ is a descendant of $Y$, by $X^{\sfi}\der{\cF'} Y^{\sfo}Z^{\sfi}$ if $Z$ is, or by $X^{\sfo}\der{\cF'} Y^{\sfo}Z^{\sfo}$ if none is. We also replace each $X\der{\cF} \tau$, for $\tau\in\Sigma\cup(\Sigma\times 2^{\captures_\var})\cup \captures_\var\setminus\{\kappa\}\cup\{\eps\}$, by $X^\sfo\der{\cF'}\tau$. If $A = S$, we replace $A$ it by $S'$.
	\item We have the rule $A^{\sfi} \to B^{\sfl} C^{\sfr}$ which was added to $P'$.
	\item We have $B^\sfl\der{\cF'}^* \hat{u}_2 \beta_1$. We get this by starting from $B\der{\cF}^*\hat{u}_2\kappa\beta_1$, and we replace $X\der{\cF} YZ$ by $X^\sfl\der{\cF'} Y^\sfl Z^\sfm$ if $E$ is a descendant of $D$, by $X^\sfl\der{\cF'} Y^\sfo Z^\sfl$ if $Z$ is, by $X^\sfo\der{\cF'} Y^\sfo Z^\sfo$ if $X$ is to the left of $D$, and by $X^\sfm\der{\cF'} Y^\sfm Z^\sfm$ if it is to the right. We also replace $X\der{\cF} \tau$ by $X^\sfo\der{\cF'}\tau$ if $X$ is to the left of $D$, and by $X^\sfm\der{\cF'}\tau$ if it is to the right. Lastly, we replace $D\der{\cF}\kappa$ by $D^\sfl\der{\cF'}\eps$.
	\item We have $C^\sfr\der{\cF'}^* \beta_2(a,\{\kappa\})\hat{v}_1$. We get this by starting from $C\der{\cF}^*\beta_2 a \hat{v}_1$, and we replace $X\der{\cF} YZ$ by $X^\sfr\der{\cF'} Y^\sfm Z^\sfr$ if $E$ is descendant of $Z$, or by $X^\sfr\der{\cF'} Y^\sfr Z^\sfo$ if $Y$ is, by $X^\sfm\der{\cF'} Y^\sfm Z^\sfm$ if $X$ is to the left of $E$, and by $X^\sfr\der{\cF'} Y^\sfr Z^\sfr$ if it is to the right. We also replace $X\der{\cF} \tau$ by $X^\sfm\der{\cF'}\tau$ if $X$ is to the left of $E$, and by $X^\sfo\der{\cF'}\tau$ if it is to the right. Lastly, we replace $E\der{\cF} a$ by $E^\sfr\der{\cF'}(a, \{\kappa\})$.
	\item We have $\delta'\der{\cF'}^*\hat{v}_2$, which we obtain from $\delta\der{\cF}^*\hat{v}_2$ by replacing each $X \der{\cF} YZ$ by $X^\sfo\der{\cF'} Y^\sfo Z^\sfo$, and each $X\der{\cF}\tau$ by $X^\sfo\der{\cF'}\tau$.
\end{enumerate}
In the end, we get the following leftmost derivation from $\cF'$:
\begin{align*}
	S  &\der{\cF'}^* 
	\hat{u}_1 A^\sfi \delta'\ (\text{or } \hat{u}_1 S' \delta') \\
	&\der{\cF'} 
	\hat{u}_1 B^\sfl C^\sfr \delta' \\
	&\der{\cF'}^* 
	\hat{u}_1 \hat{u}_2 \beta_1 C^\sfr \delta' \\
	&\der{\cF'}^*
	\hat{u}_1 \hat{u}_2 \beta_1 \beta_2 (a, \{\kappa\}) \hat{v}_1 \delta' \\
	&\der{\cF'}^* 
	\hat{u}_1 \hat{u}_2 \beta_1 \beta_2 (a, \{\kappa\}) \hat{v}_1 \hat{v}_2 = \proc_\kappa(\hat{r}),
\end{align*}
which proves that $\proc(\hat{r})\in L(\cF')$. 

We will prove that for every $\hat{s} = \hat{u}(a, T)\hat{v}\in L(\cF')$ where $\kappa\in T$ there is $\hat{r}$ such that $\proc_\kappa(\hat{r}) = \hat{s}$ in a similar way. We argue that any leftmost derivation that produces $\hat{s}$ has the following form:
\begin{align*}
	S'  &\der{\cF'}^* 
	\hat{u}_1 A^\sfi \delta_1\ \  (\text{or } \hat{u}_1 S' \delta_1)\\
	&\der{\cF'} 
	\hat{u}_1 B^\sfl C^\sfr \delta_1\\
	&\der{\cF'}^* 
	\hat{u}_1 \hat{u}_2 D^\sfl \delta_2 C^\sfr \delta_1 \\
	&\der{\cF'}
	\hat{u}_1 \hat{u}_2 \delta_2 C^\sfr \delta_1 \\
	&\der{\cF'}^* 
	\hat{u}_1 \hat{u}_2 \beta_1 C^\sfr \delta_1 \\
	&\der{\cF'}^*
	\hat{u}_1 \hat{u}_2 \beta_1 \beta_2 (a, T) \hat{v}_1 \delta_1 \\
	&\der{\cF'}^* 
	\hat{u}_1 \hat{u}_2 \beta_1 \beta_2 (a, T) \hat{v}_1 \hat{v}_2 = \hat{s},
\end{align*}
where $\delta\in V_\sfo^*$, $\delta'\in V_\sfm^*$, $\beta_1,\beta_2\in (\captures_\var\setminus\{\kappa\})^*$, $\hat{u} = \hat{u}_1\hat{u}_2$ and $\hat{v} = \hat{v}_1\hat{v}_2$. The reasoning goes as follows:
\begin{itemize}
	\item We know that $S'\der{\cF'}^*\hat{u}\beta(a, T)\hat{v}\in L(\cF')$. If $X\der{\cF'}(a, T)$ and $\kappa\in T$, then $X\in V_\sfr$.
	\item From the way $\cF'$ was built, there must be a production $X\der{\cF'} YZ$ in $S'\der{\cF'}^*\hat{s}$ such that $X\in V_\sfi$ (or $X = S'$), $Y\in V_\sfl$ and $Z\in V_\sfr$, as it is the only way to derive a nonterminal in $V_\sfr$. Let $A^\sfi$ (or $S'$), $B^\sfl$ and $C^\sfr$ be these $X$, $Y$ and $Z$ respectively.
	\item Seeing the rules in $P'$ we note that every string of terminals that is derivable from $B^\sfl$ is of the form $\hat{w}\beta$, where $\hat{w}\in\hat{\Sigma}_\kappa$ and $\beta\in (\captures_\var\setminus\{\kappa\})^*$. Furthermore, this string satisfies that there is a production $X\der{\cF'}\eps$ for some $X\in V_\sfl$ such that this $\eps$ is exactly at the left of where $\beta$ begins. Let $\hat{u}_2$ be this $\hat{w}$, let $D^\sfl$ be this $X$, and let $\beta_1$ be this $\beta$.
	\item Likewise, we note that $C^\sfr$ always derives a string of terminals of the form $\beta(a', T')\hat{w}$ for some $\beta\in (\captures_\var\setminus\{\kappa\})^*$ and $\hat{w}\in\hat{\Sigma}_\kappa$. Let $\beta_2$ be this $\beta$ and let $\hat{v}_1$ be this $\hat{w}$.
	\item Lastly, let $S' \der{\cF'}^* \hat{u}_1 A^\sfi \delta_1$ (or $\hat{u}_1 S' \delta_1$) be the one that derives $\hat{s}$. From the rules in $P'$, we note that $\delta_1$ is composed solely of nonterminals in $V_\sfo$.
\end{itemize}

An important point that can be seen from this reasoning is that for each instance $X \neq S'$ that appears in the derivation $S'\der{\cF'}^*\hat{s}$, we can deduce the set $V_{\mathsf{scr}}$ for which $X\in V_{\mathsf{scr}}$, among the options $\mathsf{scr}\in\{\sfo, \sfi, \sfl, \sfm, \sfr\}$, by seeing its position in the derivation. To be precise, this is given from how $X$ relates to the instances $D^\sfl\der{}\eps$, and to $E^\sfr\der{}(a, T)$, for the nonterminal $E^\sfr\in V_\sfr$ that satisfies this. (1) If $X$ is to the left of $D^\sfl$, then $X\in V_\sfo$, (2) if $D^\sfl$ is a descendant of $X$, but $E^\sfr$ is not, then $X\in V_\sfl$, (3) if $X$ is to the right of $D^\sfl$, and is to the left of $E^\sfr$, then $X\in V_\sfm$, (4) if $E^\sfr$ is a descendant of $X$, but $D^\sfl$ is not, then $X\in X_\sfr$, (5) if $X$ is to the right of $E^\sfr$, then $X\in V_\sfo$, and (6) if both $D^\sfl$ and $E^\sfr$ are descendants of $X$, then $X\in V_\sfi$. We bring attention to the fact that in this paragraph we referred only to the instances of $D^\sfl$ and $E^\sfr$ on the derivations mentioned above.

Another, more important point, is this reasoning gives us the derivation presented above. This derivation is translated into the following derivation in $\cF$:
\begin{align*}
	S  &\der{\cF}^* 
	\hat{u}_1 A \delta_1'\ \  (\text{or } \hat{u}_1 S \delta_1')\\
	&\der{\cF} 
	\hat{u}_1 BC\delta_1'\\
	&\der{\cF}^* 
	\hat{u}_1 \hat{u}_2 D \delta_2' C \delta_1' \\
	&\der{\cF}
	\hat{u}_1 \hat{u}_2 \kappa \delta_2' C\delta_1' \\
	&\der{\cF}^* 
	\hat{u}_1 \hat{u}_2 \kappa \beta_1 C \delta_1' \\
	&\der{\cF}
	\hat{u}_1 \hat{u}_2 \kappa \beta_1 \beta_2 (a, T\setminus\{\kappa\}) \hat{v}_1 \delta_1',\ \ \ \ \text{ or}\\
	&\ \ \ \ \ \ \ \ \ \, \, \,
	\hat{u}_1 \hat{u}_2 \kappa \beta_1 \beta_2 a \hat{v}_1 \delta_1',\\
	&\der{\cF}^*
	\hat{u}_1 \hat{u}_2 \kappa \beta_1 \beta_2 (a, T\setminus\{\kappa\}) \hat{v}_1 \hat{v}_2 = \hat{r},\ \ \ \ \text{ or}\\
	&\ \ \ \ \ \ \ \ \ \, \, \,
	\hat{u}_1 \hat{u}_2 \kappa \beta_1 \beta_2 a \hat{v}_1 \hat{v}_2 = \hat{r},
\end{align*}
Where $\delta_1'$ and $\delta_2'$ are obtained by replacing each $X^\sfm$ by $X$ in $\delta_1$ and $\delta_2$, respectively. It is direct to see that this is a valid derivation since for every production $X^\mathsf{x} \der{\cF'} Y^\mathsf{y}  Z^\mathsf{z}$ there exists a valid production  $X \der{\cF} Y Z$, for any $\mathsf{x}, \mathsf{y}, \mathsf{z}\in\{\sfi, \sfo, \sfl, \sfm, \sfr\}$. Furthermore, for the production $D^\sfl\der{\cF'}\eps$ there exists $D\der{\cF}\kappa$, and for $C^\sfr\der{\cF'}(a, T)$ there exists $C\der{\cF} (a, T\setminus\{\kappa\})$ if $T \neq \{\kappa\}$, and $C\der{\cF} a$ if $T = \{\kappa\}$. Further, note that $\proc_\kappa(\hat{r}) = \hat{s}$. We conclude that $\hat{r}\in L(\cF)$ for some $\hat{r}$ such that $\proc_\kappa(\hat{r}) = \hat{s}$.\\

From the arguments above, we obtain that for each annotated ref-word $\hat{r}\in L(\cF)$ there exists an equivalent annotated ref-word $\hat{t}\in L(\cF)$, given by $\hat{t} = \proc_\kappa(\hat{r})$. Furthermore, we showed that for each annotated ref-word $\hat{t}\in L(\cF')$ there exists an equivalent $\hat{r}\in L(\cF)$. This implies that $\cF$ and $\cF'$ are equivalent.\\

Now, assume that $\cF$ is unambiguous. We will prove that $\cF'$ is unambiguous as well. Consider an annotated ref-word $\hat{t}\in L(\cF')$ and consider two sequences $\cS_1$ and $\cS_2$ which define the derivation $S'\der{\cF'}^*\hat{t}$. We showed above how to translate these sequences into sequences $\cS_1'$ and $\cS_2'$ which define the derivation $S\der{\cF}^*	\hat{r}$, for some $\hat{r}$ such that $\hat{t} = \proc_\kappa(\hat{r})$. Since $\cF$ is unambiguous, these sequences are equal. Assume now that $\cS_1$ and $\cS_2$ are not equal, but since their translations into $\cF$ are the same, then it must be that for some production $X\der{\cF}YZ$ or $X\der{\cF}\tau$, there must be two productions $X^\mathsf{x_1} \der{\cF'} Y^\mathsf{y_1}  Z^\mathsf{z_1}$ and $X^\mathsf{x_2} \der{\cF'} Y^\mathsf{y_2}  Z^\mathsf{z_2}$, or $X^\mathsf{x_1} \der{\cF'}\tau$ and $X^\mathsf{x_2} \der{\cF'}\tau$ at the same position, for some $(\mathsf{x_1}, \mathsf{y_1}, \mathsf{z_1}) \neq (\mathsf{x_2}, \mathsf{y_2}, \mathsf{z_2})$. We note that this is not possible since we argued that for a given derivation $S'\der{\cF'}^*\hat{r}$, the set in which each nonterminal instance belongs, among $V_\sfo$, $V_\sfi$, $V_\sfl$, $V_\sfm$, $V_\sfr$, is fixed by its relation to certain instances of $D^\sfl$ and $E^\sfr$. We conclude that $\cF'$ is unambiguous.

Assume $\cF'$ is unambiguous. We will prove that $\cF$ is unambiguous as well. Likewise, consider an annotated ref-word $\hat{r}\in L(\cF)$, and consider two sequences $\cS_1$ and $\cS_2$ which define the derivation $S\der{\cF}^*\hat{t}$. We showed above how to convert these sequences into $\cS_1'$ and $\cS_2'$ which define the derivation $S'\der{\cF'}^*\proc_\kappa(\hat{r})$. Since $\cF'$ is unambiguous, then it must hold that $\cS_1' = \cS_2'$. Note that the translation we showed consisted in replacing productions of the form $X\der{\cF}YZ$ by $X^\mathsf{x} \der{\cF'} Y^\mathsf{y}  Z^\mathsf{z}$, $X\der{\cF}\kappa$ by $X^\sfl\der{\cF'}\eps$, $X\der{\cF}a$ by $X^\sfr\der{\cF}(a, \{\kappa\})$ (or $X\der{\cF}(a, T)$ by $X^\sfr\der{\cF'}(a, T\cup \{\kappa\})$) for a single fixed production, and $X\der{\cF}\tau$ by $X^\mathsf{x}\der{\cF'}\tau$ in any other case, for some $\mathsf{x}, \mathsf{y}, \mathsf{z}\in \{\sfo, \sfi, \sfl, \sfm, \sfr\}$. Therefore, the sequence $S_1$ from which $S_1'$ was obtained is uniquely defined, from which we deduce that $S_1 = S_2$, and we conclude that $\cF$ is unambiguous.\\

At the end of the procedure, we obtain an extraction grammar with annotations $\cF^\dagger = (V^\dagger, \Sigma, \var, P^\dagger, S^\dagger)$ such that there are no rules of the form $X\to\kappa$ in $P^\dagger$, for any $\kappa\in\captures_\var$. From this, we obtain the annotated grammar $\cG = (V^\dagger, \Sigma, \Omega_\var, P^\dagger, S^\dagger)$ which is equivalent to $\cH$. Furthermore, $\cG$ is unambiguous if and only if $\cH$ is unambiguous.

With respect to the running time of building $\cG$, note that in each iteration of the algorithm, by starting on an extraction grammar with annotations $\cF$ with a set of rules $P$, the resulting $\cF'$ has a set of rules $P'$ with a size of $9|P|$. Since this step is repeated twice for each variable $x\in \var$ (once for each variable operation), 
the total running time is $\cO(9^{2|\var|}(3^{2|\var|}|\cH|^2)) = \cO(9^{3|\var|}|\cH|^2)$.
%
%
%
