In this proof, we will use the notion of PDAnn introduced in
Section~\ref{gram:sec:linear}, and we will use 
Proposition~\ref{gram:prop:grammar-pdann}, which is also stated in
Section~\ref{gram:sec:linear}.

%

%
%
%
%
%
%
%
%
%
%
%
%

To prove Theorem~\ref{gram:thm:profileu-iou}, we introduce a general-purpose normal form on PDAnn, where,
intuitively, the only choices that can be made during a run are between the \emph{types}
of transition to apply. 
% This is similar to Lemma~1 in Chapter~\ref{ch1}.

\begin{definition}
  A PDAnn $\cP$ is \emph{deterministic-modulo-profile} if it satisfies the following conditions:

  \begin{enumerate}
    \item for each state $p$ there is at most one push transition that
      starts on $p$, formally $\card{\{q, \gamma \in Q \times \Gamma \mid
      (p, q, \gamma) \in \Delta\}} \leq 1$
    \item for each state $p$ and stack symbol $\gamma$ there is at most
      one pop transition that starts on $p,\gamma$, formally $\card{\{q
      \in Q \mid (p, \gamma, q) \in \Delta\}} \leq 1$
    \item for each state $p$, letter $a$, and output $\oout \in \Omega$, there is at most one read-write transition that starts on
      $p,a,\oout$, formally, we have $\card{\{q \in Q \mid (p,(a,\oout),q) \in \Delta\}}
      \leq 1$.
      \item for each state $p$ and letter $a$, there is at most one read transition that starts on
      $p,a$, formally $\card{\{q \in Q \mid (p,a,q) \in \Delta\}}
      \leq 1$.
  \end{enumerate}
\end{definition}

\begin{lemma}\label{gram:lem:pdtonechoice}
 %
        Let $\cP$ be a PDAnn. We can build an equivalent PDAnn $\cP'$ which is
        deterministic-modulo-profile. The transformation takes exponential time,
        i.e., time $\cO(2^{|\cP|^c})$ for some $c > 0$.

        Further, on any string~$w$, there is an accepting run of~$\cP$ on~$w$ with
        profile $\pi$ iff there is an accepting run of~$\cP'$ on~$w$ with the same profile.
\end{lemma}

\begin{proof}
  The proof is similar to the determinization of visibly pushdown
  automata (see Section~\ref{nested:sec:eval}, also Proposition~\ref{nested:vpawo:det}).

  Given $\cP = (Q, \Sigma, \Omega, \Gamma, \Delta, q_0, F)$,
        we build $\cP' = (Q', \Sigma, \Omega, \Gamma', \Delta', S_I, F')$ as
        follows.
	We build $Q' = 2^{Q\times Q}$, intuitively denoting a set of pairs of
        states $(p, q)$ of $\cP$ such that $\cP$ can be at state $q$ at this
        point if it was at state $p$ when the topmost stack symbol was pushed. 
        We build $\Gamma' = 2^{Q\times\Gamma\times Q}$,
        intuitively specifying the sets of possible stack symbols and
        remembering the state just after the previous stack symbol was pushed
        and the state just after that symbol was pushed.
	We build $S_I = \{(q_0,q_0)\}$, meaning that initially we are at
        the initial state~$q_0$ and were here when the stack was initialized.
        We build
        $F' = \{S\mid (q_0,q)\in S\text{ for some }q\in F \}$,
        meaning that we accept when $\cP$ reaches a final
        state and we were at the initial state when the stack was initialized.
	Let $\Delta'$ be defined as follows:
	\begin{itemize}
          \item The (unique) push transition from a state $S \in Q'$ 
                  makes $\cP'$ push a stack symbol $S'$ and move to a state $T$,
                  intuitively defined as follows. For every pair $(p, p')$
                  of~$S$ and push transition $(p', q, \gamma) \in \Delta$ in the original PDAnn,
                  we can move to state $(q, q)$ and push on the stack the symbol
                  $(p, \gamma, q)$. The stack symbol $S'$ is the set of all
                  possible stack symbols that can be pushed in this way, and
                  $T$ is the set of all possible states that can be reached in
                  this way.

                  Formally, 
                  for every $S\in Q'$ we include $(S, S', T)$ in $\Delta'$, where:
		\begin{align*}
			T &= \{(p,\gamma,q)\mid (p,p')\in S \text{ and } (p',q,\gamma)\in\Delta\text{ for some }p,p',q\in Q, \gamma\in\Gamma \},\\
			S'&= \{(q,q)\mid(p,p')\in S\text{ and }(p',q,\gamma)\in\Delta\text{ for some }p,p',q\in Q,  \gamma\in\Gamma\}
		\end{align*}
              \item The (unique) pop transition from a state $S \in Q'$ and topmost
                stack symbol $T \in \Gamma'$ makes $\cP'$ move to a state $T'$
                intuitively defined as follows. For every pair $(p', q')$ of~$S$,
                we consider all triples $(p, \gamma, p')$ of the topmost stack
                symbol $T$, and if the original PDAnn had a pop transition $(q',
                \gamma, q) \in \Delta$, then we can pop the topmost stack symbol
                and go to the state $(p, q)$. The new state $T'$ is the set of
                all pairs $(p, q)$ that can be reached in this way.

                  Formally, 
                  for every $(S, T)\in Q'\times\Gamma'$ we include $(S, T, S')$ in $\Delta'$, where:
                  {\small
		\begin{align*}
			S' &= \{(p,q)\mid(p,\gamma,p')\in T\text{ and }(p',q')\in S\text{ and }(q',\gamma,q)\in\Delta \text{ for some }p,p',q,q'\in Q,\gamma\in\Gamma\},
		\end{align*}
                }

              \item The (unique) read-write transition from a state $S \in Q'$ on a letter
                  $a \in \Sigma$ and output $\oout \in \Omega$ makes
                  $\cP'$ move to a state $S'$ intuitively defined as follows: we
                  consider all pairs $(p, p')$ in $S$ and all transitions from
                  $p'$ with $a$ and $\oout$ in~$\cP$ to some state $q$, and
                  move to all possible pairs $(p', q)$.

                  Formally, for every $(S, a, \oout)\in Q'\times\Sigma\times\Omega$ we include $(S, (a, \oout), S')$ in $\Delta'$, where:
		\begin{align*}
			S' &= \{(p,q)\mid (p,p')\in S\text{ and }	(p',(a,\oout),q)\in\Delta\text{ for some }p,p',q\in Q\}.
		\end{align*}
	 \item The (unique) read transition from a state $S \in Q'$ on a letter
	$a \in \Sigma$ makes
	$\cP'$ move to a state $S'$ intuitively defined as follows: we
	consider all pairs $(p, p')$ in $S$ and all transitions from
	$p'$ with $a$ in~$\cP$ to some state $q$, and
	move to all possible pairs $(p', q)$.
	
	Formally, for every $(S, a)\in Q'\times\Sigma$ we include $(S, a, S')$ in $\Delta'$, where:
	\begin{align*}
		S' &= \{(p,q)\mid (p,p')\in S\text{ and }	(p',a,q)\in\Delta\text{ for some }p,p',q\in Q\}.
	\end{align*}
	\end{itemize}

        It is clear by definition that $\cP'$ is deterministic-modulo-profile,
        and it is clear that the running time of the construction satisfies the
        claimed time bound.

        We now show that $\cP$ and $\cP'$ are equivalent.

        Now, for the forward
        direction, let us first assume without loss of generality that whenever
        $\cP$ makes a push transition then the stack symbol that it pushes is
        annotated with the state reached just after the push. Then we will show
        that every instantaneous description that can be reached by $\cP$ can be
        reached by $\cP'$ by induction on the run. Specifically, we show by
        induction on the length of the run $\rho$ the following claim: if $\cP$
        has a run $\rho$ on a string $w$ that produces $\mu$ from an initial state $q_0 \in T$ to an instantaneous
        description $(q, i), \alpha$, with $\alpha =
        \gamma_0 p_0, \ldots, \gamma_m p_m$ being the sequence of the stack
        symbols and states annotating them, then $\cP'$ has a run $\rho'$ on $w$
        from $S_I$ to an instantaneous description $(S, i), \alpha'$ with
        $\alpha' = T_0 \ldots T_m$ such that $T_0$ contains $(q_0, \gamma_0,
        p_0)$, $T_1$ contains $(p_0, \gamma_1, p_1)$, ..., $T_m$ contains
        $(p_{m-1}, \gamma_m, p_m)$ and $S$ contains $(p_m, q)$; further $\rho$
        and $\rho'$ have the same profile.

        The base case of an empty run on a string is immediate: if $\cP$ has an
        empty run from an initial state $q_0$, then it reaches the instantaneous
        description with $(q_0, 0)$ and the empty stack, and then $\cP'$ then
        has an empty run reaching the instantaneous description $(S, 0)$ with the
        empty stack and $S$ indeed contains $(q_0, q_0)$. 

        For the induction case, assume that $\cP$ has a non-empty run $\rho_+$ on a string $w$ that produces $\mu$.
        First, write $\rho_+$ as a run $\rho$ followed by one single transition
        of $\cP$. We know $\cP$ has a run $\rho$ on $w$ which produces $\mu$ from an initial
        state $q_0$ to an instantaneous description $(q, i), \alpha$,
        with $\alpha = \gamma_0 p_0, \ldots, \gamma_m
        p_m$. By the induction hypothesis, we know that $\cP'$ has a run
        $\rho'$ on $w$ from $(q_0, q_0)$ to an instantaneous description $(S,
        i), \alpha'$ with $\alpha' = T_0 \ldots T_m$ such that $T_0$ contains
        $(q_0, \gamma_0, p_0)$, $T_1$ contains $(p_0, \gamma_1, p_1)$, ...,
        $T_m$ contains $(p_{m-1}, \gamma, p_m)$ and $S$ contains $(p_m, q)$; and
        $\rho'$ and $\rho$ have the same profile. We
        now distinguish on the type of the transition used to extend $\rho$ to
        $\rho_+$.

        If that transition is a read-write transition $(q, (a, \oout), q')$, we consider
        the read-write transition of~$\cP'$ labeled with $(a,\oout)$ from~$T$,
        and call $S'$ the state that $\cP'$ reaches.
        As $(p_m, q) \in S$ and $(q, (a, \oout), q') \in \Delta$, we know that
        $(p_m, q') \in S'$. Thus, $\cP'$ can read $(a,\oout)$ and
        reach a suitable state $S'$ and position $i+1$ and the stacks are
        unchanged so the claim is proven.
        
        If that transition is a read transition $(q, a, q')$, we follow an analogous reasoning.

        If that transition is a push transition $(q, q', \gamma)$, the position
        of $\cP$ is unchanged and the new stack is extended by $\gamma$
        annotated with state $q'$. Consider the push transition of $\cP'$
        from~$q$, and call $S'$ the state reached and $T = T_{m+1}$ the stack symbol that is
        pushed. As $(p_m, q) \in S$ and $(q, q', \gamma) \in \Delta$, we know
        that $T$ contains $(p_m, \gamma, q')$, and $S'$ contains $(q', q')$,
        which is what we needed to show.

        If that transition is a pop transition, $(q, \gamma_m, q')$, the position
        of $\cP$ is unchanged and the topmost stack symbol is removed. Consider
        the topmost stack symbol $T_m$ and the transition of $\cP'$ that pops
        it from $S$, and call $S'$ the state that we reach. We know that $S$
        contains $(p_m, q)$ and $T_m$ contains $(p_{m-1}, \gamma_m, p_m)$ and
        $(q, \gamma_m, q') \in \Delta$, so $S'$ contains $(p_{m-1}, q')$, which
        is what we needed to show.

        Note that, in all four cases, the profile of $\rho_+$ and $\rho_+'$ is
        the same, because this was true of $\rho$ and $\rho'$, and the type of
        transition done to extend $\rho'$ to~$\rho_+'$ is the same as the type
        of transition done to extend $\rho$ to~$\rho_+$.

        The inductive claim is therefore shown, and thus if $\cP$ has a run $\rho$ on some string $w$ that produces $\mu$ starting at some initial state $q_0$ and
        ending at state $q$, then $\cP'$ has a run $\rho'$
        on $w$ which produces $\mu$ and ending at a state of the form $(q_0,
        q)$ for $q_0$ and having same profile. Thus, if $\rho$ is accepting then $q$ is final for
        $\cP$ and $(q_0, q)$ is final for $\cP'$ so $\rho'$ is accepting.
        This concludes the forward implication.

        We now show the backward implication, and show it again by induction,
        again assuming that $\cP$ annotates the symbols of its stack with the
        state reached just after pushing them. We
        show by induction on the length of a run $\rho'$ the following claim: if
        $\cP'$ has a run $\rho'$ on a string $w$ that produces $\mu$from its initial state to an instantaneous
        description $(S, i), \alpha'$ with $\alpha' =
        T_0, \ldots, T_m$ being the sequence of the stack
        symbols, then for any choice of elements $(q_0, \gamma_0, p_0) \in T_0$,
        $(p_0, \gamma_1, p_1) \in T_1$, ..., $(p_{m-1}, \gamma_m, p_m) \in T_m$
        and $(p_m, q) \in S$ it holds that $\cP$ has a run $\rho$ on $w$ producing $\mu$
        from some initial state $q_0$ to the instantaneous description $(q, i), \alpha$ with
        $\alpha = \gamma_0 q_0, \ldots, \gamma_m q_m$ (writing next to each
        stack symbol the state that annotates it), and $\rho'$ and $\rho$ have
        the same profile.

        The base case of an empty run on a string is again immediate: if $\cP'$ has an
        empty run from its initial state, then it reaches the instantaneous
        description with $(S_I, 0)$ and empty stack, and then $\cP$
        has an empty run from any initial state $q_0$ to $q_0$ so that indeed
        $S_I$ contains $(q_0, q_0)$.

        For the induction case, assume that $\cP'$ has a non-empty run $\rho'_+$ on $w$ which produces $\mu$,
        We write again $\rho'_+$ as a run $\rho'$ followed by one single transition
        of $\cP'$. We know $\cP'$ has a run $\rho'$ on $w$ which produces $\mu$ from the initial
        state $S_I$ to an instantaneous description $(S, i), \alpha'$,
        with $\alpha = T_0 \ldots T_m$.
        By the induction hypothesis, we know that for any choice of elements
        $(q_0, \gamma_0, p_0) \in T_0$,
        $(p_0, \gamma_1, p_1) \in T_1$, ..., $(p_{m-1}, \gamma_m, p_m) \in T_m$
        and $(p_m, q) \in S$, 
        then $\cP$ has a run $\rho$ on $w$ which produces $\mu$
        from some initial state $q_0$ to the instantaneous description $(q, i), \alpha$ with
        $\alpha = \gamma_0 q_0, \ldots, \gamma_m q_m$, and $\rho$ and $\rho'$
        have the same profile.
        We now distinguish on the type of transition used to extend $\rho'$
        to~$\rho'_+$. 

        If the last transition is a read-write transition $(S, (a, \oout), S')$ with $S'$
        defined as in the construction, consider any choice of 
        $(q_0, \gamma_0, p_0) \in T_0$,
        $(p_0, \gamma_1, p_1) \in T_1$, ..., $(p_{m-1}, \gamma_m, p_m) \in T_m$
        and $(p_m, q') \in S'$, and then there must be some state $p''$ such that
        $(p'', (a, \oout), q) \in \Delta$ and $(p_m, p'') \in S$. Using the
        induction hypothesis but picking $(p_m, p'') \in S$, we obtain a run
        $\rho$ of~$\cP$ on $w$ which produces $\mu$, with the correct stack and ending at
        position $i$ on state $p''$, which we can extend by the read transition
        $(p'', (a, \oout), q)$ to reach state $q$ at position $i+1$ without
        touching the stack, proving the result.
        
        If the last transition is a read transition $(S, a, S')$ with $S'$
        defined as in the construction, we follow an analogous reasoning.

        If the last transition is a push transition $(S, S', T)$ with $T$ defined as
        in the construction, consider any choice of 
        $(q_0, \gamma_0, p_0) \in T_0$,
        $(p_0, \gamma_1, p_1) \in T_1$, ..., $(p_{m-1}, \gamma_m, p_m) \in T_m$,
        $(p_m, \gamma_{m+1}, p_{m+1}) \in T_{m+1}$ and $(p_{m+1}, q') \in S'$. We
        know that we must have $q' = p_{m+1}$, 
        %
        and that there must be
        some state $p''$ and push transition $(p'', p_{m+1}, \gamma_{m+1})$ and
        pair $(p_m, p'')$ in $S$. Using the induction hypothesis but picking
        $(p_m, p'') \in S$, we obtain a run $\rho$ of~$\cP$ on $w$ which produces $\mu$ with topmost stack symbol $\gamma_m$, ending at
        state $p''$, which we can extend with the push transition $(p'',
        p_{m+1}, \gamma_{m+1})$ to obtain the desired stack and reach state
        $p_{m+1} = q'$, proving the result.

        If the last transition is a pop transition $(S, T, S')$ with $S'$ defined
        as in the construction, consider any choice of 
        $(q_0, \gamma_0, p_0) \in T_0$,
        $(p_0, \gamma_1, p_1) \in T_1$, ..., $(p_{m-2}, \gamma_{m-1}, p_{m-1})
        \in T_{m-1}$, and $(p_{m-1}, q) \in S'$. We know that there is a pair
        $(p', q') \in S$ and a triple $(p_{m-1}, \gamma_m, p')$ in $T_m$ and a
        pop transition $(q', \gamma_m, q)$ in $\Delta$. Applying the induction
        hypothesis, we get a run $\rho$ of $\cP$ on $w$ which produces $\mu$ and with topmost stack symbol $\gamma_m$ annotated with state
        $p'$ and ending at state $q'$. The pop transition $(q', \gamma_m, q)$
        allows us to extend this run to reach state~$q$ and remove the topmost
        stack symbol, while the rest of the stack is correct, proving the
        result.

        Again, we have ensured that $\rho$ is extended to $\rho_+$ with the same
        transition as the transition used to extend $\rho'$ to~$\rho_+'$,
        ensuring that $\rho_+$ and $\rho_+'$ have same profile.
        This concludes the proof of the backward induction, ensuring that
        if $\cP'$ has a run from $S_I$ to some final state $S$ reading a string $w$ and producing $\mu$, and having $(q_0, q_{\mathrm{f}})$
        with $q_{\mathrm{f}} \in
        F$ in $S$, then $\cP$ has a run reading $w$ which produces $\mu$ going from $q_0$ to the final state $q_{\mathrm{f}}$. This concludes the
        backward implication and completes the proof.
\end{proof}

%
%
%
%
%
%
%
%
%
%
%
%
%
%
%
We can now show Theorem~\ref{gram:thm:profileu-iou} via
Proposition~\ref{gram:prop:grammar-pdann}, using also the notion of \emph{profiled
PDAnn} defined in Section~\ref{gram:sec:linear}:

  Let $\cG$ be a rigid annotated grammar. Using Proposition~\ref{gram:prop:grammar-pdann}, we
  transform it in polynomial time to a profiled PDAnn $\cP$.
  Using Lemma~\ref{gram:lem:pdtonechoice}, we
  build in exponential time an equivalent PDAnn $\cP'$ satisfying the conditions of the lemma.

  We know that $\cP'$ is still profiled. Indeed, if we assume by
  contradiction that there is a string~$w$ on which $\cP'$ has two accepting runs
  with different profiles, then by the last condition of
  Lemma~\ref{gram:lem:pdtonechoice}, the same is true of~$\cP$, contradicting the
  fact that $\cP$ is profiled.

  Now, we claim that $\cP'$ is necessarily also unambiguous. To see why,
  consider two accepting runs~$\rho$ and~$\rho'$ of~$\cP'$ on some string $w$. Since
  $\cP'$ is profiled, $\rho$ and $\rho'$ must have the same
  profile. But now, the conditions of Lemma~\ref{gram:lem:pdtonechoice} ensure that,
  knowing the input string $w$ and profile, the runs $\rho$ and $\rho'$ are completely
  determined. Specifically, this is an immediate
  induction on the run. The base case is that there is only one initial state,
  so both $\rho$ and $\rho'$ must have the same initial state. Now, assuming by
  induction that the runs so far are identical and have the same stack, there
  are three cases. First, if the profile tells us that both runs make a push
  transition, the symbol pushed and state reached are determined by the last
  states of the runs so
  far, which are identical by inductive hypothesis. Second, if the profile tells us that both runs make
  a read-write transition (or read transition), the state reached is determined by the input and output
  symbols (or just the input symbol), and by the last states of the run so far, which are identical by
  inductive hypothesis. Third, if the profile tells
  us that both runs make a pop transition, the state reaches is determined by
  the last state of the run so far, and the topmost symbol of the stack, which
  are identical by inductive hypothesis. This concludes the inductive proof.

  Thus, for any two accepting runs $\rho$ and $\rho'$ on the string~$w$ which produce the same output, they must identical. Thus, $\cP'$ is unambiguous.
  We use Theorem~\ref{gram:prop:grammar-pdann} to transform $\cP'$ back into an
  annotated grammar, which is still rigid and unambiguous, and equivalent to the
  original rigid annotated grammar $\cG$. The overall complexity of the
  transformation is in $\cO((2^{(|\cG|^c)^{c'}})^{c''})$ for some $c, c', c'' > 0$,
  so it is in $\cO(2^{|\cG|^d})$ for some $d>0$ overall, and the time complexity
  is as stated.
