%!TEX root = main.tex

\subsection{Proof of Proposition \ref{nested:ds:lindelay}}

Let $\D = (\Sigma, V, I, \ell, r, \lambda)$ be a $k$-bounded \dsabbr\ and $v\in V$. We will show that the set $\L_{\D}(v)$ can be enumerated with output-linear delay. To show that this is possible we use a data structure we call an {\em output tree}. This is a dynamic binary tree $T$ which appends itself to an \dsabbr \ $\cD$. We define it as follows: If $v$ is a leaf node in $\cD$, then $v$ is an output tree of $\cD$. If $T$ is an output tree and $v$ is a union node, then $T' = v(T)$ is an output tree of $\cD$. If $T_1$ and $T_2$ are output trees and $v$ is a product node, then $v(T_1,T_2)$ is an output tree of $\cD$. In either case, we say that $T$ is rooted in $v$, and we notate it by ${\sf root}(T) = v$. For an output tree $T$ we define the functions ${\sf child}_T, {\sf lchild}_T$ and ${\sf rchild}_T$ as follows: If $v(T')$ is a subtree of $T$, then ${\sf child}_T(v) = T'$. If $v(T_1,T_2)$ is a subtree of $T$, then ${\sf lchild}_T(v) = T_1$ and ${\sf rchild}_T(v) = T_2$. These functions are not defined in any other case.

\begin{definition}
	Let $\D = (\Sigma, V, I, \ell, r, \lambda)$ be an \dsabbr. An output tree $T$ of $\cD$ is full if for each node $v$ in $T$ the following hold: If $v$ is an union node in $\cD$, then ${\sf child}_T(v)$ is either rooted in $\ell(v)$ or in $r(v)$. If $v$ is a product node in $\cD$, then ${\sf lchild}_T(v)$ is rooted in $\ell(v)$ and ${\sf rchild}_T(v)$ is rooted in $r(v)$.
\end{definition}

We define the function ${\sf print}(T)$ as follows: If $v$ is a leaf node $v$, then ${\sf print}(T) = \lambda(v)$. If $T = v(T')$ then ${\sf print}(T) = {\sf print}(T')$. If $T = v(T_1, T_2)$ then ${\sf print}(T) = {\sf print}(T_1)\cdot {\sf print}(T_2)$.

\begin{lemma}\label{nested:appendix:output-tree-print}
	Let $\cD$ be an \dsabbr and let $v$ be a node in $\cD$. For a full output tree $T$ of $\cD$ rooted on $v$ it holds that ${\sf print}(T) \in \L_{\cD}(v)$.
\end{lemma}
\begin{proof}
	We prove this by induction on the size of $T$. The case $T = v$ where $v$ is a leaf node is trivial. If $T = v(T')$, $v$ is an union node, so the proof follows since ${\sf print}(T)$ is equal to ${\sf print}(T')$ which is either in $\L(\ell(v))$ or $\L(r(v))$, and therefore in $L(v)$. If $T = v(T_1,T_2)$ then $v$ is a product node. We have that ${\sf print}(T_1)\in\L(\ell(v))$ and ${\sf print}(T_2)\in\L(r(v))$, from which it follows that ${\sf print}(T)\in\L(v)$.
\end{proof}

\begin{lemma}\label{nested:appendix:output-tree-unique}
	Let $\cD$ be an unambiguous \dsabbr and let $v$ be a node in $\cD$. For each $\mu\in\L_{\cD}(v)$ there exists exactly one full output tree $T_{\mu}$ of $\cD$ rooted in $v$ such that ${\sf print}(T) = \mu$.
\end{lemma}
\begin{proof}
	Let ${\sf reach_{\cD}}(v)$ be the number of nodes reachable from $v$ in $\cD$, including itself. We will prove this lemma by induction in ${\sf reach_{\cD}}(v)$. If ${\sf reach_{\cD}}(v) = 1$, then $v$ is a leaf node and the proof follows directly since the only output tree rooted in $v$ is $v$ itself. Assume that it holds for every node $v$ such that ${\sf reach_{\cD}}(v)< s$. Let $v$ be a node such that ${\sf reach_{\cD}}(v) = s$ and let $\mu\in\L(v)$. If $v$ is a union node suppose without loss of generality that $\mu\in\L(\ell(v))$. Note that since $\cD$ is unambiguous we have that $\mu\not\in\L(r(v))$. If $T_{\mu} = v(T')$ and $T'$ was rooted in $r(v)$, Lemma~\ref{nested:appendix:output-tree-print} would imply that ${\sf print}(T_{\mu}) = {\sf print}(T') \in \L(r(v))$ which leads to a contradiction. Therefore, $T_{\mu}$ could only be of the form $v(T')$ where $T'$ is rooted in $\ell(v)$. From our hypothesis, there exists only one full output tree $T_{\mu}'$ such that ${\sf print}(T_{\mu}') = \mu$, so the proof follows from taking $T_{\mu} = v(T_{\mu}')$. If $v$ is a product node note that any full output tree $T$ rooted in $v$ is of the form $v(T_1,T_2)$, where $T_1$ and $T_2$ are rooted in $\ell(v)$ and $r(v)$ respectively. Since $\cD$ is unambiguous, there exists only two strings $\mu_1$ and $\mu_2$ such that $\mu = \mu_1\cdot\mu_2$ and $\mu_1\in\L(\ell(v))$ and $\mu_2\in\L(r(v))$. Let $T_{\mu_1}$ and $T_{\mu_2}$ be the only full output trees that are rooted in $\ell(v)$ and $r(v)$ respectively for which the hypothesis hold. The proof follows by taking $T_{\mu} = v(T_{\mu_1},T_{\mu_2})$.
\end{proof} 

For an \dsabbr $\cD$ and node $v$ we define a total order over the full output trees rooted in $v$ recursively: If $v$ is a leaf node there exists only one tree rooted in $v$ so the order is trivial. If $v$ is a union node then let $T_1 = v(T_1')$ and $T_2 = v(T_2')$ be full output trees. We have that $T_1 < T_2$ if and only if ${\sf root}(T_1') = \ell(v)$ and ${\sf root}(T_2') = r(v)$, or $T_1' < T_2'$. If $v$ is a product node then let $T = v(T_1,T_2)$ and $T' = v(T_1',T_2')$. We have that $T < T'$ if and only if $T_1 < T_1'$, or $T_1 = T_1'$ and $T_2 < T_2'$.

For an \dsabbr $\cD$ and an output tree $T$ on $\cD$ we define the operation ${\sf tilt}(T)$ as follows: If $T = v$, then ${\sf tilt}(T) = v$. If $T = v(T')$ where ${\sf root}(T') = \ell(v)$, then ${\sf tilt}(T) = v({\sf tilt}(T))$. If $T = v(T')$ where ${\sf root}(T') = r(v)$, then ${\sf tilt}(T) = {\sf tilt}(T')$. If $T = v(T_1,T_2)$, then ${\sf tilt}(T) = v({\sf tilt}(T_1),{\sf tilt}(T_2))$. Intuitively, what this operation does is to bypass any union node in $T$ whose child is a right child in $\cD$.

\begin{definition}
	For an \dsabbr $\cD$, an output tree $T$ of $\cD$ is left-tilted if it can be obtained as $T = {\sf tilt}(T')$ where $T'$ is a full output tree.
\end{definition}

Two left-tilted output trees can be seen in Figure~\ref{nested:fig-output-trees}. The first tree in the figure is also full. Note that since the root could be a union node whose child is a right child, the root of ${\sf tilt}(T)$ could be a different node than the root of $T$. We also notice the following result.

\begin{lemma}\label{nested:appendix:enum-first}
	Let $\cD$ \dsabbr with a node $v$. The first tree $T$ in the ordered sequence of full output trees rooted in $v$ is also left-tilted. In other words, ${\sf tilt}(T) = T$.
\end{lemma}
\begin{proof}
	We define the operation ${\sf build}(v)$ as follows. If $v$ is a leaf node, then ${\sf build}(v) = v$. If $v$ is a union node then ${\sf build}(v) = v({\sf build}(\ell(v))$. If $v$ is a product node then ${\sf build}(v) = v({\sf build}(\ell(v)),{\sf build}(r(v)))$. Let $T'$ be a different full output tree rooted in $v$. A straightforward induction shows that $T < T'$.
\end{proof}

\begin{lemma}
	Let $\D$ be an \dsabbr with an output tree $T$. We have that ${\sf print}({\sf tilt}(T)) = {\sf print}(T)$.
\end{lemma}
\begin{proof}
	The proof follows by a straightforward induction on the tree. 
\end{proof}

\input{dsenumalg2}

\begin{figure}[t]
	\centering
	\begin{tikzpicture}[->,>=stealth',roundnode/.style={circle,draw,inner sep=1.2pt},squarednode/.style={rectangle,inner sep=3pt}]
		\node [squarednode] (0) at (0, 0) {$a_1$};
		\node [squarednode] (1) at (2, 0) {$a_2$};
		\node [squarednode] (2) at (4, 1) {$a_3$};
		\node [squarednode] (3) at (1, 1) {$\cup$};
		\node [squarednode] (4) at (2, 2) {$\odot$};
		\node [squarednode] (5) at (3, 3) {$\cup$};
		\node [roundnode] (6) at (0, 0.3) {};
		%\node [roundnode] (7) at (2, 0.3) {};
		\node [roundnode] (8) at (4, 1.3) {};
		\node [roundnode] (9) at (1, 1.3) {};
		\node [roundnode] (10) at (2, 2.3) {};
		\node [roundnode] (11) at (3, 3.3) {};
		\draw (3.south west) to (0);
		\draw (3.south east) to (1);
		\draw (4.south west) to (3);
		\draw (4.south east) to (2);
		\draw (5.south west) to (4);
		\draw (5.south east) to (2);
		\draw [dashed, -] [out=200,in=70] (9) to (6);
		\draw [dashed, -] [out=200,in=70] (10) to (9);
		\draw [dashed, -] [out=-20,in=130] (10) to (8);
		\draw [dashed, -] [out=200,in=70] (11) to (10);
	\end{tikzpicture}
	\begin{tikzpicture}[->,>=stealth',roundnode/.style={circle,draw,inner sep=1.2pt},squarednode/.style={rectangle,inner sep=3pt}]
		\node [squarednode] (0) at (0, 0) {$a_1$};
		\node [squarednode] (1) at (2, 0) {$a_2$};
		\node [squarednode] (2) at (4, 1) {$a_3$};
		\node [squarednode] (3) at (1, 1) {$\cup$};
		\node [squarednode] (4) at (2, 2) {$\odot$};
		\node [squarednode] (5) at (3, 3) {$\cup$};
		%\node [roundnode] (6) at (0, 0.3) {};
		\node [roundnode] (7) at (2, 0.3) {};
		\node [roundnode] (8) at (4, 1.3) {};
		%\node [roundnode] (9) at (1, 1.3) {};
		\node [roundnode] (10) at (2, 2.3) {};
		\node [roundnode] (11) at (3, 3.3) {};
		\draw (3.south west) to (0);
		\draw (3.south east) to (1);
		\draw (4.south west) to (3);
		\draw (4.south east) to (2);
		\draw (5.south west) to (4);
		\draw (5.south east) to (2);
		\draw [dashed, -] [out=-150,in=110] (10) to (7);
		\draw [dashed, -] [out=-20,in=130] (10) to (8);
		\draw [dashed, -] [out=200,in=70] (11) to (10);
	\end{tikzpicture}
	\caption{An example iteration of an output tree. The subjacent \dsabbr\ $\D$ is represented by solid edges, and the output tree with curve dashed lines. The next tree would be the single node $v$ for which $\lambda(v) = a_3$.}
	\label{nested:fig-output-trees}
\end{figure}


We are ready to discuss the enumeration algorithm. Our algorithm receives an unambiguous $k$-bounded \dsabbr $\cD$ along with one of its nodes $v$ and prints the elements in $\L_{\cD}(v)$ one by one. The way this is done is by generating the sequence of left-tilted output trees ${\sf tilt}(T_1),\ldots,{\sf tilt}(T_m)$ for which $T_1 < \cdots < T_m$ is the complete sequence of full output trees rooted in $v$. After generating each tree $T$, the procedure outputs the string ${\sf print}(T)$ which can be easily done with a depth-first traversal on the tree. The procedure is detailed in Algorithm~\ref{nested:alg:dsenum2}.

The procedure {\sc BuildTree} builds a completely embedded output tree rooted in $u$. The procedure {\sc NextTree} receives a tree rooted in $u$ and recursively builds the next tree in the sequence ${\sf tilt}(T_1),\ldots,{\sf tilt}(T_m)$ for which $T_1 < \cdots < T_m$ is the sequence of full output trees rooted in $u$. 

We can deduce the following from Lemma~\ref{nested:appendix:enum-first}:
\begin{corollary}
	Let $\D$ be an \dsabbr and let $v$ be one of its nodes. {\sc BuildTree}$(\cD,v)$ builds a full output tree $T$ that is the first in the ordered sequence of full output trees rooted in $v$.
\end{corollary}

We prove the correctness of the algorithm in the following results.

\begin{lemma}\label{nested:appendix:enum-correctlemma}
	Let $\D$ be an \dsabbr and let $v$ be one of its nodes. Let $T_1<\ldots<T_m$ be the sequence of full output trees rooted in $v$. If the procedure {\sc NextTree} receives $(\cD,{\sf tilt}(T_i))$ it returns ${\sf tilt}(T_{i+1})$, or $\emptyset$ if $i = m$.
\end{lemma}
\begin{proof}
	We prove this by induction in ${\sf reach}_{\cD}(v)$. If $v$ is a leaf node, the sequence consists only of the tree $v$, so the proof follows directly. 
	Assume it holds for nodes $v'$ such that ${\sf reach}_{\cD}(v') < s$ and let $v$ be such that ${\sf reach}_{\cD}(v) = s$. 
	If $v$ is a union node notice that there exists an $e$ such that the sequence of full output trees rooted in $v$ is $T_1<\ldots<T_e<T_{e+1}<\ldots<T_m$ where $T_e = v(T_e')$ and $T_{e+1} = v(T_{e+1}')$, and ${\sf root}(T_e') = \ell(v)$ and ${\sf root}(T_{e+1}') = r(v)$. 
	If $i < e$ or $i > e$, then the proof follows by induction. Otherwise, if $i = e$, note that the procedure {\sc BuildTree}$(\cD,r(v))$ builds the first full output tree rooted in $r(v)$, which is $T_{e+1}'$, and is equal to ${\sf tilt}(T_{e+1})$. 
	If $v$ is a product node the proof follows straightforwardly by induction over the algorithm.
\end{proof}

From the previous results, correctness of the algorithm follows:

\begin{claim}
	{\sc Enumerate} receives an \dsabbr $\D$ and one of its nodes $v$ and outputs all of the elements in $\L_{\cD}(v)$ one by one without repetition.
\end{claim}
\begin{proof}
	Let $T_1 <\cdots <T_m$ be the sequence of full output trees rooted in $v$. The algorithm starts by generating $T_1 = {\sf tilt}(T_1)$ as proven by Corollary~\ref{nested:appendix:enum-first}. Then on each step $i$, the algorithm iterates $T$ as ${\sf tilt}(T_i)$ to transform it into ${\sf tilt}(T_i)$, as proven by Lemma~\ref{nested:appendix:enum-correctlemma}. In each step, an element in $\L_{\cD}(v)$ is given as output as proven by Lemma~\ref{nested:appendix:output-tree-print}. Moreover, the sequence $T_1 <\cdots <T_m$ allows the set $\L_{\cD}(v)$ to be produced exhaustively without repetitions, as proven by Lemma~\ref{nested:appendix:output-tree-unique}.
\end{proof}

The following results ensure that each tree in the sequence can be generated efficiently.

\begin{lemma}\label{nested:appendix:next-tree-almost-linear-delay}
	Let $\cD$ be an \dsabbr, let $v$ one of its nodes, and let $T_1<\cdots<T_m$ be the sequence of full output trees rooted in $v$. If the procedure {\sc NextTree} receives $(\cD,T)$ it returns the tree $T'$ in at most $c(\vert T\vert + \vert T'\vert)$ time, for some constant $c$.
\end{lemma}
\begin{proof}
	We choose $c$ as a factor of the number of steps that are taken in {\sc NextTree} without taking into account recursion. That is, the time that it takes to run steps 19-33 without calls. A first observation that we make is that {\sc BuildTree} builds a tree $T$  in time at most $c\vert T\vert$, since each call to {\sc BuildTree} takes less than $c$ steps, and exactly one call to {\sc BuildTree} is done per node in $T$. We prove the lemma by induction on the tree. If $v$ is a leaf node, then the proof is trivial. If $v$ is a product node, let $T = v(T_1,T_2)$, and let $T'$ be the output of {\sc NextTree} such that $T' = v(T_1',T_2')$ or $T' = \emptyset$. If the call in line 22 returns an nonempty tree, then the procedure takes time $c + c(\vert T_2\vert + \vert T_2'\vert)$. Otherwise, line 22 takes time $c\vert T_2\vert$. Then, if the call in line 24 returns a nonempty tree, it takes time $c(\vert T_1\vert + \vert T_1'\vert)$, and then the call in line 27 takes time $c\vert T_2'\vert$; otherwise, it takes time $c\vert T_1\vert$. In each of the routes where $T'$ is not empty, the execution time is bounded by $c(\vert T_1\vert + \vert T_2\vert + \vert T_1'\vert + \vert T_2'\vert+1) \leq c(\vert T\vert + \vert T' \vert)$, and if $T' = \emptyset$, it is bounded by $c(\vert T_1\vert + \vert T_2\vert + 1) = c\vert T\vert$ which proves the statement. If $v$ is a union node, let $T = v(T')$ and let $T_{\out}$ be the output of {\sc NextTree}. If the call in line 30 returns a nonempty tree, it takes time $c(\vert T'\vert + \vert T_{\out}'\vert)$, where $T_{\out} = v(T_{\out}')$, and the procedure takes total time $c + c(\vert T'\vert + \vert T_{\out}'\vert)\leq c(\vert T\vert + \vert T_{\out}\vert)$, which proves the statement. Otherwise,  the call in line 30 takes time $c\vert T'\vert$, and then line 32 takes time $c\vert T_{\out}\vert$, which adds to a total time $c + c(\vert T'\vert + \vert T_{\out}\vert)= c(\vert T\vert + \vert T_{\out}\vert)$, which also proves the statement.
\end{proof}

\begin{lemma}\label{nested:appendix:tree-size}
	Let $\cD$ be a $k$-bounded \dsabbr and $T$ be a left-tilted output tree in $\cD$. The size of $T$ is at most $2k\vert {\sf print}(T)\vert$.
\end{lemma}
\begin{proof}
	Note that $\vert {\sf print}(T)\vert$ is equal to the number of leaves in $T$. Since $T$ is left-tilted, then  for each union node $v$ in $T$ we have that ${\sf child}_{T}(v)$ is rooted in $\ell(v)$. We also have that $\cD$ is $k$-bounded, so there are at most $k$ nodes between each pair of product nodes in $T$. We know that a binary tree with $e$ leaves has $2e-1$ nodes and $2e-2$ edges. Therefore, if we replace each edge by $k-1$ nodes we obtain a tree whose size is an upper bound for the size of $T$, and the proof follows.
\end{proof}

From these lemmas we obtain a result that ensures nearly output-linear delay.

\begin{claim}\label{nested:appendix:ds-almost-linear-delay}
	Let $\cD$ be a $k$-bounded \dsabbr and let $v$ be a node in $\cD$. For some sequence $\mu_1,\ldots,\mu_m$ that contains exactly the elements in $\L_{\cD}(v)$ without repetition, {\sc Enumerate} can produce each element $\mu_i$ for $i\in[2,m]$ with delay $c(\vert \mu_{i-1}\vert + \vert \mu_i\vert)$, and $\mu_1$ with delay $c\vert\mu_1\vert$, where $c$ is a constant.
\end{claim}
\begin{proof}
	The sequence in question is the one given by the total order $T_1<\cdots <T_m$ of total output trees rooted in $v$, for which $\mu_i = {\sf print}(T_i)$. Let $c'$ be the constant in Lemma~\ref{nested:appendix:next-tree-almost-linear-delay} and let $d$ be a constant such that ${\sf print}(T)$ can be produced in time $d\vert T \vert$. We have that {\sc BuildTree} can build a tree $T$ in size. In Lemma~\ref{nested:appendix:next-tree-almost-linear-delay} it is shown that the first tree $T_1$ in the sequence can be generated in time $c'\vert T_1\vert$, and in Lemma~\ref{nested:appendix:tree-size} we show that $\vert T_1\vert \leq 2k\vert \mu_1\vert$. From this, it follows that $\mu_1$ can be produced in time $2k(c'+d)\vert \mu_1\vert$. For each $i\in[2,m]$ Lemma~\ref{nested:appendix:next-tree-almost-linear-delay} shows that $T_i$ can be generated in time $c'(\vert T_{i-1}\vert + \vert T_i\vert)$. We can bound this number by $2kc'(\vert \mu_{i-1}\vert + \vert \mu_i\vert)$ using Lemma~\ref{nested:appendix:tree-size}. Printing the output takes time $d\vert T_i\vert$, so the total time is $2kc'(\vert \mu_{i-1}\vert + \vert \mu_i\vert) + 2kd\vert \mu_i\vert$, which is bounded by $2k(c'+d)(\vert \mu_{i-1}\vert + \vert \mu_i\vert)$. We conclude the proof by taking $c = 2k(c'+d)$.
\end{proof}

We optimize this result to obtain the desired statement.

\begin{proposition}[Proposition~\ref{nested:ds:lindelay}]
	Fix $k\in\nat$. Let $\cD$ be an unambiguous and $k$-bounded \dsabbr. Then the set $\L_{\cD}(v)$ can be enumerated with output-linear delay for any node $v$ in $\cD$.
\end{proposition}
\begin{proof}
	Let $c$ be the constant from Claim~\ref{nested:appendix:ds-almost-linear-delay}, and let $\mu_1,\ldots,\mu_m$ be the elements in $\L_{\cD}(v)$ in the order that the algorithm from Claim~\ref{nested:appendix:ds-almost-linear-delay} produces them.  We have that $\mu_1$ can be produced in $c|\mu_1|$ steps, whereas each other $\mu_i$ can be produced in $c(|\mu_{i-1}|+|\mu_i|)$ steps. 
	Our algorithm consists in printing the output set in order in an auxiliary tape, and to simply wait $2c\cdot|\mu_i|$ steps to print each output $\mu_i$ to the actual output tape.
	To see why this is possible to do, note that each output $\mu_i$ will be printed in the auxiliary tape after at most $c|\mu_1| + c(|\mu_1|+|\mu_2|) + c(|\mu_2|+|\mu_3|) + \cdots + c(|\mu_{i-1}|+|\mu_i|)$ steps, which is less than $2c\cdot(|\mu_1|+|\mu_2|+\cdots+|\mu_i|)$. This guarantees that at the moment each output $\mu_i$ need to be printed in the output tape, it will be available in the auxiliary tape.
	Since this clearly works with output-linear delay, the statement follows.
	%We utilize this algorithm as an oracle to retrieve each output $\mu_i$ in order. After each output is obtained, it is printed in an auxiliary tape instead of the output tape. Then the oracle is called to obtain the following outputs $\mu_{i+1},\mu_{i+2},\ldots$. Once $c'\vert\mu_i\vert$ steps have been taken in the oracle, $\mu_i$ is printed to the output tape. After this, the call to obtain $\mu_{i+1}$ is resumed. This output is then obtained $c(\vert\mu_{i}\vert + \vert\mu_{i+1}\vert)$ steps after the oracle was called, but only $c\vert\mu_{i+1}\vert$ steps after $\mu_i$ was printed. From this, we have that for each $i\in[1,m]$, printing $\mu_i$ takes time $(2c'+d)\vert \mu_i\vert$, for a $d$ that arises from the modifications to the algorithm presented here. By taking the constant $c = (2c'+d)$ we obtain output-linear delay.
\end{proof}


\subsection{Proof of Theorem \ref{nested:theo:data-structure}}

%Let $\D = (\Sigma, V, I, \ell, r, \lambda)$ and recall how operators $\add$, $\prod$ and $\union$ are defined:
%
%For $(\D',v) := \add(\D,a)$ define $V' := V \cup \{v'\}$, $I' := I$, and $\lambda'(v') = a$. One can easily check that $\L_{\D'}(v') = \{a\}$ as expected. 
%
%For $(\D',v') := \prod(\D,v_1, v_2)$ define $V' := V \cup\{v'\}$, $I' := I \cup\{v\}$, $\ell'(v') :=v_1$, $r'(v') = v_2$, and $\lambda'(v') = \odot$. 
%
%Recall also that for a node $v\in V$ is safe if $\odepth{\D}(v) \leq 1$ and if $\odepth_{\D}(v) = 1$, then $\odepth_{\D}(r(v))\leq 1$.
%
%If $v_3$ and $v_4$ are safe nodes, define $(\D',v') := \union(\D,v_3, v_4)$ and produce a safe node $v'$. For this define $(\D',v) = \union(\D,v_3,v_4)$ as follows.
%\begin{itemize}
%	\item  If $v_3$ or $v_4$ are output nodes then $V' := V\cup\{v'\}$, $I' := I\cup\{v'\}$, and $\lambda(v') := \cup$. Moreover, if $v_3$ is the output node, then $\ell'(v') := v_3$ and $r'(v') := v_4$. Otherwise, we connect $\ell'(v') := v_4$ and~$r'(v') := v_3$.
%	\item If $v_3$ and $v_4$ are not output nodes (i.e. both are union nodes), then $V' := V \cup\{v',v'',v^*\}$, $I' := I\cup\{v',v'',v^*\}$, $\ell'(v') := \ell(v_3)$, $r'(v') := v''$, and $\lambda'(v') := \cup$; $\ell'(v'') := \ell(v_4)$, $r'(v'') := v^*$, and $\lambda'(v'') := \cup$; $\ell'(v^*) := r(v_3)$, $r'(v^*) := r(v_4)$, and $\lambda'(v^*) := \cup$.
%\end{itemize}

The construction of the operators and the reasoning why each partial result $(\D',v')$ is 2-bounded is stated in the paper. 
By adding the condition that $\D'$ is unambiguous we can deduce that $\L_{\D'}(v')$ can be enumerated with output-linear delay using Proposition~\ref{nested:ds:lindelay}.


\subsection{Proof of Theorem \ref{nested:theo:data-structure-eps}}

In \dsepsabbr, $\eps$-nodes are treated quite particularly. 
For a given \dsepsabbr $\cD$, we require that any node $v\in\cD$ satisfies exactly one of the following:
(1) $\lambda(v) \neq \eps$ and for any node $u$ which is reachable from $v$ it holds that $\lambda(u) \neq \eps$, (2) $\lambda(v) = \eps$ or (3) $\lambda(v) = \cup$, $\lambda(\ell(v)) = \eps$, and $r(v)$ satisfies (1). 
In other words, $\eps$ can only be child of a union node with in-degree 0.
For the rest of the proof, we will refer to a node $v$ that satisfies each case as a node such that (1) $\eps\not\in\cL_{\cD}(v)$, (2) $\lambda(v) = \eps$ or (3) $v$ is {\em in Case 3}, respectively.
Note that this construction ensures that if $\eps\in\cL_{\cD}(v)$, it can be retrieved in constant time.

With these conditions in mind, we can address output-depth, $k$-bounded and safeness. The definition of output-depth is unchanged for nodes $v$ for which $\eps\not\in\cL_{\cD}(v)$, if $\lambda(v) = \eps$, then $\odepth(v) = 0$, and if $v$ is in Case 3, $\odepth(v) = 1$. The definition of $k$-bounded is unchanged. The definition of safe nodes is unchanged except for the additional restriction that a node $v$ can only be safe if $\eps\not\in\cL_{\cD}(v)$.
\begin{claim}\label{nested:appendix:eps-enum}
For a $k$-bounded unambiguous \dsepsabbr $\D$, the set $\cL_{\cD}(v)$ can be enumerated with output-linear delay for every node $v$ in $\cD$.
\end{claim}
\begin{proof}
	To prove this, we formalize the idea behind the construction of an \dsepsabbr.
	Let $\D_{v}$ be the \dsepsabbr induced by the nodes that are reachable from $v$. Formally, let $V_v$ be this set of nodes. 
	Then $\D_{v} = (\Sigma, V_v, I_v, \ell_v, r_v, \lambda_v)$ where $I_v = I \cap V_v$, and also $\ell_v$, $r_v$ and $\lambda$ are the functions $\ell$, $r$ and $\lambda$ induced by $V_v$. 
	It is straightforward to check that $\L_{\D_v}(v) = \L_{\D}(v)$.
	Note that if $\eps\not\in\cL_{\cD}(v)$, then $\cD_v$ is a regular \dsabbr, and if $v$ is in Case 3, then $\cD_{r(v)}$ is a regular \dsabbr as well.
	Furthermore, if $\cD$ is unambiguous and $k$-bounded, then the \dsabbr in each of these cases is also unambiguous and $k$-bounded.
	From here, the proof follows straightforwardly by using Proposition~\ref{nested:ds:lindelay} over these \dsabbr.
\end{proof}

One last notion we make use of is $\eps$-safe nodes. For a given \dsepsabbr $\cD$ and $v\in\cD$ we say that $v$ is $\eps$-safe if either (1) $v$ is safe, (2) $\lambda(v) = \eps$, or (3) $v$ is in Case 3 and $r(v)$ is safe.

We define the operations $\add$, $\prod$ and $\union$ over $\D$ to return a pair $(\D',v')$ such that $\D' = (\Sigma, V', I', \ell', r', \lambda')$ as follows:

For $\add(\D,a)\to(\D',v')$ we define $V' := V \cup \{v'\}$, $I' := I$, and $\lambda'(v') = a$.

Assume $v_1$ and $v_2$ are $\eps$-safe. 
Further, assume that for every word in $w\in\L_{\D}(v_1)\cdot\L_{\D}(v_2)$ there exist only two non-empty words $w_1$ and $w_2$ such that $w_1\in\L_{\D}(v_1)$, $w_2\in\L_{\D}(v_2)$ and $w = w_1w_2$.
Since both $v_1$ and $v_2$ may fall in one of three cases, we define $\prod(\D,v_1,v_2) \to (\D',v')$ by separating into nine cases, of which the first six are straightforward: 
\begin{itemize}
	\item If $\eps\not\in\L_{\D}(v_1)$ and $\eps\not\in\L_{\D}(v_2)$,  we use the construction given for a regular \dsabbr.
	\item If $\eps\not\in\L_{\D}(v_1)$ and $\lambda(v_2) = \eps$, we define $v' = v_1$, and $\D' = \D$.
	\item If $\lambda(v_1) = \eps$ and $\eps\not\in\L_{\D}(v_2)$, we define $v' = v_2$, and $\D' = \D$.
	\item If $\lambda(v_1) = \eps$ and $\lambda(v_2) = \eps$, we define $v' = v_1$, and $\D' = \D$.
	\item If $\lambda(v_1) = \eps$ and $v_2$ is in Case 3, we define $v' = v_2$, and $\D' = \D$.
	\item If $v_1$ is in Case 3 and $\lambda(v_2) = \eps$, we define $v' = v_1$, and $\D' = \D$.
\end{itemize}

%!TEX root = main.tex

\begin{figure}[t]
	\centering
\begin{tikzpicture}[->,>=stealth',roundnode/.style={circle,draw,inner sep=1.2pt},squarednode/.style={rectangle,inner sep=2pt}]
	\node [squarednode] (0) at (1, 2.75) {$\cup$};
	\node [squarednode] (6) at (1.55, 2.75) {$= v_a'$};
	\node [squarednode] (1) at (0.5, 1) {$v_1$};
	\node [squarednode] (2) at (2, 1.75) {$\odot$};
	\node [squarednode] (3) at (2.25, 1) {$v_2$};
	\node [squarednode] (4) at (1.5, 0) {$\eps$};
	\node [squarednode] (5) at (3, 0) {$r(v_2)$};
	\draw (0) to[out=-135,in=135] (2);
	\draw (0) to[out=-45,in=75] (1);
	\draw (2) to[out=-45,in=90] (5);
	\draw (3) to[out=-135,in=60] (4);
	\draw (3) to[out=-45,in=120] (5);
	\draw (2) to[out=-135,in=45] (1);
\end{tikzpicture}
\hspace{1em}
\begin{tikzpicture}[->,>=stealth',roundnode/.style={circle,draw,inner sep=1.2pt},squarednode/.style={rectangle,inner sep=2pt}]
		\node [squarednode] (0) at (2.5, 2.75) {$\cup$};
		\node [squarednode] (6) at (3, 2.75) {$= v_b'$};
\node [squarednode] (1) at (3, 1) {$v_2$};
\node [squarednode] (2) at (2, 1.75) {\raisebox{-7pt}{$\odot$}};
\node [squarednode] (3) at (0.75, 1) {$v_1$};
\node [squarednode] (4) at (0, 0) {$\eps$};
\node [squarednode] (5) at (1.5, 0) {$r(v_1)$};
	\draw (0) to[out=-45,in=90] (1);
	\draw (0) to[out=-135,in=90] (2);
	\draw (2) to[out=-135,in=90] (5);
	\draw (3) to[out=-135,in=60] (4);
	\draw (3) to[out=-45,in=120] (5);
	\draw (2) to[out=-45,in=135] (1);
\end{tikzpicture}
\hspace{1em}
\begin{tikzpicture}[->,>=stealth',roundnode/.style={circle,draw,inner sep=1.2pt},squarednode/.style={rectangle,inner sep=2pt}]
		\node [squarednode] (0) at (1.5, 3.25) {$\cup$};
			\node [squarednode] (11) at (2, 3.25) {$= v_c'$};
\node [squarednode] (1) at (0.75, 2.45) {$\eps^*$};
\node [squarednode] (2) at (2.25, 2.45) {$\textsf{union}$};
\node [squarednode] (3) at (3, 1.75) {$\cup$};
\node [squarednode] (4) at (2, 1) {$\odot$};
\node [squarednode] (5) at (3.25, 1) {$v_2$};
\node [squarednode] (6) at (2.5, 0) {$\eps$};
\node [squarednode] (7) at (4, 0) {$r(v_2)$};
\node [squarednode] (8) at (0.25, 1) {$v_1$};
\node [squarednode] (9) at (-0.5, 0) {$\eps$};
\node [squarednode] (10) at (1, 0) {$r(v_1)$};
\draw (0) to[out=-135,in=60] (1);
\draw (0) to[out=-45,in=120] (2);
\draw (2) to[out=-135,in=90] (10);
\draw (2) to[out=-45,in=135] (3);
\draw (4) to[out=-135,in=45] (10);
\draw (4) to[out=-45,in=135] (7);
\draw (5) to[out=-135,in=60] (6);
\draw (5) to[out=-45,in=120] (7);
\draw (3) to[out=-45,in=90] (7);
\draw (3) to[out=-135,in=40]  (4);
\draw (8) to[out=-45,in=120] (10);
\draw (8) to[out=-135,in=60]  (9);
\end{tikzpicture}
	\caption{Gadgets for $\prod$ as defined for an \dsepsabbr. Nodes $v_a'$, $v_b'$, and $v_c'$ correspond to $v'$ as defined for cases (a), (b), and (c), respectively.}
	\label{nested:fig-prod-multi-gadget}
\end{figure}

The other three cases are more involved and they are presented graphically in Figure~\ref{nested:fig-prod-multi-gadget}. 
Formally, they are defined as follows:

\begin{itemize}
	\item[(a)] If $\eps\not\in\L_{\D}(v_1)$ and $v_2$ is in Case 3, then $V' = V\cup\{v',v''\}$, $I' = I\cup\{v',v''\}$, $\ell'(v') = v_1$, $r'(v') = v''$, $\ell(v'') = v_1$,  $r(v'') = r(v_2)$, $\lambda'(v') = \cup$ and $\lambda'(v'') = \odot$. 
	\item[(b)] If $v_1$ is in Case 3 and $\eps\not\in\L_{\D}(v_1)$, then $V' = V\cup\{v',v''\}$, $I' = I\cup\{v',v''\}$, $\ell'(v') = v''$, $r'(v') = v_2$, $\ell(v'') = r(v_1)$,  $r(v'') = v_2$, $\lambda'(v') = \cup$ and $\lambda'(v'') = \odot$. 
	\item[(c)] If both $v_1$ and $v_2$ are in Case 3, we do a slightly more delicate construction. 
	First, we define a $\cD''$ with $V'' = V\cup\{v^{3},v^{4}\}$, $I'' = I\cup\{v^{3},v^{4}\}$, $\ell''(v^3) = v^4$, $r''(v^3) = r(v_2)$, $\ell''(v^4) = r(v_1)$, $r''(v^4) = r(v_2)$, $\lambda''(v^3) = \cup$, $\lambda''(v^4) = \odot$.
	Now, let $(\cD^3, v^2) \gets \union(\cD'', r(v_1), v_3)$.
	Lastly, let $V' = V^3 \cup \{v^{*}, v'\}$, $I' = I^3\cup\{v'\}$, $\ell'(v') = v^*$, $r(v') = v_2$, $\lambda(v') = \cup$ and $v^* = \eps$.
\end{itemize}
Note that the $\union$ operation in case (c) does not recurse since $r(v_1)$ is safe. In particular, it does not reach any $\eps$-leaf.

Assume $v_1$ and $v_2$ are $\eps$-safe nodes. 
Further, assume that $\L_{\D}(v_1) \setminus \{\eps\}$ and $\L_{\D}(v_2)\setminus \{\eps\}$ are disjoint. 
We define $\union(\D,v_1, v_2) \to (\D',v')$ as follows:
\begin{itemize}
	\item If $\eps\not\in\L_{\D}(v_1)$ and $\eps\not\in\L_{\D}(v_2)$, we use the construction given for a regular \dsabbr.
	\item If $\eps\not\in\L_{\D}(v_1)$ and $\lambda(v_2) = \eps$, we define $V' = V \cup\{v'\}$, $I' = I\cup\{v'\}$ and $\lambda(v') = \cup$. We connect $\ell(v') = v_2$ and $r(v') = v_1$.
	\item If $\eps\not\in\L_{\D}(v_1)$ and $v_2$ is in Case 3, let $(\D'',v'') = \union(\D,v_1,r(v_2))$ as defined for a regular \dsabbr. We define $V' = V'' \cup\{v'\}$, $I' = I''\cup\{v'\}$ and $\lambda'(v') = \cup$ where $\lambda'$ is an extension of $\lambda''$. We connect $\ell'(v') = \ell(v_2)$ and $r'(v') = v''$.
	\item If $\lambda(v_1) = \eps$ and $\eps\not\in\L_{\D}(v_2)$, we define $V' = V \cup\{v'\}$, $I' = I\cup\{v'\}$ and $\lambda(v') = \cup$. We connect $\ell(v') = v_1$ and $r(v') = v_2$.
	\item If $\lambda(v_1) = \eps$ and $\lambda(v_2) = \eps$, we define $\D' = \D$ and $v' = v_1$.
	\item If $\lambda(v_1) = \eps$ and $v_2$ is in Case 3, we define $\D' = \D$ and $v' = v_2$.
	\item If $v_1$ is in Case 3 and $\eps\not\in\L_{\D}(v_2)$, let $(\D'',v'') = \union(\D,r(v_1),v_2)$ as defined for a regular \dsabbr.  We define $V' = V'' \cup\{v'\}$, $I' = I''\cup\{v'\}$ and $\lambda'(v') = \cup$ where $\lambda'$ is an extension of $\lambda''$. We connect $\ell'(v') = \ell(v_2)$ and $r'(v') = v''$. (*)
	\item If $v_1$ is in Case 3 and $\lambda(v_2) = \eps$, we define $\D' = \D$ and $v' = v_1$.
	\item If both $v_1$ and $v_2$ are in Case 3, let $(\D',v') = \union(\D,r(v_1),v_2)$ by using the construction of case (*).
\end{itemize}

Whenever $\D''$ is mentioned it is assumed to be equal to $(\Sigma, V'', I'', \ell'', r'', \lambda'')$.

It is straightforward to check that each operation behaves as expected. 
That is, if $\add(\D,a)\to(\D',v')$, then $\L_{\D}(v') = \{a\}$, if $\prod(\D,v_1,v_2)\to(\D',v')$, then $\L_{\D}(v') = \L_{\D}(v_1)\cdot\L_{\D}(v_2)$, and if $\union(\D,v_1,v_2)\to(\D',v')$, then $\L_{\D}(v_1)\cup\L_{\D}(v_2)$. 
Moreover, if both $v_1$ and $v_2$ are $\eps$-safe, then the resulting node $v'$ is $\eps$-safe as well for each operation.

Note that each operation falls into a fixed number of cases which can be checked exhaustively, and each construction has a fixed size, so they take constant time. 
Furthermore, each operation is fully persistent.

Finally, let $(\D',v')$ be a partial result obtained from applying the operations $\add$, $\prod$ and $\union$ such that $\D'$ is unambiguous. 
The proof follows from Claim~\ref{nested:appendix:eps-enum}.