\begin{proof}
	For the sake of simplification, assume that $\cT$ is I/O-unambiguous on subruns as well. Formally, we extend the condition so that for every well-nested word $w$, span $\spanc{i}{j}$ and $\mu\in\Omega^*$, there exists only one run $\rho\in\Runs(\cT,w)$ such that $\mu = \out(\rho\spanc{i}{j})$.
	Towards a contradiction, we assume that $\cD$ is not duplicate-free. Therefore, at least one of these conditions must hold: (1) there is some union node $v$ in $\cD$ for which $\L_{\cD}(\ell(v))$ and $\L_{\cD}(r(v))$ are not disjoint, or (2) there is some product node $v$ for which there are at least two ways to decompose some $\mu\in\L_{\cD}(v)$ in non-empty strings $\mu_1$ and $\mu_2$ such that $\mu = \mu_1\cdot\mu_2$ and $\mu_1\in\L(\ell(v))$ and $\mu_2\in\L_{\D}(r(v))$. 
	
	Assume the first condition is true and let $v$ be a union node that satisfies it, and let $k$ be the step in which it was added to $\cD$. If this node was added on {\sc OpenStep}, then the node $v$ represents a subset of the subruns defined in condition 1 of Lemma~\ref{nested:vpt:steps}. Consider two different iterations of lines \ref{nested:line35}-\ref{nested:line36} on step $k$ where two nodes $v$ and $v'$ were united for which there is an element $\mu\in\L_{\cD}(v)\cap\L_{\cD}(v')$. Since these nodes were assigned to $T_{p,\gamma,q}$ on different iterations, the states $p'$ that were being considered must have been different. Therefore, if $\llevel(k) = \spanc{i}{j}$, $\mu = \out(\rho\spanc{i}{k}) = \out(\rho'\spanc{i}{k})$ for two runs $\rho$ and $\rho'$ where the $(k-1)$-th state is different. This violates the condition that $\cT$ is I/O-unambiguous. If this node was added on {\sc CloseStep}, we can follow an analogous argument. Note that union nodes created on a $\prod$ operation are duplicate-free by construction (see Theorem~\ref{nested:theo:data-structure-eps}).
	
	Assume now that the second condition is true and let $v$ be a node for which the condition holds and let $k$ be the step where it was created. We note that this node could not have been created in {\sc OpenStep} since the only step that creates product nodes is line \ref{nested:line36}, where $v_{\lambda}$ has the label $(\oout,k)$, and $S_{p,p'}$ is connected to nodes that were created in a previous step, so all of the elements $\mu\in\L(S_{p,p'})$ only contain pairs $(\oout,j)$ where $j < k$. We can follow a similar argument to prove that this node could not have been created in line \ref{nested:line45} of {\sc CloseStep}. We now have that $v$ was created in line \ref{nested:line44} of {\sc OpenStep}, and therefore $\ell(v) = T^{k-1}_{p,\gamma,q}$ and $r(v) = S^{k-1}_{p',q'}$ unless either of these indices were empty. However, that is not possible since we assumed that the step where $v$ was created was $k$, and if either were empty, no node would have been created. Now let $\mu\in\L(v)$ be such that there exist strings $\mu_1,\mu_1'\in\L(T^{k-1}_{p,\gamma,q})$ and $\mu_2,\mu_2'\in\L(S^{k-1}_{p',q'})$ such that $\mu = \mu_1\mu_2 = \mu_1'\mu_2'$ and $\mu_1 \neq \mu_1'$. Without loss of generality, let $\mu''$ be the non-empty suffix in $\mu_1$ such that $\mu_1'\mu'' = \mu_1$. Here we reach a contradiction since $\mu''$ is a prefix of $\mu_2$ and thus it must contain a pair $(\oout,j)$ such that and $j \in\llevel(k)$ and $j\in\clevel(k)$, which is not possible.
	%The fact that all nodes in $\cD$ are $\eps$-safe follows easily from Theorem~\ref{nested:theo:data-structure-eps}.
\end{proof}
