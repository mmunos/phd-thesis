\begin{proof}
For dealing with $\eps$-nodes, we need to revisit the notions of output depth, $k$-bounded, and safeness. For the sake of presentation, it is simpler first to introduce the notion of $\eps$-safe nodes to then revisit all previous definitions. Specifically, for a given \dsepsabbr $\D = (\infAlph, V, \ell, r, \lambda)$ we say that $v \in V$ is \emph{$\eps$-safe} if it is exactly one of the following situations:
\begin{enumerate}
	\item $\lambda(v) = \eps$, or 
	\item $\lambda(v) \neq \eps$, $v$ is a safe node, and  $\lambda(u) \neq \eps$ for any node $u$ which is reachable from $v$, or 
	\item $\lambda(v) = \cup$, $\lambda(\ell(v)) = \eps$, and $r(v)$ satisfies (2). 
\end{enumerate}
In other words, $\eps$ can only occur as the left child of an $\eps$-safe node or being the node itself. 

From now on, we can assume that we will only work with $\eps$-safe nodes as the input for enumerating or for operating them with other ($\eps$-safe) nodes. If this is the case, then the enumeration with output-linear delay and the notions of output depth and $k$-boundedness are similar to before. Therefore, we dedicate the rest of the proof to revisit the operations $\prod$ and $\union$ when the inputs are $\eps$-safe nodes (the $\add$ operation is straightforward similar to the case without $\eps$). 

Assume $v_1$ and $v_2$ are $\eps$-safe. 
Since both $v_1$ and $v_2$ may fall in one of three cases above, we define $\prod(\D,v_1,v_2) \to (\D',v')$ by separating into nine cases, of which the first six are straightforward: 
\begin{itemize}
	\item If $\eps\not\in\cL_{\D}(v_1)$ and $\eps\not\in\cL_{\D}(v_2)$,  we use the construction given for a regular \dsabbr.
	\item If $\eps\not\in\cL_{\D}(v_1)$ and $\lambda(v_2) = \eps$, we define $v' = v_1$, and $\D' = \D$.
	\item If $\lambda(v_1) = \eps$ and $\eps\not\in\cL_{\D}(v_2)$, we define $v' = v_2$, and $\D' = \D$.
	\item If $\lambda(v_1) = \eps$ and $\lambda(v_2) = \eps$, we define $v' = v_1$, and $\D' = \D$.
	\item If $\lambda(v_1) = \eps$ and $v_2$ is in case (3), we define $v' = v_2$, and $\D' = \D$.
	\item If $v_1$ is in case (3) and $\lambda(v_2) = \eps$, we define $v' = v_1$, and $\D' = \D$.
\end{itemize}

%!TEX root = main.tex

\begin{figure}[t]
	\centering
\begin{tikzpicture}[->,>=stealth',roundnode/.style={circle,draw,inner sep=1.2pt},squarednode/.style={rectangle,inner sep=2pt}]
	\node [squarednode] (0) at (1, 2.75) {$\cup$};
	\node [squarednode] (6) at (1.55, 2.75) {$= v_a'$};
	\node [squarednode] (1) at (0.5, 1) {$v_1$};
	\node [squarednode] (2) at (2, 1.75) {$\odot$};
	\node [squarednode] (3) at (2.25, 1) {$v_2$};
	\node [squarednode] (4) at (1.5, 0) {$\eps$};
	\node [squarednode] (5) at (3, 0) {$r(v_2)$};
	\draw (0) to[out=-135,in=135] (2);
	\draw (0) to[out=-45,in=75] (1);
	\draw (2) to[out=-45,in=90] (5);
	\draw (3) to[out=-135,in=60] (4);
	\draw (3) to[out=-45,in=120] (5);
	\draw (2) to[out=-135,in=45] (1);
\end{tikzpicture}
\hspace{1em}
\begin{tikzpicture}[->,>=stealth',roundnode/.style={circle,draw,inner sep=1.2pt},squarednode/.style={rectangle,inner sep=2pt}]
		\node [squarednode] (0) at (2.5, 2.75) {$\cup$};
		\node [squarednode] (6) at (3, 2.75) {$= v_b'$};
\node [squarednode] (1) at (3, 1) {$v_2$};
\node [squarednode] (2) at (2, 1.75) {\raisebox{-7pt}{$\odot$}};
\node [squarednode] (3) at (0.75, 1) {$v_1$};
\node [squarednode] (4) at (0, 0) {$\eps$};
\node [squarednode] (5) at (1.5, 0) {$r(v_1)$};
	\draw (0) to[out=-45,in=90] (1);
	\draw (0) to[out=-135,in=90] (2);
	\draw (2) to[out=-135,in=90] (5);
	\draw (3) to[out=-135,in=60] (4);
	\draw (3) to[out=-45,in=120] (5);
	\draw (2) to[out=-45,in=135] (1);
\end{tikzpicture}
\hspace{1em}
\begin{tikzpicture}[->,>=stealth',roundnode/.style={circle,draw,inner sep=1.2pt},squarednode/.style={rectangle,inner sep=2pt}]
		\node [squarednode] (0) at (1.5, 3.25) {$\cup$};
			\node [squarednode] (11) at (2, 3.25) {$= v_c'$};
\node [squarednode] (1) at (0.75, 2.45) {$\eps^*$};
\node [squarednode] (2) at (2.25, 2.45) {$\textsf{union}$};
\node [squarednode] (3) at (3, 1.75) {$\cup$};
\node [squarednode] (4) at (2, 1) {$\odot$};
\node [squarednode] (5) at (3.25, 1) {$v_2$};
\node [squarednode] (6) at (2.5, 0) {$\eps$};
\node [squarednode] (7) at (4, 0) {$r(v_2)$};
\node [squarednode] (8) at (0.25, 1) {$v_1$};
\node [squarednode] (9) at (-0.5, 0) {$\eps$};
\node [squarednode] (10) at (1, 0) {$r(v_1)$};
\draw (0) to[out=-135,in=60] (1);
\draw (0) to[out=-45,in=120] (2);
\draw (2) to[out=-135,in=90] (10);
\draw (2) to[out=-45,in=135] (3);
\draw (4) to[out=-135,in=45] (10);
\draw (4) to[out=-45,in=135] (7);
\draw (5) to[out=-135,in=60] (6);
\draw (5) to[out=-45,in=120] (7);
\draw (3) to[out=-45,in=90] (7);
\draw (3) to[out=-135,in=40]  (4);
\draw (8) to[out=-45,in=120] (10);
\draw (8) to[out=-135,in=60]  (9);
\end{tikzpicture}
	\caption{Gadgets for $\prod$ as defined for an \dsepsabbr. Nodes $v_a'$, $v_b'$, and $v_c'$ correspond to $v'$ as defined for cases (a), (b), and (c), respectively.}
	\label{nested:fig-prod-multi-gadget}
\end{figure}

The other three cases are more involved and they are presented graphically in Figure~\ref{nested:fig-prod-multi-gadget}. 
Formally, they are defined as follows:
\begin{itemize}
	\item[(a)] If $\eps\not\in\cL_{\D}(v_1)$ and $v_2$ is in case (3), then $V' = V\cup\{v',v''\}$, $\ell'(v') = v''$, $r'(v') = v_1$, $\ell(v'') = v_1$,  $r(v'') = r(v_2)$, $\lambda'(v') = \cup$ and $\lambda'(v'') = \odot$. 
	\item[(b)] If $v_1$ is in case (3) and $\eps\not\in\cL_{\D}(v_1)$, then $V' = V\cup\{v',v''\}$,  $\ell'(v') = v''$, $r'(v') = v_2$, $\ell(v'') = r(v_1)$,  $r(v'') = v_2$, $\lambda'(v') = \cup$ and $\lambda'(v'') = \odot$. 
	\item[(c)] If both $v_1$ and $v_2$ are in case (3), we do a slightly more delicate construction. 
	First, we define a $\cD''$ with $V'' = V\cup\{v^{3},v^{4}\}$, $\ell''(v^3) = v^4$, $r''(v^3) = r(v_2)$, $\ell''(v^4) = r(v_1)$, $r''(v^4) = r(v_2)$, $\lambda''(v^3) = \cup$, $\lambda''(v^4) = \odot$.
	Now, let $(\cD^3, v^2) \gets \union(\cD'', r(v_1), v_3)$.
	Lastly, let $V' = V^3 \cup \{v^{*}, v'\}$, $\ell'(v') = v^*$, $r(v') = v_2$, $\lambda(v') = \cup$ and $v^* = \eps$.
\end{itemize}
Note that the $\union$ operation in case (c) does not recurse since $r(v_1)$ is safe. In particular, it does not reach any $\eps$-leaf.

For the $\union$-operation, assume $v_1$ and $v_2$ are $\eps$-safe and
%Further, assume that $\cL_{\D}(v_1) \setminus \{\eps\}$ and $\cL_{\D}(v_2)\setminus \{\eps\}$ are disjoint. 
define $\union(\D,v_1, v_2) \to (\D',v')$ as:
\begin{itemize}
	\item If $\eps\not\in\cL_{\D}(v_1)$ and $\eps\not\in\cL_{\D}(v_2)$, we use the construction given for a regular \dsabbr.
	\item If $\eps\not\in\cL_{\D}(v_1)$ and $\lambda(v_2) = \eps$, we define $V' = V \cup\{v'\}$ and $\lambda'(v') = \cup$. We connect $\ell'(v') = v_2$ and $r'(v') = v_1$.
	\item If $\eps\not\in\cL_{\D}(v_1)$ and $v_2$ is in case (3), let $(\D'',v'') = \union(\D,v_1,r(v_2))$ as defined for a regular~\dsabbr. We define $V' = V'' \cup\{v'\}$, and $\lambda'(v') = \cup$ where $\lambda'$ is an extension of $\lambda''$. We connect $\ell'(v') = \ell(v_2)$ and $r'(v') = v''$.
	\item If $\lambda(v_1) = \eps$ and $\eps\not\in\cL_{\D}(v_2)$, we define $V' = V \cup\{v'\}$ and $\lambda'(v') = \cup$. We connect $\ell'(v') = v_1$ and $r'(v') = v_2$.
	\item If $\lambda(v_1) = \eps$ and $\lambda(v_2) = \eps$, we define $\D' = \D$ and $v' = v_1$.
	\item If $\lambda(v_1) = \eps$ and $v_2$ is in case (3), we define $\D' = \D$ and $v' = v_2$.
	\item (*) If $v_1$ is in case (3) and $\eps\not\in\cL_{\D}(v_2)$, let $(\D'',v'') = \union(\D,r(v_1),v_2)$ as defined for a regular \dsabbr.  We define $V' = V'' \cup\{v'\}$ and $\lambda'(v') = \cup$ where $\lambda'$ is an extension of $\lambda''$. We connect $\ell'(v') = \ell(v_2)$ and $r'(v') = v''$. 
	\item If $v_1$ is in case (3) and $\lambda(v_2) = \eps$, we define $\D' = \D$ and $v' = v_1$.
	\item If both $v_1$ and $v_2$ are in case (3), let $(\D',v') = \union(\D,v_1,r(v_2))$ by using the construction of the case marked with (*).
\end{itemize}
%Whenever $\D''$ is mentioned it is assumed to be equal to $(\Sigma, V'', I'', \ell'', r'', \lambda'')$.
It is straightforward to check that each operation behaves as expected. 
%That is, if $\add(\D,a)\to(\D',v')$, then $\cL_{\D}(v') = \{a\}$, if $\prod(\D,v_1,v_2)\to(\D',v')$, then $\cL_{\D}(v') = \cL_{\D}(v_1)\cdot\cL_{\D}(v_2)$, and if $\union(\D,v_1,v_2)\to(\D',v')$, then $\cL_{\D}(v_1)\cup\cL_{\D}(v_2)$. 
Moreover, if both $v_1$ and $v_2$ are $\eps$-safe, then the resulting node $v'$ is $\eps$-safe as well for each operation.

We finish this proof by noticing that each operation falls into a fixed number of cases that can be checked exhaustively, and each construction has a fixed size, so they take constant time. 
Furthermore, each operation is fully persistent as expected.


%Finally, let $(\D',v')$ be a partial result obtained from applying the operations $\add$, $\prod$ and $\union$ such that $\D'$ is duplicate-free. 
%The proof follows from Claim~\ref{nested:appendix:eps-enum}.



%	
%		\cristian{PREVIOUS VERSION }
%	For dealing with $\eps$-nodes, we treat them in a particular way. Specifically, for a given \dsepsabbr $\cD$ we require that any node $v\in\cD$ is in exactly one of the following situations:
%	\begin{enumerate}
%		\item $\lambda(v) \neq \eps$ and for any node $u$ which is reachable from $v$ it holds that $\lambda(u) \neq \eps$, 
%		\item $\lambda(v) = \eps$, or 
%		\item $\lambda(v) = \cup$, $\lambda(\ell(v)) = \eps$, and $r(v)$ satisfies (1). 
%	\end{enumerate}
%	In other words, $\eps$ can only be child of a union node with in-degree $0$.
%	
%
%	For the rest of the proof, we will refer to a node $v$ that satisfies each case as a node such that (1) $\eps\not\in\cL_{\cD}(v)$, (2) $\lambda(v) = \eps$ or (3) $v$ is {\em in Case 3}, respectively.
%	Note that this construction ensures that if $\eps\in\cL_{\cD}(v)$, it can be retrieved in constant time.
%	
%	
%	One last notion we make use of is $\eps$-safe nodes. For a given \dsepsabbr $\cD$ and $v\in\cD$ we say that $v$ is $\eps$-safe if either (1) $v$ is safe, (2) $\lambda(v) = \eps$, or (3) $v$ is in Case 3 and $r(v)$ is safe.
%	
%	We define the operations $\add$, $\prod$ and $\union$ over $\D$ to return a pair $(\D',v')$ such that $\D' = (\Sigma, V', I', \ell', r', \lambda')$ as follows:
%	
%	For $\add(\D,a)\to(\D',v')$ we define $V' := V \cup \{v'\}$, $I' := I$, and $\lambda'(v') = a$.
%	
%	Assume $v_1$ and $v_2$ are $\eps$-safe. 
%	Further, assume that for every word in $w\in\cL_{\D}(v_1)\cdot\cL_{\D}(v_2)$ there exist only two non-empty words $w_1$ and $w_2$ such that $w_1\in\cL_{\D}(v_1)$, $w_2\in\cL_{\D}(v_2)$ and $w = w_1w_2$.
%	Since both $v_1$ and $v_2$ may fall in one of three cases, we define $\prod(\D,v_1,v_2) \to (\D',v')$ by separating into nine cases, of which the first six are straightforward: 
%	\begin{itemize}
%		\item If $\eps\not\in\cL_{\D}(v_1)$ and $\eps\not\in\cL_{\D}(v_2)$,  we use the construction given for a regular \dsabbr.
%		\item If $\eps\not\in\cL_{\D}(v_1)$ and $\lambda(v_2) = \eps$, we define $v' = v_1$, and $\D' = \D$.
%		\item If $\lambda(v_1) = \eps$ and $\eps\not\in\cL_{\D}(v_2)$, we define $v' = v_2$, and $\D' = \D$.
%		\item If $\lambda(v_1) = \eps$ and $\lambda(v_2) = \eps$, we define $v' = v_1$, and $\D' = \D$.
%		\item If $\lambda(v_1) = \eps$ and $v_2$ is in Case 3, we define $v' = v_2$, and $\D' = \D$.
%		\item If $v_1$ is in Case 3 and $\lambda(v_2) = \eps$, we define $v' = v_1$, and $\D' = \D$.
%	\end{itemize}
%	
%	%!TEX root = main.tex

\begin{figure}[t]
	\centering
\begin{tikzpicture}[->,>=stealth',roundnode/.style={circle,draw,inner sep=1.2pt},squarednode/.style={rectangle,inner sep=2pt}]
	\node [squarednode] (0) at (1, 2.75) {$\cup$};
	\node [squarednode] (6) at (1.55, 2.75) {$= v_a'$};
	\node [squarednode] (1) at (0.5, 1) {$v_1$};
	\node [squarednode] (2) at (2, 1.75) {$\odot$};
	\node [squarednode] (3) at (2.25, 1) {$v_2$};
	\node [squarednode] (4) at (1.5, 0) {$\eps$};
	\node [squarednode] (5) at (3, 0) {$r(v_2)$};
	\draw (0) to[out=-135,in=135] (2);
	\draw (0) to[out=-45,in=75] (1);
	\draw (2) to[out=-45,in=90] (5);
	\draw (3) to[out=-135,in=60] (4);
	\draw (3) to[out=-45,in=120] (5);
	\draw (2) to[out=-135,in=45] (1);
\end{tikzpicture}
\hspace{1em}
\begin{tikzpicture}[->,>=stealth',roundnode/.style={circle,draw,inner sep=1.2pt},squarednode/.style={rectangle,inner sep=2pt}]
		\node [squarednode] (0) at (2.5, 2.75) {$\cup$};
		\node [squarednode] (6) at (3, 2.75) {$= v_b'$};
\node [squarednode] (1) at (3, 1) {$v_2$};
\node [squarednode] (2) at (2, 1.75) {\raisebox{-7pt}{$\odot$}};
\node [squarednode] (3) at (0.75, 1) {$v_1$};
\node [squarednode] (4) at (0, 0) {$\eps$};
\node [squarednode] (5) at (1.5, 0) {$r(v_1)$};
	\draw (0) to[out=-45,in=90] (1);
	\draw (0) to[out=-135,in=90] (2);
	\draw (2) to[out=-135,in=90] (5);
	\draw (3) to[out=-135,in=60] (4);
	\draw (3) to[out=-45,in=120] (5);
	\draw (2) to[out=-45,in=135] (1);
\end{tikzpicture}
\hspace{1em}
\begin{tikzpicture}[->,>=stealth',roundnode/.style={circle,draw,inner sep=1.2pt},squarednode/.style={rectangle,inner sep=2pt}]
		\node [squarednode] (0) at (1.5, 3.25) {$\cup$};
			\node [squarednode] (11) at (2, 3.25) {$= v_c'$};
\node [squarednode] (1) at (0.75, 2.45) {$\eps^*$};
\node [squarednode] (2) at (2.25, 2.45) {$\textsf{union}$};
\node [squarednode] (3) at (3, 1.75) {$\cup$};
\node [squarednode] (4) at (2, 1) {$\odot$};
\node [squarednode] (5) at (3.25, 1) {$v_2$};
\node [squarednode] (6) at (2.5, 0) {$\eps$};
\node [squarednode] (7) at (4, 0) {$r(v_2)$};
\node [squarednode] (8) at (0.25, 1) {$v_1$};
\node [squarednode] (9) at (-0.5, 0) {$\eps$};
\node [squarednode] (10) at (1, 0) {$r(v_1)$};
\draw (0) to[out=-135,in=60] (1);
\draw (0) to[out=-45,in=120] (2);
\draw (2) to[out=-135,in=90] (10);
\draw (2) to[out=-45,in=135] (3);
\draw (4) to[out=-135,in=45] (10);
\draw (4) to[out=-45,in=135] (7);
\draw (5) to[out=-135,in=60] (6);
\draw (5) to[out=-45,in=120] (7);
\draw (3) to[out=-45,in=90] (7);
\draw (3) to[out=-135,in=40]  (4);
\draw (8) to[out=-45,in=120] (10);
\draw (8) to[out=-135,in=60]  (9);
\end{tikzpicture}
	\caption{Gadgets for $\prod$ as defined for an \dsepsabbr. Nodes $v_a'$, $v_b'$, and $v_c'$ correspond to $v'$ as defined for cases (a), (b), and (c), respectively.}
	\label{nested:fig-prod-multi-gadget}
\end{figure}
%	
%	The other three cases are more involved and they are presented graphically in Figure~\ref{nested:fig-prod-multi-gadget}. 
%	Formally, they are defined as follows:
%	
%	\begin{itemize}
%		\item[(a)] If $\eps\not\in\cL_{\D}(v_1)$ and $v_2$ is in Case 3, then $V' = V\cup\{v',v''\}$, $I' = I\cup\{v',v''\}$, $\ell'(v') = v_1$, $r'(v') = v''$, $\ell(v'') = v_1$,  $r(v'') = r(v_2)$, $\lambda'(v') = \cup$ and $\lambda'(v'') = \odot$. 
%		\item[(b)] If $v_1$ is in Case 3 and $\eps\not\in\cL_{\D}(v_1)$, then $V' = V\cup\{v',v''\}$, $I' = I\cup\{v',v''\}$, $\ell'(v') = v''$, $r'(v') = v_2$, $\ell(v'') = r(v_1)$,  $r(v'') = v_2$, $\lambda'(v') = \cup$ and $\lambda'(v'') = \odot$. 
%		\item[(c)] If both $v_1$ and $v_2$ are in Case 3, we do a slightly more delicate construction. 
%		First, we define a $\cD''$ with $V'' = V\cup\{v^{3},v^{4}\}$, $I'' = I\cup\{v^{3},v^{4}\}$, $\ell''(v^3) = v^4$, $r''(v^3) = r(v_2)$, $\ell''(v^4) = r(v_1)$, $r''(v^4) = r(v_2)$, $\lambda''(v^3) = \cup$, $\lambda''(v^4) = \odot$.
%		Now, let $(\cD^3, v^2) \gets \union(\cD'', r(v_1), v_3)$.
%		Lastly, let $V' = V^3 \cup \{v^{*}, v'\}$, $I' = I^3\cup\{v'\}$, $\ell'(v') = v^*$, $r(v') = v_2$, $\lambda(v') = \cup$ and $v^* = \eps$.
%	\end{itemize}
%	Note that the $\union$ operation in case (c) does not recurse since $r(v_1)$ is safe. In particular, it does not reach any $\eps$-leaf.
%	
%	Assume $v_1$ and $v_2$ are $\eps$-safe nodes. 
%	Further, assume that $\cL_{\D}(v_1) \setminus \{\eps\}$ and $\cL_{\D}(v_2)\setminus \{\eps\}$ are disjoint. 
%	We define $\union(\D,v_1, v_2) \to (\D',v')$ as follows:
%	\begin{itemize}
%		\item If $\eps\not\in\cL_{\D}(v_1)$ and $\eps\not\in\cL_{\D}(v_2)$, we use the construction given for a regular \dsabbr.
%		\item If $\eps\not\in\cL_{\D}(v_1)$ and $\lambda(v_2) = \eps$, we define $V' = V \cup\{v'\}$, $I' = I\cup\{v'\}$ and $\lambda(v') = \cup$. We connect $\ell(v') = v_2$ and $r(v') = v_1$.
%		\item If $\eps\not\in\cL_{\D}(v_1)$ and $v_2$ is in Case 3, let $(\D'',v'') = \union(\D,v_1,r(v_2))$ as defined for a regular \dsabbr. We define $V' = V'' \cup\{v'\}$, $I' = I''\cup\{v'\}$ and $\lambda'(v') = \cup$ where $\lambda'$ is an extension of $\lambda''$. We connect $\ell'(v') = \ell(v_2)$ and $r'(v') = v''$.
%		\item If $\lambda(v_1) = \eps$ and $\eps\not\in\cL_{\D}(v_2)$, we define $V' = V \cup\{v'\}$, $I' = I\cup\{v'\}$ and $\lambda(v') = \cup$. We connect $\ell(v') = v_1$ and $r(v') = v_2$.
%		\item If $\lambda(v_1) = \eps$ and $\lambda(v_2) = \eps$, we define $\D' = \D$ and $v' = v_1$.
%		\item If $\lambda(v_1) = \eps$ and $v_2$ is in Case 3, we define $\D' = \D$ and $v' = v_2$.
%		\item If $v_1$ is in Case 3 and $\eps\not\in\cL_{\D}(v_2)$, let $(\D'',v'') = \union(\D,r(v_1),v_2)$ as defined for a regular \dsabbr.  We define $V' = V'' \cup\{v'\}$, $I' = I''\cup\{v'\}$ and $\lambda'(v') = \cup$ where $\lambda'$ is an extension of $\lambda''$. We connect $\ell'(v') = \ell(v_2)$ and $r'(v') = v''$. (*)
%		\item If $v_1$ is in Case 3 and $\lambda(v_2) = \eps$, we define $\D' = \D$ and $v' = v_1$.
%		\item If both $v_1$ and $v_2$ are in Case 3, let $(\D',v') = \union(\D,r(v_1),v_2)$ by using the construction of case (*).
%	\end{itemize}
%	
%	Whenever $\D''$ is mentioned it is assumed to be equal to $(\Sigma, V'', I'', \ell'', r'', \lambda'')$.
%	
%	It is straightforward to check that each operation behaves as expected. 
%	That is, if $\add(\D,a)\to(\D',v')$, then $\cL_{\D}(v') = \{a\}$, if $\prod(\D,v_1,v_2)\to(\D',v')$, then $\cL_{\D}(v') = \cL_{\D}(v_1)\cdot\cL_{\D}(v_2)$, and if $\union(\D,v_1,v_2)\to(\D',v')$, then $\cL_{\D}(v_1)\cup\cL_{\D}(v_2)$. 
%	Moreover, if both $v_1$ and $v_2$ are $\eps$-safe, then the resulting node $v'$ is $\eps$-safe as well for each operation.
%	
%	Note that each operation falls into a fixed number of cases which can be checked exhaustively, and each construction has a fixed size, so they take constant time. 
%	Furthermore, each operation is fully persistent.
%	
%	Finally, let $(\D',v')$ be a partial result obtained from applying the operations $\add$, $\prod$ and $\union$ such that $\D'$ is duplicate-free. 
%	The proof follows from Claim~\ref{nested:appendix:eps-enum}.
\end{proof}
