%!TEX root = main.tex

\subsection{Proof of Lemma \ref{nested:vpt:steps}}

	We will prove the lemma by induction on $k$. 
	The case $k = 0$ is trivial since $\clevel(0) = \spanc{0}{0}$, $S^0_{p,q}$ is empty and $\llevel(0)$ is not defined. 
	We assume that statements 1 and 2 of the lemma are true for $k-1$ and below. 
	
	If $a_k\in\opS$, the algorithm proceeds into {\sc OpenStep} to build $S^k$ and $T^k$. 
	Statement 1 can be proved trivially since $\clevel(k) = \spanc{k}{k}$, similarly as for the base case.
	For statement 2 let $\llevel(k) = \spanc{i}{k-1}$, and consider a run $\rho\in\Runs(\cT,w\spanc{1}{k})$ such that $\rho\spanc{i}{k}$ starts on $p$ and ends on $q$ for some $p,q$ and $\gamma$, and let $p'$ be its second-to-last state. 
	Since $a_k$ is an open symbol, then the string $a_{i+1}\cdots a_{k-1}$ is well-nested, so it holds that $\clevel(k-1) = \spanc{i}{k-1}$. 
	Therefore, from our hypothesis it holds that $\L_{\D}(S^{k-1}_{p,p'})$ contains $\out(\rho\spanc{i}{k-1})$, and so, $\out(\rho\spanc{i}{k})$ is included in $\L_{\D}(T^{k}_{p,\gamma,q})$ at some iteration of $T^{k}_{p,\gamma,q}$ at line 36. 
	To show that every element in $\L_{\D}(T^k_{p,\gamma,q})$ corresponds to some run $\rho\in\Runs(\cT,w\spanc{1}{k})$, we note that the only step that modifies $T^k_{p,\gamma,q}$ is line 36, which is reached only when a valid subrun from $i$ to $k$ can be constructed.
	
	If $a_k\in\clS$, the algorithm proceeds into {\sc CloseStep} to build $S^k$ and $T^k$. 
	Let $\clevel(k) = \spanc{j}{k}$.
	In this case, statement 2 can be deduced directly from the hypothesis since $j < k$ and the table on the top of $T^k$ is the same as $T^{j}$.
	To prove statement 1 notice that since $a_k$ is a close symbol it holds that $\clevel(k-1) = \spanc{j'}{k-1}$ and $\llevel(k-1) = \spanc{j}{j'-1}$ for some $j'$. 
	Consider a run $\rho\in\Runs(\cT,w)$ such that $\rho\spanc{j}{k}$ starts on $p$, ends on $q$, and the last symbol pushed onto the stack is $\gamma$.
	This run can be subdivided in three subruns from $p$ to $p'$, from $p'$ to $q'$, and a transition from $q'$ to $q$ as it is illustrated in Figure~\ref{nested:fig:delta-schema} (Right). 
	The first  two subruns correspond to $\rho\spanc{j}{j'+1}$ and $\rho\spanc{j'}{k-1}$, for which $\out(\rho\spanc{j}{j'+1})\in\L_{\D}(T^{k-1}_{p,\gamma,q})$ and $\out(\rho\spanc{j'}{k-1})\in\L_{\D}(S^{k-1}_{p',q'})$.
	Therefore, $\out(\rho\spanc{j}{k}) \in \L_{\D}(S^k_{p,q})$ at some iteration of line 47.
	To show that every element in $S^k_{p,q}$ corresponds to some run $\rho\in\Runs(\cT, w\spanc{1}{k})$, note that the only line at which $S^k_{p,q}$ is modified are is line 47, which is reached only when a valid run from $j$ to $k$ has been constructed.
	
\subsection{Proof of Theorem \ref{nested:eval:prep}}

This theorem is a straightforward consequence of Lemma~\ref{nested:vpt:steps}.

\subsection{Proof of Lemma \ref{nested:eval:unambiguous}}

\begin{proof}
	For the sake of simplification, assume that $\cT$ is I/O-unambiguous on subruns as well. Formally, we extend the condition so that for every well-nested word $w$, span $\spanc{i}{j}$ and $\mu\in\Omega^*$ there exists only one run $\rho\in\Runs(\cT,w)$ such that $\mu = \out(\rho\spanc{i}{j})$.
	Towards a contradiction, we assume that $\cD$ is not I/O-unambiguous. Therefore, at least one of these conditions must hold: (1) There is some union node $v$ in $\cD$ for which $\L_{\cD}(\ell(v))$ and $\L_{\cD}(r(v))$ are not disjoint, or (2) there is some product node $v$ for which there are at least two ways to decompose some $\mu\in\L_{\cD}(v)$ in non-empty strings $\mu_1$ and $\mu_2$ such that $\mu = \mu_1\cdot\mu_2$ and $\mu_1\in\L(\ell(v))$ and $\mu_2\in\L_{\D}(r(v))$. 
	
	Assume the first condition is true and let $v$ be an union node that satisfies it, and let $k$ be the step in which it was added to $\cD$. If this node was added on {\sc OpenStep}, then the node $v$ represents a subset of the subruns defined in condition 1 of Lemma~\ref{nested:vpt:steps}. Consider two different iterations of lines 35-36 on step $k$ where two nodes $v$ and $v'$ were united for which there is an element $\mu\in\L_{\cD}(v)\cap\L_{\cD}(v')$. Since these nodes were assigned to $T_{p,\gamma,q}$ on different iterations, the states $p'$ that were being considered must have been different. Therefore, if $\llevel(k) = \spanc{i}{j}$, $\mu = \out(\rho\spanc{i}{k}) = \out(\rho'\spanc{i}{k})$ for two runs $\rho$ and $\rho'$ where the $(k-1)$-th state is different. This violates the condition that $\cT$ is unambiguous. If this node was added on {\sc CloseStep}, we can follow an analogous argument. Note that union nodes created on a $\prod$ operation are unambiguous by construction (see Theorem~\ref{nested:theo:data-structure-eps}).
	
	Assume now that the second condition is true and let $v$ be a node for which the condition holds and let $k$ be the step where it was created. We note that this node could not have been created in {\sc OpenStep} since the only step that creates product nodes is line 36, where $v_{\lambda}$ has the label $(\oout,k)$, and $S_{p,p'}$ is connected to nodes that were created in a previous step, so all of the elements $\mu\in\L(S_{p,p'})$ only contain pairs $(\oout,j)$ where $j < k$. We can follow a similar argument to prove that this node could not have been created in line 45 of {\sc CloseStep}. We now have that $v$ was created in line 44 of {\sc OpenStep}, and therefore $\ell(v) = T^{k-1}_{p,\gamma,q}$ and $r(v) = S^{k-1}_{p',q'}$ unless either of these indices were empty. However, that is not possible since we assumed that the step where $v$ was created was $k$, and if either were empty, no node would have been created. Now let $\mu\in\L(v)$ be such that there exist strings $\mu_1,\mu_1'\in\L(T^{k-1}_{p,\gamma,q})$ and $\mu_2,\mu_2'\in\L(S^{k-1}_{p',q'})$ such that $\mu = \mu_1\mu_2 = \mu_1'\mu_2'$ and $\mu_1 \neq \mu_1'$. Without loss of generality, let $\mu''$ be the non-empty suffix in $\mu_1$ such that $\mu_1'\mu'' = \mu_1$. Here we reach a contradiction since $\mu''$ is a prefix of $\mu_2$ and thus it must contain a pair $(\oout,j)$ such that and $j \in\llevel(k)$ and $j\in\clevel(k)$, which is not possible.
	
	The fact that all nodes in $\cD$ are $\eps$-safe carries easily from Theorem~\ref{nested:theo:data-structure-eps}.
\end{proof}