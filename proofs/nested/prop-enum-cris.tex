% !TeX spellcheck = en_US


\begin{proof}
Let $\D = (\infAlph, V, \ell, r, \lambda)$ be a duplicate-free and $k$-bounded \dsabbr{}.
The algorithm that we present is a depth-first traversal of the DAG, done in a recursive fashion to ensure that after retrieving some output $w$, the next one $w'$ can be printed in $O(k\cdot (|w| + |w'|))$ time. The entire procedure is detailed in Algorithm~\ref{nested:alg:enumeration}.

%!TEX root = ../../Thesis.tex


\algdef{SE}[DOWHILE]{Do}{doWhile}{\algorithmicdo}[1]{\algorithmicwhile\ #1}%
\algnewcommand{\LineComment}[1]{\State \(\triangleright\) #1}



\begin{algorithm*}[t]
	\caption{Enumeration over a node $u$ from some ECS $\cD = (\infAlph, V, \ell, r, \lambda)$.}\label{nested:alg:enumeration}
	\smallskip
	\begin{varwidth}[t]{0.50\textwidth}
		\begin{algorithmic}[1]
			\LineComment{Bottom node iterator $\uit_\infAlph$} \label{nested:alg1bottom1}
			\Procedure{{create}}{$v$}\ \ \ \ \Comment{Assume $\lambda(v) \in \infAlph$} \label{nested:alg1bottom2}
			\State $\bnodelabel \gets v$ \label{nested:alg1bottom3}
			\State $\hasnext \gets {\bf true}$ \label{nested:alg1bottom4}
			\EndProcedure	
			
			\State
			
			\Procedure{{next}}{}
			\If{$\hasnext = \bf{true}$}
			\State $\hasnext \gets \bf{false}$
			\State \Return $\bf{true}$
			\EndIf
			\State \Return $\bf{false}$
			\EndProcedure
			
			\State
			
			\Procedure{{print}}{}
			\State ${\bf print}(\lambda(\bnodelabel))$
			\EndProcedure
			
			\State
			
			\LineComment{Product node iterator $\uit_\odot$}
			\Procedure{{create}}{$v$}\ \ \ \ \Comment{Assume $\lambda(v) = \odot$}
			\State $u \gets v$
			\State $\leftit \gets \textsc{create}(\ell(u))$
			\State $\leftit.\textsc{next}$
			\State $\rightit \gets \textsc{create}(r(u))$
			\EndProcedure	
			
			\State
			
			\Procedure{{next}}{}
			\If{$\rightit.\textsc{next} = \bf{false}$}
			\If{$\leftit.\textsc{next} = \bf{false}$}
			\State \Return $\bf{false}$
			\EndIf
			\State $\rightit \gets \textsc{create}(r(u))$
			\State $\rightit.\textsc{next}$
			\EndIf
			\State \Return $\bf{true}$
			\EndProcedure
			
			\State
			
			\Procedure{{print}}{}
			\State $\leftit.\textsc{print}$
			\State $\rightit.\textsc{print}$
			\EndProcedure
			\algstore{myalg}
		\end{algorithmic}
	\end{varwidth} \hfill
	\begin{varwidth}[t]{0.50\textwidth}
		\begin{algorithmic}[1]
			\algrestore{myalg}
			\LineComment{Union node iterator $\uit_\cup$}
			\Procedure{{create}}{$v$}\ \ \ \ \Comment{Assume $\lambda(v) = \cup$}
			\State $\ustack\gets\text{\sf push}(\ustack, v)$
			\State $\ustack\gets\textsc{traverse}(\ustack)$
			\State $\uit \gets \textsc{create}(\text{\sf top}(\ustack))$
			\EndProcedure	
			
			\State
			
			\Procedure{{next}}{}
			\If{$\uit.\textsc{next} = \bf{false}$}
			\State $\ustack\gets\text{\sf pop}(\ustack)$
			\If{$\text{\sf length}(\ustack) = 0$}
			\State \Return $\bf{false}$
			\ElsIf{$\lambda(\text{\sf top}(\ustack)) = \cup$}
			\State $\ustack\gets\textsc{traverse}(\ustack)$
			\EndIf
			\State $\uit \gets \textsc{create}(\text{\sf top}(\ustack))$
			\State $\uit.\textsc{next}$
			\EndIf
			\State \Return $\bf{true}$
			\EndProcedure
			
			\State
			
			\Procedure{{print}}{}
			\State $\uit.\textsc{print}$
			\EndProcedure
			
			\State
			
			\Procedure{{traverse}}{$\ustack$}
			\While{$\lambda(\text{\sf top}(\ustack)) = \cup$}
			\State $u \gets \text{\sf top}(\ustack)$
			\State $\ustack \gets \text{\sf pop}(\ustack)$
			\State $\ustack\gets\text{\sf push}(\ustack, r(u))$
			\State $\ustack\gets\text{\sf push}(\ustack, \ell(u))$
			\EndWhile
			\State \Return $\ustack$
			\EndProcedure
			
			\State
			
			\Procedure{{enumerate}}{$v$}
			\State $\uit \gets \textsc{create}(v)$
			\While{$\uit.\textsc{next} = \bf{true}$}
				\State $\uit.\textsc{print}$
			\EndWhile
			\EndProcedure
		\end{algorithmic}
	\end{varwidth}
	\smallskip
\end{algorithm*}


To simplify the presentation of the algorithm, we use the interface of an \emph{iterator} that, given a node $v$, it contains all information and methods to enumerate the outputs $\sem{\D}(v)$. Specifically, an iterator $\uit$ must implement the following three methods:
\[
\begin{array}{rclrclrcl}
	\textsc{create}(v) & \!\!\!\!\rightarrow\!\!\!\! & \uit   \ \ \ \ \ \	 & \uit.\textsc{next} & \!\!\!\! \rightarrow \!\!\!\! & b  \ \ \ \ \ \ \ &  \uit.\textsc{print}  & \!\!\!\!\rightarrow\!\!\!\! & \emptyset
\end{array}
\]
where $v$ is a node, $b$ is either {\bf true} or {\bf false}, and $\emptyset$ means ``no output''. 
The first method, \textsc{create}, receives a node $v$ and creates an iterator $\uit$ of the type of $v$. We will implement three types of iterators, one for each node type: bottom, product, and union nodes. The second method, $\uit.\textsc{next}$, moves the iterator to the next output, returning {\bf true} if, and only if, there is an output to print. Then the last method, $\uit.\textsc{print}$, write the current output pointed by $\uit$ to the output registers. 
We assume that, after creating an iterator $\uit$, one must first call $\uit.\textsc{next}$ to move to the first output before printing. Furthermore, if $\uit.\textsc{next}$ outputs {\bf false}, then the behavior of $\uit.\textsc{print}$ is undefined. Note that one can call $\uit.\textsc{print}$ several times, without calling $\uit.\textsc{next}$, and the iterator will write the same output each time in the output registers. 

Assume we can implement the iterator interface for each node type. Then the procedure $\textsc{Enumerate}(v)$ in Algorithm~\ref{nested:alg:enumeration} (lines 61-64) shows how to enumerate the set $\sem{\D}(v)$ by using an iterator $\uit$ for $v$. In the following, we show how to implement the iterator interface for each node type and how the size of the following output bounds the delay between two outcomes.

We start by presenting the iterator $\uit_\infAlph$ for a bottom node $v$ (lines 1-13), called a \emph{bottom node iterator}. We assume that each $\uit_\infAlph$ has internally two values $\bnodelabel$ and $\hasnext$, where $\bnodelabel$ is a reference to $v$ and $\hasnext$ is a boolean variable. The purpose of a bottom node iterator is only to print $\lambda(u)$. For this goal, when we create $\uit_\infAlph$, we initialize $u$ equal to $v$ and $\hasnext = {\bf true}$ (lines 3-4). Then, when we call $\uit_\infAlph.\textsc{next}$ for the first time, we swap $\hasnext$ from {\bf true} to {\bf false} and output {\bf true} (i.e., there is one output ready to be returned). Then any following call to $\uit_\infAlph.\textsc{next}$ will be false (lines 6-10). Finally, the $\uit_\infAlph.\textsc{print}$ writes the value $\lambda(u)$ to the output registers (lines 12-13). Here, we assume the existence of a method {\bf print} on the RAM model for writing to the output registers. 

For a product node, we present a \emph{product node iterator} $\uit_\odot$ in Algorithm~\ref{nested:alg:enumeration} (lines 15-32). This iterator receives a product node $v$ with $\lambda(v) = \odot$ and stores a reference of $v$, called $u$, and two iterators $\leftit$ and $\rightit$, for iterating through the left and right nodes $\ell(u)$ and $r(u)$, respectively. 
The \textsc{create} method initializes $u$ with $v$, creates the iterators $\leftit$ and $\rightit$, and calls $\leftit.\textsc{next}$ to be prepared for the first call of $\uit_\odot.\textsc{next}$ (lines 16-20).
The idea of $\uit_\odot.\textsc{next}$ is to fix one output for the left node $\ell(u)$ and iterate over all outputs of $r(u)$ (lines 22-28). When we stop enumerating all outputs of $\sem{\D}(r(u))$, we move to the next output of $\leftit$, and iterate again over all $\sem{\D}(r(u))$ (lines 24-27).
For printing, we recursively call first the printing method of $\leftit$, and then the one of $\rightit$ (lines 30-32).

The most involved case is the \emph{union node iterator} $\uit_\cup$ (lines 33-59). Similarly to a product node iterator, it receives a union node $v$ with $\lambda(v) = \cup$ and keeps a \emph{stack} of nodes $\ustack$ and an iterator~$\uit$. We assume the standard implementation of a stack with the native methods $\text{\sf push}$, $\text{\sf pop}$, $\text{\sf top}$, and $\text{\sf length}$ over stacks -- the first three define the standard operations over stacks, and $\text{\sf length}$ counts the elements in it. 
The purpose of the stack is to perform a \emph{depth-first-search} traversal of all union nodes below $v$, reaching all possible output nodes $u$ such that there is a path of union nodes between $v$ and $u$. If the top node of $\ustack$ is an output node, then $\uit$ is an iterator for that node, which enumerates all their outputs.  

For performing the \emph{depth-first-search} traversal of union nodes, we use the auxiliary method $\textsc{traverse}(\ustack)$ (lines 53-59). While the top node $u$ in $\ustack$ is a union node, it pops $u$ and pushes its right and left nodes in that order. 
This method will eventually reach an output node (i.e., a non-union node) at the top of the stack and end. 
It is important to note that $\textsc{traverse}(\ustack)$ takes $\cO(k)$-steps, given that the ECS is $k$-bounded. Then if $k$ is fixed, the  $\textsc{traverse}$ procedure takes constant time. In Figure~\ref{nested:fig-enum-stacks}, we illustrate the evolution of a stack $\ustack$ inside a union node iterator when we call $\textsc{traverse}(\ustack)$ several times. 
\begin{figure}[t]
	\centering
	%!TEX root = ../main/main.tex
\newcommand{\figfourscaleval}{0.83}
\begin{tikzpicture}[scale=\figfourscaleval,->,>=stealth',roundnode/.style={circle,draw,inner sep=1.2pt},squarednode/.style={rectangle,inner sep=3pt}]
	
	\node [squarednode] (num) at (0, 5) {1.};
	\node [squarednode] (0) at (1, 0) {$a_3$};
	%	\node [roundnode] (0b) at (1, 0.3) {};
	\node [squarednode] (1) at (3, 0) {$a_4$};
	%	\node [roundnode] (1b) at (3, 0.3) {};
	\node [squarednode] (2) at (2, 1) {$\cup$};
	%	\node [roundnode] (2b) at (2, 1.3) {};
	\node [squarednode] (3) at (0, 1) {$a_2$};
	%	\node [roundnode] (3b) at (0, 1.3) {};
	\node [squarednode] (4) at (1, 2) {$\cup$};
	%	\node [roundnode] (4b) at (1, 2.3) {};
	\node [squarednode] (5) at (3, 2) {$a_5$};
	%	\node [roundnode] (5b) at (3, 2.3) {};
	\node [squarednode] (6) at (2, 3) {$\cup$};
	%	\node [roundnode] (6b) at (2, 3.3) {};
	\node [squarednode] (7) at (0, 3) {$a_1$};
	\node [squarednode] (8) at (1, 4) {$\cup$};
	\node [squarednode] (9) at (3, 4) {$a_6$};
	%	\node [roundnode] (9b) at (3, 4.3) {};
	\node [squarednode] (10) at (2, 5) {$\cup$};
	%	\node [roundnode] (12b) at (1, 6.3) {};
	
	\draw (2) to (0);
	\draw (2) to (1);
	\draw (4) to (2);
	\draw (4) to (3);
	\draw (6) to (4);
	\draw (6) to (5);
	\draw (8) to (6);
	\draw (8) to (7);
	\draw (10) to (8);
	\draw (10) to (9);
	
	\draw[dashed] (1.2,5) to (10);
	
\end{tikzpicture}
\hspace{1em}
\begin{tikzpicture}[scale=\figfourscaleval,->,>=stealth',roundnode/.style={circle,draw,inner sep=1.2pt},squarednode/.style={rectangle,inner sep=3pt}]
	
	\node [squarednode] (num) at (0, 5) {2.};
	\node [squarednode] (0) at (1, 0) {$a_3$};
	%	\node [roundnode] (0b) at (1, 0.3) {};
	\node [squarednode] (1) at (3, 0) {$a_4$};
	%	\node [roundnode] (1b) at (3, 0.3) {};
	\node [squarednode] (2) at (2, 1) {$\cup$};
	%	\node [roundnode] (2b) at (2, 1.3) {};
	\node [squarednode] (3) at (0, 1) {$a_2$};
	\node [squarednode] (4) at (1, 2) {$\cup$};
	\node [squarednode] (5) at (3, 2) {$a_5$};
	%	\node [roundnode] (5b) at (3, 2.3) {};
	\node [squarednode] (6) at (2, 3) {$\cup$};
	\node [squarednode] (7) at (0, 3) {$a_1$};
	%	\node [roundnode] (7b) at (0, 3.3) {};
	\node [squarednode] (8) at (1, 4) {$\cup$};
	%	\node [roundnode] (8b) at (1, 4.3) {};
	\node [squarednode] (9) at (3, 4) {$a_6$};
	%	\node [roundnode] (9b) at (3, 4.3) {};
	\node [squarednode] (10) at (2, 5) {$\cup$};
	%	\node [roundnode] (12b) at (1, 6.3) {};
	
	\draw (2) to (0);
	\draw (2) to (1);
	\draw (4) to (2);
	\draw (4) to (3);
	\draw (6) to (4);
	\draw (6) to (5);
	\draw (8) to (6);
	\draw (8) to (7);
	\draw (10) to (8);
	\draw (10) to (9);
	
	\draw [dashed] (3,4.7) to (9);
	\draw [dashed,out=-90,in=0] (9) to (6);
	\draw [dashed] (6) to (7);
	
\end{tikzpicture}
\hspace{1em}
\begin{tikzpicture}[scale=\figfourscaleval,->,>=stealth',roundnode/.style={circle,draw,inner sep=1.2pt},squarednode/.style={rectangle,inner sep=3pt}]
	
	\node [squarednode] (num) at (0, 5) {3.};
	\node [squarednode] (0) at (1, 0) {$a_3$};
	\node [squarednode] (1) at (3, 0) {$a_4$};
	%	\node [roundnode] (1b) at (3, 0.3) {};
	\node [squarednode] (2) at (2, 1) {$\cup$};
	\node [squarednode] (3) at (0, 1) {$a_2$};
	%	\node [roundnode] (3b) at (0, 1.3) {};
	\node [squarednode] (4) at (1, 2) {$\cup$};
	%	\node [roundnode] (4b) at (1, 2.3) {};
	\node [squarednode] (5) at (3, 2) {$a_5$};
	%	\node [roundnode] (5b) at (3, 2.3) {};
	\node [squarednode] (6) at (2, 3) {$\cup$};
	\node [squarednode] (7) at (0, 3) {$a_1$};
	%	\node [roundnode] (7b) at (0, 3.3) {};
	\node [squarednode] (8) at (1, 4) {$\cup$};
	%	\node [roundnode] (8b) at (1, 4.3) {};
	\node [squarednode] (9) at (3, 4) {$a_6$};
	%	\node [roundnode] (9b) at (3, 4.3) {};
	\node [squarednode] (10) at (2, 5) {$\cup$};
	%	\node [roundnode] (12b) at (1, 6.3) {};
	
	\draw (2) to (0);
	\draw (2) to (1);
	\draw (4) to (2);
	\draw (4) to (3);
	\draw (6) to (4);
	\draw (6) to (5);
	\draw (8) to (6);
	\draw (8) to (7);
	\draw (10) to (8);
	\draw (10) to (9);
	
	\draw [dashed] (3,4.7) to (9);
	\draw [dashed] (9) to (5);
	\draw [dashed,out=-90,in=0] (5) to (2);
	\draw [dashed] (2) to (3);
	
\end{tikzpicture}
\hspace{1em}
\begin{tikzpicture}[scale=\figfourscaleval,->,>=stealth',roundnode/.style={circle,draw,inner sep=1.2pt},squarednode/.style={rectangle,inner sep=3pt}]
	
	\node [squarednode] (num) at (0, 5) {4.};
	\node [squarednode] (0) at (1, 0) {$a_3$};
	%	\node [roundnode] (0b) at (1, 0.3) {};
	\node [squarednode] (1) at (3, 0) {$a_4$};
	\node [squarednode] (2) at (2, 1) {$\cup$};
	%	\node [roundnode] (2b) at (2, 1.3) {};
	\node [squarednode] (3) at (0, 1) {$a_2$};
	%	\node [roundnode] (3b) at (0, 1.3) {};
	\node [squarednode] (4) at (1, 2) {$\cup$};
	%	\node [roundnode] (4b) at (1, 2.3) {};
	\node [squarednode] (5) at (3, 2) {$a_5$};
	%	\node [roundnode] (5b) at (3, 2.3) {};
	\node [squarednode] (6) at (2, 3) {$\cup$};
	\node [squarednode] (7) at (0, 3) {$a_1$};
	%	\node [roundnode] (7b) at (0, 3.3) {};
	\node [squarednode] (8) at (1, 4) {$\cup$};
	%	\node [roundnode] (8b) at (1, 4.3) {};
	\node [squarednode] (9) at (3, 4) {$a_6$};
	%	\node [roundnode] (9b) at (3, 4.3) {};
	\node [squarednode] (10) at (2, 5) {$\cup$};
	%	\node [roundnode] (12b) at (1, 6.3) {};
	
	\draw (2) to (0);
	\draw (2) to (1);
	\draw (4) to (2);
	\draw (4) to (3);
	\draw (6) to (4);
	\draw (6) to (5);
	\draw (8) to (6);
	\draw (8) to (7);
	\draw (10) to (8);
	\draw (10) to (9);
	
	\draw [dashed] (3,4.7) to (9);
	\draw [dashed] (9) to (5);
	\draw [dashed,out=-90,in=45] (5) to (1);
	\draw [dashed] (1) to (0);
	
\end{tikzpicture}
	\caption{Evolution of the stack $\ustack$ (represented by dashed arrows) for an iterator over the topmost union node in the figure. The underlying ECS contains only union nodes and six bottom nodes. The first figure is $\ustack$ after calling $\ustack \gets {\sf push}(\ustack, v)$, the second is after calling $\ustack \gets\textsc{Traverse}(\ustack)$. The last two figures represent successive calls to ${\sf pop}(\ustack), \ustack \gets\textsc{Traverse}(\ustack)$.}
	\label{nested:fig-enum-stacks}
\end{figure}

The methods of a union node iterator $\uit_\cup$ are then straightforward. For \textsc{create} (lines 34-37), we push $v$ into $\ustack$ (line 35) and traverse $\ustack$, finding the first rightmost output node from $v$ (line~36). Then we build the iterator $\uit$ of this output node for being ready to start enumerating their outputs (line~37). 
For \textsc{next}, we consume all outputs by calling $\uit.\textsc{next}$ (line 40). When there are no more outputs (lines 41-47), we pop the top node from $\ustack$ and check if the stack is empty or not (lines~41-42). If this is the case, there are no more outputs and we output {\bf false}. Instead, if $\ustack$ is non-empty but the top node $\text{\sf top}(\ustack)$ is a union node, we apply the \textsc{traverse} method for finding the rightmost output node from $\text{\sf top}(\ustack)$ (lines 44-45). When the procedure is done, we know that the top node is an output node, and then we create an iterator and move to its first output (lines~46-47). 
For \textsc{print}, we call the print method of $\uit$ which is ready to write the current output (lines 50-51). 

For proving the correctness of the enumeration procedure, since $\D$ is duplicate-free, one can verify that $\textsc{Enumerate}(v)$ in Algorithm~\ref{nested:alg:enumeration} enumerates all the set $\sem{\D}(v)$, one by one, and without repetitions. For the delay between outputs, since $\D$ is $k$-bounded, it is straightforward to prove by induction that, if $w_0$ is the first output, then:
\begin{itemize}
	\item $\textsc{create}(v)$ takes time $\cO(k\cdot |w_0|)$,
	\item $\textsc{next}$ takes time $\cO(k\cdot |w_0|)$ for the first call, and $\cO(k\cdot |w|+|w'|)$ for the next call where $w$ and $w'$ are the previous and next outputs, respectively, and
	\item $\textsc{print}$ takes time $\cO(k\cdot |w|)$ where $w$ is the current output to be printed.  
\end{itemize}
Overall, $\textsc{Enumerate}(v)$ in Algorithm~\ref{nested:alg:enumeration} have delay $O(k\cdot (|w| + |w'|))$ to write the next output $w'$ in the output register, after printing the previous output $w$. 

We end by pointing out that the existence of an enumeration algorithm $\enumE$ with delay $O(k\cdot (|w| + |w'|))$ between any consecutive outputs $w$ and $w'$, implies the existence of an enumeration algorithm $\enumE'$ with output-linear delay as defined in Section~\ref{nested:sec:enum}. We start noting that $k$ is a fixed value and then the delay of $\enumE$ only depends on $|w| + |w'|$. For depending only on the next output $w'$, one can perform the following strategy for $\enumE'$: start by running $\enumE$, enumerate the first output $w_0$, proceed $k \cdot |w_0|$ more steps of $\enumE$, stop, and print the separator symbol $\#$. Then continue running $\enumE$ to prepare the next output $w_1$, proceed $k |w_1|$ more steps, stop, and print the separator symbol $\#$. By repeating this enumeration process, one can verify that the delay between the $i$-th output $w_i$ and the $(i+1)$-th output $w_{i+1}$ is $O(|w_{i+1}|)$. Therefore, $\enumE'$ has output-linear delay. 
\end{proof}