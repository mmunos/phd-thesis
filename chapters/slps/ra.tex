%!TEX root = main.tex

In this section, we present the definition of \rtname, which is an extension of regular automata to produce outputs. We use \rtnames as our computational model to represent queries with output. 
	
	A \emph{\rtname} (\rt for short) is a tuple $\cA = (Q, \Sigma, \oalph, \Delta, \qinit, F)$ where $Q$ is a state set, $\Sigma$ is an input alphabet, $\oalph$ is an output alphabet, $\qinit \subseteq Q$ and $F \subseteq Q$ and $\Delta \subseteq Q \times (\Sigma \times \oalph) \times Q \,\cup\, Q \times \Sigma \times Q $ is the transition relation, which contains {\em read-write transitions} of the form $(p, (a, \oout), q)\in Q\times(\Sigma\times\oalph)\times Q$, and {\em read transitions} of the form  $(p, a, q)\in Q\times\Sigma\times Q$.
	
	As usual for transducers, a symbol $a \in \Sigma$ is an input symbol that the machine reads and $\oout \in \oalph$ is a symbol that indicates what the machine prints in an output tape.
	A run $\rho$ of $\cA$ over a document $d = a_1a_2\ldots a_n \in\Sigma^*$ is a sequence of the form
	$
	%\rho = (q_1, \sigma_1) \xrightarrow{s_1/\!\ooutscr_1} (q_2, \sigma_2) \xrightarrow{s_2/\!\ooutscr_2} \ldots  \xrightarrow{s_n/\!\ooutscr_n} (q_{n+1}, \sigma_{n+1})
	\rho = q_0 \xrightarrow{b_1} q_1 \xrightarrow{b_2} \ldots \xrightarrow{b_n} q_{n}
	$
	where $q_0 \in I$, and for each $i\in [1,n]$ one of the following holds: (1) $b_i = a_i$ and there is a transition $(q_{i-1}, a_i, q_i)\in \Delta$ or (2) $b_i = (a_i, \oout)$ and there is a transition $(q_{i-1}, a_i, \oout, q_i)\in \Delta$. We say that $\rho$ is accepting if $q_n\in F$.
	
	Finally, we define the {\em annotation} of $\rho$ as:
	$
	\ann(\rho) =  \ann(b_1)\cdot \ldots \cdot \ann(b_n)
	$
	such that $\ann(b_i) = (\oout, i)$ if $b_i = (a, \oout)$, and $\ann(b_i) = \eps$ otherwise, for each $i\in[1, n]$.
	%Note that in $\nu = \oout_1 \cdots \oout_n$, $\oeps$ is a symbol that may be present in the word, whereas $\eps$ indicates an empty word. 
	Given a \rtname $\cA$ and a document $d \in\Sigma^*$, we define the set $\sem{\cA}(d)$ of all outputs of $\cA$ over $d$ as:
	%$$
	%\sem{\cA}(w) \ = \ \{ \out(\rho) \, \mid \, \text{$\rho$ is an accepting run of $\cA$ over $w$}\}.
	%$$
	$
	\sem{\cA}(d) = \{ \ann(\rho) \, \mid \, \text{$\rho$ is an accepting run of $\cA$ over $d$}\}.
	$
	
	
	We say that a \rt $\cA = (Q, \Sigma, \oalph, \Delta, \qinit, F)$ is %\emph{input/output deterministic} (I/O-deterministic for short) 
	{\it deterministic} if $|\qinit| = 1$ and $\Delta$ is a partial function of the form $\Delta: (Q \times \Sigma \, \cup\, Q\times (\Sigma\times\oalph)) \to Q$.
	%for each triple $(p, a, \oout) \in Q \times \opS \times \oalph$,  there is at most one tuple $(q, \mathtt{x}) \in Q \times \Gamma$ such that $(p, a, \oout, q, \mathtt{x})\in\Delta$, 
	%for each tuple $(p, a, \oout, \mathtt{x})\in Q \times \clS \times \oalph \times \Gamma$ there is at most one $q\in Q$ such that $(p, a, \oout, \mathtt{x}, q)\in\Delta$, 
	%and for each triple  $(p, a, \oout) \in Q \times \noS \times \oalph$ there is at most one $q\in Q$ such that $(p, a, \oout, q) \in \Delta$. 
	On the other hand, we say that $\cA$ is 
	%\emph{input/output unambiguous} (I/O-unambiguous for short) 
	{\it unambiguous} if for every $d\in \Sigma^*$ and every $\nu \in \sem{\cA}(d)$ there is exactly one accepting run $\rho$ of $\cA$ over $d$ such that $\nu = \ann(\rho)$. 
	Note that an 
	%I/O-deterministic 
	deterministic \rt is also 
	%I/O-unambiguous.
	unambiguous. The definition of deterministic is in line with the notion of I/O-deterministic variable automata of~\cite{FlorenzanoRUVV20} and unambiguous is a generalization of this idea that is enough for the purpose of our enumeration algorithm. Actually, one can easily show that for every \rt $\cA$ there exists an equivalent deterministic \rt and, therefore, an equivalent unambiguous \rt. 
	\begin{lemma}\label{slps:ra:det}
		For every \rtname $\cA$ there exists an deterministic \rtname $\cA'$ such that $\sem{\cA}(d) = \sem{\cA'}(d)$ for every $d\in\Sigma^*$.
	\end{lemma}
	
	We are interested on the following enumeration problem for \rts. Let $\bigcal{C}$ be a class of \rt (e.g. deterministic \rt). 
	
	\begin{center}
		\framebox{
			\begin{tabular}{rl}
				\textbf{Problem:} & $\enumraa[\bigcal{C}]$\\
				\textbf{Input:} & a \rt $\cA \in \bigcal{C}$ and $d\in \Sigma^*$ \\
				\textbf{Output:} & Enumerate $\sem{\cA}(d)$
			\end{tabular}
		}
	\end{center}

