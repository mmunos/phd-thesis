%!TEX root = ../../Thesis.tex

In this chapter, we study the problem of enumerating results from a query over a compressed document. The model we use for compression are straight-line programs (SLPs), which are defined by a context-free grammar that produces a single string. For our queries, we use a model called Annotated Automata, an extension of regular automata that allows annotations on letters. This model extends the notion of regular spanners as it allows arbitrarily long outputs. 

The main result in this chapter is an algorithm that evaluates such a query by enumerating all results with output-linear delay after a preprocessing phase which takes linear time on the size of the SLP, and cubic time over the size of the automaton. We achieve this through a persistent data structure named Enumerable Compact Sets with Shifts which guarantees output-linear delay under certain restrictions.
These results imply constant-delay enumeration algorithms in the context of regular spanners. 

Further, we use an extension of annotated automata which utilizes succinctly encoded annotations to save an exponential factor from previous results that dealt with constant-delay enumeration over variable-set automata.
Lastly, we extend our results to allow complex document editing while maintaining the constant delay guarantee.

%\paragraph{Earlier version} 
%This paper is an extended version of the article ``{\it Constant-delay enumeration for SLP-compressed documents}\,'' that was published in the 26th International Conference on Database Theory (ICDT 2023).
%We have made a number of changes for this version. 
%The main one is that we have included full proofs for all of our results. 
%One in particular, the enumeration algorithm of the Shift-ECS data structure, was completely redone by introducing the use of iterators.
%This change makes the algorithm easier to follow and better suited for future implementation.
%Most importantly, it fixes a subtle mistake that was present in the previous algorithm. 
%This new version also includes additional examples and explanatory figures in an effort to improve readability.

\paragraph{Outline of the chapter} 
In Section~\ref{slps:sec:setting} we introduce the setting and its corresponding enumeration problem. 
In Section~\ref{slps:sec:ecs}, we present our data structure for storing and enumerating the outputs, and in Section~\ref{slps:sec:evaluation} we show the evaluation algorithm. 
Section~\ref{slps:sec:spanners} offers the application of the algorithmic results to document spanners, plus an extension for compressed annotation schemes, and Section~\ref{slps:sec:edits} shows how to extend these results to deal with complex document editing. 
%We finish the chapter with future work in Section~\ref{slps:sec:conclusions}. 
