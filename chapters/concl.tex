In this thesis, we developed a framework based on MSO to handle enumeration queries -- called the {\em Annotators} framework. As a proof of concept, we used it to build three different models that perform output-linear enumeration: (1) MSO queries over nested documents, (2) context-free grammar queries over strings, and (3) regular queries over SLP-compressed documents. These results are given with several extensions: (1) allows one-pass-streaming computation over the document, (2) is given for three different fragments of context-free grammars, each with tight complexity bounds, and (3) are shown to be compliant with other useful frameworks that work on compressed documents. Furthermore, the work improves on other known results in the literature, especially from the perspective of document spanners.

We attribute these achievements to two main factors:

The use of an intuitive model for the query instances, namely, the Annotator framework. While document spanners have been studied extensively and with reasonable success, the task of enumerating span assignments can be simplified, in our opinion, a lot by forgetting about the spans themselves. Annotated automata and their derivatives can be seen as a version of document spanners in which a ever-present difficulty, which is having multiple symbols per document position, is completely avoided.

Using a modular data structure which is almost fully independent to the query language to represent the enumerable output data. Although the algorithm itself that builds the structure can be derived from already known results for circuits, the Enumerable Compact Sets and Enumerable Compact Sets with Shifts data structures streamline the building process significantly. The differences between our framework and previous algorithms for circuits are most patent in Chapter~\ref{ch1}, as it is used to update the complete data structure in constant (data-independent) time.
