Streaming query evaluation~\cite{altinel2000efficient,babcock2002models} is the task of processing queries over data streams in one pass and with a limited amount of resources. This approach is especially useful on the web, where servers share data, and they have to extract the relevant content as they receive it. For structuring the data, the de facto structure on the web is nested documents, like XML or JSON. For querying, servers use languages designed for these purposes, like XPath, XQuery, or JSON query languages.
As an illustrative example, suppose our data server (e.g., a Web API) is continuously receiving XML documents like:
\smallskip
\begin{center}
	\texttt{
		<doc> <a> <b/> <c/> <b/> </a> <c> <b/> <b/> </c> </doc> ...}
\end{center}
\smallskip
%\vspace{-1mm}
and for each document it has to evaluate the query $\cQ = \texttt{//a/b}$ (i.e., to extract all $b$-tags that are surrounded by an $a$-tag). The streaming query evaluation problem consists of reading these documents and finding all $b$-tags without storing the entire document on memory, that is, by making one pass over the data and spending constant time per tag. In our example, we need to retrieve the 3rd and 5th tags as soon as the last tag $\texttt{</doc>}$ is received. One could consider here that the server has to read an infinite stream and perform the query evaluation continuously, where it must enumerate partial outputs as soon as one of the XML documents ends.
%Therefore, the streaming evaluation of these queries over nested documents requires reading the data and producing all the answers as efficiently as possible.}

Researchers have studied the streaming query evaluation problem in the past, focusing on reducing the processing time or memory usage (see, e.g. \cite{BarYossefFJ07}). Hence, they spent less effort on understanding the enumeration time of such a problem, with respect to delay guarantees between outputs.
Constant-delay enumeration is a new notion of efficiency for retrieving outputs~\cite{DurandG07,Segoufin13}.
Given an instance of the problem, a constant-delay enumeration algorithm performs a preprocessing phase over the instance to build some indices and then continues with an enumeration phase. It retrieves each output, one by one, taking a delay that is constant between any two consecutive outcomes. These algorithms provide a strong guarantee of efficiency since a user knows that, after the preprocessing phase, {she} will access the output as if {the algorithm had} already computed {it}. These techniques have attracted researchers' attention, finding sophisticated solutions to several query evaluation problems~\cite{BaganDG07,BerkholzGS20,Bagan06,AmarilliBJM17,FlorenzanoRUVV20,AmarilliBMN19}. 

In this chapter, we investigate the streaming query evaluation problem over nested documents by including enumeration guarantees, like constant delay. We study the evaluation of queries given by \vpannnames (\vpanns) over nested documents.  These machines are an ``output extension'' of visibly pushdown automata, %, a model for streaming evaluation over nested documents. 
{and have the same expressive power as MSO over nested documents. 
In particular, \vpanns can define queries like $\cQ$ above or any fragment of query languages for XML or JSON included in MSO.}
Therefore, \vpanns allow considering the streaming query evaluation from a more general perspective, without getting married to a specific language (e.g., XPath). 
%Furthermore, they are general enough to understand the regular behavior of such query languages and, at the same time, they are expressive enough for defining outputs. 

We study the evaluation of \vpann over a nested document in a streaming fashion. Specifically, we want to find a streaming algorithm that reads the document sequentially and spends as little time as possible per input symbol. 
Furthermore, whenever needed, the algorithm can enumerate all outputs with output-linear delay. %, a refined notion of constant-delay for varied size results.  
The main contribution in this chapter is an algorithm with such characteristics for the class of unambiguous \vpanns. We can extend this algorithm to all \vpanns by determinization (i.e., in data complexity). 
Regarding memory usage, we bound the amount of memory used in terms of the nesting of the document and the output weight. We show that our algorithm is worst-case optimal in the sense that there are instances where the maximum amount of memory required by any streaming algorithm is at least one of these two measures.
Finally, we present some examples that show how our result can be applied to the streaming evaluation of XML and JSON query languages.
Further, we show an application of our results in the context of information extraction by document spanners~\cite{FaginKRV15}.

%We base our approach on a fully-persistent data structure~\cite{driscoll1986making} called an Enumerable Compact Set (ECS) and an algorithm that mimics the abstract machine's execution. We present this algorithm as a sequence of pseudo-code instructions that could be of special interest for practical purposes. Moreover, we believe that the presentation helps to think on further optimizations and better understand constant-delay algorithms over nested documents.  

%Our main result applies to the streaming evaluation of XML and JSON query languages. 
%In the appendix, we also show an application in the context of document spanners~\cite{FaginKRV15}.}
