% !TeX spellcheck = en_US
%!TEX root = ../../Thesis.tex

We start by stating the definitions of well-nested words and visibly pushdown automata. Further, we state some basic definitions that will be useful throughout the chapter. 

\paragraph{Well-nested words and streams} Besides the usual definitions for strings, we will work over a {\em structured alphabet} $\Sigma = (\opS, \clS, \noS)$ comprised of three disjoint sets $\opS$, $\clS$, and $\noS$ that contain {\it open}, {\it close}, and {\it neutral} symbols respectively\footnote{In \cite{AlurM04,FiliotRRST18} these sets are named \emph{call}, \emph{return}, and \emph{local}, respectively.}. 
Furthermore, we will denote symbols in $\opS$, $\clS$ or $\noS$ by $\op{a}$, $\cl{a}$, and~$a$, respectively.
On the other hand, we will use $s$ to denote any symbol in $\opS$, $\clS$, or $\noS$.
The set of {\em well-nested words}
%\footnote{In~\cite{AlurM04}, Alur et al. consider a slightly more general set of nested words, where there could be close symbols at the beginning of the word that are not open and open symbols at the end that are never close. We can extend our setting to support this generalization at the cost of considering these border cases. For the sake of simplicity, we restrict our work to nested words without major loss of generality.} 
over $\Sigma$, denoted as $\wnS$, is defined as the smallest set satisfying the following rules: 
$\noS \cup \{\eps\} \subseteq \wnS$,
if $w_1, w_2 \in \wnS$ then $w_1 \cdot w_2 \in \wnS$, and if $w \in \wnS$ and $\op{a} \in \opS$ and $\cl{b} \in \clS$ then $\op{a}\cdot w\cdot\cl{b} \in \wnS$. 
In addition, we will work with prefixes of well-nested words, that we call {\em prefix-nested words}. We denote the set of prefixes of $\wnS$ as $\pwnS$.
Sometimes, we will use $w[i]$ to refer to the $i$-th symbol in a word $w$.

\paragraph{Visibly pushdown automata} A \emph{visibly pushdown automaton}~\cite{AlurM04} (\vpa) is a tuple:
$$
\cA = (Q, \Sigma, \Gamma, \Delta, \qinit, F)
$$ 
where $Q$ is a finite set of states, $\Sigma = (\opS, \clS, \noS)$ is the structured input alphabet, $\Gamma$ is the stack alphabet, $\qinit \subseteq Q$ is a set of initial states, $F\subseteq Q$ is a set of final states, and
$$
\Delta \ \subseteq \ (Q\times\opS\times Q\times\Gamma) \,\cup \, (Q\times\clS\times\Gamma\times Q) \, \cup \, (Q\times\noS\times Q)
$$ 
is the transition relation.
A transition $(q,\op{a},q',\gamma)$ is a {\em push-transition} where on reading $\op{a}\in\opS$, the symbol $\gamma$ is pushed onto the stack and the current state switches from $q$ to $q'$. Conversely, $(q,\cl{a},\gamma,q')$ is a {\em pop-transition} where on reading $\cl{a}\in\clS$ from the input and $\gamma$ from the top of the stack, the current state changes from $q$ to $q'$, and the symbol $\gamma$ is popped. Lastly, we say that $(q,a,q')$ is a {\em neutral transition} if $a\in\noS$, where there is no stack operation.

A \emph{stack} is a finite sequence $\sigma$ over $\Gamma$ where the top of the stack is the first symbol on~$\sigma$. For a well-nested word $w = s_1 \cdots s_n$ in $\wnS$, a \emph{run} of $\cA$ on $w$ is a sequence:
$$
\rho = (q_1,\sigma_1) \trans{s_1} (q_2,\sigma_2) \trans{s_2} \ldots \trans{s_n} (q_{n+1},\sigma_{n+1}),
$$ 
where each $q_j \in Q$ and $\sigma_j\in\Gamma^{*}$ for every $j\in[1,n]$, $q_1\in \qinit$, $\sigma_1 = \eps$, and for every $i\in[1,n]$ the following holds: 
(1) if $s_{i}\in\opS$, then there is $\gamma\in\Gamma$ such that $(q_i,s_{i},q_{i+1},\gamma) \in \Delta$ and $\sigma_{i+1} = \gamma\sigma_i$, 
(2) if $s_{i}\in\clS$, then there is $\gamma\in\Gamma$ such that $(q_i,s_{i},\gamma,q_{i+1}) \in \Delta$ and $\sigma_{i} = \gamma\sigma_{i+1}$, and
(3) if $s_{i}\in\noS$, then $(q_i,s_{i},q_{i+1}) \in \Delta$ and $\sigma_{i+1} = \sigma_i$.
A run $\rho$ is \emph{accepting} if $q_{n+1}\in F$. A well-nested word $w\in\wnS$ is \emph{accepted by} a \vpa $\cA$ if there is an accepting run of $\cA$ on $w$. The \emph{language} $\cL(\cA)$ is the set of well-nested words accepted by $\cA$. Note that if $\rho$ is an accepting run of $\cA$ on a well-nested word $w$, then $\sigma_{n+1} = \eps$. A set of well-nested words $\cL \subseteq \wnS$ is called a \emph{visibly pushdown language} if there exists a \vpa $\cA$ such that $\cL = \cL(\cA)$.

A \vpa $\cA = (Q, \Sigma, \Gamma, \delta, \qinit, F)$ is said to be \emph{deterministic} if $|\qinit| =1$ and $\delta$ is a function subset of the set $(Q\,\times\opS \to Q\times \Gamma) \cup
(Q\times\clS\times\Gamma \to Q)\cup
(Q\times\noS \to Q)$. We also say that $\cA$ is \emph{unambiguous} if, for every $w \in \cL(\cA)$, there exists exactly one accepting run of $\cA$ on $w$. In~\cite{AlurM04}, it is shown that for every \vpa there exists an equivalent deterministic \vpa of at most exponential size. 

\paragraph{Streams} A {\em stream} $\Stream = s_1 s_2 \cdots$ is an infinite sequence where $s_i\in \opS \cup \clS \cup \noS$. Given a stream $\Stream = s_1 s_2 \ldots$ and positions $i,j\in\nat \setminus \{0\}$ such that $i \leq j$, the word $\Stream[i,j]$ is the sequence $s_i s_{i+1} \cdots s_j$. We also use this notation to refer to contiguous subsequences of infinite sequences that are not composed of symbols in $\Sigma$. For a stream $\Stream$, we will always assume that for each $i \in \nat \setminus \{0\}$, the word $\Stream[1,i]$ is a prefix of some nested word (i.e., it can be completed to form a nested word). For our streaming algorithms, we also consider the method $\yield{\Stream}$ which can be called to access each element of $\Stream$ sequentially.