%!TEX spellcheck = en_US
%!TEX root = ../../Thesis.tex

%\cristian{También dividir esta sección en subsecciones, ya que esta muy larga.}

This section presents a data structure, called \dsnamebigcaps{} (\dsabbr{}), which is the cornerstone of our enumeration algorithm for \vpann. This data structure is strongly inspired by the work in~\cite{AmarilliBJM17,AmarilliBMN19}. Indeed, \dsabbr{} can be considered a refinement of the d-DNNF circuits used in~\cite{AmarilliBJM17} or of the set circuits used in~\cite{AmarilliBMN19}.
Several papers~\cite{OlteanuZ15,AmarilliBJM17,AmarilliBMN19pods,Torunczyk20}   have considered circuits-like structures for encoding outputs and enumerate them with constant delay. The novelty of ECS is twofold. First, we use ECS for solving a streaming evaluation problem. Although people have studied streaming query evaluation with enumeration before~\cite{IdrisUV17,BerkholzKS17}, this is the first work that uses a circuit-like data structure in an online setting. 
Second and more important, there is a difference in performance if we compare ECS to the previous approaches.
In offline evaluation, constant-delay algorithms usually create an initial circuit from the input, making several passes over the structure, building indices, and then running the enumeration process. Given time restrictions for the online evaluation, we cannot create a circuit and do this linear-time preprocessing before enumerating. On the contrary, we must extend the circuit-like data structure for each data item in constant time and then be ready to start the enumeration. This requirement justifies the need for a new data structure for representing and enumerating outputs.  Therefore, ECS differs from previous proposals because each operation must take constant time, and we can run the enumeration process with output-linear delay, at any time and without any further preprocessing. In the following, we present \dsabbr{} step-by-step to use them later in the next section.


%The main difference between \dsabbr{} and the knowledge-compilation approach is that we see a ``circuit'' as a data structure and treat it as such. With this, we can avoid using the heavy notation of circuits, or applying special operations to convert them, or computing indices to enumerate the output. Instead, we use \dsabbr{} to simplify the presentation and apply all these reductions and optimizations at once, improving the conceptual understanding of the structure. The conceptual contributions of this approach are to understand that we need a succinct data structure to store the outputs, we need to retrieve these outputs with output-linear delay, and we need the data structure to be fully-persistent~\cite{driscoll1986making}, that is, it always preserves the previous version of itself when it is modified. In fact, this last property is crucial for our update algorithm to manage different runs that share part of the same output. In the following, we present \dsabbr{} step-by-step to use it later in the next section.

Let $\infAlph$ be a (possibly infinite) alphabet. We define an \emph{\dsnamebigcaps{}} (\dsabbr) as a tuple:
$$
\D = (\infAlph, V, \ell, r, \lambda)
$$ 
such that $V$ is a finite set of nodes, $\ell\colon V\to V$ and $r\colon V\to V$ are the {\em left} and {\em right} partial functions, and $\lambda\colon V\to\infAlph\cup\{\cup,\odot\}$ is a labeling function. For every node $v\in V$, both $\ell(v)$ and $r(v)$ are defined if $\lambda(v)\in\{\cup,\odot\}$, and neither is defined otherwise. Further, we assume that the directed graph $(V, \{(v,\ell(v)),(v,r(v))\mid v\in V\})$ is acyclic. Note how this implies that nodes labeled by $\infAlph$ are bottom nodes in the DAG and nodes labelled by $\cup$ or $\odot$ are inner nodes. We will use the terms {\it inner} and {\it bottom} to refer to such nodes, respectively. Furthermore, we say that $v$ is a {\em product node} if $\lambda(v) = \odot$, and a {\em union node} if $\lambda(v) = \cup$. We define the size of $\D$ as $|\D| = |V|$.
For each node $v$ in $\D$, we associate a set of words $\sem{\D}(v)$ recursively as follows: (1) $\sem{\D}(v) = \{a\}$ whenever $\lambda(v) = a\in\infAlph$, (2) $\sem{\D}(v) = \sem{\D}(\ell(v)) \cup \sem{\D}(r(v))$ whenever $\lambda(v) = \cup$, and (3) $\sem{\D}(v) = \sem{\D}(\ell(v)) \cdot \sem{\D}(r(v))$ whenever $\lambda(v) = \odot$,
where $L_1 \cdot L_2 = \{w_1\cdot w_2 \mid w_1\in L_1\text{ and }w_2\in L_2\}$.

Since the size of the set $\sem{\D}(v)$ can be exponential with respect to $|\D|$, we say that $\D$ is a \emph{compact} representation of $\sem{\D}(v)$ for any $v \in V$.
Although $\sem{\D}(v)$ is very large, the goal is to enumerate all of its elements efficiently. Specifically, we consider the following problem:
%\vspace{.1cm}
\begin{center}
	\framebox{
		\begin{tabular}{rl}
			\textbf{Problem:} & $\enumds$\\
			\textbf{Input:} & A \dsabbr{} $\D = (\infAlph, V, \ell, r, \lambda)$ and $v \in V$.  \\
			\textbf{Output:} & Enumerate the set $\sem{\D}(v)$ without repetitions. \\
		\end{tabular}
	}
\end{center}
\vspace{.1cm}
Additionally, we want to solve $\enumds$ with output-linear delay. 
A reasonable strategy to enumerate $\sem{\cD}(v)$ is to do a sequence of traversals on the structure, each of which builds a different output from the set. 
However, to be able to do this without repetitions and output-linear delay, we need to guarantee two conditions: first that one can obtain every output from $\cD$ in only one way and, second, that union nodes are {\it close} to a bottom node or a product node, in the sense that we can always reach one of them in a bounded number of steps. To ensure that these conditions hold, we impose two restrictions on an ECS:
\begin{enumerate}
	\item We say that $\D$ is \emph{duplicate-free} if $\D$ satisfies the following two properties: (1) for every union node $v$ it holds that $\sem{\D}(\ell(v))$ and $\sem{\D}(r(v))$ are disjoint, and (2) for every product node $v$ and for every $w \in \sem{\D}(v)$, there exists a unique way to decompose $w = w_1 \cdot w_2$ such that $w_1 \in \sem{\D}(\ell(v))$ and $w_2 \in \sem{\D}(r(v))$. 
	
	\item We define the notion of \emph{$k$-bounded}~\dsabbr{} as follows. 
	Given a \dsabbr{} $\D$, define the (left) output-depth of a node $v\in V$, denoted by $\odepth_{\D}(v)$, recursively as follows:
	$\odepth_{\D}(v) = 0$ whenever $\lambda(v)\in\infAlph$ or $\lambda(v) = \odot$, and $\odepth_{\D}(v) = \odepth_{\D}(\ell(v))+1$ whenever $\lambda(v) = \cup$.
	Then, for $k\in\nat$ we say that $\D$ is $k$-bounded if $\odepth_{\D}(v)\leq k$ for all $v\in V$.
\end{enumerate}



Given the definition of output depth, we say that $v$ is an output node of $\D$ if $v$ is a bottom node or a product node. 
Note that if $\D$ only has output nodes, then it is 0-bounded, and one can easily check that $\sem{\D}(v)$ can be enumerated with output-linear delay.
Indeed, for a fixed $k$ the same happens with every duplicate-free and $k$-bounded \dsabbr{}.


\begin{proposition}\label{nested:ds:lindelay}
	Fix $k\in\nat$. Let $\D = (\infAlph, V, \ell, r, \lambda)$ be a duplicate-free and $k$-bounded \dsabbr{}. Then one can enumerate the set $\sem{\D}(v)$ with output-linear delay for any $v\in V$.
\end{proposition}
% !TeX spellcheck = en_US


\begin{proof}
Let $\D = (\infAlph, V, \ell, r, \lambda)$ be a duplicate-free and $k$-bounded \dsabbr{}.
The algorithm that we present is a depth-first traversal of the DAG, done in a recursive fashion to ensure that after retrieving some output $w$, the next one $w'$ can be printed in $O(k\cdot (|w| + |w'|))$ time. The entire procedure is detailed in Algorithm~\ref{nested:alg:enumeration}.

%!TEX root = ../../Thesis.tex


\algdef{SE}[DOWHILE]{Do}{doWhile}{\algorithmicdo}[1]{\algorithmicwhile\ #1}%
\algnewcommand{\LineComment}[1]{\State \(\triangleright\) #1}



\begin{algorithm*}[t]
	\caption{Enumeration over a node $u$ from some ECS $\cD = (\infAlph, V, \ell, r, \lambda)$.}\label{nested:alg:enumeration}
	\smallskip
	\begin{varwidth}[t]{0.50\textwidth}
		\begin{algorithmic}[1]
			\LineComment{Bottom node iterator $\uit_\infAlph$} \label{nested:alg1bottom1}
			\Procedure{{create}}{$v$}\ \ \ \ \Comment{Assume $\lambda(v) \in \infAlph$} \label{nested:alg1bottom2}
			\State $\bnodelabel \gets v$ \label{nested:alg1bottom3}
			\State $\hasnext \gets {\bf true}$ \label{nested:alg1bottom4}
			\EndProcedure	
			
			\State
			
			\Procedure{{next}}{}
			\If{$\hasnext = \bf{true}$}
			\State $\hasnext \gets \bf{false}$
			\State \Return $\bf{true}$
			\EndIf
			\State \Return $\bf{false}$
			\EndProcedure
			
			\State
			
			\Procedure{{print}}{}
			\State ${\bf print}(\lambda(\bnodelabel))$
			\EndProcedure
			
			\State
			
			\LineComment{Product node iterator $\uit_\odot$}
			\Procedure{{create}}{$v$}\ \ \ \ \Comment{Assume $\lambda(v) = \odot$}
			\State $u \gets v$
			\State $\leftit \gets \textsc{create}(\ell(u))$
			\State $\leftit.\textsc{next}$
			\State $\rightit \gets \textsc{create}(r(u))$
			\EndProcedure	
			
			\State
			
			\Procedure{{next}}{}
			\If{$\rightit.\textsc{next} = \bf{false}$}
			\If{$\leftit.\textsc{next} = \bf{false}$}
			\State \Return $\bf{false}$
			\EndIf
			\State $\rightit \gets \textsc{create}(r(u))$
			\State $\rightit.\textsc{next}$
			\EndIf
			\State \Return $\bf{true}$
			\EndProcedure
			
			\State
			
			\Procedure{{print}}{}
			\State $\leftit.\textsc{print}$
			\State $\rightit.\textsc{print}$
			\EndProcedure
			\algstore{myalg}
		\end{algorithmic}
	\end{varwidth} \hfill
	\begin{varwidth}[t]{0.50\textwidth}
		\begin{algorithmic}[1]
			\algrestore{myalg}
			\LineComment{Union node iterator $\uit_\cup$}
			\Procedure{{create}}{$v$}\ \ \ \ \Comment{Assume $\lambda(v) = \cup$}
			\State $\ustack\gets\text{\sf push}(\ustack, v)$
			\State $\ustack\gets\textsc{traverse}(\ustack)$
			\State $\uit \gets \textsc{create}(\text{\sf top}(\ustack))$
			\EndProcedure	
			
			\State
			
			\Procedure{{next}}{}
			\If{$\uit.\textsc{next} = \bf{false}$}
			\State $\ustack\gets\text{\sf pop}(\ustack)$
			\If{$\text{\sf length}(\ustack) = 0$}
			\State \Return $\bf{false}$
			\ElsIf{$\lambda(\text{\sf top}(\ustack)) = \cup$}
			\State $\ustack\gets\textsc{traverse}(\ustack)$
			\EndIf
			\State $\uit \gets \textsc{create}(\text{\sf top}(\ustack))$
			\State $\uit.\textsc{next}$
			\EndIf
			\State \Return $\bf{true}$
			\EndProcedure
			
			\State
			
			\Procedure{{print}}{}
			\State $\uit.\textsc{print}$
			\EndProcedure
			
			\State
			
			\Procedure{{traverse}}{$\ustack$}
			\While{$\lambda(\text{\sf top}(\ustack)) = \cup$}
			\State $u \gets \text{\sf top}(\ustack)$
			\State $\ustack \gets \text{\sf pop}(\ustack)$
			\State $\ustack\gets\text{\sf push}(\ustack, r(u))$
			\State $\ustack\gets\text{\sf push}(\ustack, \ell(u))$
			\EndWhile
			\State \Return $\ustack$
			\EndProcedure
			
			\State
			
			\Procedure{{enumerate}}{$v$}
			\State $\uit \gets \textsc{create}(v)$
			\While{$\uit.\textsc{next} = \bf{true}$}
				\State $\uit.\textsc{print}$
			\EndWhile
			\EndProcedure
		\end{algorithmic}
	\end{varwidth}
	\smallskip
\end{algorithm*}


To simplify the presentation of the algorithm, we use the interface of an \emph{iterator} that, given a node $v$, it contains all information and methods to enumerate the outputs $\sem{\D}(v)$. Specifically, an iterator $\uit$ must implement the following three methods:
\[
\begin{array}{rclrclrcl}
	\textsc{create}(v) & \!\!\!\!\rightarrow\!\!\!\! & \uit   \ \ \ \ \ \	 & \uit.\textsc{next} & \!\!\!\! \rightarrow \!\!\!\! & b  \ \ \ \ \ \ \ &  \uit.\textsc{print}  & \!\!\!\!\rightarrow\!\!\!\! & \emptyset
\end{array}
\]
where $v$ is a node, $b$ is either {\bf true} or {\bf false}, and $\emptyset$ means ``no output''. 
The first method, \textsc{create}, receives a node $v$ and creates an iterator $\uit$ of the type of $v$. We will implement three types of iterators, one for each node type: bottom, product, and union nodes. The second method, $\uit.\textsc{next}$, moves the iterator to the next output, returning {\bf true} if, and only if, there is an output to print. Then the last method, $\uit.\textsc{print}$, write the current output pointed by $\uit$ to the output registers. 
We assume that, after creating an iterator $\uit$, one must first call $\uit.\textsc{next}$ to move to the first output before printing. Furthermore, if $\uit.\textsc{next}$ outputs {\bf false}, then the behavior of $\uit.\textsc{print}$ is undefined. Note that one can call $\uit.\textsc{print}$ several times, without calling $\uit.\textsc{next}$, and the iterator will write the same output each time in the output registers. 

Assume we can implement the iterator interface for each node type. Then the procedure $\textsc{Enumerate}(v)$ in Algorithm~\ref{nested:alg:enumeration} (lines 61-64) shows how to enumerate the set $\sem{\D}(v)$ by using an iterator $\uit$ for $v$. In the following, we show how to implement the iterator interface for each node type and how the size of the following output bounds the delay between two outcomes.

We start by presenting the iterator $\uit_\infAlph$ for a bottom node $v$ (lines 1-13), called a \emph{bottom node iterator}. We assume that each $\uit_\infAlph$ has internally two values $\bnodelabel$ and $\hasnext$, where $\bnodelabel$ is a reference to $v$ and $\hasnext$ is a boolean variable. The purpose of a bottom node iterator is only to print $\lambda(u)$. For this goal, when we create $\uit_\infAlph$, we initialize $u$ equal to $v$ and $\hasnext = {\bf true}$ (lines 3-4). Then, when we call $\uit_\infAlph.\textsc{next}$ for the first time, we swap $\hasnext$ from {\bf true} to {\bf false} and output {\bf true} (i.e., there is one output ready to be returned). Then any following call to $\uit_\infAlph.\textsc{next}$ will be false (lines 6-10). Finally, the $\uit_\infAlph.\textsc{print}$ writes the value $\lambda(u)$ to the output registers (lines 12-13). Here, we assume the existence of a method {\bf print} on the RAM model for writing to the output registers. 

For a product node, we present a \emph{product node iterator} $\uit_\odot$ in Algorithm~\ref{nested:alg:enumeration} (lines 15-32). This iterator receives a product node $v$ with $\lambda(v) = \odot$ and stores a reference of $v$, called $u$, and two iterators $\leftit$ and $\rightit$, for iterating through the left and right nodes $\ell(u)$ and $r(u)$, respectively. 
The \textsc{create} method initializes $u$ with $v$, creates the iterators $\leftit$ and $\rightit$, and calls $\leftit.\textsc{next}$ to be prepared for the first call of $\uit_\odot.\textsc{next}$ (lines 16-20).
The idea of $\uit_\odot.\textsc{next}$ is to fix one output for the left node $\ell(u)$ and iterate over all outputs of $r(u)$ (lines 22-28). When we stop enumerating all outputs of $\sem{\D}(r(u))$, we move to the next output of $\leftit$, and iterate again over all $\sem{\D}(r(u))$ (lines 24-27).
For printing, we recursively call first the printing method of $\leftit$, and then the one of $\rightit$ (lines 30-32).

The most involved case is the \emph{union node iterator} $\uit_\cup$ (lines 33-59). Similarly to a product node iterator, it receives a union node $v$ with $\lambda(v) = \cup$ and keeps a \emph{stack} of nodes $\ustack$ and an iterator~$\uit$. We assume the standard implementation of a stack with the native methods $\text{\sf push}$, $\text{\sf pop}$, $\text{\sf top}$, and $\text{\sf length}$ over stacks -- the first three define the standard operations over stacks, and $\text{\sf length}$ counts the elements in it. 
The purpose of the stack is to perform a \emph{depth-first-search} traversal of all union nodes below $v$, reaching all possible output nodes $u$ such that there is a path of union nodes between $v$ and $u$. If the top node of $\ustack$ is an output node, then $\uit$ is an iterator for that node, which enumerates all their outputs.  

For performing the \emph{depth-first-search} traversal of union nodes, we use the auxiliary method $\textsc{traverse}(\ustack)$ (lines 53-59). While the top node $u$ in $\ustack$ is a union node, it pops $u$ and pushes its right and left nodes in that order. 
This method will eventually reach an output node (i.e., a non-union node) at the top of the stack and end. 
It is important to note that $\textsc{traverse}(\ustack)$ takes $\cO(k)$-steps, given that the ECS is $k$-bounded. Then if $k$ is fixed, the  $\textsc{traverse}$ procedure takes constant time. In Figure~\ref{nested:fig-enum-stacks}, we illustrate the evolution of a stack $\ustack$ inside a union node iterator when we call $\textsc{traverse}(\ustack)$ several times. 
\begin{figure}[t]
	\centering
	%!TEX root = ../main/main.tex
\newcommand{\figfourscaleval}{0.83}
\begin{tikzpicture}[scale=\figfourscaleval,->,>=stealth',roundnode/.style={circle,draw,inner sep=1.2pt},squarednode/.style={rectangle,inner sep=3pt}]
	
	\node [squarednode] (num) at (0, 5) {1.};
	\node [squarednode] (0) at (1, 0) {$a_3$};
	%	\node [roundnode] (0b) at (1, 0.3) {};
	\node [squarednode] (1) at (3, 0) {$a_4$};
	%	\node [roundnode] (1b) at (3, 0.3) {};
	\node [squarednode] (2) at (2, 1) {$\cup$};
	%	\node [roundnode] (2b) at (2, 1.3) {};
	\node [squarednode] (3) at (0, 1) {$a_2$};
	%	\node [roundnode] (3b) at (0, 1.3) {};
	\node [squarednode] (4) at (1, 2) {$\cup$};
	%	\node [roundnode] (4b) at (1, 2.3) {};
	\node [squarednode] (5) at (3, 2) {$a_5$};
	%	\node [roundnode] (5b) at (3, 2.3) {};
	\node [squarednode] (6) at (2, 3) {$\cup$};
	%	\node [roundnode] (6b) at (2, 3.3) {};
	\node [squarednode] (7) at (0, 3) {$a_1$};
	\node [squarednode] (8) at (1, 4) {$\cup$};
	\node [squarednode] (9) at (3, 4) {$a_6$};
	%	\node [roundnode] (9b) at (3, 4.3) {};
	\node [squarednode] (10) at (2, 5) {$\cup$};
	%	\node [roundnode] (12b) at (1, 6.3) {};
	
	\draw (2) to (0);
	\draw (2) to (1);
	\draw (4) to (2);
	\draw (4) to (3);
	\draw (6) to (4);
	\draw (6) to (5);
	\draw (8) to (6);
	\draw (8) to (7);
	\draw (10) to (8);
	\draw (10) to (9);
	
	\draw[dashed] (1.2,5) to (10);
	
\end{tikzpicture}
\hspace{1em}
\begin{tikzpicture}[scale=\figfourscaleval,->,>=stealth',roundnode/.style={circle,draw,inner sep=1.2pt},squarednode/.style={rectangle,inner sep=3pt}]
	
	\node [squarednode] (num) at (0, 5) {2.};
	\node [squarednode] (0) at (1, 0) {$a_3$};
	%	\node [roundnode] (0b) at (1, 0.3) {};
	\node [squarednode] (1) at (3, 0) {$a_4$};
	%	\node [roundnode] (1b) at (3, 0.3) {};
	\node [squarednode] (2) at (2, 1) {$\cup$};
	%	\node [roundnode] (2b) at (2, 1.3) {};
	\node [squarednode] (3) at (0, 1) {$a_2$};
	\node [squarednode] (4) at (1, 2) {$\cup$};
	\node [squarednode] (5) at (3, 2) {$a_5$};
	%	\node [roundnode] (5b) at (3, 2.3) {};
	\node [squarednode] (6) at (2, 3) {$\cup$};
	\node [squarednode] (7) at (0, 3) {$a_1$};
	%	\node [roundnode] (7b) at (0, 3.3) {};
	\node [squarednode] (8) at (1, 4) {$\cup$};
	%	\node [roundnode] (8b) at (1, 4.3) {};
	\node [squarednode] (9) at (3, 4) {$a_6$};
	%	\node [roundnode] (9b) at (3, 4.3) {};
	\node [squarednode] (10) at (2, 5) {$\cup$};
	%	\node [roundnode] (12b) at (1, 6.3) {};
	
	\draw (2) to (0);
	\draw (2) to (1);
	\draw (4) to (2);
	\draw (4) to (3);
	\draw (6) to (4);
	\draw (6) to (5);
	\draw (8) to (6);
	\draw (8) to (7);
	\draw (10) to (8);
	\draw (10) to (9);
	
	\draw [dashed] (3,4.7) to (9);
	\draw [dashed,out=-90,in=0] (9) to (6);
	\draw [dashed] (6) to (7);
	
\end{tikzpicture}
\hspace{1em}
\begin{tikzpicture}[scale=\figfourscaleval,->,>=stealth',roundnode/.style={circle,draw,inner sep=1.2pt},squarednode/.style={rectangle,inner sep=3pt}]
	
	\node [squarednode] (num) at (0, 5) {3.};
	\node [squarednode] (0) at (1, 0) {$a_3$};
	\node [squarednode] (1) at (3, 0) {$a_4$};
	%	\node [roundnode] (1b) at (3, 0.3) {};
	\node [squarednode] (2) at (2, 1) {$\cup$};
	\node [squarednode] (3) at (0, 1) {$a_2$};
	%	\node [roundnode] (3b) at (0, 1.3) {};
	\node [squarednode] (4) at (1, 2) {$\cup$};
	%	\node [roundnode] (4b) at (1, 2.3) {};
	\node [squarednode] (5) at (3, 2) {$a_5$};
	%	\node [roundnode] (5b) at (3, 2.3) {};
	\node [squarednode] (6) at (2, 3) {$\cup$};
	\node [squarednode] (7) at (0, 3) {$a_1$};
	%	\node [roundnode] (7b) at (0, 3.3) {};
	\node [squarednode] (8) at (1, 4) {$\cup$};
	%	\node [roundnode] (8b) at (1, 4.3) {};
	\node [squarednode] (9) at (3, 4) {$a_6$};
	%	\node [roundnode] (9b) at (3, 4.3) {};
	\node [squarednode] (10) at (2, 5) {$\cup$};
	%	\node [roundnode] (12b) at (1, 6.3) {};
	
	\draw (2) to (0);
	\draw (2) to (1);
	\draw (4) to (2);
	\draw (4) to (3);
	\draw (6) to (4);
	\draw (6) to (5);
	\draw (8) to (6);
	\draw (8) to (7);
	\draw (10) to (8);
	\draw (10) to (9);
	
	\draw [dashed] (3,4.7) to (9);
	\draw [dashed] (9) to (5);
	\draw [dashed,out=-90,in=0] (5) to (2);
	\draw [dashed] (2) to (3);
	
\end{tikzpicture}
\hspace{1em}
\begin{tikzpicture}[scale=\figfourscaleval,->,>=stealth',roundnode/.style={circle,draw,inner sep=1.2pt},squarednode/.style={rectangle,inner sep=3pt}]
	
	\node [squarednode] (num) at (0, 5) {4.};
	\node [squarednode] (0) at (1, 0) {$a_3$};
	%	\node [roundnode] (0b) at (1, 0.3) {};
	\node [squarednode] (1) at (3, 0) {$a_4$};
	\node [squarednode] (2) at (2, 1) {$\cup$};
	%	\node [roundnode] (2b) at (2, 1.3) {};
	\node [squarednode] (3) at (0, 1) {$a_2$};
	%	\node [roundnode] (3b) at (0, 1.3) {};
	\node [squarednode] (4) at (1, 2) {$\cup$};
	%	\node [roundnode] (4b) at (1, 2.3) {};
	\node [squarednode] (5) at (3, 2) {$a_5$};
	%	\node [roundnode] (5b) at (3, 2.3) {};
	\node [squarednode] (6) at (2, 3) {$\cup$};
	\node [squarednode] (7) at (0, 3) {$a_1$};
	%	\node [roundnode] (7b) at (0, 3.3) {};
	\node [squarednode] (8) at (1, 4) {$\cup$};
	%	\node [roundnode] (8b) at (1, 4.3) {};
	\node [squarednode] (9) at (3, 4) {$a_6$};
	%	\node [roundnode] (9b) at (3, 4.3) {};
	\node [squarednode] (10) at (2, 5) {$\cup$};
	%	\node [roundnode] (12b) at (1, 6.3) {};
	
	\draw (2) to (0);
	\draw (2) to (1);
	\draw (4) to (2);
	\draw (4) to (3);
	\draw (6) to (4);
	\draw (6) to (5);
	\draw (8) to (6);
	\draw (8) to (7);
	\draw (10) to (8);
	\draw (10) to (9);
	
	\draw [dashed] (3,4.7) to (9);
	\draw [dashed] (9) to (5);
	\draw [dashed,out=-90,in=45] (5) to (1);
	\draw [dashed] (1) to (0);
	
\end{tikzpicture}
	\caption{Evolution of the stack $\ustack$ (represented by dashed arrows) for an iterator over the topmost union node in the figure. The underlying ECS contains only union nodes and six bottom nodes. The first figure is $\ustack$ after calling $\ustack \gets {\sf push}(\ustack, v)$, the second is after calling $\ustack \gets\textsc{Traverse}(\ustack)$. The last two figures represent successive calls to ${\sf pop}(\ustack), \ustack \gets\textsc{Traverse}(\ustack)$.}
	\label{nested:fig-enum-stacks}
\end{figure}

The methods of a union node iterator $\uit_\cup$ are then straightforward. For \textsc{create} (lines 34-37), we push $v$ into $\ustack$ (line 35) and traverse $\ustack$, finding the first rightmost output node from $v$ (line~36). Then we build the iterator $\uit$ of this output node for being ready to start enumerating their outputs (line~37). 
For \textsc{next}, we consume all outputs by calling $\uit.\textsc{next}$ (line 40). When there are no more outputs (lines 41-47), we pop the top node from $\ustack$ and check if the stack is empty or not (lines~41-42). If this is the case, there are no more outputs and we output {\bf false}. Instead, if $\ustack$ is non-empty but the top node $\text{\sf top}(\ustack)$ is a union node, we apply the \textsc{traverse} method for finding the rightmost output node from $\text{\sf top}(\ustack)$ (lines 44-45). When the procedure is done, we know that the top node is an output node, and then we create an iterator and move to its first output (lines~46-47). 
For \textsc{print}, we call the print method of $\uit$ which is ready to write the current output (lines 50-51). 

For proving the correctness of the enumeration procedure, since $\D$ is duplicate-free, one can verify that $\textsc{Enumerate}(v)$ in Algorithm~\ref{nested:alg:enumeration} enumerates all the set $\sem{\D}(v)$, one by one, and without repetitions. For the delay between outputs, since $\D$ is $k$-bounded, it is straightforward to prove by induction that, if $w_0$ is the first output, then:
\begin{itemize}
	\item $\textsc{create}(v)$ takes time $\cO(k\cdot |w_0|)$,
	\item $\textsc{next}$ takes time $\cO(k\cdot |w_0|)$ for the first call, and $\cO(k\cdot |w|+|w'|)$ for the next call where $w$ and $w'$ are the previous and next outputs, respectively, and
	\item $\textsc{print}$ takes time $\cO(k\cdot |w|)$ where $w$ is the current output to be printed.  
\end{itemize}
Overall, $\textsc{Enumerate}(v)$ in Algorithm~\ref{nested:alg:enumeration} have delay $O(k\cdot (|w| + |w'|))$ to write the next output $w'$ in the output register, after printing the previous output $w$. 

We end by pointing out that the existence of an enumeration algorithm $\enumE$ with delay $O(k\cdot (|w| + |w'|))$ between any consecutive outputs $w$ and $w'$, implies the existence of an enumeration algorithm $\enumE'$ with output-linear delay as defined in Section~\ref{nested:sec:enum}. We start noting that $k$ is a fixed value and then the delay of $\enumE$ only depends on $|w| + |w'|$. For depending only on the next output $w'$, one can perform the following strategy for $\enumE'$: start by running $\enumE$, enumerate the first output $w_0$, proceed $k \cdot |w_0|$ more steps of $\enumE$, stop, and print the separator symbol $\#$. Then continue running $\enumE$ to prepare the next output $w_1$, proceed $k |w_1|$ more steps, stop, and print the separator symbol $\#$. By repeating this enumeration process, one can verify that the delay between the $i$-th output $w_i$ and the $(i+1)$-th output $w_{i+1}$ is $O(|w_{i+1}|)$. Therefore, $\enumE'$ has output-linear delay. 
\end{proof}
The enumeration algorithm above does not require any preprocessing over $\D$. By this proposition, from now we assume that all \dsabbr{} are duplicate-free and $k$-bounded for some fixed~$k$.



%The enumeration algorithm of the previous proposition does not require any preprocessing over $\D$ and the main idea is to perform some sort of DFS traversal over the nodes. Given that $\D$ is unambiguous, we know that this procedure will not enumerate any output twice. Moreover, given that $\D$ is $k$-bounded, we also know that on a \edit{fixed} number of steps it will find something to output, which is necessary to bound the delay. 
%Therefore, from now we assume that all \dsabbr{} are unambiguous and $k$-bounded for some \edit{fixed}~$k$.

\paragraph{Operations allowed by an ECS}
The next step is to provide a set of operations that allow extending a \dsabbr{} $\D$ while maintaining $k$-boundedness. Furthermore, we require these operations to be \emph{fully-persistent}: a data structure is called fully-persistent if every version can be both accessed and modified~\cite{driscoll1986making}. In other words, the previous version of the data structure is always available after each operation.
To satisfy the last requirement, the strategy will consist in extending $\D$ to $\D'$ for each operation, by adding new nodes and maintaining the previous nodes unmodified. Then $\cL_{\D'}(v) = \cL_{\D}(v)$ for each node $v \in V$, and so, the data structure will be fully-persistent. 

%\cristian{Martin: Cambie la presentación/definición de las operaciones de $O := \mathsf{op}(I)$ a $\mathsf{op}(I) \rightarrow O$ explicando la notación. Por favor, revisa esta notación y actualiza la presentación de las operaciones en el apendice (cuando definimos $\epsilon$-ECS.} 
Fix a \dsabbr{} $\D = (\infAlph, V, \ell, r, \lambda)$. In the following, we say that $\D' = (\infAlph, V', \ell', r', \lambda')$ is an \emph{extension} of $\D$ if, and only if, $\mathsf{X} \subseteq \mathsf{X}'$ for every $\mathsf{X} \in \{V, \ell, r, \lambda\}$. Further, we write $\mathsf{op}(I) \rightarrow O$ to define the signature of an operation $\mathsf{op}$ where $I$ is the input and $O$ is the output. 
Then for any $a \in \infAlph$ and $v_1, \ldots, v_4 \in V$, we define the operations:
\[
\begin{array}{rclrclrcl}
\add(\D,a) & \!\!\!\!\rightarrow\!\!\!\! & (\D',v')   \ \ \ \ \ \	 & \prod(\D,v_1, v_2) & \!\!\!\! \rightarrow \!\!\!\! & (\D',v')   \ \ \ \ \ \ \ &  \union(\D,v_3, v_4) & \!\!\!\!\rightarrow\!\!\!\! & (\D',v')
\end{array}
\]
\noindent such that $\D'$ is an extension of $\D$  and $v' \in V' \setminus V$ is a fresh node such that $\cL_{\D'}(v') = \{a\}$, $\cL_{\D'}(v') = \cL_{\D}(v_1) \cdot \cL_{\D}(v_2)$, and $\cL_{\D'}(v') = \cL_{\D}(v_3) \cup \cL_{\D}(v_4)$, respectively.
We assume that the $\union$ and $\prod$ satisfy properties (1) and (2) of a duplicate-free \dsabbr{}, namely, $\cL_{\D}(v_1)$ and $\cL_{\D}(v_2)$ are disjoint and, for every $w \in \sem{\D}(v_3) \cdot \sem{\D}(v_4)$, there exists a unique way to decompose $w = w_1 \cdot w_2$ such that $w_1 \in \sem{\D}(v_3)$ and $w_2 \in \sem{\D}(v_4)$. 
%Furthermore, we would like that each operation takes constant time and preserves $k$-boundedness.


Next, we show how to implement these operations, where the case of $\add$ and $\prod$ are straightforward. For $ \add(\D,a) \rightarrow (\D',v')$ define $V' := V \cup \{v'\}$ and $\lambda'(v') = a$. One can easily check that $\L_{\D'}(v') = \{a\}$ as expected. For $\prod(\D,v_1, v_2) \rightarrow (\D',v')$ we proceed in a similar way: define $V' := V \cup\{v'\}$, $\ell'(v') :=v_1$, $r'(v') = v_2$, and $\lambda'(v') = \odot$. 
Then $\L_{\D'}(v') = \sem{\D}(v_1)\cdot\sem{\D}(v_2)$.
Furthermore, one can check that each operation takes constant time, $\D'$ is a valid \dsabbr{} (i.e. duplicate-free and $k$-bounded), and the operations are fully-persistent (i.e. the previous version $\D$ is available).

To define the union, we need to be a bit more careful to guarantee output-linear delay, specifically, the  $k$-bounded property.
For $v\in V$, we say that $v$ is {\em safe} if (1) $\odepth_{\D}(v) \leq 1$, and (2) if $\odepth_{\D}(v) = 1$, then $\odepth_{\D}(r(v)) \leq 1$. In other words, $v$ is safe if either $v$ is an output node, or its left child is an output node and its right child is either an output node or has output depth 1.
Note that a leaf or a product node are safe nodes by definition and, thus, the $\add$ and $\prod$ operations always produce safe nodes. 

The strategy then is to show that, if $v_3$ and $v_4$ are safe nodes, then we can implement the operation $\union(\D,v_3, v_4) \rightarrow (\D',v')$ and produce a safe node $v'$. To reach this goal, define $(\D',v')$ as follows:
\begin{itemize}
	\item  If $v_3$ or $v_4$ are output nodes then $V' := V\cup\{v'\}$ and $\lambda(v') := \cup$. Moreover, if $v_3$ is the output node, then $\ell'(v') := v_3$ and $r'(v') := v_4$. Otherwise, we connect $\ell'(v') := v_4$ and~$r'(v') := v_3$.
	\item If $v_3$ and $v_4$ are not output nodes (i.e. both are union nodes), then $V' := V \cup\{v',u_1,u_2\}$, $\ell'(v') := \ell(v_3)$, $r'(v') := u_1$, and $\lambda'(v') := \cup$; $\ell'(u_1) := \ell(v_4)$, $r'(u_1) := u_2$, and $\lambda'(u_1) := \cup$; $\ell'(u_2) := r(v_3)$, $r'(u_2) := r(v_4)$, and $\lambda'(u_2) := \cup$.
\end{itemize}
This gadget\footnote{Note that a similar trick was used in~\cite{AmarilliBJM17} for computing an index over a circuit.} is depicted in Figure~\ref{nested:fig-union-def}. % and it is known as Louis' trick.
%(also used in~\cite{AmarilliBJM17}).
This construction has several properties.
First, one can easily check that $\L_{\D'}(v') = \sem{\D}(v_1)\cup\sem{\D}(v_2)$ and so its semantics is well-defined. 
Second, $\union$ can be computed in constant time in $|\D|$ given that we only need to add three fresh nodes, and the operation is fully-persistent given that we connect them without modifying the nodes of~$\D$. 
Third, the produced node $v'$ is safe in $\D'$, although nodes $u_1$ and $u_2$ are not necessarily safe. 
Finally, $\D'$ is 2-bounded whenever $\D$ is $2$-bounded. This is straightforward to verify for the first case when $v_3$ or $v_4$ are output nodes. For the second case (i.e., Figure~\ref{nested:fig-union-def}), recall that $v_3$ and $v_4$ are safe nodes, which means that $\ell(v_3)$ and $\ell(v_4)$ are output nodes, and then $\odepth_{\D'}(v') = \odepth_{\D'}(u_1) = 1$. Further, given that $v_3$ is safe, we know that $\odepth_{\D}(r(v_3)) \leq 1$, so $\odepth_{\D'}(u_2)\leq 2$. Given that the output depths of all fresh nodes in $\D'$ are bounded by $2$ and $\D$ is $2$-bounded, then we conclude that $\D'$ is $2$-bounded~as~well.

\begin{figure}[t]
	\centering
	\begin{tikzpicture}[->,>=stealth',roundnode/.style={circle,inner sep=1pt},squarednode/.style={rectangle,inner sep=2pt}, scale=1.1]
	\node[squarednode] (0) at (0, 0) {$\ell(v_3)$};
	\node[squarednode] (1) at (2, 0) {$r(v_3)$};
	\node[roundnode] (2) at (1, 0.65) {$v_3$};
	\node[squarednode] (3) at (4, 0) {$\ell(v_4)$};
	\node[squarednode] (4) at (6, 0) {$r(v_4)$};
	\node[roundnode] (5) at (5, 0.65) {$v_4$};
	\node[roundnode] (6) at (5, 1.3) {$u_2$};
	\node[roundnode] (7) at (4, 1.95) {$u_1$};
	\node[roundnode] (8) at (3, 2.6) {$v'$};
	\draw[dashed] (2) to[out=-135,in=45] (0);
	\draw (2) to[out=-45,in=135] (1);
	\draw[dashed] (5) to[out=-135,in=45] (3);
	\draw (5) to[out=-45,in=135] (4);
	\draw (6) to[out=-45,in=90] (4);
	\draw[dashed] (6) to[out=-135,in=45] (1);
	\draw (7) to[out=-45,in=135] (6);
	\draw[dashed] (7) to[out=-135,in=90] (3);
	\draw (8) to[out=-45,in=135] (7);
	\draw[dashed] (8) to[out=-135,in=90] (0);
	
	\node[color=white] (offset) at (0,-0.35) {.};
	\end{tikzpicture}
	\caption{Gadget for $\union(\D,v_3,v_4)$. Nodes $v',u_1,u_2,v_3$ and $v_4$ are labeled as $\cup$. Dashed and solid lines denote the mappings in $\ell'$ and $r'$ respectively.}
	\label{nested:fig-union-def}
	\vspace{-3mm}
\end{figure}

By the previous discussion, if we start with a \dsabbr\ $\D$ which is 2-bounded (or empty) and we apply the $\add$, $\prod$, or $\union$ operators between safe nodes (which also produce safe nodes), then the result is 2-bounded as well. Finally, by Proposition~\ref{nested:ds:lindelay}, the result can be enumerated with output-linear delay.

\begin{theorem}\label{nested:theo:data-structure}
	The operations $\add$, $\prod$, and $\union$ require constant time and are fully-persistent. Furthermore, if we start from an empty \dsabbr\ $\D$ and apply $\add$, $\prod$, or $\union$ operations over safe nodes, the partial results $(\D', v')$ satisfy that $v'$ is always a safe node and the set $\L_{\D'}(v)$ can be enumerated with output-linear delay for every node $v$.
\end{theorem}  

We want to remark that restricting the operations over safe nodes does not limit the final user of the data structure. Indeed, since we will usually start from an empty \dsabbr\ and apply these operations over previously returned nodes, the whole algorithm will always use safe nodes during its computation, satisfying the conditions of Theorem~\ref{nested:theo:data-structure}. 

\paragraph{Extending ECS with $\eps$-nodes}
For technical reasons, our algorithm of the next section needs a slight extension of \dsabbr{} by allowing leaves that produce the empty string $\eps$. Let $\eps\not\in\infAlph$ be a symbol representing the empty string (i.e. $w\cdot\eps = \eps\cdot w = w$). We define an \dsname{} with $\eps$ (called \dsepsabbr) as a tuple $\D = (\infAlph, V, \ell, r, \lambda)$ defined identically to a \dsabbr{} except that:
\[
\lambda\colon V\to\infAlph\cup\{\cup, \odot,\eps\}
\] 
and $\lambda(v)\in\{\cup,\odot\}$ if, and only if, both $\ell(v)$ and $r(v)$ are defined. Further, $\sem{\D}(v) = \{\epsilon\}$ whenever $\lambda(v) = \eps$. The duplicate-free restriction is the same for \dsepsabbr and one has to slightly extend $k$-boundedness to consider $\epsilon$-nodes. %, giving an ``infinite'' depth to a node if its left node can produce $\epsilon$.}
However, to support the $\prod$ and $\union$ operations in constant time and to maintain the $k$-boundedness invariant, we need to extend the notion of safe nodes, called \emph{$\epsilon$-safe}, and the gadgets for $\prod$ and $\union$. We summarize all these details in the proof of the following theorem which is a straightforward extension of Theorem~\ref{nested:theo:data-structure}.

%Given space restrictions, we show these extensions in the appendix and state here the main result, that will be used in the next section.
 
\begin{theorem}\label{nested:theo:data-structure-eps}
	The operations $\add$, $\prod$, and $\union$ over \dsepsabbr{} take constant time and are fully-persistent. Furthermore, if we start from an empty \dsepsabbr{} $\D$ and apply $\add$, $\prod$, and $\union$ over $\epsilon$-safe nodes, the partial results $(\D', v')$ satisfy that $v'$ is always an $\epsilon$-safe node and the set $\L_{\D'}(v)$ can be enumerated with output-linear delay for every node $v$.
\end{theorem}  

\begin{proof}
For dealing with $\eps$-nodes, we need to revisit the notions of output depth, $k$-bounded, and safeness. For the sake of presentation, it is simpler first to introduce the notion of $\eps$-safe nodes to then revisit all previous definitions. Specifically, for a given \dsepsabbr $\D = (\infAlph, V, \ell, r, \lambda)$ we say that $v \in V$ is \emph{$\eps$-safe} if it is exactly one of the following situations:
\begin{enumerate}
	\item $\lambda(v) = \eps$, or 
	\item $\lambda(v) \neq \eps$, $v$ is a safe node, and  $\lambda(u) \neq \eps$ for any node $u$ which is reachable from $v$, or 
	\item $\lambda(v) = \cup$, $\lambda(\ell(v)) = \eps$, and $r(v)$ satisfies (2). 
\end{enumerate}
In other words, $\eps$ can only occur as the left child of an $\eps$-safe node or being the node itself. 

From now on, we can assume that we will only work with $\eps$-safe nodes as the input for enumerating or for operating them with other ($\eps$-safe) nodes. If this is the case, then the enumeration with output-linear delay and the notions of output depth and $k$-boundedness are similar to before. Therefore, we dedicate the rest of the proof to revisit the operations $\prod$ and $\union$ when the inputs are $\eps$-safe nodes (the $\add$ operation is straightforward similar to the case without $\eps$). 

Assume $v_1$ and $v_2$ are $\eps$-safe. 
Since both $v_1$ and $v_2$ may fall in one of three cases above, we define $\prod(\D,v_1,v_2) \to (\D',v')$ by separating into nine cases, of which the first six are straightforward: 
\begin{itemize}
	\item If $\eps\not\in\L_{\D}(v_1)$ and $\eps\not\in\L_{\D}(v_2)$,  we use the construction given for a regular \dsabbr.
	\item If $\eps\not\in\L_{\D}(v_1)$ and $\lambda(v_2) = \eps$, we define $v' = v_1$, and $\D' = \D$.
	\item If $\lambda(v_1) = \eps$ and $\eps\not\in\L_{\D}(v_2)$, we define $v' = v_2$, and $\D' = \D$.
	\item If $\lambda(v_1) = \eps$ and $\lambda(v_2) = \eps$, we define $v' = v_1$, and $\D' = \D$.
	\item If $\lambda(v_1) = \eps$ and $v_2$ is in case (3), we define $v' = v_2$, and $\D' = \D$.
	\item If $v_1$ is in case (3) and $\lambda(v_2) = \eps$, we define $v' = v_1$, and $\D' = \D$.
\end{itemize}

%!TEX root = main.tex

\begin{figure}[t]
	\centering
\begin{tikzpicture}[->,>=stealth',roundnode/.style={circle,draw,inner sep=1.2pt},squarednode/.style={rectangle,inner sep=2pt}]
	\node [squarednode] (0) at (1, 2.75) {$\cup$};
	\node [squarednode] (6) at (1.55, 2.75) {$= v_a'$};
	\node [squarednode] (1) at (0.5, 1) {$v_1$};
	\node [squarednode] (2) at (2, 1.75) {$\odot$};
	\node [squarednode] (3) at (2.25, 1) {$v_2$};
	\node [squarednode] (4) at (1.5, 0) {$\eps$};
	\node [squarednode] (5) at (3, 0) {$r(v_2)$};
	\draw (0) to[out=-135,in=135] (2);
	\draw (0) to[out=-45,in=75] (1);
	\draw (2) to[out=-45,in=90] (5);
	\draw (3) to[out=-135,in=60] (4);
	\draw (3) to[out=-45,in=120] (5);
	\draw (2) to[out=-135,in=45] (1);
\end{tikzpicture}
\hspace{1em}
\begin{tikzpicture}[->,>=stealth',roundnode/.style={circle,draw,inner sep=1.2pt},squarednode/.style={rectangle,inner sep=2pt}]
		\node [squarednode] (0) at (2.5, 2.75) {$\cup$};
		\node [squarednode] (6) at (3, 2.75) {$= v_b'$};
\node [squarednode] (1) at (3, 1) {$v_2$};
\node [squarednode] (2) at (2, 1.75) {\raisebox{-7pt}{$\odot$}};
\node [squarednode] (3) at (0.75, 1) {$v_1$};
\node [squarednode] (4) at (0, 0) {$\eps$};
\node [squarednode] (5) at (1.5, 0) {$r(v_1)$};
	\draw (0) to[out=-45,in=90] (1);
	\draw (0) to[out=-135,in=90] (2);
	\draw (2) to[out=-135,in=90] (5);
	\draw (3) to[out=-135,in=60] (4);
	\draw (3) to[out=-45,in=120] (5);
	\draw (2) to[out=-45,in=135] (1);
\end{tikzpicture}
\hspace{1em}
\begin{tikzpicture}[->,>=stealth',roundnode/.style={circle,draw,inner sep=1.2pt},squarednode/.style={rectangle,inner sep=2pt}]
		\node [squarednode] (0) at (1.5, 3.25) {$\cup$};
			\node [squarednode] (11) at (2, 3.25) {$= v_c'$};
\node [squarednode] (1) at (0.75, 2.45) {$\eps^*$};
\node [squarednode] (2) at (2.25, 2.45) {$\textsf{union}$};
\node [squarednode] (3) at (3, 1.75) {$\cup$};
\node [squarednode] (4) at (2, 1) {$\odot$};
\node [squarednode] (5) at (3.25, 1) {$v_2$};
\node [squarednode] (6) at (2.5, 0) {$\eps$};
\node [squarednode] (7) at (4, 0) {$r(v_2)$};
\node [squarednode] (8) at (0.25, 1) {$v_1$};
\node [squarednode] (9) at (-0.5, 0) {$\eps$};
\node [squarednode] (10) at (1, 0) {$r(v_1)$};
\draw (0) to[out=-135,in=60] (1);
\draw (0) to[out=-45,in=120] (2);
\draw (2) to[out=-135,in=90] (10);
\draw (2) to[out=-45,in=135] (3);
\draw (4) to[out=-135,in=45] (10);
\draw (4) to[out=-45,in=135] (7);
\draw (5) to[out=-135,in=60] (6);
\draw (5) to[out=-45,in=120] (7);
\draw (3) to[out=-45,in=90] (7);
\draw (3) to[out=-135,in=40]  (4);
\draw (8) to[out=-45,in=120] (10);
\draw (8) to[out=-135,in=60]  (9);
\end{tikzpicture}
	\caption{Gadgets for $\prod$ as defined for an \dsepsabbr. Nodes $v_a'$, $v_b'$, and $v_c'$ correspond to $v'$ as defined for cases (a), (b), and (c), respectively.}
	\label{nested:fig-prod-multi-gadget}
\end{figure}

The other three cases are more involved and they are presented graphically in Figure~\ref{nested:fig-prod-multi-gadget}. 
Formally, they are defined as follows:
\begin{itemize}
	\item[(a)] If $\eps\not\in\L_{\D}(v_1)$ and $v_2$ is in case (3), then $V' = V\cup\{v',v''\}$, $\ell'(v') = v''$, $r'(v') = v_1$, $\ell(v'') = v_1$,  $r(v'') = r(v_2)$, $\lambda'(v') = \cup$ and $\lambda'(v'') = \odot$. 
	\item[(b)] If $v_1$ is in case (3) and $\eps\not\in\L_{\D}(v_1)$, then $V' = V\cup\{v',v''\}$,  $\ell'(v') = v''$, $r'(v') = v_2$, $\ell(v'') = r(v_1)$,  $r(v'') = v_2$, $\lambda'(v') = \cup$ and $\lambda'(v'') = \odot$. 
	\item[(c)] If both $v_1$ and $v_2$ are in case (3), we do a slightly more delicate construction. 
	First, we define a $\cD''$ with $V'' = V\cup\{v^{3},v^{4}\}$, $\ell''(v^3) = v^4$, $r''(v^3) = r(v_2)$, $\ell''(v^4) = r(v_1)$, $r''(v^4) = r(v_2)$, $\lambda''(v^3) = \cup$, $\lambda''(v^4) = \odot$.
	Now, let $(\cD^3, v^2) \gets \union(\cD'', r(v_1), v_3)$.
	Lastly, let $V' = V^3 \cup \{v^{*}, v'\}$, $\ell'(v') = v^*$, $r(v') = v_2$, $\lambda(v') = \cup$ and $v^* = \eps$.
\end{itemize}
Note that the $\union$ operation in case (c) does not recurse since $r(v_1)$ is safe. In particular, it does not reach any $\eps$-leaf.

For the $\union$-operation, assume $v_1$ and $v_2$ are $\eps$-safe and
%Further, assume that $\L_{\D}(v_1) \setminus \{\eps\}$ and $\L_{\D}(v_2)\setminus \{\eps\}$ are disjoint. 
define $\union(\D,v_1, v_2) \to (\D',v')$ as:
\begin{itemize}
	\item If $\eps\not\in\L_{\D}(v_1)$ and $\eps\not\in\L_{\D}(v_2)$, we use the construction given for a regular \dsabbr.
	\item If $\eps\not\in\L_{\D}(v_1)$ and $\lambda(v_2) = \eps$, we define $V' = V \cup\{v'\}$ and $\lambda'(v') = \cup$. We connect $\ell'(v') = v_2$ and $r'(v') = v_1$.
	\item If $\eps\not\in\L_{\D}(v_1)$ and $v_2$ is in case (3), let $(\D'',v'') = \union(\D,v_1,r(v_2))$ as defined for a regular~\dsabbr. We define $V' = V'' \cup\{v'\}$, and $\lambda'(v') = \cup$ where $\lambda'$ is an extension of $\lambda''$. We connect $\ell'(v') = \ell(v_2)$ and $r'(v') = v''$.
	\item If $\lambda(v_1) = \eps$ and $\eps\not\in\L_{\D}(v_2)$, we define $V' = V \cup\{v'\}$ and $\lambda'(v') = \cup$. We connect $\ell'(v') = v_1$ and $r'(v') = v_2$.
	\item If $\lambda(v_1) = \eps$ and $\lambda(v_2) = \eps$, we define $\D' = \D$ and $v' = v_1$.
	\item If $\lambda(v_1) = \eps$ and $v_2$ is in case (3), we define $\D' = \D$ and $v' = v_2$.
	\item (*) If $v_1$ is in case (3) and $\eps\not\in\L_{\D}(v_2)$, let $(\D'',v'') = \union(\D,r(v_1),v_2)$ as defined for a regular \dsabbr.  We define $V' = V'' \cup\{v'\}$ and $\lambda'(v') = \cup$ where $\lambda'$ is an extension of $\lambda''$. We connect $\ell'(v') = \ell(v_2)$ and $r'(v') = v''$. 
	\item If $v_1$ is in case (3) and $\lambda(v_2) = \eps$, we define $\D' = \D$ and $v' = v_1$.
	\item If both $v_1$ and $v_2$ are in case (3), let $(\D',v') = \union(\D,v_1,r(v_2))$ by using the construction of the case marked with (*).
\end{itemize}
%Whenever $\D''$ is mentioned it is assumed to be equal to $(\Sigma, V'', I'', \ell'', r'', \lambda'')$.
It is straightforward to check that each operation behaves as expected. 
%That is, if $\add(\D,a)\to(\D',v')$, then $\L_{\D}(v') = \{a\}$, if $\prod(\D,v_1,v_2)\to(\D',v')$, then $\L_{\D}(v') = \L_{\D}(v_1)\cdot\L_{\D}(v_2)$, and if $\union(\D,v_1,v_2)\to(\D',v')$, then $\L_{\D}(v_1)\cup\L_{\D}(v_2)$. 
Moreover, if both $v_1$ and $v_2$ are $\eps$-safe, then the resulting node $v'$ is $\eps$-safe as well for each operation.

We finish this proof by noticing that each operation falls into a fixed number of cases that can be checked exhaustively, and each construction has a fixed size, so they take constant time. 
Furthermore, each operation is fully persistent as expected.


%Finally, let $(\D',v')$ be a partial result obtained from applying the operations $\add$, $\prod$ and $\union$ such that $\D'$ is duplicate-free. 
%The proof follows from Claim~\ref{nested:appendix:eps-enum}.



%	
%		\cristian{PREVIOUS VERSION }
%	For dealing with $\eps$-nodes, we treat them in a particular way. Specifically, for a given \dsepsabbr $\cD$ we require that any node $v\in\cD$ is in exactly one of the following situations:
%	\begin{enumerate}
%		\item $\lambda(v) \neq \eps$ and for any node $u$ which is reachable from $v$ it holds that $\lambda(u) \neq \eps$, 
%		\item $\lambda(v) = \eps$, or 
%		\item $\lambda(v) = \cup$, $\lambda(\ell(v)) = \eps$, and $r(v)$ satisfies (1). 
%	\end{enumerate}
%	In other words, $\eps$ can only be child of a union node with in-degree $0$.
%	
%
%	For the rest of the proof, we will refer to a node $v$ that satisfies each case as a node such that (1) $\eps\not\in\cL_{\cD}(v)$, (2) $\lambda(v) = \eps$ or (3) $v$ is {\em in Case 3}, respectively.
%	Note that this construction ensures that if $\eps\in\cL_{\cD}(v)$, it can be retrieved in constant time.
%	
%	
%	One last notion we make use of is $\eps$-safe nodes. For a given \dsepsabbr $\cD$ and $v\in\cD$ we say that $v$ is $\eps$-safe if either (1) $v$ is safe, (2) $\lambda(v) = \eps$, or (3) $v$ is in Case 3 and $r(v)$ is safe.
%	
%	We define the operations $\add$, $\prod$ and $\union$ over $\D$ to return a pair $(\D',v')$ such that $\D' = (\Sigma, V', I', \ell', r', \lambda')$ as follows:
%	
%	For $\add(\D,a)\to(\D',v')$ we define $V' := V \cup \{v'\}$, $I' := I$, and $\lambda'(v') = a$.
%	
%	Assume $v_1$ and $v_2$ are $\eps$-safe. 
%	Further, assume that for every word in $w\in\L_{\D}(v_1)\cdot\L_{\D}(v_2)$ there exist only two non-empty words $w_1$ and $w_2$ such that $w_1\in\L_{\D}(v_1)$, $w_2\in\L_{\D}(v_2)$ and $w = w_1w_2$.
%	Since both $v_1$ and $v_2$ may fall in one of three cases, we define $\prod(\D,v_1,v_2) \to (\D',v')$ by separating into nine cases, of which the first six are straightforward: 
%	\begin{itemize}
%		\item If $\eps\not\in\L_{\D}(v_1)$ and $\eps\not\in\L_{\D}(v_2)$,  we use the construction given for a regular \dsabbr.
%		\item If $\eps\not\in\L_{\D}(v_1)$ and $\lambda(v_2) = \eps$, we define $v' = v_1$, and $\D' = \D$.
%		\item If $\lambda(v_1) = \eps$ and $\eps\not\in\L_{\D}(v_2)$, we define $v' = v_2$, and $\D' = \D$.
%		\item If $\lambda(v_1) = \eps$ and $\lambda(v_2) = \eps$, we define $v' = v_1$, and $\D' = \D$.
%		\item If $\lambda(v_1) = \eps$ and $v_2$ is in Case 3, we define $v' = v_2$, and $\D' = \D$.
%		\item If $v_1$ is in Case 3 and $\lambda(v_2) = \eps$, we define $v' = v_1$, and $\D' = \D$.
%	\end{itemize}
%	
%	%!TEX root = main.tex

\begin{figure}[t]
	\centering
\begin{tikzpicture}[->,>=stealth',roundnode/.style={circle,draw,inner sep=1.2pt},squarednode/.style={rectangle,inner sep=2pt}]
	\node [squarednode] (0) at (1, 2.75) {$\cup$};
	\node [squarednode] (6) at (1.55, 2.75) {$= v_a'$};
	\node [squarednode] (1) at (0.5, 1) {$v_1$};
	\node [squarednode] (2) at (2, 1.75) {$\odot$};
	\node [squarednode] (3) at (2.25, 1) {$v_2$};
	\node [squarednode] (4) at (1.5, 0) {$\eps$};
	\node [squarednode] (5) at (3, 0) {$r(v_2)$};
	\draw (0) to[out=-135,in=135] (2);
	\draw (0) to[out=-45,in=75] (1);
	\draw (2) to[out=-45,in=90] (5);
	\draw (3) to[out=-135,in=60] (4);
	\draw (3) to[out=-45,in=120] (5);
	\draw (2) to[out=-135,in=45] (1);
\end{tikzpicture}
\hspace{1em}
\begin{tikzpicture}[->,>=stealth',roundnode/.style={circle,draw,inner sep=1.2pt},squarednode/.style={rectangle,inner sep=2pt}]
		\node [squarednode] (0) at (2.5, 2.75) {$\cup$};
		\node [squarednode] (6) at (3, 2.75) {$= v_b'$};
\node [squarednode] (1) at (3, 1) {$v_2$};
\node [squarednode] (2) at (2, 1.75) {\raisebox{-7pt}{$\odot$}};
\node [squarednode] (3) at (0.75, 1) {$v_1$};
\node [squarednode] (4) at (0, 0) {$\eps$};
\node [squarednode] (5) at (1.5, 0) {$r(v_1)$};
	\draw (0) to[out=-45,in=90] (1);
	\draw (0) to[out=-135,in=90] (2);
	\draw (2) to[out=-135,in=90] (5);
	\draw (3) to[out=-135,in=60] (4);
	\draw (3) to[out=-45,in=120] (5);
	\draw (2) to[out=-45,in=135] (1);
\end{tikzpicture}
\hspace{1em}
\begin{tikzpicture}[->,>=stealth',roundnode/.style={circle,draw,inner sep=1.2pt},squarednode/.style={rectangle,inner sep=2pt}]
		\node [squarednode] (0) at (1.5, 3.25) {$\cup$};
			\node [squarednode] (11) at (2, 3.25) {$= v_c'$};
\node [squarednode] (1) at (0.75, 2.45) {$\eps^*$};
\node [squarednode] (2) at (2.25, 2.45) {$\textsf{union}$};
\node [squarednode] (3) at (3, 1.75) {$\cup$};
\node [squarednode] (4) at (2, 1) {$\odot$};
\node [squarednode] (5) at (3.25, 1) {$v_2$};
\node [squarednode] (6) at (2.5, 0) {$\eps$};
\node [squarednode] (7) at (4, 0) {$r(v_2)$};
\node [squarednode] (8) at (0.25, 1) {$v_1$};
\node [squarednode] (9) at (-0.5, 0) {$\eps$};
\node [squarednode] (10) at (1, 0) {$r(v_1)$};
\draw (0) to[out=-135,in=60] (1);
\draw (0) to[out=-45,in=120] (2);
\draw (2) to[out=-135,in=90] (10);
\draw (2) to[out=-45,in=135] (3);
\draw (4) to[out=-135,in=45] (10);
\draw (4) to[out=-45,in=135] (7);
\draw (5) to[out=-135,in=60] (6);
\draw (5) to[out=-45,in=120] (7);
\draw (3) to[out=-45,in=90] (7);
\draw (3) to[out=-135,in=40]  (4);
\draw (8) to[out=-45,in=120] (10);
\draw (8) to[out=-135,in=60]  (9);
\end{tikzpicture}
	\caption{Gadgets for $\prod$ as defined for an \dsepsabbr. Nodes $v_a'$, $v_b'$, and $v_c'$ correspond to $v'$ as defined for cases (a), (b), and (c), respectively.}
	\label{nested:fig-prod-multi-gadget}
\end{figure}
%	
%	The other three cases are more involved and they are presented graphically in Figure~\ref{nested:fig-prod-multi-gadget}. 
%	Formally, they are defined as follows:
%	
%	\begin{itemize}
%		\item[(a)] If $\eps\not\in\L_{\D}(v_1)$ and $v_2$ is in Case 3, then $V' = V\cup\{v',v''\}$, $I' = I\cup\{v',v''\}$, $\ell'(v') = v_1$, $r'(v') = v''$, $\ell(v'') = v_1$,  $r(v'') = r(v_2)$, $\lambda'(v') = \cup$ and $\lambda'(v'') = \odot$. 
%		\item[(b)] If $v_1$ is in Case 3 and $\eps\not\in\L_{\D}(v_1)$, then $V' = V\cup\{v',v''\}$, $I' = I\cup\{v',v''\}$, $\ell'(v') = v''$, $r'(v') = v_2$, $\ell(v'') = r(v_1)$,  $r(v'') = v_2$, $\lambda'(v') = \cup$ and $\lambda'(v'') = \odot$. 
%		\item[(c)] If both $v_1$ and $v_2$ are in Case 3, we do a slightly more delicate construction. 
%		First, we define a $\cD''$ with $V'' = V\cup\{v^{3},v^{4}\}$, $I'' = I\cup\{v^{3},v^{4}\}$, $\ell''(v^3) = v^4$, $r''(v^3) = r(v_2)$, $\ell''(v^4) = r(v_1)$, $r''(v^4) = r(v_2)$, $\lambda''(v^3) = \cup$, $\lambda''(v^4) = \odot$.
%		Now, let $(\cD^3, v^2) \gets \union(\cD'', r(v_1), v_3)$.
%		Lastly, let $V' = V^3 \cup \{v^{*}, v'\}$, $I' = I^3\cup\{v'\}$, $\ell'(v') = v^*$, $r(v') = v_2$, $\lambda(v') = \cup$ and $v^* = \eps$.
%	\end{itemize}
%	Note that the $\union$ operation in case (c) does not recurse since $r(v_1)$ is safe. In particular, it does not reach any $\eps$-leaf.
%	
%	Assume $v_1$ and $v_2$ are $\eps$-safe nodes. 
%	Further, assume that $\L_{\D}(v_1) \setminus \{\eps\}$ and $\L_{\D}(v_2)\setminus \{\eps\}$ are disjoint. 
%	We define $\union(\D,v_1, v_2) \to (\D',v')$ as follows:
%	\begin{itemize}
%		\item If $\eps\not\in\L_{\D}(v_1)$ and $\eps\not\in\L_{\D}(v_2)$, we use the construction given for a regular \dsabbr.
%		\item If $\eps\not\in\L_{\D}(v_1)$ and $\lambda(v_2) = \eps$, we define $V' = V \cup\{v'\}$, $I' = I\cup\{v'\}$ and $\lambda(v') = \cup$. We connect $\ell(v') = v_2$ and $r(v') = v_1$.
%		\item If $\eps\not\in\L_{\D}(v_1)$ and $v_2$ is in Case 3, let $(\D'',v'') = \union(\D,v_1,r(v_2))$ as defined for a regular \dsabbr. We define $V' = V'' \cup\{v'\}$, $I' = I''\cup\{v'\}$ and $\lambda'(v') = \cup$ where $\lambda'$ is an extension of $\lambda''$. We connect $\ell'(v') = \ell(v_2)$ and $r'(v') = v''$.
%		\item If $\lambda(v_1) = \eps$ and $\eps\not\in\L_{\D}(v_2)$, we define $V' = V \cup\{v'\}$, $I' = I\cup\{v'\}$ and $\lambda(v') = \cup$. We connect $\ell(v') = v_1$ and $r(v') = v_2$.
%		\item If $\lambda(v_1) = \eps$ and $\lambda(v_2) = \eps$, we define $\D' = \D$ and $v' = v_1$.
%		\item If $\lambda(v_1) = \eps$ and $v_2$ is in Case 3, we define $\D' = \D$ and $v' = v_2$.
%		\item If $v_1$ is in Case 3 and $\eps\not\in\L_{\D}(v_2)$, let $(\D'',v'') = \union(\D,r(v_1),v_2)$ as defined for a regular \dsabbr.  We define $V' = V'' \cup\{v'\}$, $I' = I''\cup\{v'\}$ and $\lambda'(v') = \cup$ where $\lambda'$ is an extension of $\lambda''$. We connect $\ell'(v') = \ell(v_2)$ and $r'(v') = v''$. (*)
%		\item If $v_1$ is in Case 3 and $\lambda(v_2) = \eps$, we define $\D' = \D$ and $v' = v_1$.
%		\item If both $v_1$ and $v_2$ are in Case 3, let $(\D',v') = \union(\D,r(v_1),v_2)$ by using the construction of case (*).
%	\end{itemize}
%	
%	Whenever $\D''$ is mentioned it is assumed to be equal to $(\Sigma, V'', I'', \ell'', r'', \lambda'')$.
%	
%	It is straightforward to check that each operation behaves as expected. 
%	That is, if $\add(\D,a)\to(\D',v')$, then $\L_{\D}(v') = \{a\}$, if $\prod(\D,v_1,v_2)\to(\D',v')$, then $\L_{\D}(v') = \L_{\D}(v_1)\cdot\L_{\D}(v_2)$, and if $\union(\D,v_1,v_2)\to(\D',v')$, then $\L_{\D}(v_1)\cup\L_{\D}(v_2)$. 
%	Moreover, if both $v_1$ and $v_2$ are $\eps$-safe, then the resulting node $v'$ is $\eps$-safe as well for each operation.
%	
%	Note that each operation falls into a fixed number of cases which can be checked exhaustively, and each construction has a fixed size, so they take constant time. 
%	Furthermore, each operation is fully persistent.
%	
%	Finally, let $(\D',v')$ be a partial result obtained from applying the operations $\add$, $\prod$ and $\union$ such that $\D'$ is duplicate-free. 
%	The proof follows from Claim~\ref{nested:appendix:eps-enum}.
\end{proof}




	




