% !TeX spellcheck = en_US
 
In this chapter, we are interested in the following streaming enumeration problem for a class $\mathcal{C}$ of \vpann (e.g. I/O-deterministic \vpann).
\begin{center}
	\framebox{
		\begin{tabular}{rl}
			\textbf{Problem:} & $\enumvpann[\mathcal{C}]$\\
			\textbf{Input:} & A \vpann $\cT \in \mathcal{C}$ and $w\in \wnS$ \\
			\textbf{Output:} & Enumerate $\sem{\cT}(w)$
		\end{tabular}
	}
\end{center}
The main result of the chapter is that for the class of I/O-unambiguous \vpann, the streaming full-enumeration version of this problem can be solved efficiently. %.with a streaming evaluation algorithm and with strong guarantees of efficiency. 
\begin{theorem}\label{nested:theo:main}
	The streaming full-enumeration problem of $\enumvpann$ for I/O-unambiguous \vpann can be solved with update-time $\cO(|Q|^2\vert\Delta\vert)$ and output-linear delay. For the class of all \vpann, it can be solved with update-time $\cO(2^{|Q|^2\vert\Delta\vert})$ and output-linear \nolinebreak delay. 
\end{theorem} 
% \cristian{Cuando tengamos escrito el algoritmo y tengamos exactamente el costo del update-time, hay que actualizar las funciones en este teorema.}
The general case is basically a consequence of Proposition~\ref{nested:vpawo:det} and the enumeration algorithm for I/O-unambiguous \vpann. For both cases, if the \vpann is fixed (i.e., in data complexity), then the update-time of the streaming algorithm is constant. In Sections~\ref{nested:sec:ds} and~\ref{nested:sec:eval}, we present this algorithm. For the rest of this section, we discuss some further details of this result.

\paragraph{$\Delta$-enumeration} The natural question is how we move from full-enumeration to $\Delta$-enumeration. In fact, we can have $\Delta$-enumeration with a slight loss of efficiency by solving the full-enumeration problem. Specifically, we can show that for any I/O-unambiguous \vpann $\cT$ there is an I/O-unambiguous \vpann $\cT'$ of linear size with respect to $|\cT|$ that only outputs new results at each position. Then combining this construction with the algorithm of Theorem~\ref{nested:theo:main}, we derive the following algorithm for $\Delta$-enumeration of \vpann.

\begin{theorem}\label{nested:vpawo:deltamain}
	The streaming $\Delta$-enumeration problem of $\enumvpann$ for I/O-unambiguous \vpann can be solved with update-time $\cO(|Q|^2\vert\Delta\vert)$ and output-linear delay. For the general class of \vpann, it can be solved with update-time $\cO(2^{|Q|^2\vert\Delta\vert})$ and output-linear \nolinebreak delay. 
\end{theorem}

\begin{proof}
	The proof of the theorem is a consequence of Theorem~\ref{nested:theo:main} and the following lemma.
	
	\begin{lemma}\label{nested:vpann:delta}
		For every I/O-unambiguous \vpann $\cT$ there exists an I/O-unambiguous \vpann $\cT'$ such that $\br{\cT'}(w) = \br{\cT}(w) \setminus \bigcup_{i < |w|} \br{\cT}(	w[1, i])$ for every $w\in\wnS$. Furthermore, the size of $\cT'$ is linear in the size of $\cT$.
	\end{lemma}
	
	Let $\cT = (Q, \Sigma, \Gamma, \oalph, \Delta, \qinit, F)$ be an I/O-unambiguous \vpann. We construct a \vpann
	$
	\cT' = (Q', \Sigma, \Gamma, \oalph, \Delta', \qinit, F')
	$
	such that $Q' = Q \times \{1, 2\}$, $\qinit' = \qinit \times\{1\}$, $F' = F \times\{1\}$ and $\Delta'$ is as follows: 
	\begin{align*}
		\Delta' =\ & \{((p,1),\op{a},\oout,(q,1),\gamma), ((p,2),\op{a},\oout,(q,1),\gamma)\mid (p,\op{a},\oout,q,\gamma)\in\Delta\}\ \cup\\ 
		&\{((p,1),\op{a},(q,1),\gamma), ((p,2),\op{a},(q,2),\gamma)\mid (p,\op{a},q,\gamma)\in\Delta\text{ where }p\not\in F\}\ \cup\\  \displaybreak[1]
		& \{((p,1),\op{a},(q,2),\gamma), ((p,2),\op{a},(q,2),\gamma)\mid (p,\op{a},q,\gamma)\in\Delta\text{ where } p\in F\} \ \cup \\ 
		& \{((p,1),\cl{a},\oout,\gamma, (q,1)), ((p,2),\cl{a},\oout,\gamma, (q,1))\mid (p,\cl{a},\oout,\gamma,q)\in\Delta\}\ \cup\\
		&\{((p,1),\cl{a},\gamma,(q,1)), ((p,2),\cl{a},\gamma,(q,2))\mid (p,\cl{a},\gamma,q)\in\Delta\text{ where }p\not\in F\}\ \cup\\ \displaybreak[1]
		& \{((p,1),\cl{a},\gamma,(q,2)), ((p,2),\cl{a},\gamma,(q,2))\mid (p,\cl{a},\gamma,q)\in\Delta\text{ where } p\in F\} \ \cup \\
		& \{((p,1),a,\oout,(q,1)), ((p,2),a,\oout,(q,1))\mid (p,a,\oout,q)\in\Delta\}\ \cup\\
		&\{((p,1),a,(q,1)), ((p,2),a,(q,2))\mid (p,\op{a},q)\in\Delta\text{ where }p\not\in F\}\ \cup\\
		& \{((p,1),a,(q,2)), ((p,2),a,(q,2))\mid (p,\op{a},q)\in\Delta\text{ where } p\in F\} \ 
	\end{align*}

	The idea behind this construction is to separate the \vpann in two halves. Each run starts in the first half (marked 1) and once it reaches a final state, it changes into the second half (marked 2). The run then stays on the second half until it sees an output symbol that extends the current output, upon which it returns to the first half.
	It is straightforward to see that $\br{\cT'}(w) = \br{\cT}(w) \setminus \bigcup_{i < |w|} \br{\cT}(w[1, i\rangle)$.
	%It is clear that $\cT'$ is I/O-unambiguous since for each state $q\in Q'$, input symbol $a$ and $\oout\in\oalph\cup\{\eps\}$ there is at most one tuple in $\Delta'$ that has $q,a,\oout$ as its first three elements.
	%\cristian{Martin: acá esta mal y la afirmación del resultado también. Cuando haces la construcción partes de un I/O unambiguous y llegas a un I/O unambiguous. Si llegaras a un I/O deterministic, algo esta mal ya que romperias cotas de construcción de automatas, asi que es imposible que sea así. El autómata queda I/O unambiguous, no porque por cada valor $\oout$ llegas a un estado, lo cual no es cierto, si no porque se sigue cumpliendo el hecho de un solo run. Corrigue esto, actualizando el Lemma 16, como también la afirmación en el Teorema 4.}
	
%	To show that $\br{\cT'}(w) = \br{\cT}(w) \setminus \bigcup_{i < |w|} \br{\cT}(w[1, i\rangle)$ consider a $w\in \wnS$. First, we show the $\subseteq$ direction of the equality. Let $\mu\in\br{\cT'}(w)$ and consider an accepting run $\rho'$ of $\cT$ over $w$ with output $\mu$. We can construct an accepting run $\rho$ of $\cT$ over $w$ by starting from $\rho'$ and replacing any appearance of a state $(q, k)$ by $q$. From this it follows that $\mu \in \br{\cT}(w)$. Assume now that $\mu \in \br{\cT}(w[1, i\rangle)$ for some $i < |w|$. From the construction of $\Delta'$ it can be seen that the $i$-th and following states in $\rho'$ are of the form $(q, 2)$, as all of the following transitions in $\rho'$ have no output symbol. Therefore, $\rho'$ cannot be an accepting run, and we reach a contradiction, from which we conclude that $\mu \in \br{\cT}(w) \setminus \bigcup_{i < |w|} \br{\cT}(w[1, i\rangle)$. 
%	
%	Now, we show the $\supseteq$ direction. Let $\mu$ be an output in $\mu \in \br{\cT}(w) \setminus \bigcup_{i < |w|} \br{\cT}(w[1, i\rangle)$ and let $\rho$ be the accepting run of $\cT$ over $w$ with output $\mu$. It can be seen from the construction of $\Delta'$ that the run of $\cT'$ over $w$ is identical to $\rho$ except each state $q$ in $\rho$ appears as $(q, k)$ in $\rho'$. We will show that the last state in $\rho'$ is of the form $(q,2)$. Towards a contradiction, assume that it is not. Therefore, in $\rho'$ there is a transition where the first state is of the form $(p,1)$ and the second is of the form $(q,2)$, and furthermore, every transition following this one has $\eps$ as its output symbol. Let $i$ be the step where this happens. From the construction of $\Delta$ we see that the $i$-th state is in $F$, from which it follows that the run $\rho'_i$ built from the first $i$ steps in $\rho'$ is an accepting run of $\cT'$ over $w[1,i\rangle$ and that $\out(\rho'_i) = \mu$. We can do a similar process as a above and construct an accepting run of $\cT$ over $w[1,i\rangle$ that renders the same output $\mu$, which contradicts our assumption that $\mu\not\in\bigcup_{i < |w|} \br{\cT}(w[1, i\rangle)$. We conclude that $\br{\cT'}(w) = \br{\cT}(w) \setminus \bigcup_{i < |w|} \br{\cT}(w[1, i\rangle)$.
	
	To show that $\cT'$ is I/O-unambiguous, consider a $w\in \wnS$. Let $\mu \in \br{\cT'}(w)$ and consider two accepting runs $\rho_1$ and $\rho_2$ such that $\out(\rho_1) = \out(\rho_2) = \mu$. Let us build a run $\rho$ of $\cT$ over $w$ by replacing each state $(q, k)$ in $\rho_1$ by $q$. Note that starting from $\rho_2$ renders the same run because they have the same output and $\cT$ is I/O-unambiguous.  
	This implies that $\rho_1 = ((q_1,\ell_1), \sigma_1) \xrightarrow{b_1} \cdots  \xrightarrow{b_n} ((q_{n+1}, \ell_{n+1}), \sigma_{n+1})$ and $\rho_2 = ((q_1, k_1), \sigma_1) \xrightarrow{b_1} \cdots  \xrightarrow{b_n} ((q_{n+1}, k_{n+1}), \sigma_{n+1})$ for some $q_i$, $\sigma_i$ and $b_i$. Note that $\ell_1 = k_1 = 1$, and suppose there is some $i$ for which $\ell_i = k_i$ and $\ell_{i+1} \neq k_{i+1}$. This is immediately false from the construction above since from any transition of $\Delta$ that is from state $p$, only two transitions are included in $\Delta'$: one from $(p,1)$ and one from $(p,2)$. This implies that $\rho_1 = \rho_2$ so $\cT'$ is I/O-unambiguous. 
\end{proof}
We could have considered a more general definition of \vpann to produce outputs for prefix-nested words. This would be desirable for having some sort of \emph{earliest query answering}~\cite{GauwinNT09} which is important in practical scenarios. We remark that the algorithm of Theorem~\ref{nested:theo:main} can be extended for this case at the cost of making the presentation more complicated. 
For the sake of presentation, we give a brief idea of how to handle this extension after the main part of Section~\ref{nested:sec:eval}.

%For the sake of presentation, we defer this extension to the full version of this paper.  
%\cristian{Estamos en la versión full? Esto hay que arreglarlo. No es necesario que lo demostremos. Lo podemos sacar o algo podemos decir. }


\paragraph{Space lower bounds of evaluating a \vpann}  
This last subsection deals with the space used by the streaming evaluation algorithm of Theorem~\ref{nested:theo:main}. 
Indeed, this algorithm could use linear space in the worst case. In the following, we explore some lower bounds in the space needed by any algorithm and show that this bound is tight for a certain type of \vpann.

To study the minimum number of bits needed to solve $\enumvpann$ we need to introduce some definitions.
Fix a \vpann $\cT$ and $w \in \pwnS$. Let 
$\outgap(\cT, w)$ be the number of positions less than $|w|$ that appear in some output of $\sem{\cT}(w\cdot w')$ for some $w\cdot w'\in \wnS$. 
Furthermore, for a well-nested word $u$ let $\depth(u)$ be the maximum number of nesting pairs inside $u$. Formally, $\depth(u) = 0$ for $u \in  {\noS}^*$, $\depth(u_1 \cdot u_2) = \max\{\depth(u_1), \depth(u_2)\}$, and $\depth(\op{a}\cdot u\cdot\cl{b}) = \depth(u) + 1$. For $w \in \pwnS$, we define $\depth(w) = \min\{\depth(w') \mid w'\in \wnS \text{ and } w \text{ is a prefix of } w'\}$.
Below, we provide some worst-case space lower bounds for \enumvpann{} that are dependent on $\outgap(\cT, w)$ and $\depth(w)$.
\begin{proposition}\label{nested:alg:spacebound} (a) There exists a \vpann $\cT_1$ such that every streaming evaluation algorithm for \enumvpann{} with input $\cT_1$ and $\Stream$ requires $\Omega(\depth(\Stream[1, n]))$ bits of space.\\ (b) There exists a \vpann $\cT_2$ such that every streaming evaluation algorithm for \enumvpann{} with input $\cT_2$ and $\Stream$ requires $\Omega(\outgap(\cT_2, \Stream[1, n]))$ bits of space.
\end{proposition}

\begin{proof}
	(a) The existence and lower bound for $\cT_1$ is a corollary of Theorem 4.5 in~\cite{BarYossefFJ07}. The proof of this result implies that for the XPath query $Q = {\tt / / a[b\ and\ c]}$, any streaming algorithm that verifies if an XML document matches $Q$ (the problem $\textsc{booleval}_Q$) and any integer $r \geq 1$, there exists a document of depth at most $r + C$, where $C$ is a constant value, on which the algorithm requires $\Omega(r)$ bits of space.
	
	The \vpann $\cT_1$ is shown in Figure~\ref{nested:fig-vpt-depthlowerbound}. It can simulate the query $Q$ for a direct mapping $\nu$ of the documents that are constructed in~\cite{BarYossefFJ07}, where $\nu({\tt \langle a \rangle}) = \op{{\sf a}}$, $\nu({\tt \langle / a \rangle}) = \cl{{\sf a}}$, $\nu({\tt \langle b/ \rangle}) = {\sf b}$, and $\nu({\tt \langle c/ \rangle}) = {\sf c}$. Note that for any well-nested document $w$ the set $\sem{\cT}(w)$ is either empty or $\{\eps\}$. Now suppose there is a streaming evaluation algorithm $\enumE$ that solves \enumvpann{}. We can solve $\textsc{booleval}_Q$ by receiving an XML stream $\Stream$, and running $\enumE$ with input $\cT$ while applying the mapping $\nu$ to each character. Let $w$ be the resulting string. At the end of the stream, we enumerate the set $\sem{\cT}(w)$ and it will enumerate the output $\eps$ iff $\Stream$ matches $Q$. We conclude that $\enumE$ runs in $\Omega(r)$ space.
	
	\begin{figure}[t]
		\centering
		\begin{subfigure}[b]{0.45\textwidth}
	
			\begin{tikzpicture}[scale=0.7,->,>=stealth',shorten >=1pt,auto,node distance=2cm,thick,state/.style={circle,draw}, color=black, initial text= {},
				initial distance= {5mm}]
				\node[state,initial] (q0) at (0,0) {$q_0$};
				\node[state] (q1) at (3,0) {$q_1$};
				\node[state] (q2) at (5,1.2) {$q_2$};
				\node[state] (q3) at (5,-1.2) {$q_3$};
				\node[state] (q4) at (7,0) {$q_4$};
				\node[state,accepting] (q5) at (10,0) {$q_5$};
				
				\draw (q0) to[loop above] node {$*$} ();
				\draw (q1) to[loop above] node {$*$} ();
				\draw (q2) to[loop above] node {$*$} ();
				\draw (q3) to[loop above] node {$*$} ();
				\draw (q4) to[loop above] node {$*$} ();
				\draw (q5) to[loop above] node {$*$} ();
				\draw (q0) to node [above] {$\op{\sf a}/{\sf A}$} (q1);
				\draw (q1) to node [above, xshift=-3pt] {${\sf b}$} (q2);
				\draw (q1) to node [below left, xshift=3pt, yshift=-1pt] {${\sf c}$} (q3);
				\draw (q2) to node [xshift=-3pt] {${\sf c}$} (q4);
				\draw (q3) to node [below right, xshift=-3pt, yshift=1pt] {${\sf b}$} (q4);
				\draw (q4) to node [above] {$\cl{\sf a},{\sf A}$} (q5);
				
%				\draw (q0) to[in=105,out=135,loop] node [above] {$\op{a}/{\sf A}$} ();
%				\draw (q0) to[in=45,out=75,loop] node [above] {$\cl{a},{\sf A}$} ();
%				\draw (q0) to[in=-105,out=-135,loop] node [below] {$b$} ();
%				\draw (q0) to[in=-45,out=-75,loop] node [below] {$c$} ();
				
				
				
%				\draw (q0) to[loop above] node [above, align=center] {$\op{a} / {\tt A}$} (q0);
%				\draw (q0) to[loop below] node [below, align=center] {$b$} (q0);
%				\draw (q1) to[out=120,in=-120] node [left] {$\op{a} / {\tt B}$} (q2);
%				\draw (q2) to[loop left] node [left, align=center] {$\op{a}/ {\tt A}$} (q2);
%				\draw (q2) to[loop right] node [right, align=center] {$\cl{a}, {\tt A}$} (q2);
%				\draw (q2) to[in=105,out=135,loop] node {$b$} (q2);
%				\draw (q2) to[in=45,out=75,loop] node {$c$} (q2);
%				\draw (q2) to[out=-60,in=60] node [right] {$\cl{a}, {\tt B}$} (q1);
%				\draw (q1) to node [above] {$c$} (q3);
%				\draw (q3) to[loop above] node [above, align=center] {$\cl{a}, {\tt A}$} (q3);
%				\draw (q3) to[loop below] node [below, align=center] {$c$} (q3);
			\end{tikzpicture}
		\centering
			\caption{$\cT_1$}
			\label{nested:fig-vpt-depthlowerbound}
		\end{subfigure}
		\hfill
		\begin{subfigure}[b]{0.45\textwidth}
			\centering
			\begin{tikzpicture}[scale=0.7,->,>=stealth',shorten >=1pt,auto,node distance=2cm,thick,state/.style={circle,draw}, color=black, initial text= {},
				initial distance= {5mm}]
				\node[state,draw=none,scale=0.1] (in) at (-1,0) {};
				\node[state] (q0) at (0,0) {$q_0$};
				\node[state, accepting] (q1) at (3,0) {$q_1$};		
				\draw (in) to (q0);
				\draw (q0) to[loop above] node [above, align=center] {${\sf a}$} (q0);
				\draw (q0) to[loop below] node [below, align=center] {${\sf b}: {\sf x}$} (q0);
				\draw (q0) to node [above] {$\$$} (q1);
			\end{tikzpicture}
			\caption{$\cT_2$}
			\label{nested:fig-vpt-lowerbound}
		\end{subfigure}
		\caption{\vpanns used in the proof of Proposition~\ref{nested:alg:spacebound}. On $\cT_1$, a loop over a node $p$ labeled by $*$ represents the four transitions $(p,\op{{\sf a}}, p, {\sf X})$, $(p,\cl{{\sf a}}, {\sf X}, p)$, $(p,{\sf b},p)$ and $(p,{\sf c},p)$.}
		\label{nested:fig:three graphs}
	\end{figure}

	(b) The existence and lower bound for $\cT_2$ uses the main ideas of the proof of Theorem 1 in~\cite{BarYossefFJ05}. Here, the authors describe a set-computing communication complexity problem. In the problem $\mathcal P$, Alice and Bob compute a two-argument function $p(\cdot, \cdot)$, defined as follows. Alice's input is a subset $A\subseteq\{1,\ldots,k\}$, Bob's input is a bit $\beta \in \{0,1\}$, and $p(A, \beta)$ is defined to be $A$, if $\beta = 1$, and~$\emptyset$ otherwise. Proposition 1 in~\cite{BarYossefFJ05} proves that the one-way communication complexity of $\mathcal P$ is at~least~$k$.
	
	Let $\cT_2 = (Q, \Sigma, \Gamma, \oalph, \Delta, \qinit, F)$ be a \vpann such that $\noS = \{{\sf a}, {\sf b}, \$\}$, $\oalph = \{{\sf x}\}$, and $Q$, $\Delta$, $\qinit$, $F$ be as presented in Figure~\ref{nested:fig-vpt-lowerbound}. Let $w \in (\noS)^*$ be a word such that $i_1 < i_2 < \ldots < i_k \leq |w|$  are all positions of $w$ where $w[i_\ell] = b$ for every $\ell \leq k$. Then one can check that $\cT_2$ defines the following function:
	$$
	\sem{\cT_2}(w) = \begin{cases}
		\{({\sf x}, i_1) \ldots ({\sf x}, i_k)\} & 
		\text{if } w \text{ ends in } \$\\
		\emptyset &\text{otherwise}.
	\end{cases}	
	$$	
	
	Consider an arbitrary algorithm $\enumE$ that solves $\enumvpann$ with input $\cT_2$. We will now present a reduction that creates a protocol for ${\mathcal P}$ which makes use of the algorithm $\enumE$. Here, Alice receives the set $A$ and generates a word $w$ of size $k$ such that $w[i] = {\sf b}$ if $i\in A$ and $w[i] = {\sf a}$ otherwise. Alice then executes $\enumE$ on input $\cT_2$ and $w$ as the first $k$ characters of a stream. She sends the state of the algorithm to Bob, who receives the bit $\beta$, and does the following: If $\beta = 1$ he continues running $\enumE$ as if the last character of the input was $\$$. If $\beta = 0$, he stops executing $\enumE$ immediately. In either case, the output given by $\enumE$ contains all the information necessary to compute the set $p(A,\beta)$, so the reduction is correct.  This proves that $\enumE$ requires at least $k$ bits for an input of size less than $k$, and so $\enumE$ for any $n \geq 1$, requires at least $n$ bits of space in a worst-case stream $\Stream$, which is in $\Omega(\outgap(\cT_2, S[1,n]))$.
\end{proof}



In~\cite{BarYossefFJ05,BarYossefFJ07}, the authors provide lower bounds on the amount of space needed for evaluating XPath in terms of the nesting and the concurrency (see~\cite{BarYossefFJ05} for a definition). 
One can show that  the \ogapname of $\cT$ and $w$ is always above the concurrency of $\cT$ and $w$. Despite this, one can check that both notions coincide for the space lower bound given in Proposition~\ref{nested:alg:spacebound}.

The previous results show that, in the worst case, any streaming evaluation algorithm for \vpann will require space of at least the depth of the document or the \ogapname.
To show that Theorem~\ref{nested:theo:main} is optimal in the worst-case, we need to consider a further assumption of our \vpann. We say that a \vpann $\cT$ is \emph{trimmed}~\cite{caralp2015trimming} if for every $w\in \pwnS$ and every (partial) run $\rho$ of $\cT$ over $w$, there exists $w'$ and an accepting run $\rho'$ of $\cT$ over $w \cdot w'$ such that $\rho$ is a prefix of $\rho'$. 
This notion is the analog of trimmed non-deterministic automata. Similarly to Proposition~\ref{nested:vpawo:det}, one can show that for every \vpann $\cT$ there exists a trimmed I/O-deterministic \vpann $\cT'$ equivalent to $\cT$ (i.e., by extending the construction in~\cite{caralp2015trimming} to \vpann). 
The next result shows that, if the input to \enumvpann{} is a trimmed I/O-unambiguous \vpann, then the memory footprint is at most the maximum between the depth and \ogapname of the input. 

\begin{proposition} \label{nested:prop:space}
	The streaming enumeration problem of \enumvpann\ for the class of trimmed I/O-unambiguous \vpann can be solved with update-time $\cO(|Q|^2\vert\Delta\vert)$, output-linear delay,  and\linebreak $\cO(\max\{\depth(\Stream[1, n]), \outgap(\cT,\Stream[1, n])\}\times|Q|^2|\Delta|)$ space for every stream $\Stream$.
\end{proposition} 
\begin{proof}
	Before reading this proof, it is important to note that the argument assumes the understanding of Algorithm~\ref{nested:alg:preprocessing}. Furthermore, this proof uses the notation introduce in Section~\ref{nested:sec:eval}.

First of all, note that the time bounds are implied by Theorem~\ref{nested:theo:main}, so we will restrict to prove the space bounds. 
Algorithm~\ref{nested:alg:preprocessing} has an update phase and an enumeration phase, and the enumeration phase only processes the data structure that was built on the update phase, using at most linear extra space, as is explained in Section~\ref{nested:sec:ds}. As such, we will prove that Algorithm~\ref{nested:alg:preprocessing} on input $(\cT, w)$ uses $\cO((\depth(w) + \outgap(\cT,w))\times|Q|^2|\Delta|)$ space at every point in its execution, which implies the statement of the proposition, where $w = \Stream[1,n]$ for some stream $\Stream$ and $n$.

As it was explained in Section~\ref{nested:sec:eval}, Algorithm~\ref{nested:alg:preprocessing} uses a hash table $S$, and a stack $T$ that stores hash tables. The size of the stack at each point is bounded $\depth(w)$, and the size of each hash table is bounded by $|Q|^2|\Gamma|$, so the size of $S$ and $T$ combined is in $\cO(\depth(w)|Q|^2|\Delta|)$. The rest of the space used is related to the \dsepsabbr $\cD$, which we will now bound by $\cO(\outgap(\cT,w[1,k])|Q|^2|\Delta|)$ at each~step~$k$.

For every step $k$ of the algorithm, consider an \dsepsabbr $\cD^{\textsf{trim}}_k$ which is composed solely of the nodes that are reachable from of the ones stored in $S^k$, or the ones stored in some hash table in $T^k$. A simple induction argument on $k$ shows that the rest of the nodes in $\cD$ can be discarded with no effect over the correctness of the algorithm, so they are not considered in the memory used by it. Therefore, proving that at each step $| \cD^{\textsf{trim}}_k | \in \cO(\outgap(\cT,w[1,k])|Q|^2|\Delta|)$ is enough to complete the proof. 

Let $\cI$ be the set of positions less than $k$ that appear in some output of $\sem{\cT}(w[1,k] \cdot w')$ for some $w \cdot w'\in \pwnS$. We now refer to Lemma~\ref{nested:vpt:steps} since it implies that for each node $v$ stored in $S^k$ or the topmost hash table in $T^k$, each sequence in $\L_{\D}(v)$ corresponds to at least one valid run of $\cT$ over $w[1,k]$, and since $\cT$ is trimmed, each one of these runs is part of an accepting run of $\cT$ over $w[1,k] \cdot w'$, for some word $w'$. Therefore, each of the positions that appear in some of these sets is in $\cI$. Furthermore, we can use this lemma to characterize the positions in the rest of the hash tables in $T^k$, since appending any close symbol $\cl{a}$ to $w[1,k]$ will make the algorithm pop an element from $T$, which will make the next hash table the topmost. This argument can be extended to any of the hash tables in $T^k$, so in all, Lemma~\ref{nested:vpt:steps} implies that all of the positions that appear in some non-empty leaf in $\cD^{\textsf{trim}}_k $ are in $\cI$. Theorem~\ref{nested:theo:main} implies that the set of these positions corresponds exactly to $\cI$, since if there was any position in $\cI$ missing from the leaves in $\cD$, the algorithm would not be correct.

Lastly, we will show that $| \cD^{\textsf{trim}}_k | \leq | \cI | \times |Q|^2|\Delta| \times d$, where $d$ is a constant. Towards this goal, we will bound the number of $\eps$-leaves, non-empty leaves, and product nodes by $\cO(| \cI | \times |Q|^2|\Delta|)$ independently. Union nodes can be bounded by counting the other types of nodes: The only cases where a union node is created are (1) in line \ref{nested:line35}, only after a product node had been created, (2) during the creation of a product node (as described in Theorem~\ref{nested:theo:data-structure-eps}), (3) in line \ref{nested:line46}, but only whenever one of the previous lines had created either a product node or a non-empty leaf node, and (4) in line \ref{nested:line15}, which only happens once at the end of the update phase, and iterates by nodes in $S$. Thus, the number of union nodes created at this for loop at most $| \cD^{\textsf{trim}}_k |$. The number of $\eps$-nodes is at most one, owing to Theorem~\ref{nested:theo:data-structure-eps}, since its proof shows that at the end of step $k$, each of the nodes in $\cD^{\textsf{trim}}_k$ is $\eps$-safe. The number of non-empty leaves can be straightforwardly shown to be $\cO(| \cI | \times |Q|^2|\Delta|)$ since each of these leaves was introduced in some step in $\cI$, and in each one of these steps, the number of operations that the algorithm does is in $\cO(|Q|^2|\Delta|)$. 

To show a bound over the number of product nodes, consider a slight modification of Algorithm~\ref{nested:alg:preprocessing}: 
product nodes that are created in line \ref{nested:line44} are labeled with the step $k$ in which the algorithm is at the moment. 
Now, for a set of nodes $A$ let $\cD^{\textsf{trim}}_A$ be the \dsepsabbr that is obtained by removing all of the nodes that are not reachable from some node in $A$ from $\cD$.
Let $\cI_A$ be the set of positions that appear in some non-empty leaf node in $\cD^{\textsf{trim}}_A$, and let ${\mathcal P}_A$ be the set of step labels that appear in some product node in $\cD^{\textsf{trim}}_A$ excluding the steps in $\cI_A$. Also, let $V_k$ be the the set of nodes in $\cD^{\textsf{trim}}_k$. We will show by induction on $k$ that $| {\mathcal P}_A | \leq | \cI_A | - 1$ for any $A\subseteq V_k$ which contains at least one node that is not an $\eps$-node. Consider any set $A \subseteq V_k$. The first observation we make here is that we can partition the nodes in $A$ to a collection $\{A_H\}$ of sets of nodes depending on the hash table $H$ they are reachable from, given that they are in $\cD^{\textsf{trim}}_k$. Let ${\mathcal Q}_A = {\mathcal P}_A \cup \cI_A$. From Lemma~\ref{nested:vpt:steps} we get that for two different sets $A_{H_1}$ and $A_{H_2}$ in the collection, the sets ${\mathcal Q}_{H_1}$ and ${\mathcal Q}_{H_2}$ are disjoint. 
Therefore, in step $k$, if the algorithm enters $\textsc{CloseStep}$, we only need to focus on the set $A_{S}$, and if the algorithm enters $\textsc{OpenStep}$ on the set and $A_{T^k}$ (note that in this case, $S^k$ is composed only of $\eps$-nodes). The rest of the hash tables were reachable  on a previous step, so the inequality can be reached by adding up the inequalities that held in those steps. 
First, note that if none of the product nodes in $A$ were created in step $k$, then we can consider the set $B$ of nodes reachable from $A$ that were created in a previous step and notice that ${\mathcal P}_A = {\mathcal P}_B$ and $\cI_B \subseteq \cI_A$, so the statement follows since $B \subseteq V_{k-1}$. 
Also, note that if the algorithm in step $k$ enters $\textsc{OpenStep}$, all of the product nodes created in this step are directly connected to a non-$\eps$ leaf created in this same step, so the statement also follows. 
From this point on, we can assume that the algorithm enters $\textsc{CloseStep}$ on step $k$, and all of the nodes in $A$ are reachable from some node in $S^k$, and there is at least one product node in $A$ that was created in step $k$. Let $P$ be the set of product nodes in $A$ that were created on step $k$. Consider the span $\clevel(k) = \spanc{j}{k}$. 
The \textsf{prod} operation in line \ref{nested:line44} either creates a new product node, or makes $v$ reference a node that already existed in $S^{k-1}$ or the topmost table in $T^j$. 
Furthermore, if a product node is created in line \ref{nested:line44}, then Theorem~\ref{nested:theo:data-structure-eps} tells us that it must be connected to a node in $S^{k-1}$ that is not an $\eps$-node, and to a node in the topmost table in $T^j$ that is also not an $\eps$-node. Consider now the set of nodes $B$ that is made up of (1) nodes in $A$ that are reachable from $S^{k-1}$ and (2) nodes in $S^{k-1}$ that are connected to a product node in $P$. Consider also the set of nodes $C$ that is made up of (1) nodes in $A$ that are reachable from the topmost table in $T^j$, and nodes in the topmost table in $T^j$ that are connected to a node in $P$. Note that both sets $B$ and $C$ contain a non-$\eps$ node, and are composed of nodes created in a previous step, so assume that $|{\mathcal P}_B| \leq |\cI_B| -1$ and that $|{\mathcal P}_C| \leq |\cI_C| -1$. It can be seen that every node in $\cD^{\textsf{trim}}_A$ is either in $B$, $C$, or was created on step $k$, so we get that ${\mathcal P}_A = {\mathcal P}_B \cup {\mathcal P}_C \cup \{k\}$ and $\cI_A \supseteq \cI_B \cup \cI_C$. From Lemma~\ref{nested:vpt:steps} we get that ${\mathcal Q}_B$ and ${\mathcal Q}_C$ are disjoint, and putting these facts to together gives us that $| {\mathcal P}_A | =  |{\mathcal P}_B| + |{\mathcal P}_C| + 1\leq |\cI_B| + |\cI_C| - 1 \leq |\cI_A|-1$.

Having proven this statement, we can deduce that the number of product nodes in $\cD^{\textsf{trim}}_k$ is in $\cO(| \cI | \times |Q|^2|\Delta|)$ since the number of steps where they are created is bounded by $|\cI|$. Therefore, $| \cD^{\textsf{trim}}_k | \leq | \cI | \times |Q|^2|\Delta| \times d$, for some constant $d$. This concludes the proof. \hfill \qed
\end{proof}

%The proof of Proposition~\ref{nested:prop:space} is somewhat technical and depends on analyzing the main algorithm of Theorem~\ref{nested:theo:main} (which we will present in Section~\ref{nested:sec:eval}). For this reason, we defer the proof of Proposition~\ref{nested:prop:space} to the appendix, which the reader can check after we prove the algorithm for~Theorem~\ref{nested:theo:main}.

Unfortunately, the algorithm provided in Theorem~\ref{nested:theo:main} is not \emph{instance optimal}, in the sense of using the lowest number of bits needed for each specific \vpann. 
Specifically, there exist \vpanns for which only logarithmic space in $\outgap(\cT,w)$ is enough for any stream $\Stream$. For example, let $\oout$ be any output symbol and consider a \vpann $\cT$ for which the output set is $\sem{\cT}(w)  = \{(\oout, i)\mid 1\leq i\leq \vert w \vert \}$ if the last symbol in $w$ is $\$$ and the empty set otherwise. Clearly, the \ogapname of any $w$ with respect to $\cT$ is linear in $\vert w\vert$. However, one could design a streaming evaluation algorithm that has only a counter that stores the length of the input so far and produces the correct output set after reading the last symbol in $w$. The enumeration phase can easily be done with output-linear delay (i.e., by counting from $1$ to $\vert w\vert$). 
Furthermore, note that an instance optimal algorithm for the streaming enumeration problem of \vpanns will imply a solution to the \emph{weak evaluation problem}, stated by Segoufin and Vianu~\cite{SegoufinV02}, which is an open problem in the area (see \cite{Barloy21} for some recent results). We leave the study of \emph{instance optimal} streaming evaluation algorithms for future work. 
