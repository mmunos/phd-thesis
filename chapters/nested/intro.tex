% !TeX spellcheck = en_US
%!TEX root = ../../Thesis.tex

In this chapter, we present a query model named Visibly Pushdown Annotators (VPAnn), which is a computational model that extends visibly pushdown automata with outputs and has the same expressive power as Monadic Second Order over nested documents. We study the task of receiving the input data in a streaming document, and we show an algorithm which enumerates the outputs with output-linear delay after a preprocessing phase which takes data-independent time after receiving each character in the input data.

Furthermore, we show that this algorithm is worst-case optimal in terms of update-time per symbol and memory usage.

\paragraph{Outline of the chapter} 
In Section~\ref{nested:sec:prelim}, we describe the main terminology and models for the chapter. 
In Section~\ref{nested:sec:enum}, we discuss the notion of a streaming enumeration problem. 
In Section~\ref{nested:sec:vpann}, we define Visibly Pushdown Annotators (\vpann), the logical model we will use to represent queries over nested documents and study their expressive power. In Section~\ref{nested:sec:results}, we state the main result of the chapter and discuss some of its extent. In Section~\ref{nested:sec:ds}, we show the data structure used by the algorithm, called Enumerable Compact Set, which stores the output and handles the enumeration efficiently. Section~\ref{nested:sec:eval} presents and analyzes the main algorithm. In Section~\ref{nested:sec:spanners}, we link our results to document spanners. 
We close this chapter in Section~\ref{nested:relwork} by discussing some related work. 
%Finally, we discuss future work in Section~\ref{nested:sec:conclusions}. 

%\paragraph{New Material}
%This paper is an extended version of the article ``{\it Streaming enumeration on nested documents}'' that was published in the 25th International Conference on Database Theory (ICDT 2022). For this extended version, we have made several changes. First, we updated the query model from Visibly Pushdown Transducers to Visibly Pushdown Annotators, which simplifies the presentation and makes the model more similar to recent works about enumeration for automata~\cite{AmarilliJMR22, MunozR23}. We have improved the presentation of the enumeration algorithm, which in the previous version was relegated to the appendix and had a minor mistake. We have included further examples to support the explanation of the preprocessing algorithm. Lastly, the largest change is a new chapter in which we make a connection to document spanners and the extraction grammars of Peterfreund~\cite{liatpaper}.
%
%\cristian{Martin, escribe aqui la sección de new material. {\bf [HECHO]}}




%The main idea underlying the proof is to convert an accepting run of the VDT to a circuit which is used to represent all the outputs. This idea has already appeared earlier in several places: Bruno Courcelle, "Linear delay enumeration and monadic second-order logic",  Discret. Appl. Math. 2009 Dan Olteanu, Jakub Závodný, "Factorised Representations of Query Results: Size Bounds and Readability", ICDT 2012 Antoine Amarilli, Pierre Bourhis, Louis Jachiet, Stefan Mengel "A Circuit-Based Approach to Efficient Enumeration", ICALP 2017 Szymon Toruńczyk, "Aggregate Queries on Sparse Databases", PODS 2020

%\input{../appendix/relatedlist}

%Streaming query evaluation has been extensively studied in the last decades. 
%
%Some work consider streaming verification, like schema validation or schema typing, but here the output is boolean. \cite{Vianu}.
%
%Other proposals proposed practical streaming evaluation of subfragments of XPath or XQuery, but reaching constant-delay enumeration would be very unlikely. Most of this work consider outputs of fixed size (i.e., tuples). 
%
%In X enumeration aspects, like earliest query answering (e.g,  output nodes as early as possible, namely, once they will be selected in all possible continuations) or bounded delay (e.g., given the first visit of a node, find the earliest event that permits its selection).
%
%
%Visibly pushdown automata has been used for streaming evaluation, but usually for boolean queries. 
%In~\cite{FiliotGRS19,AlurFMRS20}, the authors studied the evaluation of VPT in a streaming fashion, but none of them looked at enumeration problems.
%Further extensions of this or similar models for streaming evaluation has been used, but restricted to fixed size outputs. Moreover, concepts like constant-delay were not included. Streaming evaluation with constant-delay enumeration was considered in the context of complex event processing or dynamic query evaluation. In both cases, query evaluation over nested documents it is not considered. 
%
%To the best of our knowledge, this is the first work on considering the evaluation of visibly pushdown transducers over nested documents with constant-delay enumeration.
%
%
%\input{../appendix/relatedlist}


% #######################################################################
%\cristian{Previous version.}
%
%A \emph{constant-delay enumeration algorithm} is an efficient solution to an enumeration problem: given an instance of the problem, the algorithm performs a update phase to build some indices, to then continue with an enumeration phase where it retrieves each output, one-by-one, taking constant-delay between consecutive outcomes.
%These algorithms provide a strong guarantee of efficiency since a user knows that, after the update phase, it will access the output as if we have already computed them. 
%For these reasons, constant-delay algorithms have attracted researchers' attention, finding sophisticated solutions on several query evaluation problems. Starting with Durand and Grandjean's work~\cite{DurandG07}, researchers have found constant-delay algorithms for various subclasses of conjunctive queries~\cite{BaganDG07,CarmeliZBKS20,BerkholzGS20}, FO queries over sparse structures~\cite{DurandG07,KazanaS11,SchweikardtSV18}, MSO queries over words and trees~\cite{Bagan06,AmarilliBJM17}, document spanners~\cite{FlorenzanoRUVV20,AmarilliBMN19}, among others~\cite{Segoufin13}.
%
%One could also develop constant-delay enumeration algorithms for query evaluation over nested documents. Indeed, some of the most relevant document schemas used online, such as XML and JSON, have a nested format. We can model the structure of these documents with the theory of \emph{nested words}, where we divide the alphabet between open and close tags and encode the data in a sequence of well-nested symbols. Many formalisms for processing nested documents can be understood with \emph{visibly pushdown automata} (VPA)~\cite{AlurM04}, an automata model with excellent algorithmic properties~\cite{alur2009adding}. Moreover, there have been studies of its natural extension to transducers~\cite{FiliotRRST18}, called \emph{visibly pushdown transducers} (VPT), a model for processing nested documents and producing outputs. The main advantage of these models compared to other models of nesting structures (e.g., automata over trees) is that it allows processing naturally the input in a streaming fashion, reducing the number of resources needed to fulfill its task. 
%
%This paper presents a constant-delay enumeration algorithm for evaluating VPT over nested words. Specifically, we study the combined complexity of evaluating a VPT $\cT$ over an input word $w$ and consider the notion of output-linear delay~\cite{FlorenzanoRUVV20}, a refinement of the idea of constant-delay where the delay between outputs depends only the size of two consecutive outcomes, namely, constant with respect to $\cT$ and $w$. Therefore, we provide an output-linear delay enumeration algorithm that takes preprocessing time $|\cT|^3 \cdot |w|$, namely, linear in the size of the input word, for a reasonably expressive class of VPT. 
%
%	
%Towards the end of this work, we apply our results to study efficient enumeration algorithms in information extraction. More specifically, we use our techniques to derive an output-linear delay algorithm for evaluating visibly pushdown extraction grammars, a subclass of extraction grammars~\cite{liatpaper}. This solution is the first practical algorithm (i.e., linear preprocessing) to evaluate a subclass of recursive document spanners efficiently~\cite{PeterfreundCFK19}.
%
%\smallskip
%
%\noindent \textbf{Related work.} Constant-delay algorithms have been studied for several classes of query languages and structures~\cite{Segoufin13}, as we already discussed. The approach of~\cite{Bagan06,AmarilliBJM17} is the closest to this work (see the discussion above). Algorithms for processing nested documents (e.g., XML) with one-pass~\cite{GauwinNT09} or constant-delay~\cite{BojanczykP11} have been studied previously; however, they are interested in different problems not directly related to this work. In~\cite{FiliotGRS19,AlurFMRS20}, the authors studied the evaluation of VPT in a streaming fashion, but none of them looked at enumeration problems. Very recently, extensions of document spanners with recursion have been considered~\cite{PeterfreundCFK19, liatpaper}. In these works, the question of investigating nested documents is not addressed. To the best of our knowledge, this is the first work on considering the evaluation of visibly pushdown transducers over nested documents with constant-delay enumeration.
%
%\input{../appendix/relatedlist}
