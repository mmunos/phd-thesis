%
%

\paragraph{Rigid grammars}
We first define the restricted notion of grammars that we study.
%
Consider an annotated grammar $\cG = (N, \Sigma, \Omega, P, S)$, and a string
$\gamma\in (\Sigma \cup (\Sigma \times \Omega)\cup N)^*$ of nonterminals and
of terminals which may carry an annotation in~$\Omega$. We will be interested in
the \emph{shape}
of~$\gamma$, written $\shape(\gamma)$: it is the string over $\{0, 1\}$
obtained by replacing every nonterminal of~$N$ in~$\gamma$ by~$1$ and replacing
all terminals (annotated or not) by~$0$: note that $|\shape(\gamma)| = |\gamma|$.

We then say that an annotated grammar $\cG$ is \emph{rigid}
%
%
	%
        if for every string
        $w \in \Sigma^*$, all derivations from the start symbol~$S$ of~$\cG$ to
        an annotated string $\hat{w}$ of~$w$ have the same sequence of shapes
        (remember that we only consider leftmost derivations).
        Formally, there exists a sequence $s_1, \ldots, s_k \in \{0,
        1\}^*$ depending only on~$w$ such that for any derivation 
        $S = \alpha_1 \der{\cG} \alpha_2 \der{\cG} \ldots \der{\cG} \alpha_m =
        \hat{w}$ with $\str(\hat{w}) = w$, we have $m = k$ and $\mathrm{shape}(\alpha_i) = s_i$ for all $1
        \leq i \leq k$.
	%
	%
	%
%

Intuitively, the sequence of shapes of a derivation describes the
skeleton of the corresponding derivation tree. Thus, a rigid annotated
grammar is one where, for each unannotated string, all derivation trees for all
annotations of the string are isomorphic (ignoring the labels of 
nonterminals and the annotation of terminals).

Now we restate Proposition~\ref{gram:prp:2nf} while including the angle of rigid grammars.

\begin{proposition}[Proposition~\ref{gram:prp:2nf}]
  \label{gram:prp:2nfrig}
  Given any annotated grammar\/ $\mathcal{G}$, we can compute in linear time an annotated
  grammar $\mathcal{G}'$ in 2NF such that $\mathcal{G}$ and $\mathcal{G}'$ are
  equivalent. Furthermore, if $\mathcal{G}$ is unambiguous (resp. rigid) then $\mathcal{G}'$ is unambiguous (resp. rigid) as well.
\end{proposition}
\begin{proof}
  \paragraph{Conditions 1 and 2: removing useless nonterminals}
We first perform a linear-time exploration from the terminals to mark the
nonterminals $X$ that can derive some string of terminals. The base case is if a nonterminal
$X$ has a rule $X \rightarrow \alpha$ where $\alpha$ only consists of terminals
(in particular $\alpha = \epsilon$), then we mark it.
The induction is that whenever a nonterminal $X$ has a rule $X
\rightarrow \alpha$ where $\alpha$ only consists of terminals and of
marked nonterminals, then we mark~$X$. At the end of this process, it is clear that any nonterminal
that is not known to derive a string of terminals indeed does not derive any
string, because any derivation of a string of terminals from a nonterminal $X$
would witness that all nonterminals in this derivation, including $X$, should
have been marked, which is impossible. Hence, we can remove the nonterminals
that are not marked without changing the language or
successful derivations of the grammar, and satisfy condition 1 in linear time.

Second, we perform a linear-time exploration from the start symbol $S$ to mark the
nonterminals $X$ that can be reached in a derivation from $S$. The base case is
that $S$ is marked. The induction is that whenever a nonterminal $Y$ occurs in
the right-hand side of a rule having $X$ as its left-hand side, and $X$ is
marked, then we mark $Y$. At the end of the process, if a nonterminal $X$ is
not marked, then indeed there is no derivation from $S$ that produces a string
featuring $X$, as otherwise it would witness that $X$ is marked, which is
impossible. Hence, we can again remove the nonterminals that are not marked,
the grammar and successful derivations are again unchanged, and we
satisfy condition 2 in linear time.

As the transformations here only remove nonterminals and rules that cannot
appear in a derivation, they clearly preserve unambiguity as well as rigidity.

\paragraph{Condition 3: shape of rules}
We first ensure that every right-hand side of a rule is of size $\leq 2$.
Given the annotated grammar $\mathcal{G}$, for every rule $X \rightarrow \alpha$ where
$|\alpha| > 2$, letting $\alpha = \alpha_1\cdots\alpha_n$,
we introduce $n-2$ fresh nonterminals $X_{\alpha,1},
\ldots, X_{\alpha,n-2}$, and replace the rule by the following: $X
\rightarrow \alpha_1 X_{\alpha,1}$, $X_{\alpha,1} \rightarrow \alpha_2
X_{\alpha,2}$, ..., $X_{\alpha,n-2} \rightarrow \alpha_{n-1} \alpha_{n}$.

We make sure that the right-hand side of rules of size 2 consist only of
nonterminals by introducing fresh intermediate nonterminals whenever
necessary, which rewrite to the requisite terminal.

%
%
%
It is then clear that the result satisfies condition 3, and that there is
a one-to-one correspondence between derivations in the original grammar and derivations in the rewritten grammar. To see this, note that there is an obvious one-to-one function which maps derivations from the original grammar into derivations in the new grammar, and that there is a slightly more involved function which receives a derivation in the new grammar, and builds a derivation in the original grammar by following the steps detailed above (and using the fact that each fresh nonterminal is associated to exactly one rule), which is also one-to-one. We conclude that the original grammar is unambiguous if and only if the new grammar is unambiguous.

%
The last point to check is that the arity-2 transformation preserves rigidity,
i.e., if the original annotated grammar is rigid then so is the image of the
transformation. Let $X$ be some symbol of the original grammar $\cG$, and
$w \in \Sigma^*$ be a string. Let us show that all derivations from the
corresponding
symbol $X'$ of the rewritten grammar $\cG'$ have same shape. We do so by
induction on the length of~$w$ and then on the topological order on
nonterminals. The base case of $w$ of length $0$ is clear: the possible
derivations are sequences of applications of rules of the form $Y \rightarrow Z$
in a sequence of some fixed length, followed by a rule of the form $Y
\rightarrow \epsilon$, and what can happen in the rewritten grammar is the same.

For the inductive case, as $\cG$ is rigid, we know that there must be one
fixed profile $\pi \in \{0, 1\}^k$ such that all derivations of~$w$ from~$X$ start by the
application of a rule $X \rightarrow \alpha$ where $\alpha$ corresponds to
profile~$\pi$, i.e., it has length~$k$ and its $i$-th character is a nonterminal
or terminal according to the value of the $i$-th bit of~$\pi$.
Otherwise the existence of two different right-hand-side profiles would
contradict rigidity. Furthermore, by considering the possible sub-derivations
from~$\alpha_1$ (including the empty derivation if $\alpha_1$ is a terminal), we
know that $\alpha_1$ derives some fixed prefix of~$w$ and that all
such derivations have the same sequence of profiles; otherwise we would witness
a contradiction to rigidity. By applying the same argument successively to
$\alpha_2,
\ldots, \alpha_k$, we deduce
that there must be a partition of $w =
w_1 \cdots w_k$ such that, in all derivations of~$w$ from~$X$, the derivation
applies a rule with right-hand having profile $\pi$ to produce some string $\alpha_1 \cdots
\alpha_k$, and then each $\alpha_i$ derives an annotation of~$w_i$ and
for each $i$ all possible derivations of some annotation of~$w_i$ by some $i$-th
element in the right-hand size of such a rule has the same sequence of
profiles.

As the string is nonempty we know that $k > 0$. Further, if $k = 1$ then $X$ and
the productions involving $X$ were not rewritten so we immediately conclude
either with the case of a rule $X \rightarrow \tau$ for a terminal $\tau$ or by
induction hypothesis on the nonterminals in the topological order for the case
of a rule of the form $X \rightarrow Y$. Hence, we assume that $k \geq 2$.

We know by induction that, in the rewritten grammar, the derivation from $X$
will start by rewriting $X$ to $Y_1 X_{\alpha, 1}$, the $X_{\alpha, 1}$ being
itself rewritten to $Y_2 X_{\alpha, 2}$, and so on, for some right-hand size
$\alpha$ of a rule $X \rightarrow \alpha$ having profile $\pi$. Clearly each
$Y_i$ will have to derive an annotation of the $w_i$ in the partitioning of~$w$, as
a derivation following a different partitioning would witness a derivation in
the original grammar that contradicts rigidity.
Now, the profile
$\pi$ indicates if each $Y_i$ is a nonterminal of the initial grammar or a fresh
nonterminal introduced to rewrite to a terminal. In the latter case, there is no
possible deviation in profiles. In the former case, we conclude by induction
hypothesis that each $Y_i$ derives annotations of its~$w_i$ that all have the
same profile, and we conclude that all derivations in the rewritten grammar
indeed have the same profile, concluding the proof.~\qedhere


%
%
%
%
%
%
%
%
%
%
%
%
%
%
%
%
%
%
%
%
%
%
%
%
%
%
%
%
%
%
%
%
%
%
%
%
%
%
%
%
%
%
%
%
%
%
%
%
%
%
%
%
%
%
%
%
%
%
%
%

%
%


%
%
%
%
%
%
%
%
%
%
%
%
%
%
%
%
%
%
%
%

%
%
%
%
%
%
%
%
%

%

%


\end{proof}









\paragraph{Rigidity vs unambiguity}
Unambiguity and rigidity for annotated grammars seem
incomparable: unambiguity imposes that every annotation is produced by only one
derivation, whereas rigidity imposes that all derivations across all annotations
have the same shape (but the same annotation may be obtained multiple times).

However, it turns out that, on rigid grammars, we can impose unambiguity without
loss of generality: all rigid grammars can be converted to equivalent rigid and
unambiguous grammars.

\begin{theorem}
  \label{gram:thm:profileu-iou} For any rigid grammar
	$\cG$ we can build an equivalent rigid and
	unambiguous grammar $\cG'$. The transformation runs in exponential time,
        i.e., time $\cO(2^{|\cG|^c})$ for some $c> 0$.
\end{theorem}
\begin{proof}
  In this proof, we will use the notion of PDAnn introduced in
Section~\ref{gram:sec:linear}, and we will use 
Proposition~\ref{gram:prop:grammar-pdann}, which is also stated in
Section~\ref{gram:sec:linear}.

%

%
%
%
%
%
%
%
%
%
%
%
%

To prove Theorem~\ref{gram:thm:profileu-iou}, we introduce a general-purpose normal form on PDAnn, where,
intuitively, the only choices that can be made during a run are between the \emph{types}
of transition to apply. 
% This is similar to Lemma~1 in Chapter~\ref{ch1}.

\begin{definition}
  A PDAnn $\cP$ is \emph{deterministic-modulo-profile} if it satisfies the following conditions:

  \begin{enumerate}
    \item for each state $p$ there is at most one push transition that
      starts on $p$, formally $\card{\{q, \gamma \in Q \times \Gamma \mid
      (p, q, \gamma) \in \Delta\}} \leq 1$
    \item for each state $p$ and stack symbol $\gamma$ there is at most
      one pop transition that starts on $p,\gamma$, formally $\card{\{q
      \in Q \mid (p, \gamma, q) \in \Delta\}} \leq 1$
    \item for each state $p$, letter $a$, and output $\oout \in \Omega$, there is at most one read-write transition that starts on
      $p,a,\oout$, formally, we have $\card{\{q \in Q \mid (p,(a,\oout),q) \in \Delta\}}
      \leq 1$.
      \item for each state $p$ and letter $a$, there is at most one read transition that starts on
      $p,a$, formally $\card{\{q \in Q \mid (p,a,q) \in \Delta\}}
      \leq 1$.
  \end{enumerate}
\end{definition}

\begin{lemma}\label{gram:lem:pdtonechoice}
 %
        Let $\cP$ be a PDAnn. We can build an equivalent PDAnn $\cP'$ which is
        deterministic-modulo-profile. The transformation takes exponential time,
        i.e., time $\cO(2^{|\cP|^c})$ for some $c > 0$.

        Further, on any string~$w$, there is an accepting run of~$\cP$ on~$w$ with
        profile $\pi$ iff there is an accepting run of~$\cP'$ on~$w$ with the same profile.
\end{lemma}

\begin{proof}
  The proof is similar to the determinization of visibly pushdown
  automata (see Section~\ref{nested:sec:eval}, also Proposition~\ref{nested:vpawo:det}).

  Given $\cP = (Q, \Sigma, \Omega, \Gamma, \Delta, q_0, F)$,
        we build $\cP' = (Q', \Sigma, \Omega, \Gamma', \Delta', S_I, F')$ as
        follows.
	We build $Q' = 2^{Q\times Q}$, intuitively denoting a set of pairs of
        states $(p, q)$ of $\cP$ such that $\cP$ can be at state $q$ at this
        point if it was at state $p$ when the topmost stack symbol was pushed. 
        We build $\Gamma' = 2^{Q\times\Gamma\times Q}$,
        intuitively specifying the sets of possible stack symbols and
        remembering the state just after the previous stack symbol was pushed
        and the state just after that symbol was pushed.
	We build $S_I = \{(q_0,q_0)\}$, meaning that initially we are at
        the initial state~$q_0$ and were here when the stack was initialized.
        We build
        $F' = \{S\mid (q_0,q)\in S\text{ for some }q\in F \}$,
        meaning that we accept when $\cP$ reaches a final
        state and we were at the initial state when the stack was initialized.
	Let $\Delta'$ be defined as follows:
	\begin{itemize}
          \item The (unique) push transition from a state $S \in Q'$ 
                  makes $\cP'$ push a stack symbol $S'$ and move to a state $T$,
                  intuitively defined as follows. For every pair $(p, p')$
                  of~$S$ and push transition $(p', q, \gamma) \in \Delta$ in the original PDAnn,
                  we can move to state $(q, q)$ and push on the stack the symbol
                  $(p, \gamma, q)$. The stack symbol $S'$ is the set of all
                  possible stack symbols that can be pushed in this way, and
                  $T$ is the set of all possible states that can be reached in
                  this way.

                  Formally, 
                  for every $S\in Q'$ we include $(S, S', T)$ in $\Delta'$, where:
		\begin{align*}
			T &= \{(p,\gamma,q)\mid (p,p')\in S \text{ and } (p',q,\gamma)\in\Delta\text{ for some }p,p',q\in Q, \gamma\in\Gamma \},\\
			S'&= \{(q,q)\mid(p,p')\in S\text{ and }(p',q,\gamma)\in\Delta\text{ for some }p,p',q\in Q,  \gamma\in\Gamma\}
		\end{align*}
              \item The (unique) pop transition from a state $S \in Q'$ and topmost
                stack symbol $T \in \Gamma'$ makes $\cP'$ move to a state $T'$
                intuitively defined as follows. For every pair $(p', q')$ of~$S$,
                we consider all triples $(p, \gamma, p')$ of the topmost stack
                symbol $T$, and if the original PDAnn had a pop transition $(q',
                \gamma, q) \in \Delta$, then we can pop the topmost stack symbol
                and go to the state $(p, q)$. The new state $T'$ is the set of
                all pairs $(p, q)$ that can be reached in this way.

                  Formally, 
                  for every $(S, T)\in Q'\times\Gamma'$ we include $(S, T, S')$ in $\Delta'$, where:
                  {\small
		\begin{align*}
			S' &= \{(p,q)\mid(p,\gamma,p')\in T\text{ and }(p',q')\in S\text{ and }(q',\gamma,q)\in\Delta \text{ for some }p,p',q,q'\in Q,\gamma\in\Gamma\},
		\end{align*}
                }

              \item The (unique) read-write transition from a state $S \in Q'$ on a letter
                  $a \in \Sigma$ and output $\oout \in \Omega$ makes
                  $\cP'$ move to a state $S'$ intuitively defined as follows: we
                  consider all pairs $(p, p')$ in $S$ and all transitions from
                  $p'$ with $a$ and $\oout$ in~$\cP$ to some state $q$, and
                  move to all possible pairs $(p', q)$.

                  Formally, for every $(S, a, \oout)\in Q'\times\Sigma\times\Omega$ we include $(S, (a, \oout), S')$ in $\Delta'$, where:
		\begin{align*}
			S' &= \{(p,q)\mid (p,p')\in S\text{ and }	(p',(a,\oout),q)\in\Delta\text{ for some }p,p',q\in Q\}.
		\end{align*}
	 \item The (unique) read transition from a state $S \in Q'$ on a letter
	$a \in \Sigma$ makes
	$\cP'$ move to a state $S'$ intuitively defined as follows: we
	consider all pairs $(p, p')$ in $S$ and all transitions from
	$p'$ with $a$ in~$\cP$ to some state $q$, and
	move to all possible pairs $(p', q)$.
	
	Formally, for every $(S, a)\in Q'\times\Sigma$ we include $(S, a, S')$ in $\Delta'$, where:
	\begin{align*}
		S' &= \{(p,q)\mid (p,p')\in S\text{ and }	(p',a,q)\in\Delta\text{ for some }p,p',q\in Q\}.
	\end{align*}
	\end{itemize}

        It is clear by definition that $\cP'$ is deterministic-modulo-profile,
        and it is clear that the running time of the construction satisfies the
        claimed time bound.

        We now show that $\cP$ and $\cP'$ are equivalent.

        Now, for the forward
        direction, let us first assume without loss of generality that whenever
        $\cP$ makes a push transition then the stack symbol that it pushes is
        annotated with the state reached just after the push. Then we will show
        that every instantaneous description that can be reached by $\cP$ can be
        reached by $\cP'$ by induction on the run. Specifically, we show by
        induction on the length of the run $\rho$ the following claim: if $\cP$
        has a run $\rho$ on a string $w$ that produces $\mu$ from an initial state $q_0 \in T$ to an instantaneous
        description $(q, i), \alpha$, with $\alpha =
        \gamma_0 p_0, \ldots, \gamma_m p_m$ being the sequence of the stack
        symbols and states annotating them, then $\cP'$ has a run $\rho'$ on $w$
        from $S_I$ to an instantaneous description $(S, i), \alpha'$ with
        $\alpha' = T_0 \ldots T_m$ such that $T_0$ contains $(q_0, \gamma_0,
        p_0)$, $T_1$ contains $(p_0, \gamma_1, p_1)$, ..., $T_m$ contains
        $(p_{m-1}, \gamma_m, p_m)$ and $S$ contains $(p_m, q)$; further $\rho$
        and $\rho'$ have the same profile.

        The base case of an empty run on a string is immediate: if $\cP$ has an
        empty run from an initial state $q_0$, then it reaches the instantaneous
        description with $(q_0, 0)$ and the empty stack, and then $\cP'$ then
        has an empty run reaching the instantaneous description $(S, 0)$ with the
        empty stack and $S$ indeed contains $(q_0, q_0)$. 

        For the induction case, assume that $\cP$ has a non-empty run $\rho_+$ on a string $w$ that produces $\mu$.
        First, write $\rho_+$ as a run $\rho$ followed by one single transition
        of $\cP$. We know $\cP$ has a run $\rho$ on $w$ which produces $\mu$ from an initial
        state $q_0$ to an instantaneous description $(q, i), \alpha$,
        with $\alpha = \gamma_0 p_0, \ldots, \gamma_m
        p_m$. By the induction hypothesis, we know that $\cP'$ has a run
        $\rho'$ on $w$ from $(q_0, q_0)$ to an instantaneous description $(S,
        i), \alpha'$ with $\alpha' = T_0 \ldots T_m$ such that $T_0$ contains
        $(q_0, \gamma_0, p_0)$, $T_1$ contains $(p_0, \gamma_1, p_1)$, ...,
        $T_m$ contains $(p_{m-1}, \gamma, p_m)$ and $S$ contains $(p_m, q)$; and
        $\rho'$ and $\rho$ have the same profile. We
        now distinguish on the type of the transition used to extend $\rho$ to
        $\rho_+$.

        If that transition is a read-write transition $(q, (a, \oout), q')$, we consider
        the read-write transition of~$\cP'$ labeled with $(a,\oout)$ from~$T$,
        and call $S'$ the state that $\cP'$ reaches.
        As $(p_m, q) \in S$ and $(q, (a, \oout), q') \in \Delta$, we know that
        $(p_m, q') \in S'$. Thus, $\cP'$ can read $(a,\oout)$ and
        reach a suitable state $S'$ and position $i+1$ and the stacks are
        unchanged so the claim is proven.
        
        If that transition is a read transition $(q, a, q')$, we follow an analogous reasoning.

        If that transition is a push transition $(q, q', \gamma)$, the position
        of $\cP$ is unchanged and the new stack is extended by $\gamma$
        annotated with state $q'$. Consider the push transition of $\cP'$
        from~$q$, and call $S'$ the state reached and $T = T_{m+1}$ the stack symbol that is
        pushed. As $(p_m, q) \in S$ and $(q, q', \gamma) \in \Delta$, we know
        that $T$ contains $(p_m, \gamma, q')$, and $S'$ contains $(q', q')$,
        which is what we needed to show.

        If that transition is a pop transition, $(q, \gamma_m, q')$, the position
        of $\cP$ is unchanged and the topmost stack symbol is removed. Consider
        the topmost stack symbol $T_m$ and the transition of $\cP'$ that pops
        it from $S$, and call $S'$ the state that we reach. We know that $S$
        contains $(p_m, q)$ and $T_m$ contains $(p_{m-1}, \gamma_m, p_m)$ and
        $(q, \gamma_m, q') \in \Delta$, so $S'$ contains $(p_{m-1}, q')$, which
        is what we needed to show.

        Note that, in all four cases, the profile of $\rho_+$ and $\rho_+'$ is
        the same, because this was true of $\rho$ and $\rho'$, and the type of
        transition done to extend $\rho'$ to~$\rho_+'$ is the same as the type
        of transition done to extend $\rho$ to~$\rho_+$.

        The inductive claim is therefore shown, and thus if $\cP$ has a run $\rho$ on some string $w$ that produces $\mu$ starting at some initial state $q_0$ and
        ending at state $q$, then $\cP'$ has a run $\rho'$
        on $w$ which produces $\mu$ and ending at a state of the form $(q_0,
        q)$ for $q_0$ and having same profile. Thus, if $\rho$ is accepting then $q$ is final for
        $\cP$ and $(q_0, q)$ is final for $\cP'$ so $\rho'$ is accepting.
        This concludes the forward implication.

        We now show the backward implication, and show it again by induction,
        again assuming that $\cP$ annotates the symbols of its stack with the
        state reached just after pushing them. We
        show by induction on the length of a run $\rho'$ the following claim: if
        $\cP'$ has a run $\rho'$ on a string $w$ that produces $\mu$from its initial state to an instantaneous
        description $(S, i), \alpha'$ with $\alpha' =
        T_0, \ldots, T_m$ being the sequence of the stack
        symbols, then for any choice of elements $(q_0, \gamma_0, p_0) \in T_0$,
        $(p_0, \gamma_1, p_1) \in T_1$, ..., $(p_{m-1}, \gamma_m, p_m) \in T_m$
        and $(p_m, q) \in S$ it holds that $\cP$ has a run $\rho$ on $w$ producing $\mu$
        from some initial state $q_0$ to the instantaneous description $(q, i), \alpha$ with
        $\alpha = \gamma_0 q_0, \ldots, \gamma_m q_m$ (writing next to each
        stack symbol the state that annotates it), and $\rho'$ and $\rho$ have
        the same profile.

        The base case of an empty run on a string is again immediate: if $\cP'$ has an
        empty run from its initial state, then it reaches the instantaneous
        description with $(S_I, 0)$ and empty stack, and then $\cP$
        has an empty run from any initial state $q_0$ to $q_0$ so that indeed
        $S_I$ contains $(q_0, q_0)$.

        For the induction case, assume that $\cP'$ has a non-empty run $\rho'_+$ on $w$ which produces $\mu$,
        We write again $\rho'_+$ as a run $\rho'$ followed by one single transition
        of $\cP'$. We know $\cP'$ has a run $\rho'$ on $w$ which produces $\mu$ from the initial
        state $S_I$ to an instantaneous description $(S, i), \alpha'$,
        with $\alpha = T_0 \ldots T_m$.
        By the induction hypothesis, we know that for any choice of elements
        $(q_0, \gamma_0, p_0) \in T_0$,
        $(p_0, \gamma_1, p_1) \in T_1$, ..., $(p_{m-1}, \gamma_m, p_m) \in T_m$
        and $(p_m, q) \in S$, 
        then $\cP$ has a run $\rho$ on $w$ which produces $\mu$
        from some initial state $q_0$ to the instantaneous description $(q, i), \alpha$ with
        $\alpha = \gamma_0 q_0, \ldots, \gamma_m q_m$, and $\rho$ and $\rho'$
        have the same profile.
        We now distinguish on the type of transition used to extend $\rho'$
        to~$\rho'_+$. 

        If the last transition is a read-write transition $(S, (a, \oout), S')$ with $S'$
        defined as in the construction, consider any choice of 
        $(q_0, \gamma_0, p_0) \in T_0$,
        $(p_0, \gamma_1, p_1) \in T_1$, ..., $(p_{m-1}, \gamma_m, p_m) \in T_m$
        and $(p_m, q') \in S'$, and then there must be some state $p''$ such that
        $(p'', (a, \oout), q) \in \Delta$ and $(p_m, p'') \in S$. Using the
        induction hypothesis but picking $(p_m, p'') \in S$, we obtain a run
        $\rho$ of~$\cP$ on $w$ which produces $\mu$, with the correct stack and ending at
        position $i$ on state $p''$, which we can extend by the read transition
        $(p'', (a, \oout), q)$ to reach state $q$ at position $i+1$ without
        touching the stack, proving the result.
        
        If the last transition is a read transition $(S, a, S')$ with $S'$
        defined as in the construction, we follow an analogous reasoning.

        If the last transition is a push transition $(S, S', T)$ with $T$ defined as
        in the construction, consider any choice of 
        $(q_0, \gamma_0, p_0) \in T_0$,
        $(p_0, \gamma_1, p_1) \in T_1$, ..., $(p_{m-1}, \gamma_m, p_m) \in T_m$,
        $(p_m, \gamma_{m+1}, p_{m+1}) \in T_{m+1}$ and $(p_{m+1}, q') \in S'$. We
        know that we must have $q' = p_{m+1}$, 
        %
        and that there must be
        some state $p''$ and push transition $(p'', p_{m+1}, \gamma_{m+1})$ and
        pair $(p_m, p'')$ in $S$. Using the induction hypothesis but picking
        $(p_m, p'') \in S$, we obtain a run $\rho$ of~$\cP$ on $w$ which produces $\mu$ with topmost stack symbol $\gamma_m$, ending at
        state $p''$, which we can extend with the push transition $(p'',
        p_{m+1}, \gamma_{m+1})$ to obtain the desired stack and reach state
        $p_{m+1} = q'$, proving the result.

        If the last transition is a pop transition $(S, T, S')$ with $S'$ defined
        as in the construction, consider any choice of 
        $(q_0, \gamma_0, p_0) \in T_0$,
        $(p_0, \gamma_1, p_1) \in T_1$, ..., $(p_{m-2}, \gamma_{m-1}, p_{m-1})
        \in T_{m-1}$, and $(p_{m-1}, q) \in S'$. We know that there is a pair
        $(p', q') \in S$ and a triple $(p_{m-1}, \gamma_m, p')$ in $T_m$ and a
        pop transition $(q', \gamma_m, q)$ in $\Delta$. Applying the induction
        hypothesis, we get a run $\rho$ of $\cP$ on $w$ which produces $\mu$ and with topmost stack symbol $\gamma_m$ annotated with state
        $p'$ and ending at state $q'$. The pop transition $(q', \gamma_m, q)$
        allows us to extend this run to reach state~$q$ and remove the topmost
        stack symbol, while the rest of the stack is correct, proving the
        result.

        Again, we have ensured that $\rho$ is extended to $\rho_+$ with the same
        transition as the transition used to extend $\rho'$ to~$\rho_+'$,
        ensuring that $\rho_+$ and $\rho_+'$ have same profile.
        This concludes the proof of the backward induction, ensuring that
        if $\cP'$ has a run from $S_I$ to some final state $S$ reading a string $w$ and producing $\mu$, and having $(q_0, q_{\mathrm{f}})$
        with $q_{\mathrm{f}} \in
        F$ in $S$, then $\cP$ has a run reading $w$ which produces $\mu$ going from $q_0$ to the final state $q_{\mathrm{f}}$. This concludes the
        backward implication and completes the proof.
\end{proof}

%
%
%
%
%
%
%
%
%
%
%
%
%
%
%
We can now show Theorem~\ref{gram:thm:profileu-iou} via
Proposition~\ref{gram:prop:grammar-pdann}, using also the notion of \emph{profiled
PDAnn} defined in Section~\ref{gram:sec:linear}:

  Let $\cG$ be a rigid annotated grammar. Using Proposition~\ref{gram:prop:grammar-pdann}, we
  transform it in polynomial time to a profiled PDAnn $\cP$.
  Using Lemma~\ref{gram:lem:pdtonechoice}, we
  build in exponential time an equivalent PDAnn $\cP'$ satisfying the conditions of the lemma.

  We know that $\cP'$ is still profiled. Indeed, if we assume by
  contradiction that there is a string~$w$ on which $\cP'$ has two accepting runs
  with different profiles, then by the last condition of
  Lemma~\ref{gram:lem:pdtonechoice}, the same is true of~$\cP$, contradicting the
  fact that $\cP$ is profiled.

  Now, we claim that $\cP'$ is necessarily also unambiguous. To see why,
  consider two accepting runs~$\rho$ and~$\rho'$ of~$\cP'$ on some string $w$. Since
  $\cP'$ is profiled, $\rho$ and $\rho'$ must have the same
  profile. But now, the conditions of Lemma~\ref{gram:lem:pdtonechoice} ensure that,
  knowing the input string $w$ and profile, the runs $\rho$ and $\rho'$ are completely
  determined. Specifically, this is an immediate
  induction on the run. The base case is that there is only one initial state,
  so both $\rho$ and $\rho'$ must have the same initial state. Now, assuming by
  induction that the runs so far are identical and have the same stack, there
  are three cases. First, if the profile tells us that both runs make a push
  transition, the symbol pushed and state reached are determined by the last
  states of the runs so
  far, which are identical by inductive hypothesis. Second, if the profile tells us that both runs make
  a read-write transition (or read transition), the state reached is determined by the input and output
  symbols (or just the input symbol), and by the last states of the run so far, which are identical by
  inductive hypothesis. Third, if the profile tells
  us that both runs make a pop transition, the state reaches is determined by
  the last state of the run so far, and the topmost symbol of the stack, which
  are identical by inductive hypothesis. This concludes the inductive proof.

  Thus, for any two accepting runs $\rho$ and $\rho'$ on the string~$w$ which produce the same output, they must identical. Thus, $\cP'$ is unambiguous.
  We use Theorem~\ref{gram:prop:grammar-pdann} to transform $\cP'$ back into an
  annotated grammar, which is still rigid and unambiguous, and equivalent to the
  original rigid annotated grammar $\cG$. The overall complexity of the
  transformation is in $\cO((2^{(|\cG|^c)^{c'}})^{c''})$ for some $c, c', c'' > 0$,
  so it is in $\cO(2^{|\cG|^d})$ for some $d>0$ overall, and the time complexity
  is as stated.

\end{proof}

% The transformation to impose unambiguity goes via a notion of annotated pushdown
% automata (introduced in the next section), and is inspired by the
% determinization procedure for visibly pushdown automata~\cite{alur2004visibly}, even though rigid
% grammars generally do not define visibly pushdown languages. The transformation
% comes at a cost, as it will generally blow up the size of the grammar
% exponentially.



\paragraph{Expressiveness of rigid grammars}
Armed with Theorem~\ref{gram:thm:profileu-iou}, we study what is the expressive power
of rigid grammars. For this, let us first go back to the setting without
annotations. Theorem~\ref{gram:thm:profileu-iou} tells us that for (unannotated) CFGs the rigidity
requirement is equivalent to the usual unambiguity requirement: each accepted
word has a unique derivation.
Now, for the case of an annotated grammar $\cG$,
rigidity additionally imposes the requirement
that all annotations of an input string have the same parse tree.
In particular, the language of the strings where $\cG$ accepts some
annotation must be recognizable by a rigid (unannotated) CFG, hence an unambiguous CFG (by
Theorem~\ref{gram:thm:profileu-iou}). Formally:

\begin{proposition}
  \label{gram:prp:baseunambig}
   For a rigid grammar $\cG$, let $L'$ be the set of strings with nonempty output, i.e., $L' = \{w \mid \sem{\cG}(w) \not = \emptyset\} $. Then $L'$ is
  recognized by an unambiguous CFG.
\end{proposition}
\begin{proof}
  This proof is based on extending the definitions of unambiguity and rigidness of annotated grammars over unannotated context-free grammars. Indeed, an unambiguous annotated grammar with an empty output set is just an unambiguous CFG, and a rigid annotated grammar with an empty output set is a CFG for which every derivation of a given string $w\in \Sigma^*$ has the same shape.

  Consider the (unannotated) grammar $\cG'$ obtained from~$\cG$ by removing all
  annotations on terminals, and making $\Omega = \emptyset$. It can be seen that $L(\cG') = L'$ since for each string $w$, if $w\in L(\cG')$, then there is at least one $\hat{w}\in L(\cG)$ with $\str(\hat{w}) = w$ and vice versa. Now, we claim that
  $\cG'$ is \emph{rigid}, by extending the notion onto CFGs in the obvious way. To see this, consider a string $w \in L(\cG')$; all
  derivations of~$w$ by~$\cG'$ correspond to derivations by $\cG$ of some $\hat{w}$ such that $\str(\hat{w}) =w $. Because $\cG$ is rigid, all these derivations have the
  same shape. Now, using Theorem~\ref{gram:thm:profileu-iou}, we can compute a rigid and
  unambiguous grammar $\cG''$ recognizing the same language over~$\Sigma^*$
  as~$\cG'$, i.e., $L'$. But as $L'$ is a language without output, the
  unambiguity of~$\cG''$ actually means that $\cG''$ is an unambiguous CFG. Hence, $L'$
  is recognized by an unambiguous grammar, concluding the proof.

\end{proof}

This yields concrete examples of languages (on the empty annotation alphabet) that cannot be recognized by a
rigid annotated grammar, e.g., inherently ambiguous context-free languages
such as $L_\mathrm{a} = \{a^i b^j c^k \mid i, j, k \geq 1 \land (i = j \lor j =
k)\}$ on~$\{a, b, c\}^*$~\cite{maurer1969direct}.
%
%
%
%
Proposition~\ref{gram:prp:baseunambig} also implies that we cannot decide if the language
of an annotated grammar can be expressed instead by a rigid grammar,
or if an annotated grammar is rigid:

\begin{proposition}
  \label{gram:prp:unambundec2}
  Given an unannotated grammar $\cG$, it is undecidable to determine whether $\cG$
  is rigid, and it is undecidable to determine whether there
  is some equivalent rigid grammar $\cG'$.
\end{proposition}
\begin{proof}
  We first show the undecidability of checking if an annotated grammar has an
equivalent rigid annotated grammar:

\begin{claim}
  \label{gram:prp:unambundec}
  Consider the problem, given an annotated grammar~$\cG$, of determining whether
  there exists some equivalent rigid annotated grammar equivalent to~$\cG$.
  This problem is undecidable.
\end{claim}

\begin{proof}
  We reduce from the problem of deciding whether the language $L_2$ of an input
  (unannotated) context-free grammar $\cG_2$ can be recognized by an unambiguous
  context-free grammar: this task is known to be
  undecidable~\cite{ginsburg1966ambiguity}. Consider $\cG_2$ as an annotated
  grammar (with empty annotations). Let us show that $L_2$ can be recognized by
  a rigid annotated grammar iff it can be recognized by an
  unambiguous context-free grammar, which concludes. For the forward direction,
  if $L_2$ can be recognized
  by an unambiguous context-free grammar, then that grammar is in particular
  rigid. 
  For the backward direction, if $L_2$ can be recognized by a rigid
  grammar, then
  Proposition~\ref{gram:prp:baseunambig} implies that $L_2$ can also be recognized by
  an unambiguous context-free grammar.
  Thus, we have showed that the (trivial) reduction is correct.
\end{proof}

We next show that it is undecidable to check if an input annotated grammar is
rigid:

\begin{claim}
  \label{gram:prp:unambundecb}
  Consider the problem, given an annotated grammar~$\cG$, of determining whether
  it is rigid. This problem is undecidable.
\end{claim}


\begin{proof}
  We adapt the standard proof of undecidability~\cite[Ambiguity Theorem
  2]{chomsky1959algebraic} for the problem of deciding, given an input
  unannotated grammar~$\cG$, if it is unambiguous. The reduction is from the Post
  Correspondence Problem (PCP), which is undecidable: we are given as input sequences $\alpha_1, \ldots, \alpha_n$
  and $\beta_1, \ldots, \beta_n$ of strings over some alphabet~$\Sigma$, and we ask whether
  there is a non-empty sequence of indices $i_1, \ldots, i_m$ of integers in $[1, n]$ such that
  $\alpha_{i_1} \ldots \alpha_{i_m} = \beta_{i_1} \ldots \beta_{i_m}$. Given the
  input sequences $\alpha_1, \ldots, \alpha_n$ and $\beta_1, \ldots, \beta_n$ to
  the PCP, we
  consider the alphabet $\Sigma' = \Sigma \cup \{1, \ldots, n\}$, and we
  consider the CFG having nonterminals $S$, $S_1$, and $S_2$, start symbol $S$,
  and rules $S \rightarrow S_1$, $S \rightarrow S_2$, $S_1 \rightarrow
  \epsilon$, $S_2\rightarrow \epsilon$, and for each $1 \leq
  i \leq n$ the productions $S_1 \rightarrow \alpha_i S_1 i$ and $S_2
  \rightarrow \beta_i S_2 i$.

  We claim that this grammar is ambiguous iff there is a solution to the Post
  correspondence problem. Indeed, given any solution $\alpha_{i_1} \cdots
  \alpha_{i_m} = \beta_{i_1} \ldots \beta_{i_m}$, considering the string 
$\alpha_{i_1} \cdots
  \alpha_{i_m} i_m \cdots i_1 = \beta_{i_1} \ldots \beta_{i_m} i_m \cdots i_1$,
  we can parse it with one derivation  featuring $S_1$ and one derivation
  featuring $S_2$. Conversely, if we can parse a string $w \in \Sigma^*$ with two
  different derivations, we know that there cannot be two different derivations
  featuring $S_1$. Indeed, reading the string from right to left uniquely
  identifies the possible derivations from $S_1$. The same argument applies to
  derivations featuring $S_2$. Hence, if the grammar is ambiguous, then there is
  exactly one derivation featuring $S_1$ and exactly one derivation featuring
  $S_2$. These two derivations can be used to find a solution to the Post
  correspondence problem.

  We now adapt this proof to show the undecidability of rigidity. We
  say that an input to the PCP is \emph{trivial} if there is $i$ such that
  $\alpha_i = \beta_i$. We can clearly decide in linear time, given the input to
  the PCP, if it is trivial. Hence, the PCP is also undecidable in the case
  where the PCP is non-trivial. Now, when doing the reduction above on a PCP
  instance that is not trivial, we observe that two derivations of the same string
  can never have the same sequences of shapes. Indeed, if we have two derivations of the same
  string, then as we explained one must feature $S_1$ and the other must feature
  $S_2$, and they give a solution $\alpha_{i_1} \cdots \alpha_{i_m} =
  \beta_{i_1} \ldots \beta_{i_m}$ to the PCP. Assume by contradiction that both
  derivations have the same sequences of shapes. Then, 
  it means that we have $\card{\alpha_{i_j}} = \card{\beta_{i_j}}$ for every $1
  \leq j \leq m$. In particular we have $\card{\alpha_{i_1}} =
  \card{\beta_{i_1}}$, and so we know that $\alpha_{i_1} =
  \beta_{i_1}$ and the PCP instance was trivial, a contradiction.

  Hence, let us reduce from the PCP on non-trivial instances to the problem of
  deciding whether an input annotated grammar is not rigid. Given a
  non-trivial PCP instance, we construct $\cG$ as above, but seeing it as an
  annotated grammar with no outputs. Then $\cG$ is not rigid iff there is a
  string $w$ such that the empty annotation of $w$ has two derivations that do not
  have the same sequence of shapes. But this is equivalent to $\cG$ being
  unambiguous when seen as a CFG. Indeed, for the forward direction, if $\cG$ has
  two such derivations on a string~$w$ then clearly $w$ witnesses that $\cG$ is
  ambiguous when seen as a CFG. Conversely, if $\cG$ is ambiguous when seen as a
  CFG, we have explained in the previous paragraph that the two derivations must
  have different sequences of shapes, so $\cG$ is not rigid. Hence, we
  conclude that there is a solution to the input non-trivial PCP instance iff
  $\cG$ is not rigid. This establishes that the problem is undecidable
  and concludes the proof.
\end{proof}
The proof follows from Claims~\ref{gram:prp:unambundec} and~\ref{gram:prp:unambundecb}.
\end{proof}

These undecidability results make 
rigid grammars less appealing, but note that our enumeration algorithm for such
grammars applies in particular to decidable grammar classes which are
designed to ensure rigidity. For instance, this would be the case
of grammars arising from visibly pushdown automata, which we discuss in more
detail in the next section.

%
%
%
%
%
%
%
%
%
%
%
%
%
%
%

