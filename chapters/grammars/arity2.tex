Having presented enumerable sets, we now present the normal form to
enforce on annotated grammars. Our results could be shown using the
commonly known Chomsky normal form (CNF), but we cannot always obtain
an equivalent CNF of linear size from a grammar. For this
reason, we use a variant of CNF, the
\emph{arity-two normal
form} (2NF)~\cite{lange2009cnf}, which is intuitively like CNF but without
disallowing rules of the form $X \rightarrow Y$ or $X \rightarrow \epsilon$. Formally,
  we say that an annotated grammar $(V, \Sigma, \Omega, P, S)$ is in \emph{arity-two normal
  form} (2NF) if the following hold:
  \begin{itemize}
%
%
\item Every nonterminal $X$ can derive some
string, i.e., there exists $\hat{w}\in (\Sigma \cup \Sigma\times \Omega)^*$
such that $X \ders{\cG} \hat{w}$.
\item Every nonterminal $X$ can be
reached from the start symbol $S$, i.e., there exists $\alpha,\beta \in  (V \cup
\Sigma \cup \Sigma \times \Omega)^*$ such that $S \ders{\cG}
\alpha X \beta$.
%
%
%
%
     \item For every rule $X \rightarrow \alpha$ in~$P$, we have $|\alpha| \leq
       2$, and if $\alpha = 2$ then it consists of two nonterminals.
\end{itemize}

We can easily translate annotated grammars to 2NF, as in~\cite{lange2009cnf}:

\begin{proposition}[\cite{lange2009cnf}]
  \label{gram:prp:2nf}
  Given any annotated grammar\/ $\mathcal{G}$, we can compute in linear time an annotated
  grammar $\mathcal{G}'$ in 2NF such that $\mathcal{G}$ and $\mathcal{G}'$ are
  equivalent. Furthermore, if $\mathcal{G}$ is unambiguous then $\mathcal{G}'$ is unambiguous as well.
\end{proposition}
We omit a proof of this statement here since it follows from the reference.
However, we will prove a stronger version of it in the following sections when further restrictions for grammars are introduced.

By Proposition~\ref{gram:prp:2nf}, we assume that the 
input grammar~$\cG$ is in 2NF.

We also compute in linear time some more information about $\cG$. First, we precompute which
nonterminals are \emph{nullable}, i.e., are such that $X \ders{\cG} \epsilon$: if we have a rule
$X \rightarrow \epsilon$ then $X$ is nullable, and if we have a
rule $X \rightarrow Y$ where $Y$ is nullable or $X \rightarrow YZ$ where $Y$ and
$Z$ are nullable then $X$ is also nullable. From this information, we further
compute for each nonterminal $Y$ a set $\D[Y]$ of all
nonterminals $X$ such that one of the following rules exist: a rule
$X\rightarrow Y$, a rule $X\rightarrow YZ$ where $Z$ is nullable, or a rule $X\rightarrow ZY$ where $Z$ is
nullable. We can clearly compute all of this in linear time (note that each rule contributes at most two
entries to~$\D$).

Second, as the grammar is assumed to be unambiguous, it also contains
no \emph{cycles}, %
i.e., there are no sequence of nonterminals
$X_1 \dots X_n$ such that for $1\leq i < n$, $X_i\in \D[X_{i+1}]$ and
$X_1\in \D[X_n]$. Indeed, otherwise 
%
there would be infinitely many possible derivations of some string
starting at~$X_1$, contradicting the unambiguity of~$\cG$.  Thus, we can sort the nonterminals of $\cG$
in \emph{topological order}, by which we mean that when $Y\in \D[X]$
then $Y$ is enumerated after $X$. Intuitively, when we consider a nonterminal
$X$, we want to be done with processing the nonterminals $Y$ such that $X \rightarrow
Y$ or $X \rightarrow YZ$ or $X \rightarrow ZY$ with $Z$ nullable. This order can also be computed in linear time.
%

