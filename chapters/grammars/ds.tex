The preprocessing phase of our enumeration algorithm builds data structures
representing the set of outputs to enumerate. For this, we essentially re-use
the $\epsilon$-ECS data structure of Chapter~\ref{ch1}, but for convenience
we present them in a self-contained way for our context, and
name them \emph{enumerable sets}.
We now define them and state that we can enumerate their contents
with output-linear delay (Theorem~\ref{gram:thm:enum}).
The enumeration phase of our algorithm simply enumerates the outputs of 
an enumerable set using this delay guarantee.

An \emph{enumerable set}  is a representation of a set of
strings over some alphabet $\O$.  For our case, we want strings of
$\O^*$ to describe outputs, so $\O$ consists of pairs of
%
%
annotations
with positions of the input
string~$w$, i.e.,
$\O \colonequals \Omega \times \{1, \ldots, |w|\}$.
%

The basic enumerable sets are:
\begin{itemize}
  \item $\textsf{empty}$, the empty set;
  \item $\singleton(\epsilon)$, the singleton set containing the empty string;
  \item $\singleton(x)$ for
$x \in \O$, the singleton set with the
    single-character string~$x$.
\end{itemize}

Enumerable sets can be combined using operators to form more complex
enumerable sets. The operators that we consider all take constant-time
and are {\em
fully-persistent}~\cite{driscoll1986making}. Specifically, given
enumerable sets $\dsD_1$ and $\dsD_2$, combining them creates an
enumerable set without modifying $\dsD_1$ and $\dsD_2$ (i.e., they can still be
used in other operator applications). To make this
possible, enumerable sets can share some components, e.g., some parts
of the arguments $\dsD_1$ and $\dsD_2$ can be shared in memory, and
the result can also have some parts that are shared with
$\dsD_1$ and $\dsD_2$. This is similar, e.g., to persistent lists,
where we can only extend a list by adding an element to its head: this
does not modify the original list, and returns a new
list sharing some memory with the original list.
%
%
%

The two operators to combine enumerable sets are:

\begin{itemize}
  \item The \emph{union} operator $\textsf{union}(\dsD_1, \dsD_2)$ can
    be applied if the sets represented by $\dsD_1$ and $\dsD_2$ are
    disjoint, intuitively to avoid duplicates. It returns an enumerable set representing the
    union of these sets,
    %
    %
    %
    
  \item The \emph{product} operator $\textsf{prod}(\dsD_1, \dsD_2)$
    can be applied if there are no common letters in the strings of
    the sets represented by $\dsD_1$ and $\dsD_2$, i.e., if $\dsD_1$
    (resp. $\dsD_2$) represents $S_1$ (resp. $S_2$) then the sets
    $\{x_1 \in \O \mid x_1 \text{~occurs~in~some~} w_1 \in S_1\}$ and
    $\{x_2 \in \O \mid x_2 \text{~occurs~in~some~} w_2 \in S_2\}$ are
    disjoint.  Then, the operation returns an enumerable set $\dsD$
    which represents the concatenations of the strings in~$\dsD_1$ and
    in~$\dsD_2$: formally, it represents $S_1 \cdot S_2 = \{w_1 \cdot w_2 \mid
    w_1 \in S_1, w_2 \in S_2\}$.
  %
  %
  %
\end{itemize}

%
%

It is known that enumerable sets can be enumerated efficiently:

\begin{theorem}
  \label{gram:thm:enum}
  We can implement enumerable sets such that:
\begin{itemize} \item The enumerable
        sets $\empt$, $\singleton(\epsilon)$, and $\singleton(x)$ for $x \in \O$ can be built in
  constant time;
\item The union and product
  operations can be implemented in constant time and in a fully
  persistent way;
\item Given an enumerable set,  we can enumerate the strings
      it represents with output-linear delay and memory usage linear
  in the number of instructions used to build it.
\end{itemize}
\end{theorem}

We omit a formal proof given that it follows directly from Theorem~\ref{nested:theo:data-structure-eps}.

