We now present the preprocessing phase of the enumeration algorithm, formalized as
Algorithm~\ref{gram:alg:cubic} where the  input string $w=a_1\ldots a_n$ is
assumed nonempty. 

The principle of the algorithm is the following:
\begin{principle}
  \label{gram:pri:main}
    For every triple
of the form $(i,j,X)$ with $1\leq i < j \leq n+1$ and $X\in N$, the
table cell $\tableI[i][j][X]$ will contain an enumerable set
representing the annotations of the string $a_i \cdots a_{j-1}$ that
can be derived from symbol $X$ in the grammar.
\end{principle}
%
These sets are initialized to be empty.
In lines~\ref{gram:ln:initstart}--\ref{gram:ln:initend} of the algorithm, the cells $\tableI[i][j][X]$ with
$j-i=1$ are initialized
to consider derivations via ``simple rules'' of the form $X \rightarrow a$ or
$X \rightarrow (a, \oout)$.
(For now, ignore the role of the
$\mathsf{endIn}$ table.) Note that the rules of the form $X \rightarrow
\epsilon$ are considered when defining~$\D$ and not further examined by the algorithm. At the end, line~\ref{gram:ln:last} returns the
enumerable set for the annotations of the entire string derivable from
the start symbol, i.e., the outputs of~$\cG$ on~$w$.

%
%
%
%

The main part of the algorithm consists in satisfying Principle~\ref{gram:pri:main} by adding
the annotations corresponding to ``complex'' rules (i.e., of the form
$X\rightarrow YZ$ or $X\rightarrow Y$). At the beginning of the
algorithm the cells of the table $\tableI$ might lack some annotations
corresponding to complex rules, but each cell will be considered
complete at some point during the execution, at which point it will
satisfy Principle~\ref{gram:pri:main} and will not be modified anymore.
%
We define the order in which the cells are considered complete as follows:
$(i,j,X)<(i',j',X')$ when $j<j'$ or
$(j=j' \land i>i')$ or $(j=j' \land i=i' \land X < X')$ where we order
nonterminals $X$ and $X'$ following the topological order from $\D$.

Consider the \emph{complex derivations} starting from $X$ of the string $a_i \cdots
a_{j-1}$, i.e., those that begin with a complex rule. We will see here how to
reflect them in $\tableI[i][j][X]$. There are two kinds of complex
derivations. The first kind is the derivations where we first rewrite $X$
to another nonterminal $Z$ with a rule $X \rightarrow Z$, or by
rewriting $X$ to $YZ$ or $ZY$ but where $Y$ is nullable and will be
rewritten to~$\epsilon$. In these three cases, we have
$X \in \D[Z]$. Thus, we fill the index $\tableI[i][j][X]$ with the
contents of $\tableI[i][j][Z]$, which is already complete, for
$X\in \D[Z]$ (lines \ref{gram:ln:directProdBegin}-\ref{gram:ln:directProdEnd}).
%
%

The second kind of complex derivation begins with a complex rule $X
\rightarrow YZ$ where neither $Y$ nor $Z$ will be rewritten to $\epsilon$. In this case, the set of
annotations to add into $\tableI[i][j][X]$ using this rule is the union of
products of all the $\tableI[i][k][Y]$ and $\tableI[k][j][Z]$ where $i < k <
j$. We have $(i,k,Y)<(k,j,Z)<(i,j,X)$, so we can fill $\tableI[i][j][X]$ with the product of the contents of $\tableI[i][k][Y]$ and $\tableI[k][j][Z]$,
at the moment where $\tableI[k][j][Z]$ is considered complete.
%
%
%
%

To summarize, from line~\ref{gram:ln:calc} onwards, the algorithm
considers the positions $j$ in ascending order, and 
populates all cells $\tableI[i][j][X]$ so that they are complete. To do
so, we consider the triples $(k,j,Z)$ by increasing order in our
sorting criterion, i.e., by decreasing $k$, then increasing~$Z$ in the
order of the topological sort. Whenever we consider a cell, it is
complete, and we consider its contributions to cells of the form
$\tableI[i][j][X]$ with $i=k$ using complex rules of the first kind 
(lines \ref{gram:ln:directProdBegin}-\ref{gram:ln:directProdEnd}), and if it is non-empty
we consider how to combine it with
a neighboring cell (which is also complete and non-empty) as we explained previously,
adding the results to a cell $\tableI[i][j][X]$ with $i < k$ which is not yet complete (lines
\ref{gram:ln:kind2begin}--\ref{gram:ln:kind2end}).

We now explain the optimization involving the set $\mathsf{endIn}$. It
is not necessary to achieve the cubic running time of
this section, but is required for the quadratic bound in
Section~\ref{gram:sec:quadratic}.  The optimization is that, when processing the
triple $(k,j,Z)$ and the rule $X\rightarrow YZ$, we do not test all the
possible cells $\tableI[i][k][Y]$, but only those that are
non-empty. Indeed, if $\tableI[i][k][Y]$ is empty, then the concatenation of
$\tableI[i][k][Y]$ with $\tableI[k][j][Z]$ is also empty. Thus, we maintain
the list $\mathsf{endIn}[k][Y]$ of all the~$i$'s to consider
with $i<k$, i.e., those such that $\tableI[i][k][Y]$ is non-empty. We
initialize this list to be empty, add $i$ to $\mathsf{endIn}[k][Y]$ whenever
$\tableI[i][k][Y]$ becomes non-empty (at line~\ref{gram:ln:endset1} in the base
case, or at line~\ref{gram:ln:endset2} before adding to an empty cell for
the first time). Then, we only consider the indices
$i$ of this list to combine $\tableI[i][k][Y]$ with another cell.

We now argue that our algorithm is correct, and in particular
that 
\textbf{(i)} 
%
we satisfy Principle~\ref{gram:pri:main};
that
\textbf{(ii)} all the unions are disjoint, and that \textbf{(iii)} all the
products involve enumerable sets on disjoint alphabets. 
One can establish \textbf{(i)} by showing by induction over cells that the invariant is
correct when each cell is considered complete by our algorithm (and the
cell is not changed afterwards). Knowing \textbf{(i)}, the first violation of \textbf{(ii)} would witness that
the same annotation of some factor $a_i \cdots a_{j-1}$ can be derived in two
different ways from a nonterminal $X$,
contradicting unambiguity, so there are no
violations of \textbf{(ii)}. For
\textbf{(iii)}, we simply observe that, by \textbf{(i)}, $\tableI[i][j][X]$
only contains pairs of the form $(\oout, k)$ for some $i \leq k < j$, so we can
indeed perform the product of $\tableI[i][k][Y]$ and $\tableI[k][j][Z]$.

This establishes that the algorithm is correct. Now, the running time
of the preprocessing phase of the algorithm is clearly in $\cO(n^3 |\cG|)$, because (1) the $\mathsf{endIn}$ lists
are of size $\cO(n)$ at most, and (2) the consideration of all $Z \in N$ and $X \in \D[Z]$
is in $\cO(|\cG|)$: every $X \in \D[Z]$ corresponds to a rule, so the consideration of all $Z \in N$ and
rules in CRule$[Z]$ is in $\cO(|\cG|)$.
%
The enumeration phase is then simply that of Theorem~\ref{gram:thm:enum}. Hence, we have
shown that enumeration for unambiguous annotated grammars can be achieved
with cubic time preprocessing and output-linear delay:

\begin{theorem}
  \label{gram:thm:cubic}
	Given an unambiguous annotated grammar $\cG$ and an input string
        $w$, we can enumerate $\sem{\cG}(w)$ with preprocessing in $\cO(|w|^3
        \cdot
        |\cG|)$ (hence cubic in data complexity), and output-linear delay
        (independent from~$w$ or~$\cG$). The memory usage is in $\cO(|w|^3 \cdot
        |\cG|)$.
        %
\end{theorem}	



\newcommand{\To}{\textbf{to}}
\newcommand{\Downto}{\textbf{downto}}

\begin{algorithm}[t]
	\caption{Preprocessing phase: given a 2NF
        unambiguous annotated grammar $\cG = (N, \Sigma, \Omega, P, S)$ and
        a non-empty string $w = a_1\cdots a_n$, compute an enumerable set
        representing $\sem{\cA}(w)$.}\label{gram:alg:preprocessing}
        \label{gram:alg:cubic}
		\begin{algorithmic}[1]
                  \State I $\gets$ an array $(n+1)\times (n+1)\times N$ initialized with $\empt$
                  \State  $\mathsf{endIn}\gets$ an array $(n+1)\times N$
                  initialized with empty lists
                  \State CRule $\gets$ an array such that CRule$[Z]=\{X\rightarrow YZ \in P \}$
                  \State $\D \gets $ an array as described in the presentation of 2NF

                  \For{$1 \leq i \leq n$} \label{gram:ln:initstart} 
                    \If{rule $(X \rightarrow a_i)$ in $P$}
                      \State $\tableI[i][i+1][X] \gets \union(\tableI[i][i+1][X], \singleton(\epsilon))$
                    \EndIf
                    \For{rule $(X \rightarrow (a_i, \oout))$ in $P$}
                      \State $\tableI[i][i+1][X] \gets \union(\tableI[i][i+1][X], \singleton((\oout, i)))$
                    \EndFor
                    \If{$\tableI[i][i+1][X] \neq \empt$}
                      \State $\mathsf{endIn}[i+1][X]$.append($i$)
                      \label{gram:ln:endset1}
                    \EndIf
                  \EndFor \label{gram:ln:initend} 



                  \For{$j=1$ \To{} $n+1$} \label{gram:ln:calc} 
                    \For{$k=j-1$ \Downto{} $1$}
                      \For{nonterminal $Z \in N$ in topological order}
                          \For{nonterminal $X\in \D[Z]$} \label{gram:ln:directProdBegin}                             
                            \State $\tableI[k][j][X] \gets \union(\tableI[k][j][X],
                            \tableI[k][j][Z])$ \label{gram:ln:directProdEnd}
                          \EndFor
                        \If{$\tableI[k][j][Z] \neq \empt $}
                        \label{gram:ln:kind2begin}                             
                            \For{rule $(X \rightarrow YZ)$ in CRule$[Z]$}
                              \For{$i \in\mathsf{endIn}[k][Y]$}
                                  \label{gram:ln:innermost}
                                \If{$\tableI[i][j][X]=\empt$}
                                  \label{gram:ln:innermost3}
                                  \State $\mathsf{endIn}[j][X]$.append($i$)
                                  \label{gram:ln:endset2}
                                \EndIf
                                  \label{gram:ln:innermost2}
                                \State $\tableI[i][j][X] \gets \union(\tableI[i][j][X],$
                                \State $\mbox{~}\hspace{3em} \prod(\tableI[i][k][Y], \tableI[k][j][Z]))$
                               \EndFor
                            \EndFor
                        \label{gram:ln:kind2end}                             
                         \EndIf
                    \EndFor
                  \EndFor
                  \EndFor
                  \State\Return $\tableI[1][n+1][S]$ \label{gram:ln:last}
\end{algorithmic}%
\end{algorithm}%
