
However, while regular spanners are natural, they do not capture all
possible information extraction tasks. More
expressiveness is needed for extraction over structured data (e.g., XML,
or JSON documents), over the source code of programs, or possibly over
natural language texts. We believe that a natural way to address such
tasks is to move from finite automata to \emph{context-free grammars}
(CFGs). Context-free grammars are a well-known formalism: they extend regular
expressions and are commonly used, e.g., in programming language
design. Common verification tasks on textual representations of tree
documents can be expressed using CFGs, and so can parsing tasks, e.g.,
to extract subexpressions from source code data. However, CFGs do not
describe \emph{captures}, i.e., they do not specify how to extract the
parts of interest of an input document, and thus cannot be used
directly for information extraction.

%
%

This question of information extraction with grammars was studied by Peterfreund in very recent work~\cite{Peterfreund21}. This paper proposed a formalism of \emph{extraction grammars}, which are CFGs extended via special terminals that describe the endpoints of spans.
Further, it presents an algorithm to enumerate the mappings captured by
\emph{unambiguous} extraction grammars on an input document.
However, while the algorithm achieves constant-delay, the preprocessing bound is significantly worse than in the case of regular spanners: it is quintic in the document, and exponential in the number of variables of the grammar. This complexity is also worse than CFG parsing, e.g., the standard CYK parsing algorithm runs in cubic time in the input string.

%
%
%

Our goal in this chapter is to study the enumeration problem for CFGs while
achieving better complexities. Our algorithms ensure a constant-delay
guarantee when outputs have constant size, and more generally ensure
\emph{output-linear} delay when this is not the case: the delay is linear in the
size of each produced solution. Within this delay bound, the preprocessing
time has lower complexity: it is at worse cubic in the
input document, and improves to quadratic or even linear time for restricted
classes. We achieve these results by proposing a new formalism to extend CFGs, called
\emph{annotated grammars}, on which we impose an unambiguity restriction similar
to that of~\cite{Peterfreund21}. Let us present our specific contributions.

    \paragraph{Contributions}
    Our first contribution is to introduce \emph{annotated grammars} (Section~\ref{gram:sec:models}).
    They are a natural extension of CFGs, where terminals are optionally annotated by the information that we wish to extract.
    We then study the problem, given an annotated grammar $\cG$ and document~$s$, of
    enumerating all annotations of~$s$ that are derived by~$\cG$.
This captures the enumeration problems for regular spanners~\cite{FlorenzanoRUVV18,amarilli2020constant}, nested words,
%~\cite{icdt2020nested}
and
    even the extraction grammars of~\cite{Peterfreund21} (we explain this in Section~\ref{gram:sec:spanners}). 
    %
    %
    As we explain, we aim for \emph{output-linear delay}, which is the best
    possible delay in our setting where the solutions to output may have
    non-constant size. 

    Our second contribution is to study the enumeration problem for \emph{unambiguous} annotated grammars (Section~\ref{gram:sec:cubic}),
    that do not produce multiple times the same annotation of an input string. This is a natural restriction to avoid duplicate results, %
    which is also made in~\cite{Peterfreund21}. 
    %
  For such grammars, 
%
    we present an algorithm to enumerate the annotations produced by a grammar $\cG$ on a string~$s$ with output-linear delay (independent from~$\cG$ or~$s$), after a preprocessing time of $\cO(|\cG| \cdot |s|^3)$, i.e., cubic time in~$s$, and linear time in~$\cG$. 
    %
    This improves over the result of~\cite{Peterfreund21} whose preprocessing is quintic. 
    Our algorithm has a modular design: it follows a standard design of a CFG
    parsing algorithm, but uses the abstract data structure
    of Chapter~ref{ch1}
    %
    to represent
    the sets of annotations and combine them with operators in a way that allows for output-linear enumeration. 
    We further show a conditional lower bound on the best preprocessing time that can achieve output-linear delay, by reducing from the standard task of checking membership to a CFG, and using the lower bound of~\cite{AbboudBW18}. We show that the preprocessing time must be $\Omega(|s|^{\omega-c})$ for every $c > 0$, where $\omega$ is
    %
    the Boolean matrix multiplication exponent.

    Our third contribution is to improve the preprocessing time by imposing a different requirement on grammars. Thus, we introduce \emph{rigid} annotated grammars (Section~\ref{gram:sec:quadratic}) where, for every input string, all annotations on the string are intuitively produced by parse trees that have the same shape. In contrast with general annotated grammars, we show that rigid annotated grammars can always be made unambiguous, so that our algorithm applies to them. But we also show that, under this restriction, the data complexity of our algorithm goes down from cubic to quadratic time.
    %
    Further, achieving sub-quadratic preprocessing time would imply a sub-quadratic algorithm to test membership to an unambiguous CFG, which is an open problem.
     %

  %
	%

    Our last contribution shows how we can, in certain cases, achieve linear-time preprocessing complexity and output-linear delay (Section~\ref{gram:sec:linear}). This is the complexity of enumeration for regular spanners, and is by definition the best possible. We show that the same complexity can be achieved, beyond regular spanners, for a subclass of rigid grammars, intuitively defined by a determinism requirement. We define it via the formalism of \emph{pushdown annotators} (PDAnn for short), which are the analogue of pushdown automata for CFGs, or the extraction pushdown automata of~\cite{Peterfreund21}. We show that PDAnn are equally expressive to annotated grammars, and that rigid CFGs correspond to a natural class of PDAnns where all runs have the same sequence of stack heights. Moreover, 
    we show that we can enumerate with linear-time preprocessing and output-linear delay in the case of \emph{profiled-deterministic} PDAnn, where the sequence of stack heights can be computed deterministically over the run: this generalizes regular spanners and visibly-pushdown automata.

%    This paper is the extended version of the work published at
%    PODS'22~\cite{amarilli2022efficientPODS}. It includes complete proofs of
%    all results in the appendix.

\paragraph{Related work}
We have explained how our work is set in the context of document spanners~\cite{FaginKRV15}, and in particular of enumeration results for regular spanners~\cite{FlorenzanoRUVV18,amarilli2019constant}. A recent survey of much of this literature can be found in Peterfreund's PhD thesis~\cite{peterfreund2019complexity}. The most related work to ours is the more recent introduction of extraction grammars by Peterfreund~\cite{Peterfreund21}, which we already discussed.
%Another related work is~\cite{icdt2020nested}, by some authors of the present
%paper. 
%This paper studies enumeration for spanners over nested documents,
%defined as visibly pushdown transducers. 
%

There are also some other extensions of regular spanners that are reminiscent of CFGs, e.g., core spanners (featuring equality) or generalized core spanners (with difference) already introduced in~\cite{FaginKRV15}, or Datalog evaluated over regular spanners as in~\cite{peterfreund2019recursive}. However, to our knowledge, there are no known constant-delay enumeration algorithms in these contexts.

Our study of enumeration for annotated grammars is also reminiscent of
enumeration results for queries over trees expressed as tree automata. An
algorithm for this was given by Bagan~\cite{bagan2006mso} with linear-time
preprocessing and constant-delay in data complexity, for deterministic tree
automata, and this was extended in~\cite{amarilli2019enumeration} to
nondeterministic automata. However, this is again more restricted:
evaluating a tree automaton on a tree amounts to evaluating a \emph{visibly
pushdown} automaton over a string representation of the tree, which is again
more restrictive than general context-free grammars.

%

\paragraph{Comparison to previous chapter}
In this chapter, we re-use the
enumeration data structure of Chapter~\ref{ch1}, and we consider a transducer model in
Section~\ref{gram:sec:linear} that recaptures some of the already presented results. 
However, in our
problem, we do not require visibility guarantees. This poses new technical
challenges: the underlying tree structure (i.e., the parse tree) is not known in
advance and generally not unique.

