%
%

We have presented an enumeration algorithm for annotated grammars that achieves
quadratic-time preprocessing and output linear delay on rigid annotated
grammars. We now study whether the bound can be improved even further to achieve
linear-time preprocessing and output-linear delay, which is the best possible
data complexity bound in our model.

To achieve this, it is natural to look for a class of grammars having some ``deterministic'' behavior.
Unfortunately, grammars are not convenient for this purpose, and so we move to
the equivalent model of pushdown automata.
We thus introduce \emph{pushdown annotators} and show that they are equally expressive to annotated grammars.
We present syntactic restrictions on pushdown annotators that ensure
quadratic-time preprocessing, similarly to rigid grammars.
%
Then, we propose additional deterministic conditions on pushdown annotators that allow for linear-time preprocessing.
%

\paragraph{Pushdown annotators} A \emph{pushdown annotator} (PDAnn) is a tuple
$\cP = (Q, \Sigma, \Omega, \Gamma, \Delta, \allowbreak q_0, F)$ where $Q$ is a finite set of \emph{states},
%
%
$\Sigma$ is the alphabet, $\Omega$ is a finite set of annotations, $\Gamma$ is a finite set of \emph{stack symbols}, $q_0 \in Q$ is the \emph{initial state}, and $F \subseteq Q$ are the \emph{final states}. We assume that the set $\Gamma$ of stack symbols is disjoint from $(\Sigma \cup \Sigma\times \Omega)$. Finally, $\Delta$ is a finite set of \emph{transitions} that are of the following kinds:
%
\begin{itemize}
	\item {\em Read-write transitions} of the form $(p, (a, \oout), q) \in Q\times(\Sigma\times \Omega)\times Q$, meaning that, if the next letter of the string is~$a$, the annotator can go from states~$p$ to~$q$ while reading that letter and writing the annotation~$\oout$;
	%
          %
	\item {\em Read-only transitions} of the form $(p, a, q) \in
          Q\times\Sigma\times Q$, meaning that the annotator can go from $p$ to
          $q$ while reading~$a$;
          %
	\item {\em Push transitions} of the form $(p, q, \gamma) \in Q\times (Q\times\Gamma)$, meaning that the annotator can go from $p$ to $q$ while pushing the symbol~$\gamma$ on the stack;
	\item {\em Pop transitions} of the form $(p, \gamma, q) \in (Q\times \Gamma)\times Q$, meaning that, if the topmost symbol of the stack is~$\gamma$, the annotator can go from $p$ to $q$ while removing this topmost symbol~$\gamma$.
	%
	%
\end{itemize}

%

We now give the semantics of PDAnns.  Fix a string $w = w_1 \cdots
w_n \in \Sigma^*$. A \emph{configuration} of $\cP$ over $w$ is a pair
$C = (q, i) \in Q\times [0,n]$ of the current state and position in
$w$. An \emph{instantaneous description} of $\cP$ is a pair
$(C, \alpha)$ where $C$ is a configuration and $\alpha\in \Gamma^*$
describes the stack. A \emph{run} of $\cP$ over $w$ is a sequence:

\begin{equation}  \label{gram:eq:run} %\tag{\!\dagger}
	\rho \ := \ (C_0, \alpha_0) \trans{t_1} (C_1, \alpha_1) \trans{t_2} \, \ldots \, \trans{t_m} (C_m, \alpha_m)
\end{equation}

such that $C_0 = (q_0, 0)$ and $\alpha_0 = \eps$, each $t_k$ is a transition in $\Delta$, and for each $k\in[1,m]$ the following hold: 
\begin{itemize}
	\item if $t_k$ is a read-write transition $(p, (a, \oout), q)$ or a read-only transition $(p, a, q)$, then $\alpha_k = \alpha_{k-1}$, $C_{k-1} = (p, i-1)$, $C_{k} = (q, i)$ and $a = a_i$ for some $i\in [1,n]$;
	\item if $t_k$ is a push transition $(p, q, \gamma)$, then $\alpha_k = \alpha_{k-1}\gamma$ and for some $i\in [1,n]$, $C_{k-1} = (p,i)$, $C_k = (q,i)$; and
	\item if $t_k$ is a pop transition $(p,\gamma,q)$, then  $\alpha_{k-1} = \alpha_k \gamma$ and for some $i\in [1,n]$, $C_{k-1} = (p,i)$, $C_k = (q,i)$.
\end{itemize}
We say that $\rho$ is \emph{accepting} if $(C_m, \alpha_m) = ((q_f, n), \epsilon)$ for
some $q_f\in F$.
We define the \emph{annotation} of $\rho$ as $\ann(\rho) = \ann(t_1, C_1) \cdot \cdots \cdot \ann(t_m, C_m)$ such that $\ann(t, C) = \epsilon$ if $t$ is a push, pop, or a  read-only transition $(p, a, q)$, and $\ann(t, C) = (i, \oout)$ if $t$ is a read-write transition $(p, (a, \oout), q)$ and $C = (q, i)$.
Finally, we define the function $\sem{\cP}$ that maps any $w \in \Sigma^*$ to its set of outputs:
\[
\sem{\cP}(w) =
\{\ann(\rho) \mid \text{$\rho$ is an accepting run of $\cP$ over $w$}\}.
\]
Similarly to annotated grammars, we say that $\cP$ is \emph{unambiguous} if for every $w\in \Sigma^*$ and output~$\mu$, there exists at most one accepting run $\rho$ of $\cP$ over $w$ such that $\ann(\rho) = \mu$. 

One can alternatively see a PDAnn as a \emph{pushdown transducer}~\cite{berstel2013transductions}, which is the standard way to extend automata to have an output. However, an important difference is that a PDAnn concisely represents outputs by only writing the annotations and their positions: this can be much smaller than the input string, and cannot easily be encoded as a transducer on a finite alphabet. For instance, where a PDAnn can produce an output such as $(2,\oout), (5, \oout')$, a transducer would either write $\oout \oout'$ (losing the position information) or \textvisiblespace~$\oout$~\textvisiblespace~\textvisiblespace~$\oout'$ (whose length is always linear in the input) for a special symbol~\textvisiblespace.

%
%


\paragraph{Profiled PDAnns and annotated grammars} To define the analogue
of rigid annotated grammars on PDAnn, we will study the
\emph{stack profile} (or simply \emph{profile}) of PDAnn runs, which is
informally the sequence of all stack heights.
Formally, let $\cP$ be a PDAnn, $w$ be a string, and consider a run $\rho$ of $\cT$
over~$w$ like in ($\dagger$).
The \emph{profile}~$\pi$ of $\rho$ is the sequence $\pi:= \card{\alpha_0}, \ldots,
\card{\alpha_m}$.
We then introduce \emph{profiled} PDAnns by requiring that all
accepting runs of the PDAnn on an input string have the same profile (no matter
their output). Formally, we say that a PDAnn $\cP$ is \emph{profiled} if, for every string $w$, all accepting runs of $\cT$ over~$w$ have the same profile.

As usual for context-free grammars and pushdown automata, the formalisms of annotated grammars and PDAnn have the same expressive power.
We call two annotated grammars
$\cG$ and $\cG'$ \emph{equivalent} if they define the same functions, i.e.,
$\sem{\cG} = \sem{\cG'}$, and extend this notion to PDAnn in the expected way.
We then have:
%
\begin{proposition}\label{gram:prop:grammar-pdann}
	Annotated grammars and PDAnn are equally expressive. Specifically, for any annotated grammar $\cG$, we can build an equivalent PDAnn $\cP$ in polynomial time, and vice versa.
	Further, $\cG$ is unambiguous (resp., rigid) iff $\cP$ is unambiguous (resp., profiled).
        %
\end{proposition}
\begin{proof}
  In this proof, we will need to use the standard notion of a \emph{pushdown
automaton} (PDA), whose definition has been omitted so far.
We give it here:

\begin{definition}
  \label{gram:def:pda}
  A \emph{pushdown automaton} (PDA) is a tuple $\cA = (Q, \Sigma, \Gamma, \Delta,
  q_0, F)$, where $Q$ is a finite set of \emph{states}, $\Sigma$ is the
  alphabet, $\Gamma$ is a finite alphabet of \emph{stack symbols}, $q_0 \in Q$
  is the \emph{initial state}, $F \subseteq Q$ are the \emph{final states}. We
  assume that $\Gamma$ is disjoint from~$\Sigma$. Further, $\Delta$ is a finite
  set of \emph{transitions} of the following kind:
  \begin{itemize}
    \item \emph{Read transitions} of the form $(p, a, q) \in Q \times \Sigma
      \times Q$, meaning that the automaton can go from state $p$ to
      state~$q$ while reading the letter~$a$;
    \item \emph{Push transitions} of the form $(p, q, \gamma) \in Q \times Q
      \times \Gamma$, meaning that the automaton can go from state $p$ to
      state~$q$ while pushing the symbol $\gamma$ on the stack;
    \item \emph{Pop transitions} of the form $(p, \gamma, q) \in Q \times \Gamma
      \times Q$, meaning that, if the topmost symbol of the stack is~$\gamma$,
      the automaton can go from~$p$ to~$q$ while removing this topmost
      symbol~$\gamma$.
  \end{itemize}
\end{definition}

We omit the definition of the semantics of PDAs, which are standard, and allow
us to define the \emph{language} $L(\cA)$ accepted by a PDA. It is also well-known
that CFGs and PDAs have the same expressive power, i.e., given a CFG $G$, we can
build in polynomial time a PDA $\cA$ which is \emph{equivalent} in the sense that
$L(G) = L(\cA)$, and vice-versa.

%

We will also need to use the standard notion of a \emph{deterministic} PDA (with
acceptance by final state). Formally:
\begin{definition}
  \label{gram:def:dpda}
  Let $\cA = (Q, \Sigma, \Gamma, \Delta, q_0, F)$ be a PDA. For $p \in Q$, we define the \emph{next-transitions} of $p$ as the set $\Delta(p)$ of all transitions in $\Delta$ that start on $p$, i.e.,  $\Delta(p) = \{(p, x, y) \mid (p, x, y) \in \Delta\}$.
  We say that a PDA $\cA$ is \emph{deterministic} if
%
%
%
%
%
%
%
%
%
%
%
%
%
%
%
for every state $q \in Q$, one of the following conditions hold:
\begin{itemize}
	\item[(a)] $\Delta(q) \subseteq Q \times \Sigma \times Q$ and $|\{q'\mid
          (q, a, q')\in \Delta\}| \leq 1$ for each $a\in\Sigma$. Informally, all
          applicable transitions are read transitions, and there is at most one such
          applicable transition for each letter.
	\item[(b)] $\Delta(q) \subseteq Q \times (Q \times \Gamma)$, and
          $|\Delta(q)| \leq 1$. Informally, all applicable transitions are push
          transitions, and there is at most one such transition from $q$.
	\item[(c)] $\Delta(q) \subseteq (Q \times \Gamma) \times Q$ and
          $|\{q'\mid (q, \gamma, q')\}| \leq 1$ for each $\gamma \in \Gamma$.
          Informally, all applicable
          transitions from~$q$ are pop transitions, and there is at most one
          such applicable transition for each stack symbol.
\end{itemize}
\end{definition}

It is clear that the definition ensures that, on every input string~$w$, a
deterministic PDA $\cA$ has at most one run accepting~$w$, so that we can
check in linear time in $\cA$ and $w$ if $w \in L(\cA)$. Further, it is known that
deterministic PDAs are strictly less expressive than general PDAs.


In order to prove the statement of Proposition~\ref{gram:prop:grammar-pdann}, let us first give the formal definitions needed for  the result.
We say that two annotated grammars
$\cG$ and $\cG'$ are \emph{equivalent} if they define the same functions, i.e.,
$\sem{\cG} = \sem{\cG'}$. We define equivalence in the same way for two PDAnns, or
for an annotated grammar and a PDAnn.

We first show one direction:

\begin{claim}
	\label{gram:prp:g2pdt}
	For any annotated grammar $\cG$, we can build an equivalent PDAnn $\cP$ in polynomial
	time. Further, if $\cG$ is unambiguous then so is $\cP$. Moreover, if $\cG$ is rigid, then $\cP$ is profiled.
\end{claim}

\begin{proof}
	This is a standard transformation. Let $\cG = (V, \Sigma, \Omega, P, S)$. We build a PDAnn $\cP = (Q, \Sigma, \Omega, \Gamma, \Delta, q_0, F)$ as follows: For every rule $X \rightarrow \alpha$ of
	$\cG$ and position $0 \leq i \leq \card{\alpha}$, the PDAnn $\cP$ has a state $(X,
	\alpha, i)$ in $Q$, plus a special state $q_0$ which is the only initial and only
	final state. Also, $\Gamma = Q$. The set $\Delta$ has
	the following transitions:
	\begin{itemize}
		\item A push transition $(q_0, (S, \alpha, 0), q_0)$ and a pop transition $((S, \alpha, \card{\alpha}), q_0, q_0)$, for
		every rule $S \rightarrow \alpha$.
		\item For each state $(X, \alpha, \card{\alpha})$ for a production $X
		\rightarrow \alpha$, a pop transition reading a state from the stack and
		moving to that state.
		\item For each state $(X, \alpha, i)$ where the $(i+1)$-th element of $\alpha$
		(numbered from~$1$) is a nonterminal $Y$, for every rule $Y \rightarrow \beta$,
		a pop transition pushing $(X, \alpha, i+1)$, and moving to $(Y, \beta, 0)$.
		\item For each state $(X, \alpha, i)$ where the $(i+1)$-th element of $\alpha$
		(numbered from~$1$) is a terminal $\tau$, a
		read transition moving to $(X, \alpha, i+1)$ reading the symbol of $\tau$
		and outputting the annotation of~$\tau$ (if any).
	\end{itemize}

We will show a function that maps a given leftmost derivation $S\der{\cG} \gamma_1\der{\cG}\ldots\der{\cG}\gamma_m = \hat{w}$ into a run in $\cP$. To do this, we convert is sequence of productions into a sequence of strings which has the same size as the run (minus one). These strings serve as an intermediate representation of both the derivation and the run. The process is essentially to simulate the run in $\cP$.
\begin{itemize}
	\item First, we reduce the derivation into a sequence of productions $X_1\der{}\gamma_1, X_2\der{}\gamma_2,\ldots, X_m\der{}\gamma_m$ which uniquely defines the derivation. 
	\item The alphabet in which we represent strings that produce other strings include two special markers $\downarrow$ and $\uparrow$.
	\item We start on the string $\downarrow S\uparrow$.
	\item If the current string is $\hat{u} \downarrow X \beta$, and it is the $i$-th one that has reached a string of this form, then it must hold that $X = X_i$. We follow it by $\hat{u} \downarrow \gamma_i\uparrow\beta$.
	\item If the current string is $\hat{u}\downarrow \tau\beta$, for some terminal $\tau$, we follow it by $\hat{u} \tau \downarrow \beta$.
	\item If the current string is $\hat{u}\downarrow\uparrow\beta$, then we follow it by $\hat{u}\downarrow\beta$.
	\item If the current string is $\hat{u}\downarrow$, there is no follow up.
\end{itemize}
Interestingly, this function is completely reversible, since to obtain a sequence of productions from a sequence of strings in this model, all we need to do is to remove the markers 
$\downarrow$ and $\uparrow$ and eliminate the duplicate strings that appear.
We will borrow the name splain to talk about the function which receives a string and returns one which deletes all markers. 
It is obvious that the resulting derivation is the original one. 

Furthermore, and more interestingly, we can extend the function $\shape$ to receive one of these strings and return a string in the alphabet $\{0, 1, \downarrow, \uparrow\}$. For two derivations that have the same shape, the resulting sequences have the same shape as well.

This sequence of strings represents a run in $\cP$ almost verbatim, and we only need to adapt it into a sequence of pushes, pops and reads: We make a run $\rho$ which starts on $q_0$, pushes $(X, \gamma_1, 0)$ to the stack, and moves to the state $(X, \gamma_1, 0)$. This pairs exactly to the strings $\downarrow S$ and $
\downarrow \gamma_1\uparrow$, which are the first two in the sequence. Then, we read the sequence of strings in order. If the current string is $\hat{u}\downarrow X\beta$, and this is the $i$-th time a string of this form is seen, then the current state must be $(Y, \alpha_1 X_i \alpha_2, k)$, where $|\alpha_1| = k$; we push $(Y, \alpha_1 X_i \alpha_2, k+1)$ onto the stack, and move to the state $(X_i, \gamma_i, 0)$. If the current string is $\hat{u}\downarrow a \beta$ for some $a\in \Sigma$, and the current state is $(X, \gamma, k)$, we read $\tau$, and move to the state $(X, \gamma, k)$. If the current string is $\hat{u}\downarrow (a, \oout) \beta$ for some $a\in \Sigma$, and the current state is $(X, \gamma, k)$, we read $a$, output $\oout$, and move to the state $(X, \gamma, k)$. If the current string is $\hat{u}\downarrow\uparrow\beta$, we pop the topmost state from the stack 
  %\antoine{pop the topmost state from the stack? (otherwise we never pop)}
  and we move into that state. It is straightforward to see that this run represents exactly the leftmost derivation $S\der{\cG}^* \hat{w}$, and that for each annotated string $\hat{w}\in L(\cG)$ if and only if there is a run of $\cP$ over $w$ that produces $\mu = \ann(\hat{w})$ as output.

This function is also reversible. Consider a run of $\cP$ over a string $w$ which produces $\mu$ as output. This run must start on $q_0$, and then push $q_0$ and move onto a state $(S, \alpha, 0)$ for some rule $S \to \alpha$. Thus, our first two strings in the sequence are $\downarrow S\uparrow$ and $\downarrow \alpha\uparrow\uparrow$. If the current state is $(X, \alpha, k)$ and the next transition is to push $(X, \alpha, k+1)$ onto the stack to move into the state $(Y, \gamma, 0)$, then the current string is of the form $\hat{u}\downarrow X \beta$, so we follow it by the string $\hat{u}\downarrow \gamma\uparrow\beta$. If the next transition is a pop, then the current string is $\hat{u}\downarrow\uparrow\beta$, so we follow it by $\hat{u}\downarrow\beta$. If the current transition is a read, then the current string is $\hat{u}\downarrow a\beta$ for $a\in\Sigma$, so we follow it by $\hat{u}\tau\downarrow\beta$. If the current transition is a read-write, then the current string is $\hat{u}\downarrow (a, \oout)\beta$ for $(a, \oout)\in \Sigma\times\Omega$, so we follow it by $\hat{u}(a, \oout)\downarrow\beta$. It can easily be seen that using the original function over this resulting sequence would give the original sequence back. We point that these two reversible functions mean that there is a one to one correspondence between derivations of $S\der{\cG}^* \hat{w}$ and accepting runs of $\cP$ over $w$ with output $\mu = \ann(\hat{w})$.

Similarly to the observation we made before, we notice that if we start on a sequence in the intermediate model, the profile of the resulting run $\rho$ is fully given by the shape of the sequence (at each step, the size of the stack will be equal to the number of markers $\uparrow$ present in the string).

Now assume that $\cG$ is unambiguous. Seeing that $\cP$ is unambiguous as well comes straightforwardly from the fact that the functions presented above are bijective.

Assume now that $\cG$ is rigid. Let $w$ be an unannotated string and consider two runs $\rho_1$ and $\rho_2$ of $\cP$ over $w$ which output $\mu_1$ and $\mu_2$ respectively. Convert these two runs into sequences $\cS_1$ and $\cS_2$ in the intermediate model. Note that if we convert these two sequences into derivations $S\der{\cG}^*\mu_i(w)$, they will have the same shape. We can apply the functions above to obtain the two runs $\rho_1$ and $\rho_2$ back, and note that they have the same profile. We conclude that if $\cG$ is rigid, then $\cP$ is profiled.
\end{proof}

We then show another direction:

\begin{claim}
	\label{gram:prp:pdt2g}
	For any PDAnn $\cP$, we can build an equivalent annotated grammar $\cG$ in polynomial
	time. Further, if $\cP$ is unambiguous then so is $\cG$. Moreover, if $\cP$ is profiled, then $\cG$ is rigid.
\end{claim}

\begin{proof}
	This is again a standard transformation. We first transform the input PDAnn $\cP = (Q, \Sigma, \Omega, \Gamma, \Delta, q_0, F)$ to
	accept by empty stack, i.e., to accept iff the stack is empty. To do this, we build an equivalent PDAnn $\cP' = (Q', \Sigma, \Omega, \Gamma', \Delta', q_0', F')$ where $Q' = Q\cup\{q_0', q_e, q_f\}$, $\Gamma' = \Gamma \cup \{\gamma_0\}$, $F = \{q_f\}$, and we add the following transitions to $\Delta$ to obtain $\Delta'$: A push transition $(q_0', q_0, \gamma_0)$, a pop transition $(q, \gamma_0, q_f)$ for every $q\in F$ (for runs that accept at a point where the stack is already
	empty), plus a pop transition $(q, \gamma, q_e)$ for any other $\gamma\in\Gamma$, and a pop transition $(q_e, q_0, q_f)$.
	
	This clearly ensures that there is a bijection between the accepting runs of
	$\cP$ and those of $\cP'$: given an accepting run~$\rho$ of
	$\cP$, the bijection maps it to an accepting run of $\cP'$ by
	extending it with a push transition at the beginning, and pop transitions at
	the end. Further, all accepting runs in $\cP'$ now finish with an empty
	stack, more specifically a run is accepting iff it finishes with an empty
	stack.
		
	Now, we can perform the transformation. The nonterminals of the grammar are
	triples of the form $(q, \gamma, q')$ for states $q$ and $q'$ and a stack symbol $\gamma$. Intuitively, $(p, \gamma, q')$ will derive the strings that can be
	read by the PDAnn starting from state $p$, reaching some other state $q$ with the same stack, not seeing the stack at all in the process, and then popping $\gamma$ to reach $q'$.
	
	The production rules are the following:
	
	\begin{itemize}
		\item A rule $S \to (q_0, \gamma_0, q_f)$.
		\item A rule $(p, \gamma, q') \rightarrow (q, \gamma', r)
		(r, \gamma, q')$ for every nonterminal $(p, \gamma, q')$, push transition $(p, q, \gamma')\in\Delta$ and state
		$r$.
		\item A rule $(p, \gamma, q)\to \eps$ for each pop transition $(p, \gamma, q)\in \Delta$.
		\item A rule $(p, \gamma, q')\to (a, \oout) (q, \gamma, q')$ for each read-write transition $(p, (a, \oout), q)$, and a rule $(p, \gamma, q')\to a (q, \gamma, q')$ for each read transition $(p, a, q)$, for each nonterminal $(p, \gamma, q')$.
	\end{itemize}

As we did in Proposition~\ref{gram:prp:g2pdt}, we will show a function which receives an accepting run $\rho$ over $w$ in $\cP$ with output $\mu$ and outputs a leftmost derivation $S =\alpha_1\der{\cG}\alpha_2\der{\cG}\ldots\der{\cG}\alpha_m = \mu(w)$. The way we do this is quite straightforward: There is a one-to-one correspondence between snapshots in the run to each $\alpha_i$. Indeed, it can be seen that $\alpha_i = \hat{u}(q_1, \gamma_1, q_2)(q_2, \gamma_2, q_3)\ldots(q_k, \gamma_k, q_f)$ for some string $\hat{u}\in (\Sigma\cup(\Sigma\times\Omega))^*$, some states $q_1,\ldots,q_k$ and stack symbols $\gamma_1, \ldots, \gamma_k$. Moreover, the $i$-th stack in the run is equal to $\gamma_1\gamma_2\ldots\gamma_k$, whereas each state $q_j$ is the first state that is reached after popping the respective $\gamma_{j-1}$. We see that this function is fully reversible, as each production corresponds unequivocally to a transition in particular. This implies that $\cP$ is unambiguous if, and only if $\cG$ is unambiguous.

For the next part of the proof, we bring attention to the fact that there are exactly four possible shapes on the right sides of the rules in $\cG$. Each of these directly map to some type of transition, be it the initial push transition $(q_0', q_0, \gamma_0)$, a different push transition, a pop transition, or a read (or read-write) transition. To be precise, these shapes are the strings $1$, $11$, $\eps$ and $01$ respectively. From here it can be easily seen that, while comparing a run $\rho$ to is respective derivation $S\der{\cG}^*\hat{w}$, each production in the run immediately tells which type of transition was taken, and each transition in the run immediately tells which rule (and therefore, rule shape) was used. Therefore, each derivation shape maps to exactly one stack profile and vice versa, from which we conclude that $\cP$ is profiled if, and only if, $\cG$ is rigid.
\end{proof}
	
The proof now follows from Claims~\ref{gram:prp:g2pdt} and~\ref{gram:prp:pdt2g},
\end{proof}

%
%
Let us now study the enumeration for PDAnns. We know that the problem
for unambiguous PDAnns can be solved via Proposition~\ref{gram:prop:grammar-pdann}
with cubic-time preprocessing in data complexity and output-linear delay (with
Theorem~\ref{gram:thm:cubic}). We know that
profiled PDAnns can be made unambiguous (via Proposition~\ref{gram:prop:grammar-pdann} and Theorem~\ref{gram:thm:profileu-iou}) and so that
%
we can solve enumeration for them in
quadratic-time preprocessing in data complexity and output-linear delay (using
Theorem~\ref{gram:thm:quadratic}).
We now show that, if we are given a profile of an 
%
%
unambiguous PDAnn $\cP$ on an input string~$w$, we can use it as a guide to enumerate with linear
preprocessing and output-linear delay the set $\sem{\cP}_{\pi}(w)$ of
annotations for that profile, i.e., all $\ann(\rho)$ such that $\rho$ is an
accepting run with profile $\pi$ of $\cP$ over $w$. Formally:

%
%
%

\begin{lemma}\label{gram:lem:vpaconv}
  %
	Given an unambiguous PDAnn $\cP$, there exists an enumeration algorithm
        that receives as input a string $w$ and a profile $\pi$ of $\cP$ over
        $w$, and enumerates $\sem{\cP}_{\pi}(w)$ with output-linear delay after
        linear-time preprocessing in data complexity.
\end{lemma}
\begin{proof}
  We will show a linear-time reduction to enumeration for an unambiguous VPT (see Chapter~\ref{ch1}).
We will state the necessary preliminaries to use the result presented there. We will also adapt the models slightly to fit our results better while also keeping the results trivially equivalent.

%
%

A structured alphabet is a triple $(\opS, \clS, \noS)$ consisting of three disjoint sets $\opS$, $\clS$, and $\noS$ that contain {\it open}, {\it close}, and {\it neutral} symbols respectively.
%
%
The set of {\em well-nested strings}
over $\Sigma$, denoted as $\wnS$, is defined as the closure of the following rules: 
$\noS \cup \{\eps\} \subseteq \wnS$,
if $w_1, w_2 \in \wnS\setminus\{\eps\}$ then $w_1 \cdot w_2 \in \wnS$, and if $w \in \wnS$ and $a \in \opS$ and $b \in \clS$ then $a\cdot w\cdot b \in \wnS$. 
%
%

A \emph{Visibly Pushdown Transducer} (VPT) is a tuple $\cT = (Q, \hat{\Sigma}, \Gamma, \Omega, \Delta, I, F)$ where $Q$ is a state set, $\hat{\Sigma}$ is a structured alphabet, $\Gamma$ is set of a stack symbols, $\Omega$ is the output alphabet, $I$ is the set of initial states, $F$ is the set of final states, and 
$
%
%
%
%
%
%
\Delta \subseteq  
(Q \times (\opS \cup \opS \times \Omega) \times Q \times \Gamma)  \cup 
(Q \times (\clS \cup \clS \times \Omega) \times \Gamma \times Q)  \cup 
(Q \times (\noS \cup \noS \times \Omega) \times Q)
$
is the transition relation.
A run $\rho$ of $\cT$ over a well-nested string $w = a_1 a_2\cdots a_n \in\wnSann$ %
is a sequence of the form
$
%
\rho = (q_1, \sigma_1) \xrightarrow{s_1} \ldots  \xrightarrow{s_n} (q_{n+1}, \sigma_{n+1})
$
where $q_i \in Q$, $\sigma_i\in \Gamma^{*}$, $q_1 \in I$, $\sigma_1 = \eps$, each $s_i$ is either equal to $a_i$, or to $a_i/\!\oout_i$ for some $\oout_i\in\Omega$, and for every $i\in[1,n]$ the following holds:
\begin{enumerate}
  \item If $s_{i} = a\in \opS$, then $(q_i, a,q_{i+1},\gamma) \in \Delta$, and if $s_i = a/\!\oout$, for $a\in \opS$, then $(q_i, (a,\oout),q_{i+1},\gamma) \in \Delta$, for some $\gamma\in\Gamma$ with $\sigma_{i+1} = \gamma\sigma_i$,
  \item If $s_{i} = a\in\clS$, then $(q_i, a, \gamma, q_{i+1}) \in \Delta$, and if $s_i = a/\!\oout$, for $a\in \clS$, then $(q_i, (a, \oout), \gamma, q_{i+1}) \in \Delta$, for some $\gamma\in\Gamma$ with $\sigma_i = \gamma\sigma_{i+1}$, and
  \item If $s_i = a\in\noS$, then $(p_i, a,q_{i+1})\in \Delta$, and if $s_i = a/\!\oout$ for $a\in\noS$, then $(p_i, (a,\oout),q_{i+1})\in \Delta$ with $\sigma_i = \sigma_{i+1}$. We say that the run is accepting if $q_{n+1}\in F$.
\end{enumerate}
Given a VPT $\cT$ and a $w \in\wnS$, we define the set $\sem{\cT}(w)$ of all outputs of $\cT$ over $w$ as:
$
\sem{\cT}(w) \ = \ \{ \ann(\rho) \, \mid \, \text{$\rho$ is an accepting run of $\cT$ over $w$}\}.
$
where $\ann$ for runs in VPT is defined analogously to PDAnn. That is, if $\rho = (q_1, \sigma_1) \xrightarrow{s_1} \ldots  \xrightarrow{s_n} (q_{n+1}, \sigma_{n+1})$, then $\ann(\rho) = \omega_1\ldots\omega_n$ where $\omega_i = (\oout, i)$ if $s_i = a/\!\oout$ and $\omega_i = \eps$ otherwise.

We say that $\cT$ is \emph{unambiguous} if for every $w$ and $\mu$ there is at most one accepting run $\rho$ of $\cT$ which produces $\mu$. In Chapter~\ref{ch1} these VPT are called input-output unambiguous.

The theorem we use can be stated as follows:

\begin{theorem}{(Theorem~\ref{nested:theo:main})}\label{gram:eval:prep}
	There is an algorithm that receives an unambiguous VPT $\cT = (Q, \hat{\Sigma}, \Gamma, \Omega, \Delta, q_0, F)$ and an input string $w$, and enumerates the set $\sem{\cT}(w)$ with output-linear delay after a preprocessing phase that takes $\cO(|Q|^2 \cdot \vert\Delta\vert \cdot |w|)$ time.
\end{theorem}

The rest of the proof will consist on showing a linear-time reduction from the problem of enumerating the set $\sem{\cP}_{\pi}(w)$ for an unambiguous PDAnn $\cP$ and input string $w$ to the problem of enumerating the set $\sem{\cT}(w')$ for an unambiguous VPT $\cT$, and input string $w'$.

Let $\cP = (Q, \Sigma, \Omega, \Gamma, \Delta, q_0, F)$ and let $w\in \Sigma^*$ be an input string.
Consider the structured alphabet $\hat{\Sigma} = (\{\texttt{<}\}, \{\texttt{>}\}, \Sigma)$ for some $\texttt{<}, \texttt{>}\not\in\Sigma$. Assume $\pi = \pi_1, \ldots, \pi_m$.
We construct a well-nested string $w' = b_1\cdots b_{m-1}$ where $b_i = 
\texttt{<}$ if $\pi_i > \pi_{i+1}$, $b_i = \texttt{<}$ if $\pi_i < \pi_{i+1}$, and $b_i  = w_j$ otherwise, where $i$ is the $j$-th index in which $\pi_i = \pi_{i+1}$.
We also build a table $\mathsf{Ind}$ such that $\mathsf{Ind}(i) = j$ for each of the indices in the third case.
We build a VPT $\cT = (Q, \hat{\Sigma}, \Gamma, \Omega, \Delta', I, F)$ where $I = \{q_0\}$ and we get $\Delta'$ by replacing every push transition $(p, q, \gamma)\in \Delta$ by $(p, \texttt{<}, q, \gamma)$ and every pop transition $(p,\gamma,q)\in\Delta$ by $(p, \texttt{>}, q, \gamma)$. Note that read and read-write transitions are untouched.

Let $w$ be an input string, and let $\mu$ be an output. Consider the output $\mu'$ which is obtained by shifting the indices in $\mu$ to those that correspond in $w'$. We argue that for each run $\rho$ of $\cP$ over $w$ with profile $\pi$ which produces $\mu$, there is exactly one run $\rho'$ of $\cT$ over $w'$ which produces $\mu'$, and vice versa. We see this by a straightforward induction argument on the size of the run. This immediately implies that for each output $\mu\in\sem{\cT}(w)$ there exists exactly one output $\mu'\in\sem{\cP}_{\pi}(w)$, which has its indices shifted as we mentioned.
The algorithm then consists on simulating the procedure from Theorem~\ref{gram:eval:prep} over $\cT$ and $w'$, and before producing an output $\mu$, we replace the indices to the correct ones following the table  $\mathsf{Ind}$. The time bounds are unchanged since the table $\mathsf{Ind}$ has linear size in $m$, and replacing the index on some output $\mu$ can be done linearly on $|\mu|$. We conclude that there is algorithm that enumerates the set $\sem{\cP}_{\pi}(w)$ with output-linear delay after a preprocessing that takes $\cO(|Q|^2\cdot \vert\Delta\vert\cdot |\pi|)$ time.

\end{proof}

This result implies that we could achieve linear-time enumeration over profiled
PDAnn if we could easily discover their (unique) profile.
We achieve this in \emph{deterministically profiled PDAnns}.

\paragraph{Deterministically profiled PDAnn}
%
%
%
%
%
%
%
%
%
%
%
%
%
%
%
%
%
%
%
%
%
%
%
%
%
%
%
%
%
%
%
%
Let $\cP = (Q, \Sigma, \Omega, \Gamma, \Delta, q_0, F)$ be a PDAnn. We
say that a PDAnn $\cP$ is \emph{deterministically profiled} if, for any
string $w \in \Sigma^*$, for any two partial runs $\rho$ and
$\rho'$ of $\cP$ over $w$ with the same length, $\rho$ and $\rho'$ have the same profile.

The relationship between deterministically profiled PDAnns and 
deterministic pushdown automata 
%(formally defined in Appendix~\ref{gram:apx:linear}) 
is similar to the relationship between 
profiled PDAnns and unambiguous pushdown automata (the latter relationship was
stated as
Proposition~\ref{gram:prp:baseunambig} in the context of grammars). 

Specifically:

\begin{proposition}
  \label{gram:prp:basedeterm}
  For a deterministically profiled PDAnn $\cP$, let $L'$ be the set of strings with nonempty output, i.e., $L' = \{w \mid \sem{\cP}(w) \not = \emptyset\} $. Then $L'$ is
  recognized by a deterministic pushdown automaton.
\end{proposition}
\begin{proof}
  
For this result, we use the notion of PDA (Definition~\ref{gram:def:pda}) and
deterministic PDA (Definition~\ref{gram:def:dpda}) that were presented inside a previous proof.

As we have done in previous proofs, the strategy consists on starting with a profiled-deterministic PDAnn $\cP$, and building a PDAnn $\cP'$ by eliminating the output symbols from each transition. This PDAnn behaves almost identically to a pushdown automaton $\cA$ in the sense that if $w\in L(\cA)$,  then $\sem{\cP}(w) = \{\eps\}$, and that if $w\not\in L(\cA)$ then $\sem{\cP}(w) = \emptyset$. Whenever this holds, we say that the PDAnn and the pushdown automaton are \emph{equivalent}. It is simple to see that for this $\cA$ it holds that $L(\cA) = L'$. To conclude the proof, we must show that $\cA$ can be made deterministic. Without loss of generality, we remove from $\cA$ 
%
all \emph{inaccessible states}, i.e., all states for which there is no run that goes to the state.

First, we will prove that the PDAnn $\cP'$ is profiled-deterministic. Let $w$ be a string in $\Sigma^*$ and let $\rho_1'$ and $\rho_2'$ be two partial runs of $\cP'$ over $w$ with the same profile, and with last configurations $(q, i)$ and $(q', i)$. There clearly exist partial runs $\rho_1$ and $\rho_2$ of $\cP$ over $w$ with the same profile, which can be obtained by replacing each transition by one of the transitions in $\cP$ it was replaced by. Since $\cP$ is profile-deterministic, then one of the following must hold in $\cP$: (1) $\Delta(q) \cup \Delta(q') \subseteq Q \times (\Sigma \cup \Sigma \times
\Omega) \times Q$, i.e., all transitions from~$q$ and $q'$ are read or
read-write transitions; (2) $\Delta(q) \cup \Delta(q') \subseteq Q \times (Q \times \Gamma)$, i.e.,
all transitions from~$q$ and~$q'$ are push transitions; or (3) $\Delta(q) \cup \Delta(q') \subseteq (Q \times \Gamma) \times Q$, i.e., 
all transitions from $q$ and $q'$ are pop transitions. Note that if (2) or (3) hold, then in the new PDAnn $\cP$ the condition holds again in $\cP'$ trivially since none of the transitions in $\Delta(q)$ and $\Delta(q')$ was changed. Moreover, if (1) holds, then it can be seen that all of the transitions that belonged in $Q\times (\Sigma\times\Omega)\times Q$ now belong in $Q\times \Sigma\times Q$, which also leaves the condition unchanged in $\cP'$. We conclude that $\cP'$ is profiled-deterministic.

The next step is to use Lemma~\ref{gram:lem:pdtonechoice} from $\cP'$ to obtain an equivalent PDAnn $\cP''$ which is deterministic-modulo-profile. We will argue that if we start with $\cP'$, which was profiled-deterministic, then the resulting $\cP''$ is equivalent to a pushdown automaton $\cA'$ which is also deterministic. Let $w$ be an input string in $\Sigma^*$ and let $\rho''$ be a partial run of $\cA$ over $w$ with last configuration $(S, i)$ and with topmost symbol on the stack $T$. Let us recall what $\cP''$ being deterministic-modulo-profile entails that the following conditions hold:
  \begin{enumerate}
	\item There is at most one push transition that
	starts on $S$; formally, we have:
        \[\card{\{S', T \in Q'' \times \Gamma'' \mid
                (S, S', T) \in \Delta\}} \leq 1.\]
	\item There is at most
	one pop transition that starts on $S,T$; formally, for each $\gamma$, we
        have:
        \[\card{\{S'
                \in Q \mid (S, \gamma, S') \in \Delta\}} \leq 1.\]
	\item For each letter $a$, and output $\oout \in \Omega$, there is at most one read-write transition that starts on
          $S,a,\oout$; formally, we have \[\card{\{S' \in Q'' \mid (S,(a,\oout),S') \in \Delta''\}}
        \leq 1.\]
	\item For each letter $a$, there is at most one read transition that starts on
          $S,a$; formally, we have: \[\card{\{q \in Q'' \mid (S,a,S') \in \Delta''\}}
        \leq 1.\]
\end{enumerate}
We will show that at most one of these conditions holds. Recall that in the transformation, the states of $\cP''$ are sets which contain pairs of states $(p,q)\in Q'\times Q'$, and the stack symbols are triples $(p, \gamma, q)\in Q'\times\Gamma' \times Q'$. Now, recall the claim that was proven in the lemma, on the backwards direction:

If $\cP''$ has a run $\rho''$ on a string $w$, producing output $\mu$, from its initial state to an instantaneous
description $(S, i), \alpha'$ with $\alpha' =
T_0, \ldots, T_m$ being the sequence of the stack
symbols, then for any choice of elements $(q_0, \gamma_0, p_0) \in T_0$,
$(p_0, \gamma_1, p_1) \in T_1$, ..., $(p_{m-1}, \gamma_m, p_m) \in T_m$
and $(p_m, q) \in S$ it holds that $\cP'$ has a run $\rho'$ on $w$ producing output $\mu$
from some initial state $q_0$ to the instantaneous description $(q, i), \alpha$ with
$\alpha = \gamma_0 q_0, \ldots, \gamma_m q_m$ (writing next to each
stack symbol the state that annotates it), and $\rho''$ and $\rho'$ have
the same profile. 

Since $\cP'$ is profiled-deterministic, then each run $\rho'^+$ which continues $\rho'$ by one step must have the same shape. This implies that exactly one of the following conditions must hold:
\begin{itemize}
	\item The last transition in $\rho'^+$ is a read or read-write transition. Therefore, all transitions from $q$ are either read or read-write transitions.
	\item The last transition in $\rho'^+$ is a push transition. Therefore, all transitions from $q$ are pop transitions.
	\item The last transition in $\rho'^+$ is a pop transition. Therefore, all transitions from $q$ are pop transitions.
\end{itemize}
 Assume bullet point 1 holds. Note that there are no read-write transitions in $\cP'$ so there are only read transitions. From here, we prove that only (4) is true simply by inspecting the transformation in the lemma; if (1) held, then there would be a push transition from $q$ in $\cP'$, if (2) held then there would be a pop transition from $q$ and $\gamma$ in $\cP'$, and (3) never holds. Now, assume bullet point 2 holds. From here, we prove that only (1) can be true; if (2) held, then there would be a pop transition from $q$ and $\gamma$ in $\cP'$, if (4) held, then there would be a read transition from $q$ in $\cP'$, and again, (3) is never true. Lastly, assume bullet point 3 holds. From here, we prove that only (2) can be true; if (1) held, then there would be a push transition from $q$ in $\cP'$, if (4) held, then there would be a read transition from $q$ in $\cP'$, and yet again, (3) is never true. We conclude that from the 4 points, at most one of these can be true at the same time.

%
Now we prove that the equivalent PDA $\cA$ is deterministic.
Let $q$ be a state of~$\cA$. As all states of~$\cA$ are accessible, pick $\rho''$ to be a run that reaches state~$q$. We have argued that at most one of the points in the list above is true of~$\cP'$, and it cannot be point (3). Now,
we see that (a) is equivalent to (4), that (b) is equivalent to (1) and (c) is equivalent to (2). Since only one of the conditions among (1), (2) or (4) can be true, the same holds for (a), (b) and (c), from which we conclude that $\cA$ is deterministic. This completes the proof.

\end{proof}

%
%
%
%

This result gives a concrete picture of the expressive power of 
deterministically profiled PDAnn $\cA$, i.e., as acceptors they are more powerful
than the class of
\emph{visibly pushdown
automata}~\cite{alur2004visibly}, where each alphabet letter must have a
specific effect on the profile. Deterministically profiled PDAnn are also
reminiscent of 
%
%
%
%
%
%
%
%
%
%
%
%
%
%
%
%
the \emph{height-determinism} notion introduced for pushdown automata~\cite{nowotka2007height}, but
extend this with the support of annotations.
%

%
%


%
%
%
%
%
%
%
%


%
%

%
%
%
%
%
%
%
%
%
%
%
%
%
%
%
%
%
%
%
%
%

Deterministically profiled PDAnn are designed to ensure that they have only one
profile (i.e., they are profiled), and further that their unique
profile can be constructed in linear time:

\begin{proposition}\label{gram:prp:detlinear}
  A deterministically profiled PDAnn $\cP$ is always profiled, and given a string
  $w$, the unique profile of accepting runs of $\cP$ over~$w$ can be computed in
  linear time in~$w$.
\end{proposition}
\begin{proof}
  
Consider a deterministically profiled PDAnn $\cP$. To prove that it is profiled, consider an input string $w\in \Sigma^*$. We will prove by a simple induction argument that any two runs of $\cP$ over $w$ have the same profile. The base case is trivial since the run is of length 0, and the profile up to now is composed simply of the stack size 0. Assume now that for each pair of runs $\rho$ and $\rho'$ of $\cP$ over $w$ of size $k$, that they have the same profile. We will show that for every pair of runs $\rho_1$ and $\rho_2$ over $w$ of size $k+1$, they have the same profile as well. Note that the runs $\rho_1^-$ and $\rho_2^-$ that are obtained by removing the last step have the same profile, by the hypothesis. From the definition of deterministically profiled it can be directly seen that if (1) the last transition in $\rho_1$ is a read or read-write transition, then for the runs $\rho_1^-$ and $\rho_2^-$, the only choices are read or read-write transitions, from which we deduce that the last transition in $\rho_2$ is a read or read-write transition as well, if (2) the last transition in $\rho_1$ is a push transition, then for the run $\rho_1^-$ and $\rho_2^-$ the only choice are push transitions, and therefore the last transition in $\rho_2$ has to be a push transition as well, and if (3) the last transition in $\rho_1$ is a pop transition, then for $\rho_1^-$ and $\rho_2^-$ the only choices are pop transitions, so the last transition in $\rho_2$ must be a pop transition as well. We obtain that $\rho_1$ and $\rho_2$ have the same profile, and from the induction argument, we conclude that $\cP$ is profiled.

Now, consider a deterministically profiled PDAnn $\cP$ and an input string $w$. We will prove that the unique profile of accepting runs of $\cP$ over $w$ can be computed in linear time in $|w|$. The way we do this is by using the pushdown automaton $\cA$ that was constructed in Proposition~\ref{gram:prp:basedeterm}. By inspecting the proof, it can be seen that the unique profile of $\cP$ over $w$ is maintained throughout the construction. Indeed, the first construction simply removes the output symbols, which does not affect the profile, and the second construction has an invariant that keeps the profile intact as well. Therefore, by running the automaton $\cA$ over $w$, and storing the stack sizes at each step, we obtain a profile $\pi$ which is exactly the same profile of the accepting runs of $\cP$ over $w$. To finish the proof, we only need to argue that this profile has linear size on $|w|$ (from a data complexity perspective). This follows from the fact that any run of a deterministic pushdown automaton $\cA$ over a string $w$ has $\cO(f(\cA)\times|w|)$ length, for some computable function $f$. This can be seen from a counting argument: (1) There is a maximum stack size $k$ that can be reached in an accepting run of $\cP$ over $w$ from an empty stack through $\eps$-transitions, which is given by the number of states in $\cA$. Otherwise, there are two configurations which are reachable from one another in a way such the stack, as it was at the first configuration, is not seen. This implies that there is a loop, and since $\cA$ is deterministic, $\cA$ does not accept $w$. (2) From a given stack, the maximum numbers of steps that can be taken without reading from $w$, and without seeing the topmost symbol on the stack is given by the number of possible stacks of size $k$. (3) Between a read (or read-write) transition and the next one, the maximum height difference is $k$, and if we move out of a read (or read-write) transition with a certain stack, from (2) we can see that we can only do a fixed number of steps before consuming some symbol from this stack, and therefore, the number of steps is bounded by a factor depending on $\cA$ multiplied by the size of the stack up until this point, which is linear on the number of symbols in $w$ read so far. We conclude that $w'$ has size linear on $w$, from a data complexity point of view.


\end{proof}

Together with Lemma~\ref{gram:lem:vpaconv}, this yields:

\begin{corollary}\label{gram:cor:linear-time-pdann}
	Let $\cP$ be a deterministically profiled PDAnn. Then for every string $w$
        the set $\sem{\cP}(w)$ can be enumerated with output-linear delay after
        linear-time preprocessing in data complexity.
\end{corollary}

