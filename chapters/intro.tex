%1) Big data, efficient evaluation, linear time over the data. 
%
%2) Query evaluation, query languages Q, model this as Q(D). Evaluation in |D|.
%
%3) Q(D) is only one part of the story, also we need to talk about results. You want to evaluate and enumerate the output hopefully in time f(Q,D) + |O|.
%
%4) Talk about constant delay as a more efficient notion of enumeration guarantee.
%
%5) Let's instatiate this idea in a more concrete query language, specifically, in this work a good starting point is to talk about query lanuguages based on MSO. Hablar sobre MSO, XML, JSON, bases de datos de grafos, etc.
%
%6) Spanners como un ejemplo especial y importante para este trabajo. 
%
%7) (parráfo corto) Concluir que en todos estos casos, al final, la consulta se puede expresar como un automata (en algún modelo) y los datos como un string. 
%
%8) Evaluación eficient con enumeración de automatas ya se ha estudiado harto y teniendo grandes resultados. Por ejemplo, Bagan, Antoine, Spanners, CEP. 
%
%9) Todos los resultados anteriores no consideran algún nivel de recursión en los automatas como visibly, grammars. Es sabido que visibly y gramaticas también tienen buenas propiedades algoritmicas, pero esto es solo para el caso booleano. Creemos que podemos estos resultados a automatas o datos donde hay cierto nivel de recursión.

The term {\it Big Data} is believed to have been originated in the mid-1990s, back when the Internet was utilized by some tens of millions of users. Even then, as the usage of the term suggests, there was a distinct necessity of processing tremendous amounts of data efficiently. In 2023, this number grew to over five billion users, and the sheer amount of data online has grown accordingly. It is increasingly more common that data processing tasks feature files of gigabytes or terabytes of size. These require sharply efficient solutions, to the point of making any solution that takes more than linear time useless in practice.

In database theory, it is common to talk about {\it query evaluation} as the most elementary task which extracts relevant information from data. It can be formally defined by an instance $(Q, D)$ in which $Q$ is the query and $D$ is the data. Each query $Q$ also defines a function which maps $D$ into the desired set of outputs $Q(D)$.
With this formalization, one can talk about linear-time solutions as those which take $O(|D|)$ time. 
Furthermore, it is also useful to frame results around {\it query languages}, which define classes of queries by their syntax and semantics.

When measuring the execution time of a evaluation algorithm, the classic notion is to start a hypothetical timer the moment the algorithm begins, and then stop it when the last output is printed. This is a reasonable model whenever the overall output size is small. However, for problems in which one can expect a large amount of outputs, which occur quite naturally in many query languages, it makes sense to treat the output printing phase separately. Naturally, the best execution time one can hope for the output phase is linear in the size of the entire output set, so the overall complexity that we will set as our goal is given by $f(Q,D) + |\text{Output}|$, for some function $f$.

In the realm of {\it enumeration algorithms}, there is an even finer complexity yardstick: We say that an algorithm has {\it output-linear delay} if the time that takes to print each output is linear in the size of said output. In the literature it is much more common to see {\it constant delay}, which in the sort of problems that we deal with implies that each output has constant size.

Now we can discuss the query space that we will work with in this thesis. All of our results are given for some sort of extension or improvement on Monadic Second Order over strings or trees. This query language is equivalent to regular automata, and subsumes standard queries over JSON and XML documents when it is queried over trees. Further, it can be applied to relational and graph query databases, and other sorts of common query tasks.

As an important example of a query language that is closely related to MSO we find Document Spanners. In its more fundamental definition, a Document Spanner is a function that maps a document to an output set that contains tuples of {\it spans}, which are pairs of positions in the document. They can easily model tasks of converting plain or semi-structured text into a relational database. Inside these, we find {\it Regular Spanners}, which can be directly expressed by an MSO extraction query, and were the basis of an early landmark in the literature of output-linear delay enumeration in Database Theory~\cite{FlorenzanoRUVV20}. It is not a stretch to say that this Thesis contains several extensions of this very result.

All the query tasks built around Document Spanners can be expressed by some extension of regular automata (the query), and a string (the data). A great part of the enumeration results for Document Spanners in the literature are described using the automata model, and this Thesis follows the same rationale to some extent.

Inside and outside Document Spanners, efficient enumeration for queries modeled via Automata has found great success. One of the earliest results in this realm was given in~\cite{bagan2006mso}, which was to enumerate all valid assignments for a given MSO formula. The papers~\cite{FlorenzanoRUVV20} and~\cite{amarilli2019constant}  give a parallel results which improves the delay to linear in the {\it support} of this assignment -- albeit restricted to Regular Spanners. The work in~\cite{Niewerth18} had a log-delay result for MSO queries over trees, which was later improved to output-linear delay~\cite{amarilli2020constant}. Output-linear delay enumeration for Complex Event Processing queries has also been realized by compiling them into a suitable automaton~\cite{GrezRU19,GrezR20,BucchiGQRV22}.

All of the aforementioned results deal in query languages that lack any strong form of recursion, namely, with the level of context-free grammars. The algorithmic properties of context-free grammars and pushdown automata have been widely studied, but any efficiently implementable queries have been limited to binary evaluation. The only enumeration result for CFG-modeled queries that we know of~\cite{peterfreund2019complexity} goes in the direction that we desire, although with a gap between the complexity of its binary counterpart. Our goal is to meet the enumeration results for automata and the binary decision algorithms for context-free grammars in the middle, and improve on any known results along the way.

To this end, In this thesis we present three different formalisms for information extraction and then provide an efficient enumeration scheme for queries expressed in each of them. 
These are: (1.) Automata over streams of nested documents, (2.) context-free grammars which represent annotations in documents and (3.) regular automata over compressed documents. 


%Enumeration algorithms are a refinement of evaluation algorithms in general. An evaluation problem describes the task of receiving an input and producing an output that satisfies some requirements. A typical evaluation algorithm would deal with this task by optimizing the time transcurred between receiving the input and producing the output. This is not helpful for a wide array of evaluation tasks that are of interest when processing large amounts of data. (ejemplo con numeros)



%Arguably the most fundamental problem in database research is query
%evaluation: given as input a query and data, we must find the results of the query over the data.
%Database theory research has studied the complexity of such problems for 
%decades.
%However, in some contexts, for instance over large datasets, the usual
%complexity measures are not well-suited to this study.
%Indeed, the number of query results might be so large that it is unreasonable in
%practice to produce all of them.
%Further, in theoretical terms, the complexity of an algorithm may be dominated by the cost of writing
%the full output, hiding the actual complexity of the
%computation.
%For this reason, a significant line of research on query evaluation has adopted the perspective of \emph{enumeration algorithms}.
%Instead of explicitly producing all results, the task is to \emph{enumerate} them, in any order and without repetition.
%The cost of the algorithm is then measured across two dimensions: the
%\emph{preprocessing time}, which is the time needed to read the input and
%prepare an enumeration data structure; and the \emph{delay}, the worst-case time
%elapsed between any two solutions while enumerating using the data structure.
%
%This study of enumeration algorithms has managed in some cases to achieve a
%\emph{constant-delay} guarantee. In this case, once the algorithm has
%preprocessed its input, the delay between any two outputs is \emph{constant},
%i.e., it is independent from the input. Of course, the challenge is to achieve
%this strong guarantee after a preprocessing phase that runs in a limited amount
%of time -- in particular, one that does not explicitly materialize all solutions.
%%
%Starting with the work of Durand and Grandjean~\cite{durand2007first}, researchers have designed
%constant-delay algorithms for several query evaluation problems, e.g., the evaluation of some queries over relational databases~\cite{bagan2007acyclic,kara2020trade}, query evaluation over dynamic data~\cite{berkholz2017answering,idris2017dynamic}, query evaluation over graph data~\cite{hartig2018semantics,kroll2016complexity}, among others~\cite{Segoufin13}. 
%
%One area where enumeration algorithms have been especially successful is the problem of \emph{information extraction}, studied through the lens of \emph{document spanners}~\cite{FaginKRV15}. In this data management task, the data is a textual document (i.e., a string), and the query is a declarative specification of information to extract from the text, formalized as a \emph{spanner}. The spanner describes \emph{mappings}, which are possible choices of how to map variables to substrings of the document (called spans). The enumeration problem is then to enumerate all mappings of a spanner on an input document, i.e., to enumerate efficiently all possible results for the information extraction task.
%%
%The work by Florenzano et al.~\cite{FlorenzanoRUVV18} showed that the task could be solved with preprocessing linear in the document and polynomial in a finite deterministic automaton describing the spanner, improving on a theoretical result by Bagan~\cite{bagan2006mso}; and 
%this was extended in~\cite{amarilli2019constant} to spanners described using nondeterministic automata or regular expressions.
%
%However, while regular spanners are natural, they do not capture all
%possible information extraction tasks. More
%expressiveness is needed for extraction over structured data (e.g., XML,
%or JSON documents), over the source code of programs, or possibly over
%natural language texts. 
