Acá le dedico una página y media a las personas que me acompañaron durante estos seis años (de cuatro) mientras realizaba mi doctorado y escribía esta tesis. Y por qué no, a los que me motivaron a seguir este camino.

El agradecimiento más importante es a Cristian que me recibió cuando no sabía nada y ahora puedo decir que sé algo más que nada. Me acompañó y guió sin dejarme en ningún momento, me mostró una pincelada de temas de investigación que resultaron ser profundos e interesantes a concho y que espero me van a mantener ocupado por un buen rato. También gracias a él y a Cristian Ruz por convertirme en coach y agregarle una dimensión tan llenadora a mi doctorado.

A Marcelo por haber estado ahí cuando yo sabía aún menos que nada y por el tiempo que me ha dado entonces y ahora.

Gracias a la PUC, al IMFD, a ANID y al Fondecyt de Cristian que me han mantenido con vida, y al comité de candidatura+defensa por el apoyo.

To Antoine and Louis for welcoming me in Paris twice, letting me join on some of the most inspiring research meetings of my career, and showing me what is hopefully a preview of the next few years of my life.

A la Selección de Programación Competitiva del DCC, y a toda la comunidad actual de Programación Competitiva en Chile por la motivación y el talento, por convertirse en un grupo de alumnos de los que puedo sentir orgullo de segunda mano y en un grupo de amistades con los que puedo seguir armando proyectos y juntas. 
Y de acá a Bella y Sensual por no parar de ganar hasta el final y llevarme a darle la vuelta al mundo.

A la comunidad antigua de programación competitiva, a Jorge Pérez, Cristian Ruz, a Marinkovic, a Pablo, Koco y a Lete. Ahí entendí cuál es la parte que importa de la computación.

Una mención honrosa a la máquina de café del DCC y a Magnet: ambos me dieron un empujón hacia el lado de la investigación y la academia en puntos cruciales de mi carrera.

A los de la oficina de doctorado, a Alda, Ceci, Yessenia, a Antonieta y a Cristian Ruz.

A los DCCompas por el apañe, a los cabros de siempre por estar ahí siempre y al AMQ.

Gracias a mi papá, mi mamá, a la Isa y al Luciano por todo lo demás, por no fallarme nunca y por ser mi centro.